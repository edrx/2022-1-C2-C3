% (find-LATEX "2022-1-C2-VSB.tex")
% (defun c () (interactive) (find-LATEXsh "lualatex -record 2022-1-C2-VSB.tex" :end))
% (defun C () (interactive) (find-LATEXsh "lualatex 2022-1-C2-VSB.tex" "Success!!!"))
% (defun D () (interactive) (find-pdf-page      "~/LATEX/2022-1-C2-VSB.pdf"))
% (defun d () (interactive) (find-pdftools-page "~/LATEX/2022-1-C2-VSB.pdf"))
% (defun e () (interactive) (find-LATEX "2022-1-C2-VSB.tex"))
% (defun o () (interactive) (find-LATEX "2022-1-C2-VSB.tex"))
% (defun u () (interactive) (find-latex-upload-links "2022-1-C2-VSB"))
% (defun v () (interactive) (find-2a '(e) '(d)))
% (defun d0 () (interactive) (find-ebuffer "2022-1-C2-VSB.pdf"))
% (defun cv () (interactive) (C) (ee-kill-this-buffer) (v) (g))
%          (code-eec-LATEX "2022-1-C2-VSB")
% (find-pdf-page   "~/LATEX/2022-1-C2-VSB.pdf")
% (find-sh0 "cp -v  ~/LATEX/2022-1-C2-VSB.pdf /tmp/")
% (find-sh0 "cp -v  ~/LATEX/2022-1-C2-VSB.pdf /tmp/pen/")
%     (find-xournalpp "/tmp/2022-1-C2-VSB.pdf")
%   file:///home/edrx/LATEX/2022-1-C2-VSB.pdf
%               file:///tmp/2022-1-C2-VSB.pdf
%           file:///tmp/pen/2022-1-C2-VSB.pdf
% http://angg.twu.net/LATEX/2022-1-C2-VSB.pdf
% (find-LATEX "2019.mk")
% (find-sh0 "cd ~/LUA/; cp -v Pict2e1.lua Pict2e1-1.lua Piecewise1.lua ~/LATEX/")
% (find-sh0 "cd ~/LUA/; cp -v Pict2e1.lua Pict2e1-1.lua Pict3D1.lua ~/LATEX/")
% (find-sh0 "cd ~/LUA/; cp -v C2Subst1.lua C2Formulas1.lua ~/LATEX/")
% (find-CN-aula-links "2022-1-C2-VSB" "2" "c2m221vsb" "c2vs")

% «.defs»		(to "defs")
% «.defs-T-and-B»	(to "defs-T-and-B")
% «.title»		(to "title")
% «.questao-1»		(to "questao-1")
% «.questao-1-grids»	(to "questao-1-grids")
% «.questoes-2-e-3»	(to "questoes-2-e-3")
% «.questao-4»		(to "questao-4")
% «.questao-1-gab»	(to "questao-1-gab")
% «.questao-2-gab»	(to "questao-2-gab")
% «.questao-3-gab»	(to "questao-3-gab")
% «.questao-4-gab»	(to "questao-4-gab")



% <videos>
% Video (not yet):
% (find-ssr-links     "c2m221vsb" "2022-1-C2-VSB")
% (code-eevvideo      "c2m221vsb" "2022-1-C2-VSB")
% (code-eevlinksvideo "c2m221vsb" "2022-1-C2-VSB")
% (find-c2m221vsbvideo "0:00")

\documentclass[oneside,12pt]{article}
\usepackage[colorlinks,citecolor=DarkRed,urlcolor=DarkRed]{hyperref} % (find-es "tex" "hyperref")
\usepackage{amsmath}
\usepackage{amsfonts}
\usepackage{amssymb}
\usepackage{pict2e}
\usepackage[x11names,svgnames]{xcolor} % (find-es "tex" "xcolor")
\usepackage{colorweb}                  % (find-es "tex" "colorweb")
%\usepackage{tikz}
%
% (find-dn6 "preamble6.lua" "preamble0")
%\usepackage{proof}   % For derivation trees ("%:" lines)
%\input diagxy        % For 2D diagrams ("%D" lines)
%\xyoption{curve}     % For the ".curve=" feature in 2D diagrams
%
\usepackage{edrx21}               % (find-LATEX "edrx21.sty")
\input edrxaccents.tex            % (find-LATEX "edrxaccents.tex")
\input edrx21chars.tex            % (find-LATEX "edrx21chars.tex")
\input edrxheadfoot.tex           % (find-LATEX "edrxheadfoot.tex")
\input edrxgac2.tex               % (find-LATEX "edrxgac2.tex")
%\usepackage{emaxima}              % (find-LATEX "emaxima.sty")
%
%\usepackage[backend=biber,
%   style=alphabetic]{biblatex}            % (find-es "tex" "biber")
%\addbibresource{catsem-slides.bib}        % (find-LATEX "catsem-slides.bib")
%
% (find-es "tex" "geometry")
\usepackage[a6paper, landscape,
            top=1.5cm, bottom=.25cm, left=1cm, right=1cm, includefoot
           ]{geometry}
%
\begin{document}

\catcode`\^^J=10
\directlua{dofile "dednat6load.lua"}  % (find-LATEX "dednat6load.lua")
%L dofile "Piecewise1.lua"           -- (find-LATEX "Piecewise1.lua")
%L dofile "QVis1.lua"                -- (find-LATEX "QVis1.lua")
%L dofile "Pict3D1.lua"              -- (find-LATEX "Pict3D1.lua")
%L dofile "C2Formulas1.lua"          -- (find-LATEX "C2Formulas1.lua")
%L dofile "Lazy5.lua"                -- (find-LATEX "Lazy5.lua")
%L dofile "2022-1-C2-P2.lua"         -- (find-LATEX "2022-1-C2-P2.lua")
%L Pict2e.__index.suffix = "%"
\pu
\def\pictgridstyle{\color{GrayPale}\linethickness{0.3pt}}
\def\pictaxesstyle{\linethickness{0.5pt}}
\def\pictnaxesstyle{\color{GrayPale}\linethickness{0.5pt}}
\celllower=2.5pt

% «defs»  (to ".defs")
% (find-LATEX "edrx21defs.tex" "colors")
% (find-LATEX "edrx21.sty")

\def\u#1{\par{\footnotesize \url{#1}}}

\def\drafturl{http://angg.twu.net/LATEX/2022-1-C2.pdf}
\def\drafturl{http://angg.twu.net/2022.1-C2.html}
\def\draftfooter{\tiny \href{\drafturl}{\jobname{}} \ColorBrown{\shorttoday{} \hours}}

% «defs-T-and-B»  (to ".defs-T-and-B")
% (c3m202p1p 6 "questao-2")
% (c3m202p1a   "questao-2")
\long\def\ColorOrange#1{{\color{orange!90!black}#1}}
\def\T(Total: #1 pts){{\bf(Total: #1)}}
\def\T(Total: #1 pts){{\bf(Total: #1 pts)}}
\def\T(Total: #1 pts){\ColorRed{\bf(Total: #1 pts)}}
\def\B       (#1 pts){\ColorOrange{\bf(#1 pts)}}



%  _____ _ _   _                               
% |_   _(_) |_| | ___   _ __   __ _  __ _  ___ 
%   | | | | __| |/ _ \ | '_ \ / _` |/ _` |/ _ \
%   | | | | |_| |  __/ | |_) | (_| | (_| |  __/
%   |_| |_|\__|_|\___| | .__/ \__,_|\__, |\___|
%                      |_|          |___/      
%
% «title»  (to ".title")
% (c2m221vsbp 1 "title")
% (c2m221vsba   "title")

\thispagestyle{empty}

\begin{center}

\vspace*{1.2cm}

{\bf \Large Cálculo 2 - 2022.1}

\bsk

VS extra - 31/ago/2022

\bsk

Eduardo Ochs - RCN/PURO/UFF

\url{http://angg.twu.net/2022.1-C2.html}

\end{center}

\newpage

%   ___                  _                _ 
%  / _ \ _   _  ___  ___| |_ __ _  ___   / |
% | | | | | | |/ _ \/ __| __/ _` |/ _ \  | |
% | |_| | |_| |  __/\__ \ || (_| | (_) | | |
%  \__\_\\__,_|\___||___/\__\__,_|\___/  |_|
%                                           
% «questao-1»  (to ".questao-1")
% (c2m221vsbp 2 "questao-1")
% (c2m221vsba   "questao-1")
% (c2m221p1p 8 "escadas-gab")
% (c2m221p1a   "escadas-gab")
% (c2m221p1p 7 "escadas-defs")
% (c2m221p1a   "escadas-defs")
%
%L -- (find-angg "LUA/Pict2e1-1.lua" "FromYs")
%L fryF = FromYs.fromys({0,-1,-2,-3, 3,2,1,0,-1,0,-1,0,-2,0}):getYs(1)
%L fryG = FromYs.fromys({0,-1,-2,-3, 3,2,1,0,-1,0,-1,0,-2,0}):getYs(3)
%L fryF:getypict():pgat("pgatc"):sa("fig f"):output()
%L fryF:getYpict():pgat("pgatc"):sa("fig F"):output()
%L fryG:getYpict():pgat("pgatc"):sa("fig G"):output()
%L -- fry:getYpict():pgat("pgatc"):sa("fig F"):output()
%L fryF:getYgrid(-4,4):
%L                pgat("pgatc"):sa("grid F"):output()
\pu

\unitlength=10pt

{\bf Questão 1}


\scalebox{0.7}{\def\colwidth{8.5cm}\firstcol{

\vspace*{-0.4cm}

\T(Total: 1.0 pts)

Seja
%
$$f(x) \;=\; \ga{fig f} \;.$$

a) \B (0.5 pts) Represente graficamente a função $F(x) = \Intt{2}{x}{f(t)}$.

\msk

b) \B (0.5 pts) Represente graficamente a função $G(x) = \Intt{3}{x}{f(t)}$.

\msk

Use os grids da próxima folha. Indique claramente qual desenho é a
resposta de cada item e quais desenhos são só rascunho.


}\anothercol{
}}

\newpage

% «questao-1-grids»  (to ".questao-1-grids")
% (c2m221vsbp 3 "questao-1-grids")
% (c2m221vsba   "questao-1-grids")

\def\figf{\ga{fig f} \phantom{m}}
\def\figF{\ga{fig F}}
\def\gridF{\ga{grid F}}

$\scalebox{0.85}{$
  \begin{array}{ll}
  \figf  & \figf  \\ \\
  \gridF & \gridF \\ \\
  \gridF & \gridF \\ \\
  \end{array}
  $}
$

\newpage

% «questoes-2-e-3»  (to ".questoes-2-e-3")
% (c2m221vsbp 4 "questoes-2-e-3")
% (c2m221vsba   "questoes-2-e-3")

\sa{[IP]}{\CFname{IP}{}}
\sa{[M]}{\CFname{M}{}}
\sa{[F]}{\CFname{F}{}}


\scalebox{0.55}{\def\colwidth{10cm}\firstcol{

{\bf Questão 2}

\T(Total: 3.0 pts)

\msk

Lembre que a fórmula da integração por partes é esta aqui:
%
$$\begin{array}{l}
  \ga{[IP]} \;\;=\;\; \\ \\[-7.5pt]
  \D
  \left(
    \intx{f(x)g'(x)} = f(x)g(x) - \intx{f'(x)g(x)}
  \right)
  \end{array}
$$

e que se substituirmos $f(x)$ por $x^4$ e $g(x)$ por $e^{5x}$ nela
obtemos isto:
%
$$
  \intx{x^4·5e^{5x}} = (x^4·e^{5x}) - \intx{4x^3·e^{5x}}
$$

Calcule $$\intx{x·e^{42x}}.$$

Pontuação:

\msk

a) \B (0.5 pts) pelo resultado correto,

\msk

b) \B (2.5 pts) se você conseguir escrever as contas disto no formato
que nós vimos em sala, com os `$=$'s alinhados e as justificativas
(corretas!) à direita.

}\anothercol{

{\bf Questão 3}

\T(Total: 3.0 pts)

\msk

Calcule:
%
$$\intx{(\sen x)^3(\cos x)^3}.$$

}}

\newpage

% «questao-4»  (to ".questao-4")
% (c2m221vsbp 5 "questao-4")
% (c2m221vsba    "questao-4")

{\bf Questão 4}

%L namedang("EDOVSintro", "", [[
%L    \begin{array}{rcl}
%L      \ga{[M]} &=& <EDOVSG> \\ \\[-5pt]
%L      \ga{[F]} &=& <EDOVSP> \\
%L    \end{array}
%L ]])
%L EDOVSintro:sa("FOO"):output()
\pu

\scalebox{0.55}{\def\colwidth{10cm}\firstcol{

\vspace*{-0.4cm}

\T(Total: 3.0 pts)

Lembre que nós vimos que o ``método'' para resolver EDOs com variáveis
separáveis pode ser escrito como a demonstração $\ga{[M]}$ abaixo, e a
``fórmula'' para resolver EDOs com variáveis separáveis pode ser
escrita como $\ga{[F]}$:

\bsk

$\ga{FOO}$

}\anothercol{

  Digamos que queremos resolver esta EDO:
  %
  $$ \D \frac{dy}{dx} = \frac{3x^2}{5y^4} \qquad (*)
  $$

  \msk

  a) \B (0.5 pts) Encontre a solução geral da EDO $(*)$.

  \msk

  b) \B (1.0 pts) Teste a solução do seu item (a).

  \msk

  c) \B (0.5 pts) Encontre a solução da EDO que passa pelo ponto
  $(x,y)=(2,2)$.

  \msk

  d) \B (1.0 pts) Teste a solução do seu item (c).



}}



% (setq eepitch-preprocess-regexp "^")
% (setq eepitch-preprocess-regexp "^%T ")
%
%T  (eepitch-maxima)
%T  (eepitch-kill)
%T  (eepitch-maxima)
%T myeq    : ('diff(y,x) = (3 * x^2)/(5 * y^4));
%T mysolg  : ode2 (myeq, y, x);
%T mysolgs : solve(mysolg, y);
%T sol     : mysolgs[5];
%T sol2    : subst([x=2, y=2], sol);
%T sol3    : lhs(sol2)^5 = rhs(sol2)^5;
%T solve(sol3, %c);
%T f       : rhs(sol);
%T f       : subst([%c=8], rhs(sol));
%T subst([y=f], rhs(myeq));
%T diff(f, x);

\newpage

% «questao-1-gab»  (to ".questao-1-gab")
% (c2m221vsbp 6 "questao-1-gab")
% (c2m221vsba   "questao-1-gab")

{\bf Questão 1: gabarito}

\unitlength=8pt

$$\begin{array}{rcl}
  f(x) &=& \ga{fig f} \\
  F(x) \;=\; \Intt{2}{x}{f(t)} &=& \ga{fig F} \\
  G(x) \;=\; \Intt{3}{x}{f(t)} &=& \ga{fig G} \\
  \end{array}
$$

\newpage

% «questao-2-gab»  (to ".questao-2-gab")
% (c2m221vsbp 7 "questao-2-gab")
% (c2m221vsba   "questao-2-gab")

{\bf Questão 2: gabarito}

\def\T{\textstyle}
\def\und#1#2{\underbrace{\T#1}_{\T#2}}

\bsk

$\begin{array}{rcll}
  \D \intx{\und{x}{f(x)}·\und{e^{42x}}{g'(x)}}
    &\eqnpfull{1}& \und{x}{f(x)}·\und{\frac{1}{42}e^{42x}}{g(x)} -
        \D \intx{\und{1}{f'(x)}·\und{\frac{1}{42}e^{42x}}{g(x)}} \\
    &\eqnpfull{2}& \frac{1}{42} xe^{42x} - \frac{1}{42} \D \intx{e^{42x}}    \\ \\[-6pt]
    &\eqnpfull{3}& \frac{1}{42} xe^{42x} - \frac{1}{42} \frac{1}{42} e^{42x} \\
  \end{array}
$

\bsk

(1): por \CFname{IP}{} com $f(x)=x$ e $g(x)=\frac{1}{42}e^{42x}$

\ssk

(2): por $\intx{kf(x)} = k\intx{f(x)}$, com $k=\frac{1}{42}$ e $f(x)=e^{42x}$

\ssk

(3): por $\intx{e^{42x}} = \frac{1}{42}e^{42x}$

% (find-LATEX "edrxgac2.tex" "C2" "eqnp")

\newpage

% «questao-3-gab»  (to ".questao-3-gab")
% (c2m221vsbp 8 "questao-3-gab")
% (c2m221vsba   "questao-3-gab")
% (c2m201ipscp 2 "exemplo-1")
% (c2m201ipsca   "exemplo-1")

{\bf Questão 3: gabarito}

\def\S{\sen x}
\def\C{\cos x}
\def\D{\displaystyle}
\def\und#1#2{\underbrace{#1}_{#2}}

$$\scalebox{0.9}{$
  \begin{array}[t]{l}
  \D \intx{(\S)^3 (\C)^3} \\
  \D = \;\; \intx{(\S)^3 (\C)^2 (\C)} \\
  \D = \;\; \intx{(\und{\S}{s})^3 \und{(\C)^2}{1-s^2} \und{(\C)}{\frac{ds}{dx}}} \\
  \D = \;\; \ints{s^3 (1-s^2)} \\
  \D = \;\; \ints{s^3 - s^5} \\
  \D = \;\; \frac{s^4}{4} - \frac{s^6}{6} \\
  \D = \;\; \frac{(\S)^4}{4} - \frac{(\S)^6}{6} \\
  \end{array}
  \qquad
  \begin{array}[t]{c}
  \\ \\
    \bmat{s = \sen x \\
          \frac{ds}{dx} = \cos x \\
          \sen x = s \\
          (\cos x)^2 = 1 - s^2 \\
          \cos x \, dx = ds
    }
  \end{array}
  $}
$$


% (setq eepitch-preprocess-regexp "^")
% (setq eepitch-preprocess-regexp "^%T ")
%
%T  (eepitch-maxima)
%T  (eepitch-kill)
%T  (eepitch-maxima)
%T f : sin(x)^3 * cos(x)^3 ;
%T F : integrate(f, x);
%T F2 : sin(x)^4/4 - sin(x)^6/6;
%T expand(F - F2);

\newpage

% «questao-4-gab»  (to ".questao-4-gab")
% (c2m221vsbp 9 "questao-4-gab")
% (c2m221vsba   "questao-4-gab")

{\bf Questão 4: gabarito}

Sejam:

$g(x) = 3x^2$,

$h(y) = 5y^4$,

$G(x) = x^3$,

$H(y) = y^5$,

$H^{-1}(y) = y^{1/5}$.

Isso dá a solução $f(x) = y = H^{-1}(G(x)+C_3) = (x^3+C_3)^{1/5}$.

Temos $\frac{dy}{dx}
 = f'(x)
 = \frac{1}{5}(x^3+C_3)^{-4/5}·3x^2
 = \frac{3x^2}{5(x^3+C_3)^{4/5}}
 $

e $\frac{g(x)}{h(y)}
 = \frac{3x^2}{5y^4}
 = \frac{3x^2}{5f(x)^4}
 = \frac{3x^2}{5((x^3+C_3)^{1/5})^4}
 = \frac{3x^2}{5(x^3+C_3)^{4/5}}
$,

e portanto $\frac{dy}{dx} = \frac{g(x)}{h(y)}$.

Se $(x,y) = (2,2)$ então $(x,f(x))=(2,2)$, $f(2)=2$,

$(2^3+C_3)^{1/5} = 2$,
$(2^3+C_3) = 2^5$, 
$8+C_3 = 32$, 
$C_3 = 24$. 

Testando: $f(2) = (2^3+C_3)^{1/5} = (8+24)^{1/5} = 32^{1/5} = \sqrt[5]{32} = 2$.

%T  (eepitch-maxima)
%T  (eepitch-kill)
%T  (eepitch-maxima)
%T g   : 3*x^2;
%T h   : 5*y^4;
%T f   : (x^3 + C3)^(1/5);
%T fp  : diff(f, x);
%T rt0 : g/h;
%T rt  : subst([y=f], rt0);


\newpage

{\bf Critérios de correção}


\scalebox{0.55}{\def\colwidth{10.5cm}\firstcol{

{\bf Questão 1:}

Cada uma das duas figuras vale 0.5 pontos se for feita corretamente.
Cada segmento (com $Δx=1$) inexistente ou com inclinação errada
desconta 0.1 pontos -- ou seja, uma figura com 5 ou mais 6 segmentos
errados vale 0.


\bsk

{\bf Questão 2:}

Item a: aqui eu aceitei respostas sem justificativa.

Item b: o ``formato que nós vimos em sala'' é este aqui:

{\footnotesize

%    http://angg.twu.net/LATEX/2022-1-C2-tudo.pdf#page=116
\url{http://angg.twu.net/LATEX/2022-1-C2-tudo.pdf\#page=116}

% (c2m212tudop 1)
%    http://angg.twu.net/LATEX/2021-2-C2-tudo.pdf#page=8
\url{http://angg.twu.net/LATEX/2021-2-C2-tudo.pdf\#page=8}

}

\bsk

{\bf Questão 3:}

Se a pessoa tiver escrito o resultado certo isso vale 0.3 pontos. Os
outros 2.7 pontos são pra quem conseguiu fazer as contas passo a passo
de uma forma que cada passo ficasse legível (e correto). Nós
trabalhamos exemplos bem parecidos com esse --- mas com outros
expoentes --- em sala, e um dos PDFs que eu recomendei muito que os
alunos estudassem por ele pra P2, pra VR e pra VS também tinha uma
integral de potências de senos e cossenos bem parecida com essa, então
aqui dá pra ser bem exigente na correção.




}\anothercol{


{\bf Questão 4:}

As fórmulas/figuras \CFname{M}{} e \CFname{F}{} são pra lembrar os
alunos da questão de EDOs com variáveis separáveis que eles fizeram na
P2. Aqui eu esperava que os alunos fizessem contas legíveis, feitas
passo a passo, e chegassem nos resultados corretos. Note que os itens
de ``encontre a solução'' valem só 0.5 pontos cada um e que os itens
de ``teste a solução'' valem bem mais, 1.5 pontos cada um.

}}



\GenericWarning{Success:}{Success!!!}  % Used by `M-x cv'

\end{document}

% Local Variables:
% coding: utf-8-unix
% ee-tla: "c2vs"
% ee-tla: "c2m221vsb"
% End:
