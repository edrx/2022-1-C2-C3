% (find-LATEX "2022-1-C2-somas-3.tex")
% (defun c () (interactive) (find-LATEXsh "lualatex -record 2022-1-C2-somas-3.tex" :end))
% (defun C () (interactive) (find-LATEXsh "lualatex 2022-1-C2-somas-3.tex" "Success!!!"))
% (defun D () (interactive) (find-pdf-page      "~/LATEX/2022-1-C2-somas-3.pdf"))
% (defun d () (interactive) (find-pdftools-page "~/LATEX/2022-1-C2-somas-3.pdf"))
% (defun e () (interactive) (find-LATEX "2022-1-C2-somas-3.tex"))
% (defun o () (interactive) (find-LATEX "2022-1-C2-somas-3.tex"))
% (defun l () (interactive) (find-LATEX    "Piecewise1.lua"))
% (defun l () (interactive) (find-angg "LUA/Piecewise1.lua"))
% (defun u () (interactive) (find-latex-upload-links "2022-1-C2-somas-3"))
% (defun v () (interactive) (find-2a '(e) '(d)))
% (defun d0 () (interactive) (find-ebuffer "2022-1-C2-somas-3.pdf"))
% (defun cv () (interactive) (C) (ee-kill-this-buffer) (v) (g))
%          (code-eec-LATEX "2022-1-C2-somas-3")
% (find-pdf-page   "~/LATEX/2022-1-C2-somas-3.pdf")
% (find-sh0 "cp -v  ~/LATEX/2022-1-C2-somas-3.pdf /tmp/")
% (find-sh0 "cp -v  ~/LATEX/2022-1-C2-somas-3.pdf /tmp/pen/")
%     (find-xournalpp "/tmp/2022-1-C2-somas-3.pdf")
%   file:///home/edrx/LATEX/2022-1-C2-somas-3.pdf
%               file:///tmp/2022-1-C2-somas-3.pdf
%           file:///tmp/pen/2022-1-C2-somas-3.pdf
% http://angg.twu.net/LATEX/2022-1-C2-somas-3.pdf
% (find-LATEX "2019.mk")
% (find-CN-aula-links "2022-1-C2-somas-3" "2" "c2m221somas3" "c2so3")
% (find-sh0 "cd ~/LUA/; cp -v Pict2e1.lua Pict2e1-1.lua Piecewise1.lua ~/LATEX/")

% «.defs»			(to "defs")
% «.title»			(to "title")
% «.links»			(to "links")
% «.mountains»			(to "mountains")
% «.exercicio-1»		(to "exercicio-1")
% «.imagens-figuras»		(to "imagens-figuras")
% «.imagens-figuras-2»		(to "imagens-figuras-2")
% «.exercicio-2»		(to "exercicio-2")
% «.sup-e-inf-informais»	(to "sup-e-inf-informais")
% «.exercicio-3»		(to "exercicio-3")
% «.exercicio-4»		(to "exercicio-4")
% «.exercicio-5»		(to "exercicio-5")
% «.exercicio-6»		(to "exercicio-6")
% «.metodos-nomes»		(to "metodos-nomes")
% «.exercicio-7»		(to "exercicio-7")
% «.exercicio-7-figs»		(to "exercicio-7-figs")
% «.exercicio-8»		(to "exercicio-8")
% «.exercicio-8-figs»		(to "exercicio-8-figs")
%
% «.djvuize»	(to "djvuize")



% <videos>
% Video (not yet):
% (find-ssr-links     "c2m221somas3" "2022-1-C2-somas-3")
% (code-eevvideo      "c2m221somas3" "2022-1-C2-somas-3")
% (code-eevlinksvideo "c2m221somas3" "2022-1-C2-somas-3")
% (find-c2m221somas3video "0:00")

\documentclass[oneside,12pt]{article}
\usepackage[colorlinks,citecolor=DarkRed,urlcolor=DarkRed]{hyperref} % (find-es "tex" "hyperref")
\usepackage{amsmath}
\usepackage{amsfonts}
\usepackage{amssymb}
\usepackage{pict2e}
\usepackage[x11names,svgnames]{xcolor} % (find-es "tex" "xcolor")
\usepackage{colorweb}                  % (find-es "tex" "colorweb")
%\usepackage{tikz}
%
% (find-dn6 "preamble6.lua" "preamble0")
%\usepackage{proof}   % For derivation trees ("%:" lines)
%\input diagxy        % For 2D diagrams ("%D" lines)
%\xyoption{curve}     % For the ".curve=" feature in 2D diagrams
%
\usepackage{edrx21}               % (find-LATEX "edrx21.sty")
\input edrxaccents.tex            % (find-LATEX "edrxaccents.tex")
\input edrx21chars.tex            % (find-LATEX "edrx21chars.tex")
\input edrxheadfoot.tex           % (find-LATEX "edrxheadfoot.tex")
\input edrxgac2.tex               % (find-LATEX "edrxgac2.tex")
%\usepackage{emaxima}              % (find-LATEX "emaxima.sty")
%
%\usepackage[backend=biber,
%   style=alphabetic]{biblatex}            % (find-es "tex" "biber")
%\addbibresource{catsem-slides.bib}        % (find-LATEX "catsem-slides.bib")
%
% (find-es "tex" "geometry")
\usepackage[a6paper, landscape,
            top=1.5cm, bottom=.25cm, left=1cm, right=1cm, includefoot
           ]{geometry}
%
\begin{document}

\catcode`\^^J=10
\directlua{dofile "dednat6load.lua"}  % (find-LATEX "dednat6load.lua")
%L dofile "Piecewise1.lua"           -- (find-LATEX "Piecewise1.lua")
%L Pict2e.__index.suffix = "%"
\pu
\def\pictgridstyle{\color{GrayPale}\linethickness{0.3pt}}
\def\pictaxesstyle{\linethickness{0.5pt}}


% «defs»  (to ".defs")
% (find-LATEX "edrx21defs.tex" "colors")
% (find-LATEX "edrx21.sty")

\def\u#1{\par{\footnotesize \url{#1}}}

\def\drafturl{http://angg.twu.net/LATEX/2022-1-C2.pdf}
\def\drafturl{http://angg.twu.net/2022.1-C2.html}
\def\draftfooter{\tiny \href{\drafturl}{\jobname{}} \ColorBrown{\shorttoday{} \hours}}



%  _____ _ _   _                               
% |_   _(_) |_| | ___   _ __   __ _  __ _  ___ 
%   | | | | __| |/ _ \ | '_ \ / _` |/ _` |/ _ \
%   | | | | |_| |  __/ | |_) | (_| | (_| |  __/
%   |_| |_|\__|_|\___| | .__/ \__,_|\__, |\___|
%                      |_|          |___/      
%
% «title»  (to ".title")
% (c2m221somas3p 1 "title")
% (c2m221somas3a   "title")

\thispagestyle{empty}

\begin{center}

\vspace*{1.2cm}

{\bf \Large Cálculo 2 - 2022.1}

\bsk

Aula 11: somas de retângulos

\bsk

Eduardo Ochs - RCN/PURO/UFF

\url{http://angg.twu.net/2022.1-C2.html}

\end{center}

\newpage

% «links»  (to ".links")
% (c2m221somas3p 2 "links")
% (c2m221somas3a   "links")

{\bf Links}

Este PDF complementa estes três PDFs

do semestre passado:

\ssk

{\footnotesize

% (c2m212somas1a "title")
% (c2m212somas1a "title" "Aula 4: integrais como somas de retângulos (1)")
% (c2m212somas2a "title")
% (c2m212somas2a "title" "Aula 13: integrais como somas de retângulos (2)")
% (c2m212mt1p 1 "title")
% (c2m212mt1a   "title")
%    http://angg.twu.net/LATEX/2021-2-C2-somas-1.pdf
\url{http://angg.twu.net/LATEX/2021-2-C2-somas-1.pdf}

%    http://angg.twu.net/LATEX/2021-2-C2-somas-2.pdf
\url{http://angg.twu.net/LATEX/2021-2-C2-somas-2.pdf}

%    http://angg.twu.net/LATEX/2021-2-C2-MT1.pdf
\url{http://angg.twu.net/LATEX/2021-2-C2-MT1.pdf}

}

\ssk


%  __  __                   _        _           
% |  \/  | ___  _   _ _ __ | |_ __ _(_)_ __  ___ 
% | |\/| |/ _ \| | | | '_ \| __/ _` | | '_ \/ __|
% | |  | | (_) | |_| | | | | || (_| | | | | \__ \
% |_|  |_|\___/ \__,_|_| |_|\__\__,_|_|_| |_|___/
%                                                
% «mountains»  (to ".mountains")
% (c2m221somas3p 3 "mountains")
% (c2m221somas3a   "mountains")

% (find-angg "LUA/Piecewise1.lua" "Xtoxytoy-test2")
%
%L Pict2e.bounds = PictBounds.new(v(0,0), v(23,9))
%L spec   = "(0,1)--(5,6)--(7,4)--(11,8)--(15,4)--(17,6)--(23,0)"
%L xs     = {    1,3,    6,     9,  11, 13,   16,19,    21      }
%L labely = -1
%L pws    = PwSpec.from(spec)
%L xtos   = Xtoxytoy.from(pws:fun(), xs)
%L vlines = xtos:topict("v")
%L curve  = pws:topict()
%L labels = PictList {}
%L for i,x in ipairs(xs) do
%L   labels:addputstrat(v(x,labely), "\\cell{x_"..(i-1).."}")
%L end
%L p = PictList { vlines, curve:prethickness("2pt"), labels }
%L p:pgat("pA", "mountain"):output()
\pu

\unitlength=8pt

\vspace*{-0.25cm}
\hspace*{-0.5cm}
$\scalebox{0.55}{$
 \begin{array}{ccccc}
 \mountain && \mountain && \mountain \\[20pt]
 \mountain && \mountain && \mountain \\[20pt]
 \mountain && \mountain && \mountain \\[20pt]
 \mountain && \mountain && \mountain \\[20pt]
 \end{array}
 $}
$


\newpage

%  _____                   _      _         _ 
% | ____|_  _____ _ __ ___(_) ___(_) ___   / |
% |  _| \ \/ / _ \ '__/ __| |/ __| |/ _ \  | |
% | |___ >  <  __/ | | (__| | (__| | (_) | | |
% |_____/_/\_\___|_|  \___|_|\___|_|\___/  |_|
%                                             
% «exercicio-1»  (to ".exercicio-1")
% (c2m221somas3p 4 "exercicio-1")
% (c2m221somas3a   "exercicio-1")
% (find-pdf-page "~/2022.1-C2/C2-quadros-tarde.pdf" 9)

{\bf Exercício 1.}

Seja $f(x)$ a função da página anterior ---

a página anterior tem 12 cópias do gráfico dela.

Represente graficamente cada um dos somatórios abaixo.

Cada item abaixo vai virar 8 retângulos sobre

uma das cópias do gráfico da $f(x)$.

\msk

\def\sumo{\sum_{i=1}^{8}}
\def\sumoo#1{\sumo #1 (x_i - x_{i-1})}

a) $\sumoo{f(x_i)}$

\ssk

b) $\sumoo{f(x_{i-1})}$

\ssk

c) $\sumoo{\max(f(x_{i-1}), f(x_i))}$

\ssk

d) $\sumoo{\min(f(x_{i-1}), f(x_i))}$

\ssk

e) $\sumoo{f(\frac{x_{i-1} + x_i}{2})}$

\ssk

f) $\sumoo{\frac{f(x_{i-1}) + f(x_i)}{2}}$


\newpage

%  ___                                           __ _           
% |_ _|_ __ ___   __ _  __ _  ___ _ __  ___ _   / _(_) __ _ ___ 
%  | || '_ ` _ \ / _` |/ _` |/ _ \ '_ \/ __(_) | |_| |/ _` / __|
%  | || | | | | | (_| | (_| |  __/ | | \__ \_  |  _| | (_| \__ \
% |___|_| |_| |_|\__,_|\__, |\___|_| |_|___(_) |_| |_|\__, |___/
%                      |___/                          |___/     
%
% «imagens-figuras»  (to ".imagens-figuras")
% (c2m221somas3p 5 "imagens-figuras")
% (c2m221somas3a   "imagens-figuras")

% (find-angg "LUA/Piecewise1.lua" "Xtoxytoy-test3")

{\bf Imagens de intervalos: figuras}

\bsk
\bsk

%L Pict2e.bounds = PictBounds.new(v(0,0), v(8,4))
%L 
%L cthick = "2pt"     -- curve
%L dthick = "0.25pt"  -- dots
%L sthick = "4pt"     -- segments
%L 
%L -- Curve:
%L cspec   = "(0,2)--(2,4)--(6,0)--(8,2)"
%L cpws    = PwSpec.from(cspec)
%L curve   = cpws:topict():prethickness(cthick)
%L 
%L -- Segments:
%L sspec = "(1,0)c--(2,0)--(4,0)c" ..
%L        " (1,3)c--(2,4)--(4,2)c" ..
%L        " (0,2)c--(0,4)c"
%L spws   = PwSpec.from(sspec)
%L segs   = spws:topict():prethickness(sthick):Color("Orange")
%L 
%L -- Dots:
%L dotsn = function (nsubsegs)
%L     local xs    = seqn(1, 4, nsubsegs)
%L     local dots0 = Xtoxytoy.from(cpws:fun(), xs)
%L     local dots  = dots0:topict("vhxpy"):prethickness(dthick):Color("Red")
%L     return dots
%L   end
%L 
%L PictList { curve, dotsn(1)  } :pgat("pgatc", "ImageOne")   :output()
%L PictList { curve, dotsn(3)  } :pgat("pgatc", "ImageThree") :output()
%L PictList { curve, dotsn(6)  } :pgat("pgatc", "ImageSix")   :output()
%L PictList { curve, dotsn(12) } :pgat("pgatc", "ImageTwelve"):output()
%L PictList { curve, segs }      :pgat("pgatc", "ImageSegs")  :output()
\pu

$\ImageOne$
$\ImageThree$
$\ImageSix$
$\ImageTwelve$

$\ImageSegs$


\newpage

% «imagens-figuras-2»  (to ".imagens-figuras-2")
% (c2m221somas3p 6 "imagens-figuras-2")
% (c2m221somas3a   "imagens-figuras-2")

{\bf Imagens de intervalos: figuras $→$ contas}

\scalebox{0.75}{\def\colwidth{12cm}\firstcol{

Algumas pessoas acham que isto é sempre verdade:
%
$$f([a,b]) = [f(a),f(b)].$$

\ColorRed{Não seja como elas!!!}

Nas três figuras à esquerda da página anterior temos:
%
$$\begin{array}{rcl}
  f(\{1,4\}) &=& \{f(1),f(4)\} \\
             &=& \{3,2\} \\
             &=& \{2,3\} \\
  f(\{1,2,3,4\}) &=& \{f(1),f(2),f(3),f(4)\} \\
                 &=& \{2,3,4,3\} \\
                 &=& \{2,3,4\} \\
  f([1,3]) &=& [2,4] \\{}
  [f(1),f(3)] &=& [3,2] \\
              &=& \setofst{y∈\R}{3≤y≤2} \\
              &=& ∅ \\
  \end{array}
$$

%}\anothercol{
}}


\newpage

%  _____                   _      _         ____  
% | ____|_  _____ _ __ ___(_) ___(_) ___   |___ \ 
% |  _| \ \/ / _ \ '__/ __| |/ __| |/ _ \    __) |
% | |___ >  <  __/ | | (__| | (__| | (_) |  / __/ 
% |_____/_/\_\___|_|  \___|_|\___|_|\___/  |_____|
%                                                 
% «exercicio-2»  (to ".exercicio-2")
% (c2m221somas3p 7 "exercicio-2")
% (c2m221somas3a   "exercicio-2")

{\bf Exercício 2.}

Seja $f(x)$ esta função:

\msk

%L Pict2e.bounds = PictBounds.new(v(0,0), v(8,4))
%L spec   = "(0,2)--(2,4)--(6,0)--(8,2)"
%L pws    = PwSpec.from(spec)
%L curve  = pws:topict()
%L p = PictList { curve:prethickness("2pt") }
%L p:pgat("pgatc", "falsoseno"):output()
\pu
%
$f(x) = \falsoseno$

\msk

Calcule estas imagens de intervalos:

\msk

\begin{tabular}[t]{l}
a) $f([0,1])$ \\
b) $f([1,2])$ \\
c) $f([0,2])$ \\
d) $f([2,3])$ \\
e) $f([1,3])$ \\
f) $f([0,3])$ \\
\end{tabular}
\qquad
\begin{tabular}[t]{l}
g) $f([0,4])$ \\
h) $f([4,8])$ \\
i) $f([0,8])$ \\
j) $f([1,7])$ \\
\end{tabular}

\newpage

%                  _    ___     _        __ _ 
%  ___ _   _ _ __ (_)  ( _ )   (_)_ __  / _(_)
% / __| | | | '_ \| |  / _ \/\ | | '_ \| |_| |
% \__ \ |_| | |_) | | | (_>  < | | | | |  _| |
% |___/\__,_| .__/|_|  \___/\/ |_|_| |_|_| |_|
%           |_|                               
%
% «sup-e-inf-informais»  (to ".sup-e-inf-informais")
% (c2m221somas3p 8 "sup-e-inf-informais")
% (c2m221somas3a   "sup-e-inf-informais")

{\bf Sup informal e inf informal}

% (find-pdf-page "~/2022.1-C2/C2-quadros-manha.pdf" 7)

\def\sup {\mathsf{sup}}
\def\supi{\mathsf{supi}}
\def\inf {\mathsf{inf}}
\def\infi{\mathsf{infi}}


\scalebox{0.85}{\def\colwidth{12cm}\firstcol{

O sup informal, `$\supi$', é como uma função bugada.

O $\sup$ ``de verdade'' recebe um conjuntos

de números e sempre retorna um número.

O $\supi$ recebe um conjuntos de números e às vezes

retorna um número, mas às vezes ele dá erro.

Quando o $\supi$ recebe um intervalo fechado $[a,b]$ ele

retorna a extremidade superior do intervalo: $\supi([a,b])=b$.

Quando o supi recebe um conjunto que

não é um intervalo fechado ele dá erro.

\msk

O inf informal, `$\infi$', é como o $\supi$, mas ele retorna

a extremidade inferior do intervalo: $\infi([a,b])=a$.

\msk

O sup e o inf de verdade são BEM difíceis de definir.

Nós vamos começar usando o $\supi$ e o $\infi$ e vamos deixar

pra definir o sup e o inf de verdade só quando

já tivermos bastante prática com o $\supi$ e o $\infi$.

%}\anothercol{
}}


\newpage

%  _____                   _      _         _____ 
% | ____|_  _____ _ __ ___(_) ___(_) ___   |___ / 
% |  _| \ \/ / _ \ '__/ __| |/ __| |/ _ \    |_ \ 
% | |___ >  <  __/ | | (__| | (__| | (_) |  ___) |
% |_____/_/\_\___|_|  \___|_|\___|_|\___/  |____/ 
%                                                 
% «exercicio-3»  (to ".exercicio-3")
% (c2m221somas3p 9 "exercicio-3")
% (c2m221somas3a   "exercicio-3")

{\bf Exercício 3.}

Cada um dos itens do exercício 2 pede pra

você calcular uma expressão da forma $f([a,b])$.

Para cada um dos itens do exercício 2

faça uma cópia do gráfico da função $f(x)$

e desenhe sobre esta cópia estes dois retângulos:
%
$$\begin{array}{c}
  \supi(f([a,b]))(b-a), \\
  \infi(f([a,b]))(b-a) \\
  \end{array}
$$

\newpage

%  _____                   _      _         _  _   
% | ____|_  _____ _ __ ___(_) ___(_) ___   | || |  
% |  _| \ \/ / _ \ '__/ __| |/ __| |/ _ \  | || |_ 
% | |___ >  <  __/ | | (__| | (__| | (_) | |__   _|
% |_____/_/\_\___|_|  \___|_|\___|_|\___/     |_|  
%                                                  
% «exercicio-4»  (to ".exercicio-4")
% (c2m221somas3p 10 "exercicio-4")
% (c2m221somas3a    "exercicio-4")

{\bf Exercício 4.}

Acrescente estes dois itens extras

no exercício 1 e faça eles:

\bsk

\def\sumo{\sum_{i=1}^{8}}
\def\sumoo#1{\sumo #1 (x_i - x_{i-1})}

g) $\sumoo{\supi(f([x_{i-1},x_i]))}$

\msk

h) $\sumoo{\infi(f([x_{i-1},x_i]))}$


\newpage

% «exercicio-5»  (to ".exercicio-5")
% (c2m221somas3p 11 "exercicio-5")
% (c2m221somas3a    "exercicio-5")

{\bf Exercício 5.}


\scalebox{0.9}{\def\colwidth{12cm}\firstcol{

Vou me referir a este PDF aqui, do semestre passado,

como ``Somas 1'', ou ``S1'':

%\ssk

{\footnotesize

% http://angg.twu.net/LATEX/2021-2-C2-somas-1.pdf
\url{http://angg.twu.net/LATEX/2021-2-C2-somas-1.pdf}

}

\ssk

Vou usar nomes como ``S1E4'' pra me referir aos

exercícios dele --- S1E4 é o Exercício 4 do ``Somas 1''.

\bsk

% (c2m212somas1p 9 "particoes")
% (c2m212somas1a   "particoes")
% (c2m212somas1p 10 "exercicio-4")
% (c2m212somas1a    "exercicio-4")

a) Leia as páginas 9 e 10 do S1 e faça o S1E4.

% (c2m212somas1p 11 "exercicio-5")
% (c2m212somas1a    "exercicio-5")

b) Leia a página 11 e 12 do S1 e faça o S1E5.

% (c2m212somas1p 23 "metodos-nomes")
% (c2m212somas1a    "metodos-nomes")

\def\mname#1{\text{[#1]}}

\msk

Para os próximos itens leia a página

23 do S1 e considere que $f(x) = \scalebox{0.33}{$\falsoseno$}$.

\ssk

c) Seja $P=\{0,1,4,5\}$. Represente num gráfico só $f(x)$ e $\mname{L}$.

d) Seja $P=\{0,1,4,5\}$. Represente num gráfico só $f(x)$ e $\mname{R}$.

e) Seja $P=\{0,1,4,5\}$. Represente num gráfico só $f(x)$ e $\mname{min}$.


}\anothercol{
}}


\newpage

% «exercicio-6»  (to ".exercicio-6")
% (c2m221somas3p 12 "exercicio-6")
% (c2m221somas3a    "exercicio-6")

{\bf Exercício 6.}

Vou me referir a este outro PDF do semestre

passado como ``Soma 2'', ou S2:

%\ssk

{\footnotesize

% (c2m212somas2p 1 "title")
% (c2m212somas2a   "title")
%    http://angg.twu.net/LATEX/2021-2-C2-somas-2.pdf
\url{http://angg.twu.net/LATEX/2021-2-C2-somas-2.pdf}

}

\ssk

Vou usar nomes como ``S2E5'' pra me referir aos

exercícios dele --- S2E5 é o Exercício 5 do ``Somas 2''.

\bsk

% (c2m212somas2p 35 "exercicio-13")
% (c2m212somas2a    "exercicio-13")
a) Leia as páginas 32 até 35 do S2 e faça o S2E13.

\bsk

Obs: \ColorRed{não faça} o S2E12 da página 34 do S2 ---

nos próximos exercícios você vai fazer algo

parecido com ele, mas com outra notação.


\newpage

% «metodos-nomes»  (to ".metodos-nomes")
% (c2m221somas3p 13 "metodos-nomes")
% (c2m221somas3a    "metodos-nomes")

{\bf Métodos de integração: nomes}


\scalebox{0.9}{\def\colwidth{12cm}\firstcol{

O S1 e o S2 usam estes nomes pros métodos de integração:

{\footnotesize

% (c2m212somas1p 23 "metodos-nomes")
% (c2m212somas1a    "metodos-nomes")
%    http://angg.twu.net/LATEX/2021-2-C2-somas-1.pdf#page=23
\url{http://angg.twu.net/LATEX/2021-2-C2-somas-1.pdf\#page=23}

% (c2m212somas2p 32 "metodos-nomes")
% (c2m212somas2a    "metodos-nomes")
%    http://angg.twu.net/LATEX/2021-2-C2-somas-2.pdf#page=32
\url{http://angg.twu.net/LATEX/2021-2-C2-somas-2.pdf\#page=32}

}

\ssk

Ainda não definimos o $\sup$ e o $\inf$ ``de verdade'',

então vamos usar estes nomes aqui...

\def\sumiN#1{\sum_{i=1}^N #1 (b_i-a_i)}
\def\mname#1{\text{[#1]}}
%
$$\scalebox{0.95}{$
  \begin{array}{ccl}
  \mname{L}    &=& \sumiN {f(a_i)}                    \\[2pt]
  \mname{R}    &=& \sumiN {f(b_i)}                    \\[2pt]
  \mname{Trap} &=& \sumiN {\frac{f(a_i) + f(b_i)}{2}} \\[2pt]
  \mname{M}    &=& \sumiN {f(\frac{a_i+b_i}{2})}      \\[2pt]
  \mname{min}  &=& \sumiN {\min(f(a_i), f(b_i))}      \\[2pt]
  \mname{max}  &=& \sumiN {\max(f(a_i), f(b_i))}      \\[2pt]
  \mname{infi} &=& \sumiN {\infi(f([a_i,b_i]))}        \\[2pt]
  \mname{supi} &=& \sumiN {\supi(f([a_i,b_i]))}        \\
  \end{array}
  $}
$$

}\anothercol{
}}


\newpage

% «exercicio-7»  (to ".exercicio-7")
% (c2m221somas3p 14 "exercicio-7")
% (c2m221somas3a    "exercicio-7")

{\bf Exercício 7.}

Faça o S2E12 (página 34 do S2), mas substituindo

os `sup's por `supi's e os `inf's por `infi's.

Use a função do enunciado do S2E12.

(Tem cópias dela na próxima página).

\msk

Link:

{\footnotesize

% (c2m212somas2p 32 "metodos-nomes")
% (c2m212somas2a    "metodos-nomes")
% (c2m212somas2p 34 "exercicio-12")
% (c2m212somas2a    "exercicio-12")
%    http://angg.twu.net/LATEX/2021-2-C2-somas-2.pdf#page=34
\url{http://angg.twu.net/LATEX/2021-2-C2-somas-2.pdf\#page=34}

}

\ssk


\newpage

% «exercicio-7-figs»  (to ".exercicio-7-figs")
% (c2m221somas3p 15 "exercicio-7-figs")
% (c2m221somas3a    "exercicio-7-figs")
% (c2m212somas2p 34 "exercicio-12")
% (c2m212somas2a    "exercicio-12")

%L Pict2e.bounds = PictBounds.new(v(0,0), v(11,7))
%L spec   = "(0,3)--(3,6)--(8,1)--(11,4)"
%L pws    = PwSpec.from(spec)
%L curve  = pws:topict()
%L p = PictList { curve:prethickness("2pt") }
%L p:addputstrat(v(3,6.4), "\\cell{(3,6)}")
%L p:addputstrat(v(8,0.4), "\\cell{(8,1)}")
%L p:pgat("pgatc"):preunitlength("15pt"):sa("FOO"):output()
\pu
%
\def\FOO{\scalebox{0.5}{$\ga{FOO}$}}
\def\FOO{\ga{FOO}}

% $f(x) = \FOO \smallfoo$

\vspace*{-0.25cm}
\hspace*{-0.5cm}
$\scalebox{0.40}{$
 \begin{array}{ccccccc}
 \FOO && \FOO && \FOO && \FOO \\[70pt]
 \FOO && \FOO && \FOO && \FOO \\[70pt]
 \FOO && \FOO && \FOO && \FOO \\[70pt]
 \FOO && \FOO && \FOO && \FOO \\[70pt]
 \end{array}
 $}
$



\newpage

% «exercicio-8»  (to ".exercicio-8")
% (c2m221somas3p 16 "exercicio-8")
% (c2m221somas3a    "exercicio-8")
% (c2m211mt1p 4 "questao-1")
% (c2m211mt1a   "questao-1")

{\bf Exercício 8.}

\def\mname#1{\text{[#1]}}

%L Pict2e.bounds = PictBounds.new(v(0,0), v(10,5))
%L spec   = "(0,3)--(2,1)--(8,4)--(10,0)"
%L pws    = PwSpec.from(spec)
%L curve  = pws:topict()
%L p = PictList { curve:prethickness("2pt") }
%L p:pgat("pgatc"):preunitlength("15pt"):sa("Exercicio 8 bare"):output()
%L p:addputstrat(v(2,0.4), "\\cell{(2,1)}")
%L p:addputstrat(v(8,4.3), "\\cell{(8,4)}")
%L p:pgat("pgatc"):preunitlength("15pt"):sa("Exercicio 8"):output()
\pu
%
\def\EB{\scalebox{0.6}{$\ga{Exercicio 8 bare}$}}
\def\EC{\scalebox{0.6}{$\ga{Exercicio 8}$}}

Seja:
%
$$f(x) = \EC
$$

\msk

Em cada um dos casos abaixo represente num gráfico só

a função $f$ e os dois somatórios pedidos.

\msk

a) $\mname{supi}_{[1,9]_{2^1}}$, $\mname{infi}_{[1,9]_{2^1}}$

b) $\mname{supi}_{[1,9]_{2^2}}$, $\mname{infi}_{[1,9]_{2^2}}$

c) $\mname{supi}_{[1,9]_{2^3}}$, $\mname{infi}_{[1,9]_{2^3}}$

d) $\mname{max}_{[1,9]_{2^1}}$, $\mname{min}_{[1,9]_{2^1}}$

e) $\mname{max}_{[1,9]_{2^2}}$, $\mname{min}_{[1,9]_{2^2}}$


\newpage

% «exercicio-8-figs»  (to ".exercicio-8-figs")
% (c2m221somas3p 15 "exercicio-8-figs")
% (c2m221somas3a    "exercicio-8-figs")
%$\ga{Exercicio 8 bare}$
%$\ga{Exercicio 8}$

\def\EB{\scalebox{1.0}{$\ga{Exercicio 8 bare}$}}
\def\EC{\scalebox{1.0}{$\ga{Exercicio 8}$}}


\vspace*{-0.25cm}
\hspace*{-0.5cm}
$\scalebox{0.5}{$
 \begin{array}{ccccccc}
 \EC && \EB && \EB && \EB \\[60pt]
 \EB && \EB && \EB && \EB \\[60pt]
 \EB && \EB && \EB && \EB \\[60pt]
 \EB && \EB && \EB && \EB \\[60pt]
 \end{array}
 $}
$



%\printbibliography

\GenericWarning{Success:}{Success!!!}  % Used by `M-x cv'

\end{document}

%  ____  _             _         
% |  _ \(_)_   ___   _(_)_______ 
% | | | | \ \ / / | | | |_  / _ \
% | |_| | |\ V /| |_| | |/ /  __/
% |____// | \_/  \__,_|_/___\___|
%     |__/                       
%
% «djvuize»  (to ".djvuize")
% (find-LATEXgrep "grep --color -nH --null -e djvuize 2020-1*.tex")

 (eepitch-shell)
 (eepitch-kill)
 (eepitch-shell)
# (find-fline "~/2022.1-C2/")
# (find-fline "~/LATEX/2022-1-C2/")
# (find-fline "~/bin/djvuize")

cd /tmp/
for i in *.jpg; do echo f $(basename $i .jpg); done

f () { rm -v $1.pdf;  textcleaner -f 50 -o  5 $1.jpg $1.png; djvuize $1.pdf; xpdf $1.pdf }
f () { rm -v $1.pdf;  textcleaner -f 50 -o 10 $1.jpg $1.png; djvuize $1.pdf; xpdf $1.pdf }
f () { rm -v $1.pdf;  textcleaner -f 50 -o 20 $1.jpg $1.png; djvuize $1.pdf; xpdf $1.pdf }

f () { rm -fv $1.png $1.pdf; djvuize $1.pdf }
f () { rm -fv $1.png $1.pdf; djvuize WHITEBOARDOPTS="-m 1.0 -f 15" $1.pdf; xpdf $1.pdf }
f () { rm -fv $1.png $1.pdf; djvuize WHITEBOARDOPTS="-m 1.0 -f 30" $1.pdf; xpdf $1.pdf }
f () { rm -fv $1.png $1.pdf; djvuize WHITEBOARDOPTS="-m 1.0 -f 45" $1.pdf; xpdf $1.pdf }
f () { rm -fv $1.png $1.pdf; djvuize WHITEBOARDOPTS="-m 0.5" $1.pdf; xpdf $1.pdf }
f () { rm -fv $1.png $1.pdf; djvuize WHITEBOARDOPTS="-m 0.25" $1.pdf; xpdf $1.pdf }
f () { cp -fv $1.png $1.pdf       ~/2022.1-C2/
       cp -fv        $1.pdf ~/LATEX/2022-1-C2/
       cat <<%%%
% (find-latexscan-links "C2" "$1")
%%%
}

f 20201213_area_em_funcao_de_theta
f 20201213_area_em_funcao_de_x
f 20201213_area_fatias_pizza



%  __  __       _        
% |  \/  | __ _| | _____ 
% | |\/| |/ _` | |/ / _ \
% | |  | | (_| |   <  __/
% |_|  |_|\__,_|_|\_\___|
%                        
% <make>

 (eepitch-shell)
 (eepitch-kill)
 (eepitch-shell)
# (find-LATEXfile "2019planar-has-1.mk")
make -f 2019.mk STEM=2022-1-C2-somas-3 veryclean
make -f 2019.mk STEM=2022-1-C2-somas-3 pdf

% Local Variables:
% coding: utf-8-unix
% ee-tla: "c2so3"
% ee-tla: "c2m221somas3"
% End:
