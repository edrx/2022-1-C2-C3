% (find-LATEX "2022-1-C2-dicas-pra-VSA.tex")
% (defun c () (interactive) (find-LATEXsh "lualatex -record 2022-1-C2-dicas-pra-VSA.tex" :end))
% (defun C () (interactive) (find-LATEXsh "lualatex 2022-1-C2-dicas-pra-VSA.tex" "Success!!!"))
% (defun D () (interactive) (find-pdf-page      "~/LATEX/2022-1-C2-dicas-pra-VSA.pdf"))
% (defun d () (interactive) (find-pdftools-page "~/LATEX/2022-1-C2-dicas-pra-VSA.pdf"))
% (defun e () (interactive) (find-LATEX "2022-1-C2-dicas-pra-VSA.tex"))
% (defun o () (interactive) (find-LATEX "2022-1-C2-dicas-pra-VSA.tex"))
% (defun u () (interactive) (find-latex-upload-links "2022-1-C2-dicas-pra-VSA"))
% (defun v () (interactive) (find-2a '(e) '(d)))
% (defun d0 () (interactive) (find-ebuffer "2022-1-C2-dicas-pra-VSA.pdf"))
% (defun cv () (interactive) (C) (ee-kill-this-buffer) (v) (g))
%          (code-eec-LATEX "2022-1-C2-dicas-pra-VSA")
% (find-pdf-page   "~/LATEX/2022-1-C2-dicas-pra-VSA.pdf")
% (find-sh0 "cp -v  ~/LATEX/2022-1-C2-dicas-pra-VSA.pdf /tmp/")
% (find-sh0 "cp -v  ~/LATEX/2022-1-C2-dicas-pra-VSA.pdf /tmp/pen/")
%     (find-xournalpp "/tmp/2022-1-C2-dicas-pra-VSA.pdf")
%   file:///home/edrx/LATEX/2022-1-C2-dicas-pra-VSA.pdf
%               file:///tmp/2022-1-C2-dicas-pra-VSA.pdf
%           file:///tmp/pen/2022-1-C2-dicas-pra-VSA.pdf
% http://angg.twu.net/LATEX/2022-1-C2-dicas-pra-VSA.pdf
% (find-LATEX "2019.mk")
% (find-sh0 "cd ~/LUA/; cp -v Pict2e1.lua Pict2e1-1.lua Piecewise1.lua ~/LATEX/")
% (find-sh0 "cd ~/LUA/; cp -v Pict2e1.lua Pict2e1-1.lua Pict3D1.lua ~/LATEX/")
% (find-sh0 "cd ~/LUA/; cp -v C2Subst1.lua C2Formulas1.lua ~/LATEX/")
% (find-CN-aula-links "2022-1-C2-dicas-pra-VSA" "2" "c2m221dvs" "c2dv")

% «.defs»		(to "defs")
% «.title»		(to "title")
% «.semantica»		(to "semantica")
% «.erros-sintaticos»	(to "erros-sintaticos")
% «.sobre-as-questoes»	(to "sobre-as-questoes")
%
% «.djvuize»	(to "djvuize")



% <videos>
% Video (not yet):
% (find-ssr-links     "c2m221dvs" "2022-1-C2-dicas-pra-VSA")
% (code-eevvideo      "c2m221dvs" "2022-1-C2-dicas-pra-VSA")
% (code-eevlinksvideo "c2m221dvs" "2022-1-C2-dicas-pra-VSA")
% (find-c2m221dvsvideo "0:00")

\documentclass[oneside,12pt]{article}
\usepackage[colorlinks,citecolor=DarkRed,urlcolor=DarkRed]{hyperref} % (find-es "tex" "hyperref")
\usepackage{amsmath}
\usepackage{amsfonts}
\usepackage{amssymb}
\usepackage{pict2e}
\usepackage[x11names,svgnames]{xcolor} % (find-es "tex" "xcolor")
\usepackage{colorweb}                  % (find-es "tex" "colorweb")
%\usepackage{tikz}
%
% (find-dn6 "preamble6.lua" "preamble0")
%\usepackage{proof}   % For derivation trees ("%:" lines)
%\input diagxy        % For 2D diagrams ("%D" lines)
%\xyoption{curve}     % For the ".curve=" feature in 2D diagrams
%
\usepackage{edrx21}               % (find-LATEX "edrx21.sty")
\input edrxaccents.tex            % (find-LATEX "edrxaccents.tex")
\input edrx21chars.tex            % (find-LATEX "edrx21chars.tex")
\input edrxheadfoot.tex           % (find-LATEX "edrxheadfoot.tex")
\input edrxgac2.tex               % (find-LATEX "edrxgac2.tex")
%\usepackage{emaxima}              % (find-LATEX "emaxima.sty")
%
%\usepackage[backend=biber,
%   style=alphabetic]{biblatex}            % (find-es "tex" "biber")
%\addbibresource{catsem-slides.bib}        % (find-LATEX "catsem-slides.bib")
%
% (find-es "tex" "geometry")
\usepackage[a6paper, landscape,
            top=1.5cm, bottom=.25cm, left=1cm, right=1cm, includefoot
           ]{geometry}
%
\begin{document}

\catcode`\^^J=10
\directlua{dofile "dednat6load.lua"}  % (find-LATEX "dednat6load.lua")
%L dofile "Piecewise1.lua"           -- (find-LATEX "Piecewise1.lua")
%L dofile "QVis1.lua"                -- (find-LATEX "QVis1.lua")
%L dofile "Pict3D1.lua"              -- (find-LATEX "Pict3D1.lua")
%L dofile "C2Formulas1.lua"          -- (find-LATEX "C2Formulas1.lua")
%L Pict2e.__index.suffix = "%"
\pu
\def\pictgridstyle{\color{GrayPale}\linethickness{0.3pt}}
\def\pictaxesstyle{\linethickness{0.5pt}}
\def\pictnaxesstyle{\color{GrayPale}\linethickness{0.5pt}}
\celllower=2.5pt

% «defs»  (to ".defs")
% (find-LATEX "edrx21defs.tex" "colors")
% (find-LATEX "edrx21.sty")

\def\u#1{\par{\footnotesize \url{#1}}}

\def\drafturl{http://angg.twu.net/LATEX/2022-1-C2.pdf}
\def\drafturl{http://angg.twu.net/2022.1-C2.html}
\def\draftfooter{\tiny \href{\drafturl}{\jobname{}} \ColorBrown{\shorttoday{} \hours}}



%  _____ _ _   _                               
% |_   _(_) |_| | ___   _ __   __ _  __ _  ___ 
%   | | | | __| |/ _ \ | '_ \ / _` |/ _` |/ _ \
%   | | | | |_| |  __/ | |_) | (_| | (_| |  __/
%   |_| |_|\__|_|\___| | .__/ \__,_|\__, |\___|
%                      |_|          |___/      
%
% «title»  (to ".title")
% (c2m221dvsp 1 "title")
% (c2m221dvsa   "title")

\thispagestyle{empty}

\begin{center}

\vspace*{1.2cm}

{\bf \Large Cálculo 2 - 2022.1}

\bsk

Aula 35: dicas pra VS aberta

(versão muito incompleta!)

\bsk

Eduardo Ochs - RCN/PURO/UFF

\url{http://angg.twu.net/2022.1-C2.html}

\end{center}

\newpage

% «semantica»  (to ".semantica")
% (c2m221dvsp 2 "semantica")
% (c2m221dvsa   "semantica")

{\bf Substituição: semântica}

\scalebox{0.48}{\def\colwidth{11.1cm}\firstcol{

Lembre que no curso eu insisti muito que a operação `$[:=]$' na
maioria dos casos seria usada como uma notação complementar pra coisas
que normalmente são escritas em português, e em várias situações eu
pedi pra vocês fazerem exercícios do Miranda traduzindo-os pra notação
com `$[:=]$'s... por exemplo, eu pedi pra vocês fazerem os exercícios
de regra da cadeia da seção 3.5 do Miranda --

\ssk

{\scriptsize

% (find-dmirandacalcpage 87 "3.5 A Regra da Cadeia")
%    http://hostel.ufabc.edu.br/~daniel.miranda/calculo/calculo.pdf#page=87
\url{http://hostel.ufabc.edu.br/~daniel.miranda/calculo/calculo.pdf\#page=87}

}

\ssk

traduzindo-os pra notação com `$[:=]$'...

\ssk

Eu conversei com alguns amigos meus, e todos eles (ok, confesso: todos
os dois...) disseram o seguinte: todo mundo aprende as regras de como
a substituição funciona estudando centenas de horas e descobrindo como
ela deve funcionar. Por exemplo: as pessoas sabem que a fórmula da
regra da cadeia da página 87 do Miranda tem que valer pra todas as
funções $f,g:\R→\R$ diferenciáveis em todo ponto. Vou dizer que
funções definidas em todo $\R$ e diferenciáveis em todo ponto são
``simples''. Se a gente escolhe duas funções $f$ e $g$ ``simples'',
substitui elas na fórmula da regra da cadeia, e obtém uma igualdade
que é falsa, absurda, ou ``que não compila'', então tem algo errado
com o nosso método de substituir -- o método pra fazer a substituição
que a gente achou que estava certo está errado, e a gente tem que
consertá-lo.

  {\sl Um dos indícios de que um aluno
    de Cálculo 1 estudou o suficiente é que toda vez que ele precisa
    aplicar alguma fórmula famosa ele obtém um caso particular dela do
    jeito certo}.

}\anothercol{

  Lembre que tem alguns tipos de erros que são indícios de que a
  pessoa não aprendeu algo importantíssimo de matérias anteriores, e
  ou não notou que precisava aprender aquilo depois, ou não deu bola
  pra isso...

  Muitos dos meus colegas consideram que Cálculo 2 é uma das matérias
  que têm como funcionar como filtros que não deixam passar pessoas
  que cometem certos tipos de erros gravíssimos, como $2+3=23$, ou
  como aplicar a regra da cadeia errado e obter consequências absurdas
  dela, ou como aplicar o [TFC2] ou o [MV2] e obter consequências
  absurdas deles... e ``Cálculo 2 tem que funcionar como filtro'' quer
  dizer ``Cálculo 2 tem que reprovar pessoas assim''... ${=}{/}$

  \msk

  Durante algumas aulas lá no início do curso eu tentei fazer uma
  experiência didática que talvez tenha sido uma má idéia. Foi isso
  aqui, principalmente nos dias em que a gente trabalhou o exercício
  7:

  \ssk

  {\footnotesize

  % (c2m221introp 10 "exercicio-7")
  % (c2m221introa    "exercicio-7")
  %    http://angg.twu.net/LATEX/2022-1-C2-intro.pdf#page=10
  \url{http://angg.twu.net/LATEX/2022-1-C2-intro.pdf\#page=10}

  }

  \ssk

  Naquela época eu estava tentando fazer as pessoas consideram
  operações de substituição que fossem puramente sintáticas e que
  pudessem transformar expressões que jeitos que não fizessem sentido
  matematicamente... ou seja, que fossem o contrário do ``todo mundo
  aprende as regras de como a substituição funciona estudando centenas
  de horas e descobrindo como ela deve funcionar'' de alguns
  parágrafos atrás. Bom, essa época passou, e agora a gente só está
  interessado em operações de substituição que ``funcionam do jeito
  certo'' -- ou seja, que quando você aplica elas num teorema você
  sempre obtém casos particulares verdadeiros.

}}



\newpage

% «erros-sintaticos»  (to ".erros-sintaticos")
% (c2m221dvsp 3 "erros-sintaticos")
% (c2m221dvsa   "erros-sintaticos")

{\bf Substituição: erros comuns (sintáticos)}


\scalebox{0.6}{\def\colwidth{8.5cm}\firstcol{

\begin{itemize}

\item[CMM] confundiu maiúsculas e minúsculas. Exemplo: transformar
  $f(x)$ em $F(x)$.

\item[ET] erro de tipo

\item[DA] descartou o argumento. Exemplo: escrever só ``$f$'' ao invés
  de ``$f(x)$''.

\item[DE] derivada errada. Exemplo:

  $\bsm{f(x) := (\sen x)^2 \\ f'(x) := (\cos x)^2}$

\item[DPO] descartou parte do original. Exemplo:

  $[RC]\bsm{g(x):=42 \\ g'(x):=43} = (f'(42x)·42)$

  Lembre que o [RC] é uma igualdade.

\item[LEC] lado esquerdo complicado demais. Dois exemplos:

$\bsm{f(g(x)):=(\sen x)^5}$, $\bsm{\intx{\frac1x}:=\ln|x|}$, 


\item[NS] não substituiu. Exemplo:

$(f(x)g(x))\bsm{f(x):=\sen x \\ g(x):=42x} = ((\sen x)g(x))$

\end{itemize}

}\anothercol{

\begin{itemize}

\item[NSD] não substituiu dentro (recursivamente). Exemplo:

$(f(g(x)))\bsm{f(x):=\sen x \\ g(x):=42x} = (\sen (g(x)))$


\item[SIMP] simplificou. Exemplo:

$(f'(g(x))·g'(x)) \bsm{f'(x) := \frac{1}{x} \\ g(x):=x+42 \\ g'(x):= 1}
 = \left(\frac{1}{x+42} \right)$

ao invés de ``$= \left(\frac{1}{x+42} · 1\right)$''.


\item[SP] substituiu parênteses. Pra gente `$dx$' e `$du$' são como
  `)'s, e expressões como ``$\cos x \, dx$'' são tão incompletas como
  ``$\cos x)$'', e coisas como

  $\bsm{\cos x\,dx := ds}$

  não fazem sentido. Lembre que nós passamos um tempo aprendendo a ver
  expressões como árvores e subexpressões como subárvores; quando a
  gente interpreta o `$dx$' como um `$)$' a gente vê que
  `$\cos x\,dx$' nunca corresponde a uma subárvore.

\end{itemize}

}}

\newpage

% «sobre-as-questoes»  (to ".sobre-as-questoes")
% (c2m221dvsp 4 "sobre-as-questoes")
% (c2m221dvsa   "sobre-as-questoes")

{\bf Sobre as questões da prova}

\scalebox{0.6}{\def\colwidth{10cm}\firstcol{

    Na página 2 eu fiz papel de ogro que reprova todo mundo, MAAAS a
    VS aberta só tem como aumentar a nota de quem fizer ela -- se
    alguém tirar uma nota muito baixa na VSA essa nota é ignorada -- e
    ela vai ser sobre assuntos que vale muito a pena estudar... ela
    vai ter pelo menos duas questões de ``encontre a generalização
    certa e aplique ela a este outro caso aqui'', como a questão 4
    da P2,

    \ssk

    {\footnotesize

      % (c2m221p2p 6 "edo-2a-ordem-cont")
      % (c2m221p2a   "edo-2a-ordem-cont")
      %    http://angg.twu.net/LATEX/2022-1-C2-P2.pdf#page=6
      \url{http://angg.twu.net/LATEX/2022-1-C2-P2.pdf\#page=6}

    }

    \ssk

    a continuação dela na VR,

    \ssk

    {\footnotesize

      % (c2m221vrp 4 "questao-2")
      % (c2m221vra   "questao-2")
      % http://angg.twu.net/LATEX/2022-1-C2-VR.pdf#page=4
      \url{http://angg.twu.net/LATEX/2022-1-C2-VR.pdf\#page=4}

    }

    \ssk

    A questão da VSA que vai parecida com essas daí vai ser bem mais
    fácil de fazer se você souber multiplicar números complexos. Dica
    (detalhes depois):

    \ssk

    {\footnotesize

    % (find-angg ".emacs" "c2q192")
    %    http://angg.twu.net/2019.2-C2/2019.2-C2.pdf#page=59
    \url{http://angg.twu.net/2019.2-C2/2019.2-C2.pdf\#page=59}

    }

    \ssk

    A outra questão de ``encontra a generalização certa'' vai ter a
    ver com substituição trigonométrica. Detalhes em breve também!

}\anothercol{
}}






% (c2m221tudop 2 "parts")
% (c2m221tudoa   "parts")
% http://angg.twu.net/LATEX/2022-1-C2-tudo.pdf

% (c2m212tudop 2 "parts")
% (c2m212tudoa   "parts")
% http://angg.twu.net/LATEX/2021-2-C2-tudo.pdf

% (c2m211tudop 2 "parts")
% (c2m211tudoa   "parts")
% http://angg.twu.net/LATEX/2021-1-C2-tudo.pdf

% (c2m202tudop 2 "parts")
% (c2m202tudoa   "parts")
% http://angg.twu.net/LATEX/2020-2-C2-tudo.pdf

% (c2m201tudop 2 "parts")
% (c2m201tudoa   "parts")
% http://angg.twu.net/LATEX/2020-1-C2-tudo.pdf

% (find-LATEX "2020-2-C2-tudo.tex" "parts" "P2 (segunda prova)")



%\printbibliography

\GenericWarning{Success:}{Success!!!}  % Used by `M-x cv'

\end{document}

%  ____  _             _         
% |  _ \(_)_   ___   _(_)_______ 
% | | | | \ \ / / | | | |_  / _ \
% | |_| | |\ V /| |_| | |/ /  __/
% |____// | \_/  \__,_|_/___\___|
%     |__/                       
%
% «djvuize»  (to ".djvuize")
% (find-LATEXgrep "grep --color -nH --null -e djvuize 2020-1*.tex")

 (eepitch-shell)
 (eepitch-kill)
 (eepitch-shell)
# (find-fline "~/2022.1-C2/")
# (find-fline "~/LATEX/2022-1-C2/")
# (find-fline "~/bin/djvuize")

cd /tmp/
for i in *.jpg; do echo f $(basename $i .jpg); done

f () { rm -v $1.pdf;  textcleaner -f 50 -o  5 $1.jpg $1.png; djvuize $1.pdf; xpdf $1.pdf }
f () { rm -v $1.pdf;  textcleaner -f 50 -o 10 $1.jpg $1.png; djvuize $1.pdf; xpdf $1.pdf }
f () { rm -v $1.pdf;  textcleaner -f 50 -o 20 $1.jpg $1.png; djvuize $1.pdf; xpdf $1.pdf }

f () { rm -fv $1.png $1.pdf; djvuize $1.pdf }
f () { rm -fv $1.png $1.pdf; djvuize WHITEBOARDOPTS="-m 1.0 -f 15" $1.pdf; xpdf $1.pdf }
f () { rm -fv $1.png $1.pdf; djvuize WHITEBOARDOPTS="-m 1.0 -f 30" $1.pdf; xpdf $1.pdf }
f () { rm -fv $1.png $1.pdf; djvuize WHITEBOARDOPTS="-m 1.0 -f 45" $1.pdf; xpdf $1.pdf }
f () { rm -fv $1.png $1.pdf; djvuize WHITEBOARDOPTS="-m 0.5" $1.pdf; xpdf $1.pdf }
f () { rm -fv $1.png $1.pdf; djvuize WHITEBOARDOPTS="-m 0.25" $1.pdf; xpdf $1.pdf }
f () { cp -fv $1.png $1.pdf       ~/2022.1-C2/
       cp -fv        $1.pdf ~/LATEX/2022-1-C2/
       cat <<%%%
% (find-latexscan-links "C2" "$1")
%%%
}

f 20201213_area_em_funcao_de_theta
f 20201213_area_em_funcao_de_x
f 20201213_area_fatias_pizza



%  __  __       _        
% |  \/  | __ _| | _____ 
% | |\/| |/ _` | |/ / _ \
% | |  | | (_| |   <  __/
% |_|  |_|\__,_|_|\_\___|
%                        
% <make>

 (eepitch-shell)
 (eepitch-kill)
 (eepitch-shell)
# (find-LATEXfile "2019planar-has-1.mk")
make -f 2019.mk STEM=2022-1-C2-dicas-pra-VSA veryclean
make -f 2019.mk STEM=2022-1-C2-dicas-pra-VSA pdf

% Local Variables:
% coding: utf-8-unix
% ee-tla: "c2dv"
% ee-tla: "c2m221dvs"
% End:
