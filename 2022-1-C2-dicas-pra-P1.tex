% (find-LATEX "2022-1-C2-dicas-pra-P1.tex")
% (defun c () (interactive) (find-LATEXsh "lualatex -record 2022-1-C2-dicas-pra-P1.tex" :end))
% (defun C () (interactive) (find-LATEXsh "lualatex 2022-1-C2-dicas-pra-P1.tex" "Success!!!"))
% (defun D () (interactive) (find-pdf-page      "~/LATEX/2022-1-C2-dicas-pra-P1.pdf"))
% (defun d () (interactive) (find-pdftools-page "~/LATEX/2022-1-C2-dicas-pra-P1.pdf"))
% (defun e () (interactive) (find-LATEX "2022-1-C2-dicas-pra-P1.tex"))
% (defun o () (interactive) (find-LATEX "2022-1-C2-dicas-pra-P1.tex"))
% (defun u () (interactive) (find-latex-upload-links "2022-1-C2-dicas-pra-P1"))
% (defun v () (interactive) (find-2a '(e) '(d)))
% (defun d0 () (interactive) (find-ebuffer "2022-1-C2-dicas-pra-P1.pdf"))
% (defun cv () (interactive) (C) (ee-kill-this-buffer) (v) (g))
%          (code-eec-LATEX "2022-1-C2-dicas-pra-P1")
% (find-pdf-page   "~/LATEX/2022-1-C2-dicas-pra-P1.pdf")
% (find-sh0 "cp -v  ~/LATEX/2022-1-C2-dicas-pra-P1.pdf /tmp/")
% (find-sh0 "cp -v  ~/LATEX/2022-1-C2-dicas-pra-P1.pdf /tmp/pen/")
%     (find-xournalpp "/tmp/2022-1-C2-dicas-pra-P1.pdf")
%   file:///home/edrx/LATEX/2022-1-C2-dicas-pra-P1.pdf
%               file:///tmp/2022-1-C2-dicas-pra-P1.pdf
%           file:///tmp/pen/2022-1-C2-dicas-pra-P1.pdf
% http://angg.twu.net/LATEX/2022-1-C2-dicas-pra-P1.pdf
% (find-LATEX "2019.mk")
% (find-sh0 "cd ~/LUA/; cp -v Pict2e1.lua Pict2e1-1.lua Piecewise1.lua ~/LATEX/")
% (find-sh0 "cd ~/LUA/; cp -v Pict2e1.lua Pict2e1-1.lua Pict3D1.lua ~/LATEX/")
% (find-sh0 "cd ~/LUA/; cp -v C2Subst1.lua C2Formulas1.lua ~/LATEX/")
% (find-CN-aula-links "2022-1-C2-dicas-pra-P1" "2" "c2m221dp1" "c2d1")

% «.defs»		(to "defs")
% «.title»		(to "title")
% «.C2Formulas1»	(to "C2Formulas1")
% «.introducao»		(to "introducao")
% «.underbraces»	(to "underbraces")
%
% «.djvuize»		(to "djvuize")



% <videos>
% Video (not yet):
% (find-ssr-links     "c2m221dp1" "2022-1-C2-dicas-pra-P1")
% (code-eevvideo      "c2m221dp1" "2022-1-C2-dicas-pra-P1")
% (code-eevlinksvideo "c2m221dp1" "2022-1-C2-dicas-pra-P1")
% (find-c2m221dp1video "0:00")

\documentclass[oneside,12pt]{article}
\usepackage[colorlinks,citecolor=DarkRed,urlcolor=DarkRed]{hyperref} % (find-es "tex" "hyperref")
\usepackage{amsmath}
\usepackage{amsfonts}
\usepackage{amssymb}
\usepackage{pict2e}
\usepackage[x11names,svgnames]{xcolor} % (find-es "tex" "xcolor")
\usepackage{colorweb}                  % (find-es "tex" "colorweb")
%\usepackage{tikz}
%
% (find-dn6 "preamble6.lua" "preamble0")
%\usepackage{proof}   % For derivation trees ("%:" lines)
%\input diagxy        % For 2D diagrams ("%D" lines)
%\xyoption{curve}     % For the ".curve=" feature in 2D diagrams
%
\usepackage{edrx21}               % (find-LATEX "edrx21.sty")
\input edrxaccents.tex            % (find-LATEX "edrxaccents.tex")
\input edrx21chars.tex            % (find-LATEX "edrx21chars.tex")
\input edrxheadfoot.tex           % (find-LATEX "edrxheadfoot.tex")
\input edrxgac2.tex               % (find-LATEX "edrxgac2.tex")
%\usepackage{emaxima}              % (find-LATEX "emaxima.sty")
%
%\usepackage[backend=biber,
%   style=alphabetic]{biblatex}            % (find-es "tex" "biber")
%\addbibresource{catsem-slides.bib}        % (find-LATEX "catsem-slides.bib")
%
% (find-es "tex" "geometry")
\usepackage[a6paper, landscape,
            top=1.5cm, bottom=.25cm, left=1cm, right=1cm, includefoot
           ]{geometry}
%
\begin{document}

\catcode`\^^J=10
\directlua{dofile "dednat6load.lua"}  % (find-LATEX "dednat6load.lua")
%L dofile "Piecewise1.lua"           -- (find-LATEX "Piecewise1.lua")
%L dofile "QVis1.lua"                -- (find-LATEX "QVis1.lua")
%L dofile "Pict3D1.lua"              -- (find-LATEX "Pict3D1.lua")
%L dofile "C2Formulas1.lua"          -- (find-LATEX "C2Formulas1.lua")
%L Pict2e.__index.suffix = "%"
\pu
\def\pictgridstyle{\color{GrayPale}\linethickness{0.3pt}}
\def\pictaxesstyle{\linethickness{0.5pt}}
\def\pictnaxesstyle{\color{GrayPale}\linethickness{0.5pt}}
\celllower=2.5pt

% «defs»  (to ".defs")
% (find-LATEX "edrx21defs.tex" "colors")
% (find-LATEX "edrx21.sty")

\def\u#1{\par{\footnotesize \url{#1}}}

\def\drafturl{http://angg.twu.net/LATEX/2022-1-C2.pdf}
\def\drafturl{http://angg.twu.net/2022.1-C2.html}
\def\draftfooter{\tiny \href{\drafturl}{\jobname{}} \ColorBrown{\shorttoday{} \hours}}

\def\und#1#2{\underbrace{#1}_{#2}}



%  _____ _ _   _                               
% |_   _(_) |_| | ___   _ __   __ _  __ _  ___ 
%   | | | | __| |/ _ \ | '_ \ / _` |/ _` |/ _ \
%   | | | | |_| |  __/ | |_) | (_| | (_| |  __/
%   |_| |_|\__|_|\___| | .__/ \__,_|\__, |\___|
%                      |_|          |___/      
%
% «title»  (to ".title")
% (c2m221dp1p 1 "title")
% (c2m221dp1a   "title")

\thispagestyle{empty}

\begin{center}

\vspace*{1.2cm}

{\bf \Large Cálculo 2 - 2022.1}

\bsk

Aula 31: dicas pra P1

\bsk

Eduardo Ochs - RCN/PURO/UFF

\url{http://angg.twu.net/2022.1-C2.html}

\end{center}

\newpage

%\printbibliography

%   ____ ____  _____                          _           _ 
%  / ___|___ \|  ___|__  _ __ _ __ ___  _   _| | __ _ ___/ |
% | |     __) | |_ / _ \| '__| '_ ` _ \| | | | |/ _` / __| |
% | |___ / __/|  _| (_) | |  | | | | | | |_| | | (_| \__ \ |
%  \____|_____|_|  \___/|_|  |_| |_| |_|\__,_|_|\__,_|___/_|
%                                                           
% «C2Formulas1»  (to ".C2Formulas1")
% (c2m221dp1p 2 "C2Formulas1")
% (c2m221dp1a   "C2Formulas1")
% (find-angg "LUA/C2Formulas1.lua")
% (c2m221prp 4 "C2Formulas1-test")
% (c2m221pra   "C2Formulas1-test")
% (find-LATEX "edrxgac2.tex" "C2" "Definite integrals")
\input 2022-1-C2-formulas-defs.tex

\def\pga#1{\left(\ga{#1}\right)}
\def\Sname#1#2{\ensuremath[\text{#1}#2]}
\def\CSname#1#2{{\color{DarkRed} \Sname{#1}{#2}}}
\def\CSname#1#2{{\color{Red} \Sname{#1}{#2}}}
\def\CSname#1#2{{\color{Orange} \Sname{#1}{#2}}}
\def\CSname#1#2{{\color{Orange!80!red} \Sname{#1}{#2}}}
\sa{[S0]}{\CSname{S0}{}}
\sa{[S1]}{\CSname{S1}{}}
\sa{[S2]}{\CSname{S2}{}}
\sa{[S3]}{\CSname{S3}{}}
\sa{[S4]}{\CSname{S4}{}}
\sa{[S5]}{\CSname{S5}{}}
\sa{[S6]}{\CSname{S6}{}}
\sa{[S7]}{\CSname{S7}{}}
\sa{[S8]}{\CSname{S8}{}}
\sa{[S9]}{\CSname{S9}{}}
\sa{[RC]}{\CSname{RC}{}}

\def\Expr{\ang{\mathsf{expr}}}
\def\Expr{\mathsf{expr}}
\def\Expr{α}
\def\Expr{\ColorOrange{α}}
\def\Expr{\ColorRed{α}}

\sa{MV veq} {\veq \ph{mm}}


%  ___       _                 _                       
% |_ _|_ __ | |_ _ __ ___   __| |_   _  ___ __ _  ___  
%  | || '_ \| __| '__/ _ \ / _` | | | |/ __/ _` |/ _ \ 
%  | || | | | |_| | | (_) | (_| | |_| | (_| (_| | (_) |
% |___|_| |_|\__|_|  \___/ \__,_|\__,_|\___\__,_|\___/ 
%                                                      
% «introducao»  (to ".introducao")
% (c2m221dp1p 2 "introducao")
% (c2m221dp1a    "introducao")

% (find-angg "LUA/C2Formulas1.lua" "TFC2")


{\bf Introdução}

%L SA1 = Subst.from("SA1", "\\CSname{SA}{_1}", [[
%L      f(expr1) := sen(expr1)
%L      g(expr1) := Mul(2, expr1)
%L   ]])
%L SA2 = Subst.from("SA2", "\\CSname{SA}{_2}", [[
%L      f(expr1) := sen(SA2(expr1))
%L      g(expr1) := Mul(2, SA2(expr1))
%L   ]])
%L SA1:bmatsas():output()
%L SA2:bmatsas():output()
\pu
%


\scalebox{0.45}{\def\colwidth{12cm}\firstcol{

Nos livros é bem comum encontrarmos trechos como este aqui...

Lembre que está é a fórmula da regra da cadeia:
%
% (c2m221fda "RC")
$$\ga{RC}
$$

Substituindo $f(x)$ e $g(x)$ nela por $\sen x$ e $2x$ respectivamente,

obtemos:
%
$$\D \ddx \sen(2x) = \cos(2x)·2$$

Nós precisávamos de uma notação curta e precisa pra esse

tipo de substituição, e eu introduzi essa notação aqui:
%
$$\begin{array}{c}
  \ga{[RC]} \;=\; \ga{(RC)} \\
  \ga{[RC]}
    \bmat{
      f(x):=\sen x \\
      f'(x):=\cos x \\
      g(x):=2x \\
      g'(x):=2 \\
    }
  \;=\; \left( \D \ddx \sen(2x) = \cos(2x)·2 \right) \\
  \end{array}
$$

}\anothercol{

Quando nós tentamos traduzir essa duas notações --- a original em
português e a formalizada --- pra uma operação ``que um computador
conseguisse executar'' nós vimos que a tradução não era nada óbvia.
Nós fizemos uns exercícios representando as expressões em árvores e
fazendo tudo passo a passo, testamos algumas traduções, e vimos que
se:
%
$$\begin{array}{c}
  \CSname{SA}{_1} = \ga{SA1} = \ga{SA1 lazy} \\ \\[-9pt]
  \CSname{SA}{_2} = \ga{SA2} = \ga{SA2 lazy} \\
  \end{array}
$$

Então
%
$$\begin{array}{c}
  (f(g(100))) \CSname{SA}{_1} \;=\; (\sen(g(100))) \\
  (f(g(100))) \CSname{SA}{_2} \;=\; (\sen(2·100)) \;, \\
  \end{array}
$$

e a tradução $\CSname{SA}{_2}$ é a que corresponde ao que os livros
fazem quando eles dizem ``subtitua''... a $\CSname{SA}{_1}$ é uma
substituição que ``pára pelo meio do caminho'' e não substitui tudo
que deveria.

\msk

Em termos beeem técnicos, acho que a substituição que nós queremos
usar é o ``call-by-name'',

\ssk

{\footnotesize

%    https://en.wikipedia.org/wiki/Evaluation_strategy#Call_by_name
\url{https://en.wikipedia.org/wiki/Evaluation_strategy\#Call_by_name}

}

\ssk

mas numa versão simplificada que não tem o truque pra evitar captura
de variáveis ligadas.

}}



\newpage

%  _   _           _           _                             
% | | | |_ __   __| | ___ _ __| |__  _ __ __ _  ___ ___  ___ 
% | | | | '_ \ / _` |/ _ \ '__| '_ \| '__/ _` |/ __/ _ \/ __|
% | |_| | | | | (_| |  __/ |  | |_) | | | (_| | (_|  __/\__ \
%  \___/|_| |_|\__,_|\___|_|  |_.__/|_|  \__,_|\___\___||___/
%                                                            
% «underbraces»  (to ".underbraces")
% (c2m221dp1p 3 "underbraces")
% (c2m221dp1a   "underbraces")
% (find-angg "LUA/C2Formulas1.lua" "RC-test-und")
%
%L und = function (over, under) return LExpr.from("\\und{<1>}{<2>}", over, under) end
%L 
%L substislazy = nil
%L S1 = Subst.from("S1", "\\ga{[S1]}", [[
%L       f(expr1) := sin(S1(expr1))
%L      fp(expr1) := cos(S1(expr1))
%L       g(expr1) := Mul(S1(42),S1(expr1))
%L      gp(expr1) := 42
%L   ]])
%L 
%L u = function (over, under0)
%L     local under1 = "⇒\\;"..tostring(under0)
%L     return und(over, under1)
%L   end
%L S      = function (a) return mul(2, a) end
%L S      = S1
%L _x     = u(x,      S(x))
%L _gx    = u(g(_x),  S(g(x)))
%L _fgx   = u(f(_gx), S(f(g(x))))
%L _gpx   = u(gp(_x),  S(gp(x)))
%L _fpgx  = u(fp(_gx), S(fp(g(x))))
%L _right = u(Mul(_fpgx,_gpx), S(mul(fp(g(x)),gp(x))))
%L _left  = u(ddvar(_x, _fgx), S(ddvar(x,f(g(x)))))
%L _RC    = u(eq(_left, _right), S(eq(ddvar(x,f(g(x))), mul(fp(g(x)),gp(x)))))
%L
%L substislazy = true
%L S1:bmat():topict():sa("S1") :output()
%L _RC      :topict():sa("_RC"):output()
%L substislazy = false
\pu

\def\und#1#2{\underbrace{\mathstrut #1}_{#2}}

Seja $\ga{[S1]} = \ga{S1}$.

\bsk

Também dá pra calcular $\ga{[RC]}\ga{[S1]}$ desta forma:
%
$$\ga{_RC}$$


\newpage

% (c2m221ftp 1 "title")
% (c2m221fta   "title")
% (c2m221fda   "title")
\input 2022-1-C2-formulas-defs.tex

% «MVs»  (to ".MVs")
% (c2m221ftp 5 "MVs")
% (c2m221fta   "MVs")
% (c2m221fda   "MVs")

$\scalebox{0.6}{$
   \begin{array}{l}
   \ga{[MV1]} \;=\; \ga{(MV1)} \\ \\[-5pt]
   \ga{[MV2]} \;=\; \ga{(MV2)} \\ \\[-5pt]
   \ga{[MV3]} \;=\; \ga{(MV3)}
    \quad
     \ga{[MV4]} \;=\; \ga{(MV4)} \\ \\[-5pt]
   \ga{[MVI3]} \;=\; \ga{(MVI3)}
    \quad
     \ga{[MVI4]} \;=\; \ga{(MVI4)} \\ \\[-5pt]
   \end{array}
 $}
$


\newpage

No mini-teste 2 - link:

\ssk

{\footnotesize

% (c2m221mt2p 2 "questao-E1")
% (c2m221mt2a   "questao-E1")
%    http://angg.twu.net/LATEX/2022-1-C2-MT2.pdf
\url{http://angg.twu.net/LATEX/2022-1-C2-MT2.pdf}

}

\ssk

eu pedi pra vocês justificarem um passo de um exemplo

do livro do Miranda como nesse jogo aqui,

\ssk

{\footnotesize

% (c2m221dfip 6 "o-jogo")
% (c2m221dfia   "o-jogo")
%    http://angg.twu.net/LATEX/2022-1-C2-der-fun-inv.pdf#page=6
\url{http://angg.twu.net/LATEX/2022-1-C2-der-fun-inv.pdf#page=6}

}

como se o jogador $O$ tivesse dito ``justifica essa igualdade

{\sl com um caso particular do \ga{[MV1]} ou do \ga{[MV3]}}'', onde

\ga{[MV1]} e \ga{[MV3]} são as demonstrações da mudança de variável

da página anterior... a \ga{[MV1]} e a \ga{[MV3]} são pra integral

definida, então você vai obter um caso particular que é só

``parecido'' com a igualdade que você quer justificar.



%L substislazy = true
%L S4 = Subst.from("S4", "\\CSname{S4}{}", [[
%L       g(expr1) := sen(mul(4, x))
%L      gp(expr1) := mul(4, cos(mul(4,x)))
%L   ]])
%L S5 = Subst.from("S5", "\\CSname{S5}{}", [[
%L       g(expr1) := mul(frac(1,4),sen(mul(4, x)))
%L      gp(expr1) := cos(mul(4,x))
%L   ]])
%L S6 = Subst.from("S6", "\\CSname{S6}{}", [[
%L       g(expr1) := mul(frac(1,4),sen(mul(4, expr1)))
%L      gp(expr1) := cos(mul(4,S6(expr1)))
%L      fp(expr1) := pot(paren(S6(expr1)),5)
%L   ]])
%L S7 = Subst.from("S7", "\\CSname{S7}{}", [[
%L       g(expr1) := mul(frac(1,4),sen(mul(4, expr1)))
%L      gp(expr1) := cos(mul(4,S7(expr1)))
%L      fp(expr1) := Mul(4, pot(paren(S7(expr1)),5))
%L   ]])
%L S4:defs3():output()
%L S5:defs3():output()
%L S6:defs3():output()
%L S7:defs3():output()
%L substislazy = false
%L MV2      :topict():sa("MV2 orig"):output()
%L S4(MV2)  :topict():sa("MV2 S4")  :output()
%L S5(MV2)  :topict():sa("MV2 S5")  :output()
%L S6(MV2)  :topict():sa("MV2 S6")  :output()
%L S7(MV2)  :topict():sa("MV2 S7")  :output()
\pu

$\scalebox{0.6}{$
 \begin{array}{cr}
%& \ga{[MV2]}          =  \ga{(MV2)}    \\ \\[-2pt]
 & \ga{[MV2]}          = \pga{MV2 orig} \\ \\[-5pt]
             \ga{[S4]} =  \ga{S4}      %\\ \\[-2pt]
 & \ga{[MV2]}\ga{[S4]} = \pga{MV2 S4}   \\
             \ga{[S5]} =  \ga{S5}      %\\ \\[-2pt]
 & \ga{[MV2]}\ga{[S5]} = \pga{MV2 S5}   \\
             \ga{[S6]} =  \ga{S6}      %\\ \\[-2pt]
 & \ga{[MV2]}\ga{[S6]} = \pga{MV2 S6}   \\
             \ga{[S7]} =  \ga{S7}      %\\ \\[-2pt]
 & \ga{[MV2]}\ga{[S7]} = \pga{MV2 S7}   \\
 \end{array}
 $}
$







\GenericWarning{Success:}{Success!!!}  % Used by `M-x cv'

\end{document}

%  ____  _             _         
% |  _ \(_)_   ___   _(_)_______ 
% | | | | \ \ / / | | | |_  / _ \
% | |_| | |\ V /| |_| | |/ /  __/
% |____// | \_/  \__,_|_/___\___|
%     |__/                       
%
% «djvuize»  (to ".djvuize")
% (find-LATEXgrep "grep --color -nH --null -e djvuize 2020-1*.tex")

 (eepitch-shell)
 (eepitch-kill)
 (eepitch-shell)
# (find-fline "~/2022.1-C2/")
# (find-fline "~/LATEX/2022-1-C2/")
# (find-fline "~/bin/djvuize")


%  __  __       _        
% |  \/  | __ _| | _____ 
% | |\/| |/ _` | |/ / _ \
% | |  | | (_| |   <  __/
% |_|  |_|\__,_|_|\_\___|
%                        
% <make>

 (eepitch-shell)
 (eepitch-kill)
 (eepitch-shell)
# (find-LATEXfile "2019planar-has-1.mk")
make -f 2019.mk STEM=2022-1-C2-dicas-pra-P1 veryclean
make -f 2019.mk STEM=2022-1-C2-dicas-pra-P1 pdf

% Local Variables:
% coding: utf-8-unix
% ee-tla: "c2d1"
% ee-tla: "c2m221dp1"
% End:
