% (find-LATEX "2022-1-C3-orbita.tex")
% (defun c () (interactive) (find-LATEXsh "lualatex -record 2022-1-C3-orbita.tex" :end))
% (defun C () (interactive) (find-LATEXsh "lualatex 2022-1-C3-orbita.tex" "Success!!!"))
% (defun D () (interactive) (find-pdf-page      "~/LATEX/2022-1-C3-orbita.pdf"))
% (defun d () (interactive) (find-pdftools-page "~/LATEX/2022-1-C3-orbita.pdf"))
% (defun e () (interactive) (find-LATEX "2022-1-C3-orbita.tex"))
% (defun o () (interactive) (find-LATEX "2022-1-C3-orbita.tex"))
% (defun u () (interactive) (find-latex-upload-links "2022-1-C3-orbita"))
% (defun v () (interactive) (find-2a '(e) '(d)))
% (defun d0 () (interactive) (find-ebuffer "2022-1-C3-orbita.pdf"))
% (defun cv () (interactive) (C) (ee-kill-this-buffer) (v) (g))
%          (code-eec-LATEX "2022-1-C3-orbita")
% (find-pdf-page   "~/LATEX/2022-1-C3-orbita.pdf")
% (find-sh0 "cp -v  ~/LATEX/2022-1-C3-orbita.pdf /tmp/")
% (find-sh0 "cp -v  ~/LATEX/2022-1-C3-orbita.pdf /tmp/pen/")
%     (find-xournalpp "/tmp/2022-1-C3-orbita.pdf")
%   file:///home/edrx/LATEX/2022-1-C3-orbita.pdf
%               file:///tmp/2022-1-C3-orbita.pdf
%           file:///tmp/pen/2022-1-C3-orbita.pdf
% http://angg.twu.net/LATEX/2022-1-C3-orbita.pdf
% (find-LATEX "2019.mk")
% (find-sh0 "cd ~/LUA/; cp -v Pict2e1.lua Pict2e1-1.lua Piecewise1.lua ~/LATEX/")
% (find-sh0 "cd ~/LUA/; cp -v Pict2e1.lua Pict2e1-1.lua Pict3D1.lua ~/LATEX/")
% (find-CN-aula-links "2022-1-C3-orbita" "3" "c3m221orbita" "c3or")

% «.defs»	(to "defs")
% «.title»	(to "title")
% «.links»	(to "links")
% «.orbita»	(to "orbita")
%
% «.djvuize»	(to "djvuize")



% <videos>
% Video (not yet):
% (find-ssr-links     "c3m221orbita" "2022-1-C3-orbita")
% (code-eevvideo      "c3m221orbita" "2022-1-C3-orbita")
% (code-eevlinksvideo "c3m221orbita" "2022-1-C3-orbita")
% (find-c3m221orbitavideo "0:00")

\documentclass[oneside,12pt]{article}
\usepackage[colorlinks,citecolor=DarkRed,urlcolor=DarkRed]{hyperref} % (find-es "tex" "hyperref")
\usepackage{amsmath}
\usepackage{amsfonts}
\usepackage{amssymb}
\usepackage{pict2e}
\usepackage[x11names,svgnames]{xcolor} % (find-es "tex" "xcolor")
\usepackage{colorweb}                  % (find-es "tex" "colorweb")
%\usepackage{tikz}
%
% (find-dn6 "preamble6.lua" "preamble0")
%\usepackage{proof}   % For derivation trees ("%:" lines)
%\input diagxy        % For 2D diagrams ("%D" lines)
%\xyoption{curve}     % For the ".curve=" feature in 2D diagrams
%
\usepackage{edrx21}               % (find-LATEX "edrx21.sty")
\input edrxaccents.tex            % (find-LATEX "edrxaccents.tex")
\input edrx21chars.tex            % (find-LATEX "edrx21chars.tex")
\input edrxheadfoot.tex           % (find-LATEX "edrxheadfoot.tex")
\input edrxgac2.tex               % (find-LATEX "edrxgac2.tex")
%\usepackage{emaxima}              % (find-LATEX "emaxima.sty")
%
%\usepackage[backend=biber,
%   style=alphabetic]{biblatex}            % (find-es "tex" "biber")
%\addbibresource{catsem-slides.bib}        % (find-LATEX "catsem-slides.bib")
%
% (find-es "tex" "geometry")
\usepackage[a6paper, landscape,
            top=1.5cm, bottom=.25cm, left=1cm, right=1cm, includefoot
           ]{geometry}
%
\begin{document}

\catcode`\^^J=10
\directlua{dofile "dednat6load.lua"}  % (find-LATEX "dednat6load.lua")
%L dofile "Piecewise1.lua"           -- (find-LATEX "Piecewise1.lua")
%L dofile "QVis1.lua"                -- (find-LATEX "QVis1.lua")
%L dofile "Pict3D1.lua"              -- (find-LATEX "Pict3D1.lua")
%L Pict2e.__index.suffix = "%"
\pu
\def\pictgridstyle{\color{GrayPale}\linethickness{0.3pt}}
\def\pictaxesstyle{\linethickness{0.5pt}}
\celllower=2.5pt

% «defs»  (to ".defs")
% (find-LATEX "edrx21defs.tex" "colors")
% (find-LATEX "edrx21.sty")

\def\u#1{\par{\footnotesize \url{#1}}}

\def\drafturl{http://angg.twu.net/LATEX/2022-1-C3.pdf}
\def\drafturl{http://angg.twu.net/2022.1-C3.html}
\def\draftfooter{\tiny \href{\drafturl}{\jobname{}} \ColorBrown{\shorttoday{} \hours}}



%  _____ _ _   _                               
% |_   _(_) |_| | ___   _ __   __ _  __ _  ___ 
%   | | | | __| |/ _ \ | '_ \ / _` |/ _` |/ _ \
%   | | | | |_| |  __/ | |_) | (_| | (_| |  __/
%   |_| |_|\__|_|\___| | .__/ \__,_|\__, |\___|
%                      |_|          |___/      
%
% «title»  (to ".title")
% (c3m221orbitap 1 "title")
% (c3m221orbitaa   "title")

\thispagestyle{empty}

\begin{center}

\vspace*{1.2cm}

{\bf \Large Cálculo 3 - 2022.1}

\bsk

Aulas 4 e 5: órbita

\bsk

Eduardo Ochs - RCN/PURO/UFF

\url{http://angg.twu.net/2022.1-C3.html}

\end{center}

\newpage

% «links»  (to ".links")
% (c3m221orbitap 2 "links")
% (c3m221orbitaa   "links")

Nós começamos o curso de C3 em 2022.1

usando este PDF aqui, de 2021.2:

\ssk

{\footnotesize

% (c3m212introp 3 "introducao")
% (c3m212introa   "introducao")
%    http://angg.twu.net/LATEX/2021-2-C3-intro.pdf
\url{http://angg.twu.net/LATEX/2021-2-C3-intro.pdf}

}

\msk

Depois de fazer os exercícios de desenhar

parábolas parametrizadas dele nós passamos pros

exercícios de ``adivinhar trajetórias'' daqui:


\ssk

{\footnotesize

% (c3m212vtp 7 "sobre-adivinhar-trajetorias")
% (c3m212vta   "sobre-adivinhar-trajetorias")
%    http://angg.twu.net/LATEX/2021-2-C3-vetor-tangente.pdf#page=7
\url{http://angg.twu.net/LATEX/2021-2-C3-vetor-tangente.pdf#page=7}

}

\ssk

Logo depois disso, nas aulas de 8 e 13 de abril

de 2022, eu pedi pras pessoas ``adivinharem'' a

trajetória do exercício 4 daqui fazendo só contas

à mão e aproximações no olhômetro:

\ssk

{\footnotesize

% (c3m212vtp 7 "sobre-adivinhar-trajetorias")
% (c3m212vta   "sobre-adivinhar-trajetorias")
%    http://angg.twu.net/2022.1-C3/C3-quadros.pdf#page=5
\url{http://angg.twu.net/2022.1-C3/C3-quadros.pdf\#page=5}

}

\msk

Essa trajetória dá a órbita da próxima página.

\newpage

% «orbita»  (to ".orbita")
% (c3m221orbitap 3 "orbita")
% (c3m221orbitaa   "orbita")

%L PradClass.__index.bshow0 = function (p)
%L     return p:pgat("pgat"):d():tostringp()
%L   end
%L Pict2e.bounds = PictBounds.new(v(-3,-3), v(3,3))
%L 
%L _P  = function (t) return v(   cos(t),       sin(t)) end
%L _Pt = function (t) return v(  -sin(t),       cos(t)) end
%L _Q  = function (t) return v(   cos(4*t)/2,   sin(4*t)/2) end
%L _Qt = function (t) return v(-2*sin(4*t),   2*cos(4*t)  ) end
%L _R  = function (t) return _P(t)+_Q(t) end
%L _Rt = function (t) return _Pt(t)+_Qt(t) end
%L 
%L ts   = seqn(0, 2*pi, 64)
%L r = Plot2D.new {
%L       P  = "t => _R (t)",
%L       Pt = "t => _Rt(t)",
%L       ts = ts
%L   }
%L 
%L ts_v = seqn(0, 2*pi, 6)
%L 
%L p = PictList {
%L       r:toline():Color("Orange"),
%L       r:tovectors(ts_v):Color("Red")
%L     }:prethickness("2pt")
%L p:pgat("pgat"):sa("orbita"):output()
\pu

\unitlength=25pt

$$\ga{orbita}$$



% (find-angg "LUA/Pict2e1-1.lua" "Plot2D-test2")
% (find-pdf-page "~/2022.1-C3/C3-quadros.pdf" 5)
% (find-pdf-page "~/2022.1-C3/C3-quadros.pdf" 6)


%\printbibliography

\GenericWarning{Success:}{Success!!!}  % Used by `M-x cv'

\end{document}

%  ____  _             _         
% |  _ \(_)_   ___   _(_)_______ 
% | | | | \ \ / / | | | |_  / _ \
% | |_| | |\ V /| |_| | |/ /  __/
% |____// | \_/  \__,_|_/___\___|
%     |__/                       
%
% «djvuize»  (to ".djvuize")
% (find-LATEXgrep "grep --color -nH --null -e djvuize 2020-1*.tex")

 (eepitch-shell)
 (eepitch-kill)
 (eepitch-shell)
# (find-fline "~/2022.1-C3/")
# (find-fline "~/LATEX/2022-1-C3/")
# (find-fline "~/bin/djvuize")

cd /tmp/
for i in *.jpg; do echo f $(basename $i .jpg); done

f () { rm -v $1.pdf;  textcleaner -f 50 -o  5 $1.jpg $1.png; djvuize $1.pdf; xpdf $1.pdf }
f () { rm -v $1.pdf;  textcleaner -f 50 -o 10 $1.jpg $1.png; djvuize $1.pdf; xpdf $1.pdf }
f () { rm -v $1.pdf;  textcleaner -f 50 -o 20 $1.jpg $1.png; djvuize $1.pdf; xpdf $1.pdf }

f () { rm -fv $1.png $1.pdf; djvuize $1.pdf }
f () { rm -fv $1.png $1.pdf; djvuize WHITEBOARDOPTS="-m 1.0 -f 15" $1.pdf; xpdf $1.pdf }
f () { rm -fv $1.png $1.pdf; djvuize WHITEBOARDOPTS="-m 1.0 -f 30" $1.pdf; xpdf $1.pdf }
f () { rm -fv $1.png $1.pdf; djvuize WHITEBOARDOPTS="-m 1.0 -f 45" $1.pdf; xpdf $1.pdf }
f () { rm -fv $1.png $1.pdf; djvuize WHITEBOARDOPTS="-m 0.5" $1.pdf; xpdf $1.pdf }
f () { rm -fv $1.png $1.pdf; djvuize WHITEBOARDOPTS="-m 0.25" $1.pdf; xpdf $1.pdf }
f () { cp -fv $1.png $1.pdf       ~/2022.1-C3/
       cp -fv        $1.pdf ~/LATEX/2022-1-C3/
       cat <<%%%
% (find-latexscan-links "C3" "$1")
%%%
}

f 20201213_area_em_funcao_de_theta
f 20201213_area_em_funcao_de_x
f 20201213_area_fatias_pizza



%  __  __       _        
% |  \/  | __ _| | _____ 
% | |\/| |/ _` | |/ / _ \
% | |  | | (_| |   <  __/
% |_|  |_|\__,_|_|\_\___|
%                        
% <make>

 (eepitch-shell)
 (eepitch-kill)
 (eepitch-shell)
# (find-LATEXfile "2019planar-has-1.mk")
make -f 2019.mk STEM=2022-1-C3-orbita veryclean
make -f 2019.mk STEM=2022-1-C3-orbita pdf

% Local Variables:
% coding: utf-8-unix
% ee-tla: "c3or"
% ee-tla: "c3m221orbita"
% End:
