% (find-LATEX "2022-1-C2-primitivas.tex")
% (defun c () (interactive) (find-LATEXsh "lualatex -record 2022-1-C2-primitivas.tex" :end))
% (defun C () (interactive) (find-LATEXsh "lualatex 2022-1-C2-primitivas.tex" "Success!!!"))
% (defun D () (interactive) (find-pdf-page      "~/LATEX/2022-1-C2-primitivas.pdf"))
% (defun d () (interactive) (find-pdftools-page "~/LATEX/2022-1-C2-primitivas.pdf"))
% (defun e () (interactive) (find-LATEX "2022-1-C2-primitivas.tex"))
% (defun o () (interactive) (find-LATEX "2022-1-C2-primitivas.tex"))
% (defun u () (interactive) (find-latex-upload-links "2022-1-C2-primitivas"))
% (defun v () (interactive) (find-2a '(e) '(d)))
% (defun d0 () (interactive) (find-ebuffer "2022-1-C2-primitivas.pdf"))
% (defun cv () (interactive) (C) (ee-kill-this-buffer) (v) (g))
%          (code-eec-LATEX "2022-1-C2-primitivas")
% (find-pdf-page   "~/LATEX/2022-1-C2-primitivas.pdf")
% (find-sh0 "cp -v  ~/LATEX/2022-1-C2-primitivas.pdf /tmp/")
% (find-sh0 "cp -v  ~/LATEX/2022-1-C2-primitivas.pdf /tmp/pen/")
%     (find-xournalpp "/tmp/2022-1-C2-primitivas.pdf")
%   file:///home/edrx/LATEX/2022-1-C2-primitivas.pdf
%               file:///tmp/2022-1-C2-primitivas.pdf
%           file:///tmp/pen/2022-1-C2-primitivas.pdf
% http://angg.twu.net/LATEX/2022-1-C2-primitivas.pdf
% (find-LATEX "2019.mk")
% (find-sh0 "cd ~/LUA/; cp -v Pict2e1.lua Pict2e1-1.lua Piecewise1.lua ~/LATEX/")
% (find-sh0 "cd ~/LUA/; cp -v Pict2e1.lua Pict2e1-1.lua Pict3D1.lua ~/LATEX/")
% (find-sh0 "cd ~/LUA/; cp -v Pict2e1.lua UbExpr1.lua UbExpr2.lua ~/LATEX/")
% (find-sh0 "cd ~/LUA/; cp -v C2Subst1.lua C2Formulas1.lua ~/LATEX/")
% (find-CN-aula-links "2022-1-C2-primitivas" "2" "c2m221pr" "c2pr")

% «.videos-antigos»	(to "videos-antigos")
% «.defs»		(to "defs")
% «.title»		(to "title")
% «.videos-antigos»	(to "videos-antigos")
% «.introducao»		(to "introducao")
% «.introducao-2»	(to "introducao-2")
% «.introducao-3»	(to "introducao-3")
% «.exercicio-1»	(to "exercicio-1")
% «.introducao-4»	(to "introducao-4")
% «.figura-fake-ln»	(to "figura-fake-ln")
%
% «.djvuize»		(to "djvuize")



% <videos>
% Video (not yet):
% (find-ssr-links     "c2m221pr" "2022-1-C2-primitivas")
% (code-eevvideo      "c2m221pr" "2022-1-C2-primitivas")
% (code-eevlinksvideo "c2m221pr" "2022-1-C2-primitivas")
% (find-c2m221prvideo "0:00")

% «videos-antigos»  (to ".videos-antigos")
% (c2m211tfcsa    "video-1")
% (c2m211pra      "video-1")
% (c2m211pra      "video-2")
% (c2m211pra      "video-3")
% (c2m211pra      "video-3" "F'(x)=f(x)")

\documentclass[oneside,12pt]{article}
\usepackage[colorlinks,citecolor=DarkRed,urlcolor=DarkRed]{hyperref} % (find-es "tex" "hyperref")
\usepackage{amsmath}
\usepackage{amsfonts}
\usepackage{amssymb}
\usepackage{pict2e}
\usepackage[x11names,svgnames]{xcolor} % (find-es "tex" "xcolor")
\usepackage{colorweb}                  % (find-es "tex" "colorweb")
%\usepackage{tikz}
%
% (find-dn6 "preamble6.lua" "preamble0")
%\usepackage{proof}   % For derivation trees ("%:" lines)
%\input diagxy        % For 2D diagrams ("%D" lines)
%\xyoption{curve}     % For the ".curve=" feature in 2D diagrams
%
\usepackage{edrx21}               % (find-LATEX "edrx21.sty")
\input edrxaccents.tex            % (find-LATEX "edrxaccents.tex")
\input edrx21chars.tex            % (find-LATEX "edrx21chars.tex")
\input edrxheadfoot.tex           % (find-LATEX "edrxheadfoot.tex")
\input edrxgac2.tex               % (find-LATEX "edrxgac2.tex")
%\usepackage{emaxima}              % (find-LATEX "emaxima.sty")
%
% (c2m221fta   "title")
% (c2m221fda   "title")
\input 2022-1-C2-formulas-defs.tex
%
% (find-es "tex" "geometry")
\usepackage[a6paper, landscape,
            top=1.5cm, bottom=.25cm, left=1cm, right=1cm, includefoot
           ]{geometry}
%
\begin{document}

\catcode`\^^J=10
\directlua{dofile "dednat6load.lua"}  % (find-LATEX "dednat6load.lua")
%L dofile "Piecewise1.lua"           -- (find-LATEX "Piecewise1.lua")
%L -- dofile "QVis1.lua"             -- (find-LATEX "QVis1.lua")
%L -- dofile "Pict3D1.lua"           -- (find-LATEX "Pict3D1.lua")
%L dofile "C2Formulas1.lua"          -- (find-LATEX "C2Formulas1.lua")
%L Pict2e.__index.suffix = "%"
\pu
\def\pictgridstyle{\color{GrayPale}\linethickness{0.3pt}}
\def\pictaxesstyle{\linethickness{0.5pt}}
\celllower=2.5pt

% «defs»  (to ".defs")
% (find-LATEX "edrx21defs.tex" "colors")
% (find-LATEX "edrx21.sty")

\def\u#1{\par{\footnotesize \url{#1}}}

\def\drafturl{http://angg.twu.net/LATEX/2022-1-C2.pdf}
\def\drafturl{http://angg.twu.net/2022.1-C2.html}
\def\draftfooter{\tiny \href{\drafturl}{\jobname{}} \ColorBrown{\shorttoday{} \hours}}

\def\domc{\text{domc}}


%  _____ _ _   _                               
% |_   _(_) |_| | ___   _ __   __ _  __ _  ___ 
%   | | | | __| |/ _ \ | '_ \ / _` |/ _` |/ _ \
%   | | | | |_| |  __/ | |_) | (_| | (_| |  __/
%   |_| |_|\__|_|\___| | .__/ \__,_|\__, |\___|
%                      |_|          |___/      
%
% «title»  (to ".title")
% (c2m221prp 1 "title")
% (c2m221pra   "title")

\thispagestyle{empty}

\begin{center}

\vspace*{1.2cm}

{\bf \Large Cálculo 2 - 2022.1}

\bsk

Aula 31: primitivas

\bsk

Eduardo Ochs - RCN/PURO/UFF

\url{http://angg.twu.net/2022.1-C2.html}

\end{center}

\newpage

% «videos-antigos»  (to ".videos-antigos")
% (c2m221prp 2 "videos-antigos")
% (c2m221pra   "videos-antigos")

{\bf Vídeos antigos}


\scalebox{0.8}{\def\colwidth{12.5cm}\firstcol{

Em 2021.1 eu fiz três vídeos sobre como integrar funções escada que eu
acho que ficaram muito bons. Se você ainda não é capaz de integrar uma
função escada em poucos segundos, comece por eles! Links:

\msk

{\scriptsize

% (c2m211pra      "video-1")
% (find-c2m211prvideo "14:14" "integrando funções escada")

\url{http://angg.twu.net/eev-videos/2021-1-C2-propriedades-da-integral.mp4}

\url{https://www.youtube.com/watch?v=ORfsWiwelV8}

\ssk

% (c2m211pra      "video-2")
\url{http://angg.twu.net/eev-videos/2021-1-C2-propriedades-da-integral-2.mp4}

\url{https://www.youtube.com/watch?v=MmlQTtH5jFo}

\ssk

% (c2m211pra      "video-3")
\url{http://angg.twu.net/eev-videos/2021-1-C2-propriedades-da-integral-3.mp4}

\url{https://www.youtube.com/watch?v=J97x7MNpr90}

}

\msk

A parte sobre integrar funções escada do primeiro vídeo começa no
14:14, e ela é sobre este trecho aqui do PDF (note o page=29):

\msk

{\scriptsize

% (c2m211prp 29 "integrando-escadas")
% (c2m211pra    "integrando-escadas")
%    http://angg.twu.net/LATEX/2021-1-C2-propriedades-da-integral.pdf#page=29
\url{http://angg.twu.net/LATEX/2021-1-C2-propriedades-da-integral.pdf\#page=29}

}

\msk

O segundo vídeo é sobre primitivas diferentes pra mesma função. O
terceiro vídeo é sobre integrar uma função escada resolvendo uma EDO
no olho traçando segmentos de reta com os coeficientes angulares
certos.

}\anothercol{
}}







\newpage

% «introducao»  (to ".introducao")
% (c2m221prp 2 "introducao")
% (c2m221pra   "introducao")

{\bf Introdução}

\ssk

O livro do Miranda define primitiva na página 181,

e define a integral indefinida na página seguinte:

\ssk

{\scriptsize

% (find-youtubedl-links "/sda5/videos/" nil "u4kex7hDC2o" nil "{stem}")
% (code-video "calclogvideo" "/sda5/videos/Your_calculus_prof_lied_to_you_probably-u4kex7hDC2o.webm")
% (find-calclogvideo)
% (find-calclogvideo "0:00")
% (find-calclogvideo "5:20")
% http://www.youtube.com/watch?v=u4kex7hDC2o#t=5m20s
%    http://hostel.ufabc.edu.br/~daniel.miranda/calculo/calculo.pdf#page=182
\url{http://hostel.ufabc.edu.br/~daniel.miranda/calculo/calculo.pdf\#page=182}

}

\msk

Ele faz uma gambiarra que é muito comum em livros de

Cálculo 2, que é definir primitivas e integrais indefinidas

de um jeito simples demais, que não funciona pra fórmula

$\intx{\frac1x} = \ln|x|+C$ (que aparece na p.184!!!) e funciona

bem mal pra funções escada... tem um vídeo bom sobre

isso aqui, mas é em inglês:

\ssk

``Your calculus prof lied to you (probably)''

\url{http://www.youtube.com/watch?v=u4kex7hDC2o}

\newpage

% «introducao-2»  (to ".introducao-2")
% (c2m221prp 3 "introducao-2")
% (c2m221pra   "introducao-2")

{\bf Introdução (2)}

\ssk

Existem vários jeitos de trocar essas definições de

primitiva e integral indefinida ``simples demais''

que a maioria dos livros de C2 usam por outras um

pouco melhores. O meu jeito preferido é o que eu

vou explicar nas próximas páginas.

\msk

Existe uma notação padrão pro domínio de uma

função $f$: ``$\dom(f)$''. Vamos definir o conjunto

$\domc(f)$ --- pronúncia: o ``domínio de continuidade''

da $f$ --- como o subconjunto de $\dom(f)$ que só contém

os pontos em que a $f$ é contínua...


\newpage

% «introducao-3»  (to ".introducao-3")
% (c2m221prp 4 "introducao-3")
% (c2m221pra   "introducao-3")

{\bf Introdução (3)}

Agora vamos escolher a definição de primitiva mais

simples possível que faça as soluções dos exercícios

8 e 9 daqui serem primitivas da $f$ do enunciado:

\ssk

{\footnotesize

% (c2m221tfc1p 48 "exercicio-8")
% (c2m221tfc1a    "exercicio-8")
%    http://angg.twu.net/LATEX/2022-1-C2-TFC1.pdf#page=48
\url{http://angg.twu.net/LATEX/2022-1-C2-TFC1.pdf#page=48}

}

\ssk

A definição vai ser esta aqui:

\begin{quotation}

  Uma função $F$ é uma primitiva da função $f$

  quando estas quatro condições são obedecidas:

  \ssk

  1) $\dom(F)=\dom(f)$,

  2) $F$ é contínua --- ou seja, $\domc(F) = \dom(F)$,

  3) a $F$ é derivável em todos os pontos de $\domc(f)$,

  4) para todo $x∈\domc(f)$ temos $F'(x)=f(x)$.

\end{quotation}

\newpage

% «exercicio-1»  (to ".exercicio-1")

{\bf Exercício 1.}

Entenda a definição do slide anterior e verifique que 


\newpage




% «introducao-4»  (to ".introducao-4")
% (c2m221prp 3 "introducao-4")
% (c2m221pra   "introducao-4")


\newpage

{\bf P1 e escadas}


% (c2m202escadasp 12 "exercicio-6")
% (c2m202escadas     "exercicio-6")
% (c2m202escadasp 11 "exercicio-7")
% (c2m202escadas     "exercicio-7")
% (c2m202escadasp 18 "exercicio-9")
% (c2m202escadasa    "exercicio-9")


% (c2m202escadasp 17 "primitivas-como-usar")
% (c2m202escadas     "primitivas-como-usar")

% (find-books "__analysis/__analysis.el" "miranda")
% (find-dmirandacalcpage 181 "antiderivada")
% (find-dmirandacalcpage 182 "antiderivada" "figura")
% (find-dmirandacalctext 181 "antiderivada")

% (c2m211tfcsp 16 "integral-indefinida")
% (c2m211tfcsa    "integral-indefinida")
% (c2m211tfcsp 17 "exercicio-5")
% (c2m211tfcsa    "exercicio-5")

% (find-angg "LUA/Pict2e1.lua" "fake_ln_mod_x")

% «figura-fake-ln»  (to ".figura-fake-ln")
% (c2m221prp 3 "figura-fake-ln")
% (c2m221pra   "figura-fake-ln")

%L Pict2e.bounds = PictBounds.new(v(-5,-5), v(5,5))
%L fake_ln_mod_x = function (C1, C2)
%L     return PictList {
%L       PictList({})
%L         :addline(v(-5, 3+C1),
%L                  v(-3, 3+C1),
%L                  v(-2, 2+C1),
%L                  v(-1, 0+C1),
%L                  v(-0,-3+C1))
%L         :addopendotat(v(-0,-3+C1))
%L         :Color("Orange"),
%L       PictList({})
%L         :addline(v( 5, 3+C2),
%L                  v( 3, 3+C2),
%L                  v( 2, 2+C2),
%L                  v( 1, 0+C2),
%L                  v( 0,-3+C2))
%L         :addopendotat(v( 0,-3+C2))
%L         :Color("Red"),
%L     }
%L   end
%L th = "2pt"
%L fake_ln_mod_x( 1, 1):prethickness(th):pgat("pgatc"):sa("fake ln  1  1"):output()
%L fake_ln_mod_x( 1, 0):prethickness(th):pgat("pgatc"):sa("fake ln  1  0"):output()
%L fake_ln_mod_x( 1,-1):prethickness(th):pgat("pgatc"):sa("fake ln  1 -1"):output()
%L fake_ln_mod_x( 0, 1):prethickness(th):pgat("pgatc"):sa("fake ln  0  1"):output()
%L fake_ln_mod_x( 0, 0):prethickness(th):pgat("pgatc"):sa("fake ln  0  0"):output()
%L fake_ln_mod_x( 0,-1):prethickness(th):pgat("pgatc"):sa("fake ln  0 -1"):output()
%L fake_ln_mod_x(-1, 1):prethickness(th):pgat("pgatc"):sa("fake ln -1  1"):output()
%L fake_ln_mod_x(-1, 0):prethickness(th):pgat("pgatc"):sa("fake ln -1  0"):output()
%L fake_ln_mod_x(-1,-1):prethickness(th):pgat("pgatc"):sa("fake ln -1 -1"):output()
\pu

\def\closeddot{\circle*{0.7}}
\def\opendot  {\circle*{0.8}\color{white}\circle*{0.5}}
\unitlength=5pt

\def\fak#1{\ga{fake ln #1}}
\def\fak#1{\scalebox{0.6}{$\ga{fake ln #1}$}}
%\scalebox{1.0}{$\ga{fake ln 0 0}$}$ \; ,


$a \fak{0 0} b$

$$\begin{array}{cccc}
   C_1=1  & \fak{1  -1} & \fak{1 0} & \fak{1 1} \\ \\[-5pt]
   C_1=0  & \fak{0  -1} & \fak{0 0} & \fak{0 1} \\ \\[-5pt]
   C_1=-1 & \fak{-1 -1} & \fak{0 0} & \fak{0 1} \\ \\[-5pt]
          &    C_2=-1  &    C_2=0 &      C_2=1  \\ \\[-5pt]
  \end{array}
$$

Compare com:


\newpage

% (c2m211prp 45 "TFC1-escadas")
% (c2m211pra    "TFC1-escadas")




\GenericWarning{Success:}{Success!!!}  % Used by `M-x cv'

\end{document}

%  ____  _             _         
% |  _ \(_)_   ___   _(_)_______ 
% | | | | \ \ / / | | | |_  / _ \
% | |_| | |\ V /| |_| | |/ /  __/
% |____// | \_/  \__,_|_/___\___|
%     |__/                       
%
% «djvuize»  (to ".djvuize")
% (find-LATEXgrep "grep --color -nH --null -e djvuize 2020-1*.tex")


%  __  __       _        
% |  \/  | __ _| | _____ 
% | |\/| |/ _` | |/ / _ \
% | |  | | (_| |   <  __/
% |_|  |_|\__,_|_|\_\___|
%                        
% <make>

 (eepitch-shell)
 (eepitch-kill)
 (eepitch-shell)
# (find-LATEXfile "2019planar-has-1.mk")
make -f 2019.mk STEM=2022-1-C2-primitivas veryclean
make -f 2019.mk STEM=2022-1-C2-primitivas pdf

% Local Variables:
% coding: utf-8-unix
% ee-tla: "c2pr"
% ee-tla: "c2m221pr"
% End:
