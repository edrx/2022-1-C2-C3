% (find-LATEX "2022-1-C2-algumas-t-ints.tex")
% (defun c () (interactive) (find-LATEXsh "lualatex -record 2022-1-C2-algumas-t-ints.tex" :end))
% (defun C () (interactive) (find-LATEXsh "lualatex 2022-1-C2-algumas-t-ints.tex" "Success!!!"))
% (defun D () (interactive) (find-pdf-page      "~/LATEX/2022-1-C2-algumas-t-ints.pdf"))
% (defun d () (interactive) (find-pdftools-page "~/LATEX/2022-1-C2-algumas-t-ints.pdf"))
% (defun e () (interactive) (find-LATEX "2022-1-C2-algumas-t-ints.tex"))
% (defun o () (interactive) (find-LATEX "2022-1-C2-algumas-t-ints.tex"))
% (defun u () (interactive) (find-latex-upload-links "2022-1-C2-algumas-t-ints"))
% (defun v () (interactive) (find-2a '(e) '(d)))
% (defun d0 () (interactive) (find-ebuffer "2022-1-C2-algumas-t-ints.pdf"))
% (defun cv () (interactive) (C) (ee-kill-this-buffer) (v) (g))
%          (code-eec-LATEX "2022-1-C2-algumas-t-ints")
% (find-pdf-page   "~/LATEX/2022-1-C2-algumas-t-ints.pdf")
% (find-sh0 "cp -v  ~/LATEX/2022-1-C2-algumas-t-ints.pdf /tmp/")
% (find-sh0 "cp -v  ~/LATEX/2022-1-C2-algumas-t-ints.pdf /tmp/pen/")
%     (find-xournalpp "/tmp/2022-1-C2-algumas-t-ints.pdf")
%   file:///home/edrx/LATEX/2022-1-C2-algumas-t-ints.pdf
%               file:///tmp/2022-1-C2-algumas-t-ints.pdf
%           file:///tmp/pen/2022-1-C2-algumas-t-ints.pdf
% http://angg.twu.net/LATEX/2022-1-C2-algumas-t-ints.pdf
% (find-LATEX "2019.mk")
% (find-sh0 "cd ~/LUA/; cp -v Pict2e1.lua Pict2e1-1.lua Piecewise1.lua ~/LATEX/")
% (find-sh0 "cd ~/LUA/; cp -v Pict2e1.lua Pict2e1-1.lua Pict3D1.lua ~/LATEX/")
% (find-CN-aula-links "2022-1-C2-algumas-t-ints" "2" "c2m221atis" "c2at")
% (find-CN-aula-links "2022-1-C2-formulas-defs" "2" "c2m221fd" "c2fd")
% (find-CN-aula-links "2022-1-C2-formulas-test" "2" "c2m221ft" "c2ft")

% «.videos-antigos»		(to "videos-antigos")
%
% «.defs»			(to "defs")
% «.formulas-defs»		(to "formulas-defs")
% «.title»			(to "title")
% «.historinha»			(to "historinha")
% «.exercicio-1»		(to "exercicio-1")
% «.exercicio-2»		(to "exercicio-2")
% «.exercicio-3»		(to "exercicio-3")
% «.exercicio-2»		(to "exercicio-2")
% «.x^-2»			(to "x^-2")
% «.1-then-2»			(to "1-then-2")
% «.S2-proof-1»			(to "S2-proof-1")
% «.formulas-mv-2022.1»		(to "formulas-mv-2022.1")
% «.dfi»			(to "dfi")
% «.exercicio-3»		(to "exercicio-3")
% «.mais-algumas»		(to "mais-algumas")
% «.exercicio-4»		(to "exercicio-4")
% «.juntando»			(to "juntando")
% «.um-exemplo»			(to "um-exemplo")
% «.mv-outro-exemplo»		(to "mv-outro-exemplo")
% «.substituicao-figura»	(to "substituicao-figura")
% «.links»			(to "links")
%
% «.exemplo-contas-1b»	(to "exemplo-contas-1b")
% «.S2I»	(to "S2I")
% «.S3I»	(to "S3I")
%
% «.djvuize»		(to "djvuize")



% «videos-antigos»  (to ".videos-antigos")
% (c2m212intsa    "video-1")
% (c2m212intsa    "video-2")
% (c2m212intsa    "video-3")

% Video (not yet):
% (find-ssr-links     "c2m221atis" "2022-1-C2-algumas-t-ints")
% (code-eevvideo      "c2m221atis" "2022-1-C2-algumas-t-ints")
% (code-eevlinksvideo "c2m221atis" "2022-1-C2-algumas-t-ints")
% (find-c2m221atisvideo "0:00")

\documentclass[oneside,12pt]{article}
\usepackage[colorlinks,citecolor=DarkRed,urlcolor=DarkRed]{hyperref} % (find-es "tex" "hyperref")
\usepackage{amsmath}
\usepackage{amsfonts}
\usepackage{amssymb}
\usepackage{pict2e}
\usepackage[x11names,svgnames]{xcolor} % (find-es "tex" "xcolor")
\usepackage{colorweb}                  % (find-es "tex" "colorweb")
%\usepackage{tikz}
%
% (find-dn6 "preamble6.lua" "preamble0")
%\usepackage{proof}   % For derivation trees ("%:" lines)
%\input diagxy        % For 2D diagrams ("%D" lines)
%\xyoption{curve}     % For the ".curve=" feature in 2D diagrams
%
\usepackage{edrx21}               % (find-LATEX "edrx21.sty")
\input edrxaccents.tex            % (find-LATEX "edrxaccents.tex")
\input edrx21chars.tex            % (find-LATEX "edrx21chars.tex")
\input edrxheadfoot.tex           % (find-LATEX "edrxheadfoot.tex")
\input edrxgac2.tex               % (find-LATEX "edrxgac2.tex")
%\usepackage{emaxima}              % (find-LATEX "emaxima.sty")
%
% (find-es "tex" "geometry")
\usepackage[a6paper, landscape,
            top=1.5cm, bottom=.25cm, left=1cm, right=1cm, includefoot
           ]{geometry}
%
\begin{document}

\catcode`\^^J=10
\directlua{dofile "dednat6load.lua"}  % (find-LATEX "dednat6load.lua")
%L dofile "Piecewise1.lua"           -- (find-LATEX "Piecewise1.lua")
%L dofile "QVis1.lua"                -- (find-LATEX "QVis1.lua")
%L dofile "Pict3D1.lua"              -- (find-LATEX "Pict3D1.lua")
%L Pict2e.__index.suffix = "%"
\pu
\def\pictgridstyle{\color{GrayPale}\linethickness{0.3pt}}
\def\pictaxesstyle{\linethickness{0.5pt}}
\celllower=2.5pt

% «defs»  (to ".defs")
% (c2m221atisa "defs")
% (find-LATEX "edrx21defs.tex" "colors")
% (find-LATEX "edrx21.sty")

\def\u#1{\par{\footnotesize \url{#1}}}

\def\drafturl{http://angg.twu.net/LATEX/2022-1-C2.pdf}
\def\drafturl{http://angg.twu.net/2022.1-C2.html}
\def\draftfooter{\tiny \href{\drafturl}{\jobname{}} \ColorBrown{\shorttoday{} \hours}}

\def\pfo#1{\ensuremath{\mathsf{[#1]}}}
\def\veq{\rotatebox{90}{$=$}}
\def\Rd{\ColorRed}
\def\Rq{\ColorRed{?}}
\def\D{\displaystyle}

% Difference with mathstrut
\def\Difms #1#2#3{\left. \mathstrut #3 \right|_{s=#1}^{s=#2}}
\def\Difmu #1#2#3{\left. \mathstrut #3 \right|_{u=#1}^{u=#2}}
\def\Difmx #1#2#3{\left. \mathstrut #3 \right|_{x=#1}^{x=#2}}
\def\Difmth#1#2#3{\left. \mathstrut #3 \right|_{θ=#1}^{θ=#2}}

\def\iequationbox#1#2{
    \left(
    \begin{array}{rcl}
    \D{ #1 } &=& \D{ #2 } \\
    \end{array}
    \right)
  }
\def\isubstbox#1#2#3#4#5{{
    \def\veq{\rotatebox{90}{$=$}}
    \def\ph{\phantom}
    \left(
    \begin{array}{rcl}
    \D{ #1 } &=& \D{ #2 } \\
    {\veq#3} \\
    \D{ #4 } &=& \D{ #5 } \\
    \end{array}
    \right)
  }}
\def\isubstboxT#1#2#3#4#5#6{{
    \def\veq{\rotatebox{90}{$=$}}
    \def\ph{\phantom}
    \left(
    \begin{array}{rcl}
    \multicolumn{3}{l}{\text{#6}} \\%[5pt]
    \D{ #1 } &=& \D{ #2 } \\
    {\veq#3} \\
    \D{ #4 } &=& \D{ #5 } \\
    \end{array}
    \right)
  }}
\def\isubstboxTT#1#2#3#4#5#6#7{{
    \def\veq{\rotatebox{90}{$=$}}
    \def\ph{\phantom}
    \left(
    \begin{array}{rcl}
    \multicolumn{3}{l}{\text{#6}} \\%[5pt]
    \D{ #1 } &=& \D{ #2 } \\
    {\veq#3} \\
    \D{ #4 } &=& \D{ #5 } \\
    \multicolumn{3}{l}{\text{#7}} \\%[5pt]
    \end{array}
    \right)
  }}

% Definição das fórmulas para integração por substituição.
% Algumas são pmatrizes 3x3 usando isubstbox.

\def\TFCtwo{
  \iequationbox {\Intx{a}{b}{F'(x)}}
                {\Difmx{a}{b}{F(x)}}
}
\def\TFCtwoI{
  \iequationbox {\intx{F'(x)}}
                {F(x)}
}

\def\Sone{
  \isubstbox
    {\Difmx{a}{b}{f(g(x))}}  {\Intx{a}{b}{f'(g(x))g'(x)}}
    {\ph{mmm}}
    {\Difmu{g(a)}{g(b)}{f(u)}} {\Intu{g(a)}{g(b)}{f'(u)}}
}
\def\SoneI{
  \isubstbox
    {f(g(x))} {\intx{f'(g(x))g'(x)}}
    {\ph{m}}
    {f(u)}    {\intu{f'(u)}}
}

\def\Stwo{
  \isubstboxT
    {\Difmx{a}{b}{F(g(x))}}   {\Intx{a}{b}{f(g(x))g'(x)}}
    {\ph{mmm}}
    {\Difmu{g(a)}{g(b)}{F(u)}}  {\Intu{g(a)}{g(b)}{f(u)}}
    {Se $F'(u)=f(u)$ então:}
}
\def\StwoI{
  \isubstboxT
    {F(g(x))}  {\intx{f(g(x))g'(x)}}
    {\ph{m}}
    {F(u)}     {\intu{f(u)}}
    {Se $F'(u)=f(u)$ então:}
}
\def\StwoI{
  \isubstboxTT
    {F(g(x))}  {\intx{f(g(x))g'(x)}}
    {\ph{m}}
    {F(u)}     {\intu{f(u)}}
    {Se $F'(u)=f(u)$ então:}
    {Obs: $u=g(x)$.}
}

\def\Sthree{
  \iequationbox {\Intx{a}{b}{f(g(x))g'(x)}}
                {\Intu{g(a)}{g(b)}{f(u)}}
}
\def\SthreeI{
  \iequationbox {\intx{f(g(x))g'(x)}}
                {\intu{f(u)}
                 \qquad [u=g(x)]
                }
  % [u=g(x)]
}

\def\Sthree{
  \pmat{
    \D \Intx{a}{b}{f(g(x))g'(x)} \\
    \veq \\
    \D \Intu{g(a)}{g(b)}{f(u)}
  }}

\def\SthreeI{
  \pmat{
    \D \intx{f(g(x))g'(x)} \\
       \veq \\
    \D \intu{f(u)} \\
    \text{Obs: $u=g(x)$.} \\
  }}

\def\Subst#1{\bmat{#1}}

\def\Ps  #1{\left( #1 \right) }
\def\ps  #1{     ( #1       ) }
\def\nops#1{       #1         }
\def\righte{\quad\text{e}}

\def\Rd#1{{\ColorRed{#1}}}
\def\Rdq {{\ColorRed{?}}}

% «formulas-defs»  (to ".formulas-defs")
% (c2m221ftp 1 "title")
% (c2m221fta   "title")
% (c2m221fda   "title")
\input 2022-1-C2-formulas-defs.tex



%  _____ _ _   _                               
% |_   _(_) |_| | ___   _ __   __ _  __ _  ___ 
%   | | | | __| |/ _ \ | '_ \ / _` |/ _` |/ _ \
%   | | | | |_| |  __/ | |_) | (_| | (_| |  __/
%   |_| |_|\__|_|\___| | .__/ \__,_|\__, |\___|
%                      |_|          |___/      
%
% «title»  (to ".title")
% (c2m221atisp 1 "title")
% (c2m221atisa   "title")

\thispagestyle{empty}

\begin{center}

\vspace*{1.2cm}

{\bf \Large Cálculo 2 - 2022.1}

\bsk

Aula 29: algumas técnicas de integração

\bsk

Eduardo Ochs - RCN/PURO/UFF

\url{http://angg.twu.net/2022.1-C2.html}

\end{center}

\newpage

Aviso: estou reorganizando este PDF e

reescrevendo muita coisa do semestre passado!

Ele ainda está uma bagunça!

\bsk

% (c2m221pra   "title")
% (c2m221pra   "title" "Aula nn: primitivas")



No mini-teste 3 - link:

\ssk

{\footnotesize

% (c2m212mt3p 3 "questao")
% (c2m212mt3a   "questao")
% (c2m212mt3p 4 "gabarito")
% (c2m212mt3a   "gabarito")
%    http://angg.twu.net/LATEX/2021-2-C2-MT3.pdf#page=4
\url{http://angg.twu.net/LATEX/2021-2-C2-MT3.pdf#page=4}

}

\ssk

vocês viram que quando a função $G$ ``é uma integral da $f$''

nós podemos fazer contas como esta aqui:
%
$$\Intx{2}{5}{f(x)} \;\;=\;\; G(5) - G(2)$$

Isto é um caso particular do TFC2,

que tem várias versões diferentes...

a \ColorRed{fórmula} dele é essa aqui:

$$\Intx{a}{b}{F'(x)} \;\;=\;\; \difx{a}{b}{F(x)}$$

\newpage

% «historinha»  (to ".historinha")
% (c2m221atisp 3 "historinha")
% (c2m221atisa    "historinha")

{\bf Historinha (pros meus amigos lógicos)}

\scalebox{0.45}{\def\colwidth{12cm}\firstcol{


Nas últimas aulas nós vimos um pouquinho sobre: 1) como fazer só as
``contas'' de uma demonstração, deixando pra completar as
``hipóteses'' depois, e 2) como fazer demonstrações em que cada passo
seja ``fácil de justificar''...

Isso tem a ver com um dos meus assuntos de pesquisa e com uma
tentativa de lidar com uma coisa que toda vez que acontece me deixa
com muito ódio durante anos. Vou tentar explicar.

\msk

Hoje em dia existe uma noção {\sl que todo mundo aceita} (!!!) do que
é uma demonstração ``completa'': é uma demonstração {\sl formalizada
  num proof assistant}. Esse artigo aqui explica isso bem:

\ssk

{\footnotesize

% (find-books "__comp/__comp.el" "buzzard-wp")
%    https://arxiv.org/pdf/2112.11598.pdf
\url{https://arxiv.org/pdf/2112.11598.pdf}

}

\ssk

Eu aprendi um pouquinho de dois proof assistants importantes: o Lean,
que o artigo acima descreve, e o Agda. ``Um pouquinho'' quer dizer que
eu segui metade de cada um desses tutoriais daqui:

% (find-es "lean" "natural-number-game")
% (find-es "agda" "plfa")

\ssk

{\footnotesize

\url{https://www.ma.imperial.ac.uk/~buzzard/xena/natural_number_game/}

\url{https://plfa.github.io/}

}

\ssk

Eu conheco a base teórica por trás dos proof assistants ---
``Dependent Type Theory'' --- razoavelmente bem, mas os proof
assistants em si eu conheço mal.

\msk

{\sl Praticamente todas as operações, regras e táticas que aparecem
nos dois tutoriais acima são consequência do `$[:=]$'.}

\msk

Os módulos do mathlib do Lean pra demonstrações de contas com
derivadas são bem difíceis de entender:

{\footnotesize

% (find-es "lean" "mathlib-calculus")
%    https://leanprover-community.github.io/mathlib_docs/analysis/calculus/deriv.html
\url{https://leanprover-community.github.io/mathlib_docs/analysis/calculus/deriv.html}

}

e os módulos pra contas com integrais são mais difíceis ainda. Eu acho
que isso é porque eles exigem que o usuário faça as ``contas'' com
todas as ``hipóteses'' especificadas corretamente...

}\anothercol{


O meu assunto principal de pesquisa é como formalizar demonstrações
--- por enquanto só numa área chamada Teoria de Categorias, que não
tem quase nada a ver com Cálculo --- em duas etapas, como se a gente
fizesse as ``contas'' primeiro e só acrescentasse as ``hipóteses''
depois. Tem dois artigos meus que explicam isso de jeitos bastante
legíveis:

\ssk

{\footnotesize

% (find-TH "math-b" "2022-md")
%    http://angg.twu.net/math-b.html#2022-md
%    http://angg.twu.net/math-b.html#idarct

\url{http://angg.twu.net/math-b.html\#2022-md}

\url{http://angg.twu.net/math-b.html\#idarct}

}

\msk

Acho --- sendo otimista, óbvio --- que daqui a uns anos eu vou
conseguir adaptar essas idéias pra formalizar a matéria de Cálculo 2
em Lean. E quando eu conseguir fazer isso eu vou conseguir lidar bem
melhor com uma das coisas que me dá mais ódio no meu trabalho como
professor, que são aqueles alunos que escrevem os maiores absurdos
numa prova e depois ficam insistindo AOS BERROS que tá tudo certo na
prova deles e que eu só dei nota baixa pra eles porque eu tava de
marcação com eles... aí eu vou poder responder ``{\sl se a sua
  resposta tá certa traz ela formalizada em Lean}''.

}}


\newpage


% «diferenca»  (to ".diferenca")
% (c2m211tfcsp 2 "diferenca")
% (c2m211tfcsa   "diferenca")

{\bf A operação ``diferença''}


\scalebox{0.8}{\def\colwidth{12cm}\firstcol{

Defs:
%
$$\begin{array}{rcl}
  \Difx{a}{b}{f(x)} &=& f(b) - f(a) \\
  \Difx{a}{b}{\textsf{expr}} &=&
    (\textsf{expr})[x:=b] -
    (\textsf{expr})[x:=a] \\
  \ga{[DefDif]} &=& \ga{(DefDif)}
  \end{array}
$$

Os livros costumam usar a primeira definição acima.

\bsk

% «exercicio-1»  (to ".exercicio-1")
% (c2m211tfcsp 2 "exercicio-1")
% (c2m211tfcsa   "exercicio-1")

{\bf Exercício 1.}

Expanda e simplifique o máximo possível:

\msk

\begin{tabular}[t]{cl}
a) & $\difx{4}{5}{x^2} $ \\[5pt]
b) & $\difx{5}{4}{x^2} $ \\[5pt]
c) & $\difx{4}{5}{2}   $ \\[5pt]
d) & $\dift{4}{5}{t^2} $ \\[5pt]
e) & $\dift{4}{5}{x^2} $ \\[5pt]
\end{tabular}
\qquad
\begin{tabular}[t]{cl}
f) & $\difx{2}{10}{(x^3-x^2)} $ \\[5pt]
g) & $\difx{2}{10}{x^3} -  \difx{2}{10}{x^2}$ \\[5pt]
h) & $             x^3  - (\difx{2}{10}{x^2})$ \\[5pt]
\end{tabular}

}\anothercol{
}}

\newpage

% «exercicio-2»  (to ".exercicio-2")
% (c2m211tfcsp 10 "exercicio-2")
% (c2m211tfcsa    "exercicio-2")

{\bf Exercício 2.}

\scalebox{0.9}{\def\colwidth{10cm}\firstcol{

Lembre que:
%
$$\ga{[TFC2]} \;=\; \ga{(TFC2)}$$

a) Calcule o resultado desta substituição:
%
$$\ga{[TFC2]}
  \bmat{
    F(x) := 2x^2 - \frac{x^3}{3} \\
    F'(x) := 4x - x^2 \\
    a := 0 \\
    b := 4 \\
  }
  = \Rq
$$

b) Use o resultado do item (a) pra calcular

$\Intx{0}{4}{4 - (x-2)^2}$. Dica: o resultado final

vai ser $32/3$.


}\anothercol{
}}

\newpage    

% «exercicio-3»  (to ".exercicio-3")
% (c2m211tfcsp 12 "exercicio-3")
% (c2m211tfcsa    "exercicio-3")

Vamos chamar o método do slide anterior de

``integração por TFC2 e chutar-e-testar''...

Obs: no exercício eu chutei a $F(x)$ certa.

\msk

{\bf Exercício 3.}

Integre por TFC2 e chutar-e-testar:

\msk

a) $\D \Intx{0}{π/2}{\cos x} = \Rd{?}$

\msk

b) $\D \Intx{0}{π}{\sen x} = \Rd{?}$

\msk

c) $\D \Intx{π/2}{π}{\sen x} = \Rd{?}$

\msk

d) $\D \Intx{5}{6}{\sen(2x + 3)} = \Rd{?}$



\newpage    

%  ____ ____                           __   _ 
% / ___|___ \   _ __  _ __ ___   ___  / _| / |
% \___ \ __) | | '_ \| '__/ _ \ / _ \| |_  | |
%  ___) / __/  | |_) | | | (_) | (_) |  _| | |
% |____/_____| | .__/|_|  \___/ \___/|_|   |_|
%              |_|                            
%
% «S2-proof-1»  (to ".S2-proof-1")
% (c2m212intsp 7 "S2-proof-1")
% (c2m212intsa   "S2-proof-1")

\def\TfcDois{[\text{TFC2}]}
\def\DefDif {[\text{DefDif}]}
\def\DEFDIFA #1{ \difx{a}{b}{F(x)}  #1 = #1 F(b) - F(a)       }
\def\TFCDOISA#1{ \Intx{a}{b}{F'(x)} #1 = #1 \difx{a}{b}{F(x)} }
\def\TFCP    #1{ \D \left( #1 \right) }

\sa{TFC2-ap1-S}{ F(x)  := f(g(x))       \\
                 F'(x) := f'(g(x))g'(x) \\
               }
\sa{TFC2-ap1-L}{ \Intx{a}{b}{f'(g(x))g'(x)} }
\sa{TFC2-ap1-R}{ \difx{a}{b}{f(g(x))}       }

\sa{TFC2-ap2-S}{ x := u         \\
                 b := g(b)      \\
                 a := g(a)      \\
                 F(u)  := f(u)  \\
                 F'(u) := f'(u) \\
               }
\sa{TFC2-ap2-L}{ \Intu{g(a)}{g(b)}{f'(u)} }
\sa{TFC2-ap2-R}{ \difu{g(a)}{g(b)}{f(u)}       }

\sa{DEFDIF-ap1-S}{ F(x)  := f(g(x))       \\
                 }
\sa{DEFDIF-ap2-S}{ x     := u      \\
                   F(u)  := f(u)   \\
                   a     := g(a)   \\
                   b     := g(b)   \\
                 }

\sa{TFC2-ap1}  { \TFCP{ \ga{TFC2-ap1-L}         \;\;=\;\; \ga{TFC2-ap1-R}   } }
\sa{TFC2-ap2}  { \TFCP{ \ga{TFC2-ap2-L}         \;\;=\;\; \ga{TFC2-ap2-R}   } }
\sa{DEFDIF-ap1}{ \TFCP{ \difx{a}{b}{f(g(x))}    \;\;=\;\; f(g(b)) - f(g(a)) } }
\sa{DEFDIF-ap2}{ \TFCP{ \difu{g(a)}{g(b)}{f(u)} \;\;=\;\; f(g(b)) - f(g(a)) } }

\sa{S2-primeira-versao}{
 \begin{array}{rcl}
   \D \ga{TFC2-ap1-L} &=& \D \ga{TFC2-ap1-R} \\
                      &=& f(g(b)) - f(g(a))  \\[7.5pt]
                      &=& \D \ga{TFC2-ap2-R} \\[7.5pt]
                      &=& \D \ga{TFC2-ap2-L}
 \end{array}}
%
%\vspace*{-0.75cm}
%
%\scalebox{0.65}{\def\colwidth{14cm}\firstcol{
%
%$$\begin{array}{rcl}
% \DefDif  &=& \TFCP{ \DEFDIFA{\;\;}  } \\
% \TfcDois &=& \TFCP{ \TFCDOISA{\;\;} } \\
% [20pt]
% \DefDif \bsm{\ga{DEFDIF-ap1-S}} &=& \ga{DEFDIF-ap1} \\
% \DefDif \bsm{\ga{DEFDIF-ap2-S}} &=& \ga{DEFDIF-ap2} \\
% [20pt]
% \TfcDois \bsm{\ga{TFC2-ap1-S}} &=& \ga{TFC2-ap1} \\
% \TfcDois \bsm{\ga{TFC2-ap2-S}} &=& \ga{TFC2-ap2} \\
% \end{array}
%$$
%
%$$\ga{S2-primeira-versao}$$
%
%
%%}\anothercol{
%}}

\newpage

% «formulas-mv-2022.1»  (to ".formulas-mv-2022.1")
% (c2m221atisp 8 "formulas-mv-2022.1")
% (c2m221atisa   "formulas-mv-2022.1")

{\bf Fórmulas pra mudança de variáveis (2022.1)}

% (c2m221ftp 5 "MVs")
% (c2m221fta   "MVs")
% (c2m221fda   "MVs")

\msk

$\scalebox{0.55}{$
   \begin{array}{l}
   \ga{[MV1]} \;=\; \ga{(MV1)} \\ \\[-5pt]
   \ga{[MV2]} \;=\; \ga{(MV2)} \\ \\[-5pt]
   \ga{[MV3]} \;=\; \ga{(MV3)}
    \quad
     \ga{[MV4]} \;=\; \ga{(MV4)} \\ \\[-5pt]
   \ga{[MVI3]} \;=\; \ga{(MVI3)}
    \quad
     \ga{[MVI4]} \;=\; \ga{(MVI4)} \\ \\[-5pt]
   \end{array}
 $}
$


\newpage

{\bf Uma demonstração do \ga{[MV1]}}

\scalebox{0.6}{\def\colwidth{14cm}\firstcol{

\vspace*{-0.5cm}

$$\begin{array}{rcl}
 \ga{[DefDif]}  &=& \ga{(DefDif)} \\
 \ga{[TFC2]}    &=& \ga{(TFC2)} \\
 [20pt]
 \ga{[DefDif]} \bsm{\ga{DEFDIF-ap1-S}} &=& \ga{DEFDIF-ap1} \\
 \ga{[DefDif]} \bsm{\ga{DEFDIF-ap2-S}} &=& \ga{DEFDIF-ap2} \\
 [20pt]
 \ga{[TFC2]} \bsm{\ga{TFC2-ap1-S}} &=& \ga{TFC2-ap1} \\
 \ga{[TFC2]} \bsm{\ga{TFC2-ap2-S}} &=& \ga{TFC2-ap2} \\
 \end{array}
$$

$$\ga{MV1}$$
%$\ga{S2-primeira-versao}$

}\anothercol{
}}



\newpage

% «um-exemplo»  (to ".um-exemplo")
% (c2m221atisp 16 "um-exemplo")
% (c2m221atisa    "um-exemplo")
% (c2m212intsp 13 "um-exemplo")
% (c2m212intsa    "um-exemplo")

{\bf Um exemplo de mudança de variável}

\def\P  #1{\left(    #1 \right)}
\def\Pga#1{\left(\ga{#1}\right)}

\sa{Emv1}{[\text{EMV1}]}
\sa{Emv2}{[\text{EMV2}]}
\sa{Emv3}{[\text{EMV3}]}
\sa{Emv4}{[\text{EMV4}]}
\sa{Emv5}{[\text{EMV5}]}
\sa{EMV1}{
 \begin{array}{rcl}
       \D \Intx{  a }{  b }{f'(g(x))g'(x)}
   &=& \D \difx{  a }{  b }{f (g(x))     } \\
   &=& f(g(b))            - f (g(a))       \\[7.5pt]
   &=& \D \difu{g(a)}{g(b)}{f (u)}         \\[7.5pt]
   &=& \D \Intu{g(a)}{g(b)}{f'(u)}
 \end{array}}
\sa{EMV2}{
 \begin{array}{rcl}
       \D \Intx{  a }{  b }{f'(2x)·2}
   &=& \D \difx{  a }{  b }{f (2x)     } \\
   &=& f(2b)              - f (2a)       \\[7.5pt]
   &=& \D \difu{2a}{2b}{f (u)}         \\[7.5pt]
   &=& \D \Intu{2a}{2b}{f'(u)}
 \end{array}}
\sa{EMV3}{
 \begin{array}{rcl}
       \D \Intx{  a }{  b }{\sen(2x)·2}
   &=& \D \difx{  a }{  b }{(-\cos(2x))} \\
   &=& (-\cos(2b))         -(-\cos(2a))  \\[7.5pt]
   &=& \D \difu{2a}{2b}{(-\cos(u))}         \\[7.5pt]
   &=& \D \Intu{2a}{2b}{\sen(u)}
 \end{array}}
\sa{EMV4}{
 \begin{array}{rcl}
       \D \Intu{2a}{2b}{\sen(u)}
   &=& \D \Intx{ a}{ b}{\sen(2x)·2 \,}
 \end{array}}
\sa{EMV5}{
 \begin{array}{rcl}
       \D \Intu{a  }{b  }{\sen(u)}
   &=& \D \Intx{a/2}{b/2}{2\sen(2x)}
 \end{array}}



\msk

\scalebox{0.49}{\def\colwidth{9cm}\firstcol{

$\begin{array}{rcl}
  \ga{Emv1} &=& \Pga{EMV1} \\[55pt]
  \ga{Emv2} \;\;=\;\;
            \ga{Emv1} \bmat{g(x) := 2x \\ g'(x) := 2}
            &=& \Pga{EMV2} \\[55pt]
  \ga{Emv3} \;\;=\;\;
            \ga{Emv2} \bmat{f(x) := -\cos x \\ f'(x) := \sen x}
            &=& \Pga{EMV3} \\[55pt]
  \ga{Emv4} &=& \Pga{EMV4} \\[15pt]
  \ga{Emv5} &=& \Pga{EMV5} \\
  \end{array}
$

%}\anothercol{
}}


\newpage

% «mv-outro-exemplo»  (to ".mv-outro-exemplo")
% (c2m221atisp 11 "mv-outro-exemplo")
% (c2m221atisa    "mv-outro-exemplo")

{\bf Outro exemplo de mudança de variável}

\def\P  #1{\left(    #1 \right)}
\def\Pga#1{\left(\ga{#1}\right)}

\sa{Oemv3}{[\text{OEMV3}]}
\sa{Oemv4}{[\text{OEMV4}]}
\sa{Oemv5}{[\text{OEMV5}]}
\sa{OEMV3}{
 \begin{array}{rcl}
       \D \Intx{  a }{  b }{\tan(2x)·2}
   &=& \D \difx{  a }{  b }{(f(2x))} \\
   &=& (f(2b))         -(f(2a))  \\[7.5pt]
   &=& \D \difu{2a}{2b}{(f(u))}         \\[7.5pt]
   &=& \D \Intu{2a}{2b}{\tan(u)}
 \end{array}}
\sa{OEMV4}{
 \begin{array}{rcl}
       \D \Intu{2a}{2b}{\tan(u)}
   &=& \D \Intx{ a}{ b}{\tan(2x)·2 \,}
 \end{array}}
\sa{OEMV5}{
 \begin{array}{rcl}
       \D \Intu{a  }{b  }{\tan(u)}
   &=& \D \Intx{a/2}{b/2}{2\tan(2x)}
 \end{array}}

\msk

\scalebox{0.7}{\def\colwidth{9cm}\firstcol{

Aqui a gente não substitui a $f$, só a $f'$...

Digamos que $f(x) = \Intt{c}{x}{\tan t}$,

e portanto $f'(x) = \tan x$.

$\begin{array}{rcl}
  %\ga{Emv1} &=& \Pga{EMV1} \\[55pt]
  %\ga{Emv2} \;\;=\;\;
  %          \ga{Emv1} \bmat{g(x) := 2x \\ g'(x) := 2}
  %          &=& \Pga{EMV2} \\[55pt]
  \ga{Oemv3} \;\;=\;\;
            \ga{Emv2} \bmat{f'(x) := \tan x}
            &=& \Pga{OEMV3} \\[55pt]
  \ga{Oemv4} &=& \Pga{OEMV4} \\[15pt]
  \ga{Oemv5} &=& \Pga{OEMV5} \\
  \end{array}
$

%}\anothercol{
}}

\newpage

% «substituicao-figura»  (to ".substituicao-figura")
% (c2m221atisp 12 "substituicao-figura")
% (c2m221atisa    "substituicao-figura")
% (find-angg "LUA/Piecewise1.lua" "ChangeVar-test1")

{\bf Uma figura pra mudança de variável}

%L Pict2e.bounds = PictBounds.new(v(0,0), v(4,3))
%L pi, sqrt, sin, cos = math.pi, math.sqrt, math.sin, math.cos
%L ve = Code.ve
%L cv = ChangeVar {
%L   xtou = ve " cv,x =>     x^2        ",
%L   fx   = ve " cv,x => sin(x^2) * 2*x ",
%L   fu   = ve " cv,u => sin( u )       ",
%L   utox = ve " cv,u =>  sqrt(u)       ",
%L }
%L cv:setus(0, pi, {0, 1, 2, 3})
%L cv:setpwfs()
%L cv:setcolors()
%L Pict2e.bounds = PictBounds.new(v(0,0), v(4,3))
%L ppx = PictList {
%L   cv:areasx(),
%L   cv:curvex(),
%L   -- cv:rect(cv.xs[2], cv.xs[3], cv:fx(cv.xs[2])):color("blue"),
%L   cv:xlabels(-0.35),
%L }
%L ppu = PictList {
%L   cv:areasu(),
%L   cv:curveu(),
%L   cv:ulabels(-0.35)
%L }
%L ppu:pgat("pgatc"):sa("color sin(u)")      :output()
%L ppx:pgat("pgatc"):sa("color sin(x^2)*2*x"):output()

\pu

\vspace*{-0.1cm}

$$\unitlength=50pt
  \begin{array}{rcl}
  x^2 &=& u \\
  \D \Intx{a}{b}{\sen(x^2)·2x}
  &=&
  \D \Intu{a^2}{b^2}{\sen u}
  \\
  \\
  \scalebox{0.6}{$\ga{color sin(x^2)*2*x}$}
  &&
  \scalebox{0.6}{$\ga{color sin(u)}$}
  \\
  \end{array}
$$

% Nós vimos lá atrás que cada igualdade daqui
% era verdade (com as hipóteses certas)...




% (c2m211isp 13 "hipotese")
% (c2m211isa    "hipotese")


\newpage

% «links»  (to ".links")
% (c2m221atisp 16 "links")
% (c2m221atisa    "links")

{\bf Links}

\scalebox{0.9}{\def\colwidth{13cm}\firstcol{

Aqui tem links pra como alguns livros apresentam

o método da mudança de variáveis...

% (c2m211isp 27 "exercicio-3")
% (c2m211isa    "exercicio-3")
% (c2m211isp 27 "exercicio-4")
% (c2m211isa    "exercicio-4")

\ssk

{\scriptsize

% (find-books "__analysis/__analysis.el" "martins-martins")
% (find-martinscdipage (+ 10 109) "4.2       Integral")
% (find-martinscditext (+ 10 109) "4.2       Integral")
% (find-martinscdipage (+ 10 165) "6" "Metodos de Integracao")
% (find-martinscditext (+ 10 165) "6" "Metodos de Integracao")
% (find-martinscdipage (+ 10 165) "6.1       Metodo da Substituicao")
% (find-martinscditext (+ 10 165) "6.1       Metodo da Substituicao")
% http://angg.twu.net/2021.1-C2/martins_martins__sec_6.1.pdf
\url{http://angg.twu.net/2021.1-C2/martins_martins__sec_6.1.pdf}

% (find-books "__analysis/__analysis.el" "apex-calculus")
% (find-apexcalculuspage (+ 10 263) "6.1 Substitution")
% (find-apexcalculuspage (+ 10 280)     "Exercises 6.1")
% (find-twusfile "2021.1-C2/")
% http://angg.twu.net/2021.1-C2/APEX_Calculus_Version_4_BW_secs_6.1_6.2.pdf
\url{http://angg.twu.net/2021.1-C2/APEX_Calculus_Version_4_BW_secs_6.1_6.2.pdf}

% (find-books "__analysis/__analysis.el" "thomas")
% (find-thomas11-1page (+  61  368) "5.5 Indefinite integrals and the substituion rule")
% (find-thomas11-1page (+  61  369)     "Example 1")
% (find-thomas11-1page (+  61  370)     "Example 2")
% (find-thomas11-1page (+  61  371)     "Example 3")
%    http://angg.twu.net/2020.2-C2/thomas_secoes_5.5_e_5.6.pdf
\url{http://angg.twu.net/2020.2-C2/thomas_secoes_5.5_e_5.6.pdf}

% (find-books "__analysis/__analysis.el" "ross")
% (find-books "__analysis/__analysis.el" "ross" "ross20221")
%    http://angg.twu.net/2022.1-C2/ross__elementary_analysis_secs_10_32_33_34.pdf#page=42
\url{http://angg.twu.net/2022.1-C2/ross__elementary_analysis_secs_10_32_33_34.pdf\#page=42}

}

\bsk

De todos esses o único que eu acho realmente honesto é o Ross.

\msk

Em todos os outros a variável nova é tratada à vezes como

uma variável independente, às vezes como uma variável

dependente, e às vezes como uma abreviação...

Eu levei mais de 10 anos pra entender como essa gambiarra

funcionava. Vocês vão ver um pouco sobre o meu modo

atual de entender ela em Cálculo 3, nesta parte do curso:


\msk

{\scriptsize

% (c3m212nfp 9 "regras-de-traducao")
% (c3m212nfa   "regras-de-traducao")
%    http://angg.twu.net/LATEX/2021-2-C3-notacao-de-fisicos.pdf#page=9
\url{http://angg.twu.net/LATEX/2021-2-C3-notacao-de-fisicos.pdf\#page=9}

}






}\anothercol{
}}



\newpage

% «exemplo-contas»  (to ".exemplo-contas")
% (c2m221atisp 18 "exemplo-contas")
% (c2m221atisa    "exemplo-contas")

% (c2m211isp 6 "exemplo-contas")
% (c2m211isa   "exemplo-contas")
% (c2m202isp 9 "exemplo-gamb")
% (c2m202isa   "exemplo-gamb")

{\bf Um exemplo com contas}


\scalebox{0.85}{\def\colwidth{12cm}\firstcol{

Isto aqui é um exemplo de como contas com integração

por substituição costumam ser feitas na prática:
%
$$\scalebox{0.95}{$
  \begin{array}{l}
  \D \intx{2 \cos(3x+4)} \\[8pt]
  = \;\; \D \intu {2 (\cos u) · \frac13}
    \\[8pt]
  = \;\; \D \frac23 \intu{\cos u} \\[8pt]
  = \;\; \D \frac23 \sen u \\[8pt]
  = \;\; \D \frac23 \sen (3x+4) \\
  \end{array}
  $}
$$

É necessário indicar em algum lugar que a relação

entre a variável nova e a antiga é esta: $u=3x+4$.

\ssk

Dê uma olhada nos exemplos do livro do Miranda:

\ssk

{\scriptsize

% (find-dmirandacalcpage 189 "6.2 Integração por Substituição")
%    http://hostel.ufabc.edu.br/~daniel.miranda/calculo/calculo.pdf#page=190
\url{http://hostel.ufabc.edu.br/~daniel.miranda/calculo/calculo.pdf\#page=190}

}

}\anothercol{
}}


\newpage

% «exemplo-contas-1b»  (to ".exemplo-contas-1b")
% (c2m221atisp 19 "exemplo-contas-1b")
% (c2m221atisa    "exemplo-contas-1b")

{\bf Um exemplo com contas (cont.)}

\sa{2 cos(3x+4) full}{
  \begin{array}{l}
  \D \Intx{a}{b}{2 \cos(3x+4)} \\[8pt]
  = \;\; \D \Intu{3a+4}{3b+4} {2 (\cos u) · \frac13}
    \\[8pt]
  = \;\; \D \frac23 \Intu{3a+4}{3b+4} {\cos u} \\[8pt]
  = \;\; \D \frac23 \left(\difu{3a+4}{3b+4} {(\sen u)} \right)\\[8pt]
  = \;\; \D \frac23 \left(\difx{a}{b} {(\sen (3x+4))} \right)\\
  \end{array}
}
\sa{2 cos(3x+4) thin}{
  \begin{array}{l}
  \D \intx{2 \cos(3x+4)} \\[8pt]
  = \;\; \D \intu {2 (\cos u) · \frac13}
    \\[8pt]
  = \;\; \D \frac23 \intu{\cos u} \\[8pt]
  = \;\; \D \frac23 \sen u \\[8pt]
  = \;\; \D \frac23 \sen (3x+4) \\
  \end{array}
}

Compare:
%
$$\scalebox{0.75}{$
  \left( \ga{2 cos(3x+4) full} \right) 
  \qquad
  \left( \ga{2 cos(3x+4) thin} \right) 
  $}
$$

Nós vamos tratar a versão da direita como uma abreviação da versão da
esquerda --- pra ir da esquerda pra direita a gente apaga os limites
de integração, o que é bem fácil... pra ir da direita pra esquerda a
gente precisa reconstruir os limites de integração, o que é mais
difícil.


\newpage


% «exemplo-contas-2»  (to ".exemplo-contas-2")
% (c2m211isp 7 "exemplo-contas-2")
% (c2m211isa   "exemplo-contas-2")

{\bf Outro exemplo com contas}
%
\def\S{\sen x}
\def\C{\cos x}
\def\und#1#2{\underbrace{#1}_{#2}}
%
$$\begin{array}[t]{l}
  \D \intx{(\S)^5 (\C)^3} \\
  \D = \;\; \intx{(\S)^5 (\C)^2 (\C)} \\
  \D = \;\; \intx{(\und{\S}{s})^5 \und{(\C)^2}{1-s^2} \und{(\C)}{\frac{ds}{dx}}} \\
  \D = \;\; \ints{s^5 (1-s^2)} \\
  \D = \;\; \ints{s^5 - s^7} \\
  \D = \;\; \frac{s^6}{6} - \frac{s^8}{8} \\
  \D = \;\; \frac{(\S)^6}{6} - \frac{(\S)^8}{8} \\
  \end{array}
  \qquad
  \begin{array}[t]{c}
  \\ \\
    \bmat{s = \sen x \\
          \frac{ds}{dx} = \cos x \\
          \sen x = s \\
          (\cos x)^2 = 1 - s^2 \\
          \cos x \, dx = ds
    }
  \end{array}
$$

\newpage

% «subst-int-def»  (to ".subst-int-def")
% (c2m221atisp 22 "subst-int-def")
% (c2m221atisa    "subst-int-def")
% (c2m211isp 8 "subst-int-def")
% (c2m211isa   "subst-int-def")

{\bf Substituição na integral definida}

Eu vou chamar a \ColorRed{demonstração} abaixo de \pfo{S2}.

Ela é uma série de três igualdades: o `$=$' de cima,

o `$=$' de baixo, e o `$=$' da esquerda (que é um `$\,\rotl{=}$').

Eu vou chamar o ``$F'(u)=f(u)$'' de a \ColorRed{hipótese} do \pfo{S2}.

Obs: nós \ColorRed{ainda} não acreditamos nessa demonstração...

vamos verificar as igualdades dela daqui a alguns slides.
%
% (c2m202isp 3 "def-S2-S2I")
% (c2m202isa   "def-S2-S2I")
%
$$\begin{array}{rcc}
 \pfo{S2} &=& \Stwo \\
 % \\
 % \pfo{S2I} &=& \StwoI \\
 \end{array}
$$


\newpage

% «so-alguns-simbolos»  (to ".so-alguns-simbolos")
% (c2m211isp 9 "so-alguns-simbolos")
% (c2m211isa   "so-alguns-simbolos")

Lembre que dá pra substituir só alguns símbolos...

Por exemplo:
%
\def\Stwotmp{
  \isubstboxT
    {\Difmx{a}{b}{F(2x)}}   {\Intx{a}{b}{f(2x)·2}}
    {\ph{mmm}}
    {\Difmu{2a}{2b}{F(u)}}  {\Intu{2a}{2b}{f(u)}}
    {Se $F'(u)=f(u)$ então:}
}
%
$$\scalebox{0.9}{$
  \begin{array}{c}
 \pfo{S2} \;\;=\;\; \Stwo \\
 [50pt]
 \pfo{S2}[g(x):=2x] \;\;=\;\; \Stwotmp \\
 \end{array}
  $}
$$

\newpage

% «hip-triv-true»  (to ".hip-triv-true")
% (c2m211isp 10 "hip-triv-true")
% (c2m211isa    "hip-triv-true")

Também podemos substituir o $f$ por $F'$...

E aí a hipótese passa a ser ``trivialmente verdadeira'':
%
\def\Stwotmp{
  \isubstboxT
    {\Difmx{a}{b}{F(g(x))}}   {\Intx{a}{b}{F'(g(x))g'(x)}}
    {\ph{mmm}}
    {\Difmu{g(a)}{g(b)}{F(u)}}  {\Intu{g(a)}{g(b)}{F'(u)}}
    {Se $F'(u)=F'(u)$ então:}
}
%
$$\scalebox{0.9}{$
  \begin{array}{c}
  \pfo{S2} \;\;=\;\; \Stwo \\
  [50pt]
  \pfo{S2}[f(u):=F'(u)] \;\;=\;\; \Stwotmp \\
  \end{array}
  $}
$$

\newpage

%       /\      ____  
% __  _|/\|    |___ \ 
% \ \/ /   _____ __) |
%  >  <   |_____/ __/ 
% /_/\_\       |_____|
%                     
% «x^-2»  (to ".x^-2")
% (c2m221atisp 5 "x^-2")
% (c2m221atisa   "x^-2")
% (c2m212intsp 5 "x^-2")
% (c2m212intsa   "x^-2")
% (find-angg "LUA/Piecewise1.lua" "PwFunction-testpoles")

{\bf Um caso em que o TFC2 dá um resultado errado}

\ssk

%L Pict2e.bounds = PictBounds.new(v(-4,-4), v(4,4))
%L plotdot = function (x, y) return PictList{}:addcloseddotat(v(x,y)) end
%L trunc = function (y) return min(max(-4,y),4) end
%L f = function (x) return x==0 and 4 or trunc(1/x^2) end
%L F = function (x) return x==0 and 4 or trunc(-1/x)  end
%L pwf = PwFunction.from(f, seqn(-4, 4, 64))
%L pwF = PwFunction.from(F, seqn(-4, 4, 64))
%L p = PictList {
%L   pwf:areaify(-1, 1):Color("Orange"),
%L   pwf:pw(-4, -1/2),
%L   pwf:pw(1/2, 4),
%L }
%L p:pgat("pgatc"):sa("x^-2 f"):output()
%L p = PictList {
%L   pwF:pw(-4, -1/4),
%L   pwF:pw(1/4, 4),
%L   plotdot(-1, F(-1)),
%L   plotdot(1,  F(1))
%L }
%L p:pgat("pgatc"):sa("x^-2 F"):output()
\pu

\GenericWarning{Success:}{Success!!!}  % Used by `M-x cv'

\pu

\unitlength=10pt

Se $F(x) = -x^{-1}$

então $F'(x) = x^{-2}$, e:

$$\begin{array}{rcl}
  \Intx{-1}{1}{F'(x)} &=& \difx{-1}{1}{F(x)} \\
  \Intx{-1}{1}{x^{-2}}
        & = & \difx{-1}{1}{(- x^{-1})} \\
        & = & (- 1^{-1}) - (- (-1)^{-1}) \\
        & = & -2 \\
  \end{array}
$$

$$\ga{x^-2 f}
  \;\; = \;\;
  \ga{x^-2 F}
  \qquad
  \frown
$$


\newpage

% «1-then-2»  (to ".1-then-2")
% (c2m221atisp 7 "1-then-2")
% (c2m221atisa   "1-then-2")
% (c2m212intsp 6 "1-then-2")
% (c2m212intsa   "1-then-2")

{\bf Outro caso em que o TFC2 dá um resultado errado}

% (c2m212intsp 5 "x^-2")
% (c2m212intsa   "x^-2")

%L Pict2e.bounds = PictBounds.new(v(-1,0), v(3,4))
%L specf  = "(-1,1)--(1,1)c (1,2)o--(3,2)"
%L specF  = "(0,0)c--(1,1) (1,2)--(2,4)c"
%L pwsf   = PwSpec.from(specf)
%L pwsF   = PwSpec.from(specF)
%L pf     = PictList{
%L   pwsf:topwfunction():areaify(0, 2):Color("Orange"),
%L   pwsf:topict()
%L }
%L pF     = PictList{
%L   pwsF:topict()
%L }
%L pf:pgat("pgatc"):sa("1then2 f"):output()
%L pF:pgat("pgatc"):sa("1then2 F"):output()
\pu

Isso aqui foi uma questão de prova que

metade da turma errou... $\frown$ Links:

\ssk

{\scriptsize

% (c2m202p1p 3 "questao-1")
% (c2m202p1a   "questao-1")
%    http://angg.twu.net/LATEX/2020-2-C2-P1.pdf#page=3
\url{http://angg.twu.net/LATEX/2020-2-C2-P1.pdf#page=3}

% (c2m202p1p 8 "gabarito-1")
% (c2m202p1a   "gabarito-1")
%    http://angg.twu.net/LATEX/2020-2-C2-P1.pdf#page=8
\url{http://angg.twu.net/LATEX/2020-2-C2-P1.pdf#page=8}

% (c2m211somas2p 50 "trailer")
% (c2m211somas2a    "trailer")
%    http://angg.twu.net/LATEX/2021-1-C2-somas-2.pdf#page=50
\url{http://angg.twu.net/LATEX/2021-1-C2-somas-2.pdf#page=50}

}

\msk

$$\Intx{0}{2}{f(x)}
  \;\;=\;\;
  \difx{0}{2}{F(x)}
$$
$$\ga{1then2 f}
  \;\; = \;\;
  \ga{1then2 F}
$$
$$3 \;\; = \;\; 4-0$$




\newpage

% «exercicio-1»  (to ".exercicio-1")
% (c2m221atisp 5 "exercicio-1")
% (c2m221atisa    "exercicio-1")
% (c2m212intsp 3 "exercicio-1")
% (c2m212intsa   "exercicio-1")

\scalebox{0.75}{\def\colwidth{12cm}\firstcol{

Neste semestre eu vou tentar 

explicar o TFC2 e as consequências dele ---

tipo: TODAS as técnicas de integração são

consequência do TFC2 --- com uma abordagem

diferente da do semestre passado.

\msk

Dê uma olhada nestes slides do semestre passado:

\ssk

{\footnotesize

% (c2m211tfcsp 2 "exercicio-1")
% (c2m211tfcsa   "exercicio-1")
% (c2m211tfcsp 10 "exercicio-2")
% (c2m211tfcsa    "exercicio-2")
% (c2m211tfcsp 12 "exercicio-3")
% (c2m211tfcsa    "exercicio-3")
%    http://angg.twu.net/LATEX/2021-1-C2-os-dois-TFCs.pdf
\url{http://angg.twu.net/LATEX/2021-1-C2-os-dois-TFCs.pdf}

}

Leia as páginas 2 até 4 dele,

a definição no fim da página 7,

e as páginas 10 até 12.

\msk

{\bf Exercício 1.}

Faça os exercícios 1, 2 e 3 do PDF acima ---

mas ao invés de fazer o 2 como eu pedi no semestre

passado faça esta versão modificada dele:
%
$$[\text{TFC2}] \pmat{F(x) := 2x^2 - \frac{x^3}{3} \\
                      F'(x) := 4x - x^2 \\
                      b:=4 \\
                      a:=0 }
  \;\;=\;\; \ColorRed{?}
$$

%}\anothercol{
}}








\newpage


% «exercicio-2»  (to ".exercicio-2")
% (c2m212intsp 4 "exercicio-2")
% (c2m212intsa   "exercicio-2")


{\bf Exercício 2.}

Assista este vídeo,

\ssk

{\footnotesize

\url{http://angg.twu.net/eev-videos/2021-2-C2-int-subst.mp4}

\url{https://www.youtube.com/watch?v=YbVfNi-xGNw}

}

e depois tente entender cada uma

das igualdades do slide 7.

\bsk

Dica: os `$=$'s do slide 7 têm montes

de significados diferentes dependendo

do contexto. Tente fazer uma lista de

significados e pronúncias.


\bsk

Obs: os próximos 3 slides não são

autocontidos -- você vai precisar

assistir o vídeo pra entendê-los.








\newpage


% «dfi»     (to ".dfi")
% (c2m212intsp 8 "dfi")
% (c2m212intsa   "dfi")
% «exercicio-3»  (to ".exercicio-3")
% (c2m212intsp 8      "exercicio-3")
% (c2m212intsa        "exercicio-3")

{\bf A fórmula da derivada da função inversa}

\sa{DFI1}{\left(
    \begin{array}{rcl}
           f(g(x)) &=& x \\
      \ddx f(g(x)) &=& \ddx x \;\;=\;\; 1 \\
      \ddx f(g(x)) &=& f'(g(x))g'(x) \\
      f'(g(x))g'(x) &=& 1 \\
      g'(x) &=& \frac{1}{f'(g(x))} \\
    \end{array}
  \right)}
\sa{DFI2}{\left(
    \begin{array}{rcl}
           f(g(x)) &=& x \\
      g'(x) &=& \frac{1}{f'(g(x))} \\
    \end{array}
  \right)}
\sa{Dfi1}{[\text{DFI1}]}
\sa{Dfi2}{[\text{DFI2}]}


\scalebox{0.65}{\def\colwidth{11cm}\firstcol{

$$\begin{array}{rcc}
  \ga{Dfi1} &=& \ga{DFI1} \\[40pt]
  \ga{Dfi2} &=& \ga{DFI2} \\
  \end{array}
$$

\bsk
\bsk

{\bf Exercício 3.}

\msk

a) $\ga{Dfi1} \bmat{f(y) := e^y \\
                    f'(y) := e^y \\
                    g(x) := \ln x \\
                    g'(x) := \ln' x \\
                   } = \rq$

}\anothercol{

{\bf Exercício 3 (cont.)}

\msk

b) $\def\Sqrt{\text{sqrt}}
    \ga{Dfi2} \bmat{f (y) := y^2 \\
                    f'(y) := 2y \\
                    g (x) := \Sqrt(x) \\
                    g'(x) := \Sqrt'(x) \\
                   } = \rq
   $

\msk

c) $\ga{Dfi2} \bmat{f (y) := \sen y \\
                    f'(y) := \cos y \\
                    g (x) := \arcsen(x) \\
                    g'(x) := \arcsen'(x) \\
                   } = \rq
   $

\msk

d) $\ga{Dfi2} \bmat{x     := s \\
                    f (θ) := \sen θ \\
                    f'(θ) := \cos θ \\
                    g (s) := \arcsen(s) \\
                    g'(s) := \arcsen'(s) \\
                   } = \rq
   $

\msk

e) $\ga{Dfi2} \bmat{x := c\\
                    f (θ) := \cos θ \\
                    f'(θ) := -\sen θ \\
                    g (c) := \cos^{-1}(c) \\
                    g'(c) := (\cos^{-1})'(c) \\
                   } = \rq
   $


}}


\newpage

% «mais-algumas»  (to ".mais-algumas")
% (c2m212intsp 9 "mais-algumas")
% (c2m212intsa   "mais-algumas")

{\bf Mais algumas fórmulas que não valem sempre}

$$\begin{array}{rcl}
  (\cos x)^2 + (\sen x)^2 &=& 1 \\
  %
  [10pt]
  %
  (\sen x)^2 &=& 1 - (\cos x)^2 \\
  \sqrt{(\sen x)^2} &=& \sqrt{1 - (\cos x)^2} \\
         \sen x     &=& \sqrt{1 - (\cos x)^2} \\
  (\cos x)^2 &=& 1 - (\sen x)^2 \\
  %
  [10pt]
  %
  \sqrt{(\cos x)^2} &=& \sqrt{1 - (\sen x)^2} \\
         \cos x     &=& \sqrt{1 - (\sen x)^2} \\
  \end{array}
$$

\newpage


% «exercicio-4»  (to ".exercicio-4")
% (c2m221atisp 11 "exercicio-4")
% (c2m221atisa    "exercicio-4")
% (c2m212intsp 10 "exercicio-4")
% (c2m212intsa    "exercicio-4")

{\bf Exercício 4.}

% See: (find-angg "LUA/Pict2e1-1.lua" "Plot2D-test1")
%
%L xs = seqn(0, 2*pi, 128)
%L fx = function (expr) return "x => v(x,"..expr..")" end
%L plotf = function (expr) return Plot2D.from(fx(expr), xs):toline() end
%L 
%L Pict2e.bounds = PictBounds.new(v(0,-2), v(7,2))
%L p = PictList {
%L     plotf("sin(x)"):Color("Red"),
%L     plotf("cos(x)"):Color("Orange"),
%L   }:prethickness("2pt")
%L p:pgat("pgatc"):sa("sin and cos"):output()
\pu


\scalebox{0.8}{\def\colwidth{10cm}\firstcol{

a) Escolha um número entre 42 e 99.

\ColorRed{(Se você não conseguir converse com seus colegas!!!)}

\msk

b) Escolha um $α∈\R$ tal que $\sen α<0$

e verifique se $\sen α = \sqrt{1 - (\cos α)^2}$.

Dica: escolha um $α$ para o qual você sabe $\sen α$ e $\cos α$.

\msk

c) Escolha um $β∈\R$ tal que $\cos β<0$

e verifique se $\cos β = \sqrt{1 - (\cos β)^2}$.

\msk

d) Faça uma cópia do gráfico abaixo num papel
%
\unitlength=10pt
%
$$\ga{sin and cos}
$$
%
e desenhe sobre ela os conjuntos:

\ssk

$A \;\;=\;\; \setofst{θ∈[0,2π]}{\sen θ = \sqrt{1 - (\cos θ)^2}}$,

$B \;\;=\;\; \setofst{θ∈[0,2π]}{\cos θ = \sqrt{1 - (\sen θ)^2}}$.

}\anothercol{
}}


\newpage

{\bf Juntando fórmulas estranhas}

$$\begin{array}{rcl}
  f(g(x)) &=& x \\
    g'(x) &=& \frac{1}{f'(g(x))} \\
  e^{\ln x} &=& x \\
  \ln' x &=& \frac{1}{e^{\ln x}} \\[2.5pt]
         &=& \frac{1}{x} \\
  \Intx{a}{b}{\ln' x} &=& \difx{a}{b}{\ln x} \\
  \Intx{a}{b}{\frac{1}{x}} &=& \difx{a}{b}{\ln x} \\
  \end{array}
$$

\newpage

% «juntando»  (to ".juntando")
% (c2m221atisp 13 "juntando")
% (c2m221atisa    "juntando")

{\bf Juntando fórmulas estranhas}

$$\begin{array}{rcl}
  f(g(x)) &=& x \\
    g'(x) &=& \frac{1}{f'(g(x))} \\
  \sen(\arcsen x) &=& x \\
  \arcsen' x &=& \frac{1}{\cos(\arcsen x)} \\
             &=& \frac{1}{\sqrt{(\cos(\arcsen x))^2}} \\
             &=& \frac{1}{\sqrt{1 - (\sen(\arcsen x))^2}} \\
             &=& \frac{1}{\sqrt{1 - x^2}} \\
  \Intx{a}{b}{\arcsen' x} &=& \difx{a}{b}{\arcsen x} \\
  \Intx{a}{b}{\frac{1}{\sqrt{1-x^2}}} &=& \difx{a}{b}{\arcsen x} \\
  \end{array}
$$






\newpage

% «exercicio-1»  (to ".exercicio-1")
% (c2m211isp 11 "exercicio-1")
% (c2m211isa    "exercicio-1")

{\bf Exercício 1.}

Lembre que:
%
$$\pfo{TFC2}
  \;\;=\;\;
  \Ps{
       \D \Intx{a}{b}{\ddx F(x)} \;\;=\;\; \difx{a}{b}{F(x)}
     }
$$

\msk

Calcule os resultados destas expansões:

a) $\pfo{TFC2} \bmat{F(x):=F(g(x))}$

b) $\pfo{TFC2} \bmat{x:=u} \bmat{a:=g(a) \\ b:=g(b)}$

\bsk
\bsk

...e verifique que \ColorRed{se $f(u)=F'(u)$ então}:

c) o que você obteve no (a) prova o `$=$' de cima da \pfo{S2},

d) o que você obteve no (b) prova o `$=$' de baixo da \pfo{S2},


\newpage

% «esquerda»  (to ".esquerda")
% (c2m211isp 12 "esquerda")
% (c2m211isa    "esquerda")

O `$\,\rotl{=}$' à esquerda na \pfo{S2}

é bem fácil de verificar... ó:

$$\begin{array}{rcl}
  \difx{a}{b}{F(g(x))} &=& F(g(b)) - F(g(a)) \\
                       &=& \difu{g(a)}{g(b)}{F(u)}
  \end{array}
$$

\bsk
\bsk

Se você conseguiu fazer todos os itens

do exercício 1 e conseguiu entender isso aí

então \ColorRed{agora} você entende o $\pfo{S2}$ como uma

demonstração --- você entende todas as

igualdades dele.


\newpage

% «hipotese»  (to ".hipotese")
% (c2m211isp 13 "hipotese")
% (c2m211isa    "hipotese")

{\bf Pra que serve a hipótese do \pfo{S2}?}

Ela serve pra gente lidar com `$f$'s que a gente

não sabe integrar! Por exemplo:
%
\def\Stwotmp{
  \isubstboxT
    {\Difmx{a}{b}{F(g(x))}}   {\Intx{a}{b}{\tan(g(x))\tan'(x)}}
    {\ph{mmm}}
    {\Difmu{g(a)}{g(b)}{F(u)}}  {\Intu{g(a)}{g(b)}{\tan(u)}}
    {Se $F'(u)=F'(u)$ então:}
}
%
\def\Stwotmp{
  \isubstboxT
    {\Difmx{a}{b}{F(2x)}}   {\Intx{a}{b}{\tan(2x)·2}}
    {\ph{mmm}}
    {\Difmu{g(a)}{g(b)}{F(u)}}  {\Intu{2a}{2b}{\tan(u)}}
    {\ColorRed{Se $F'(u)=\tan u$ então:}}
}
%
$$\scalebox{0.90}{$
  \begin{array}{c}
  \pfo{S2} \;\;=\;\; \Stwo \\
  [50pt]
  % \pfo{S2}[f(x):=\tan x] \;\;=\;\; \Stwotmp \\
    \pfo{S2}\bmat{f(x):=\tan x \\ g(u):=2u} \;\;=\;\; \Stwotmp \\
  \end{array}
  $}
$$

\newpage

% «S2I»  (to ".S2I")
% (c2m221atisp 28 "S2I")
% (c2m221atisa    "S2I")

{\bf Uma versão do \pfo{S2} para integrais indefinidas}

Compare... e repare no ``\ColorRed{Obs: $u = g(x)$}''.
%
\def\StwoItmp{
  \isubstboxTT
    {F(g(x))}  {\intx{f(g(x))g'(x)}}
    {\ph{m}}
    {F(u)}     {\intu{f(u)}}
    {Se $F'(u)=f(u)$ então:}
    {\ColorRed{Obs: $u=g(x)$.}}
}
%
$$\scalebox{0.90}{$
  \begin{array}{c}
  \pfo{S2} \;\;=\;\; \Stwo \\
  [50pt]
  \pfo{S2I} \;\;=\;\; \StwoItmp \\
  \end{array}
  $}
$$

\newpage

{\bf Versões sem a parte da esquerda}

Compare:
%
$$\scalebox{0.90}{$
  \begin{array}{c}
  \pfo{S2} \;\;=\;\; \Stwo \\
  [50pt]
  \pfo{S3} \;\;=\;\; \Sthree \\
  \end{array}
  $}
$$

\newpage

% «S3I»  (to ".S3I")
% (c2m221atisp 30 "S3I")
% (c2m221atisa    "S3I")

{\bf Versões sem a parte da esquerda (2)}

...e compare:
%
$$\scalebox{0.90}{$
  \begin{array}{c}
  \pfo{S2I} \;\;=\;\; \StwoI \\
  [50pt]
  \pfo{S3I} \;\;=\;\; \SthreeI \\
  \end{array}
  $}
$$

\newpage

% «encontre-a-subst»  (to ".encontre-a-subst")
% (c2m211isp 17 "encontre-a-subst")
% (c2m211isa    "encontre-a-subst")

As pessoas costumam usar variações da $\pfo{S3I}$,

geralmente sem darem um nome pra função $g(u)$...


Lembre que em vários exercícios que nós já fizemos

ficava implícito que vocês tinham que descobrir qual

era a substituição certa... por exemplo:
%
$$\begin{array}{rcl}
  \difx{4}{5}{x^2} &=& \Rdq \\[5pt]
  \Ps{\difx{a}{b}{f(x)} = f(b)-f(a)} \bmat{f(x):=\Rdq \\ a:=\Rdq \\ b:=\Rdq} &=& \Rdq
  \\[20pt]
  \Ps{\difx{a}{b}{f(x)} = f(b)-f(a)} \bmat{f(x):=x^2  \\ a:=4    \\ b:=5} &=&
  \Ps{\difx{4}{5}{x^2} = 5^2-4^2} \\
  [20pt]
  \difx{4}{5}{x^2} &=& 5^2 - 4^2 \\
  \end{array}
$$


\newpage

% «exercicio-2»  (to ".exercicio-2")
% (c2m211isp 18 "exercicio-2")
% (c2m211isa    "exercicio-2")

{\bf Exercício 2.}

Nos livros e nas notas de aula que você vai encontrar por aí

o ``\ColorRed{Obs: $u = g(x)$}'' da nossa \pfo{S3I} quase sempre aparece escrito

de (ZILHÕES DE!!!) outros jeitos, então o melhor que a gente

pode fazer é tentar encontrar as substituições que transformam

a nossa \pfo{S3I} em algo ``mais ou menos equivalente'' às

igualdades complicadas que eu mostrei no vídeo e que eu disse

que a gente iria tentar decifrar...

\msk

Nos itens a e b deste exercício você vai tentar encontrar

as substituições --- que eu vou escrever como `$[\Rdq]$' --- que

transformam a $\pfo{S3I}$ em algo ``mais ou menos equivalente''

às igualdades da direita.

\newpage

% «exercicio-2-cont»  (to ".exercicio-2-cont")
% (c2m211isp 19 "exercicio-2-cont")
% (c2m211isa    "exercicio-2-cont")

{\bf Exercício 2 (cont.)}

Encontre as substituições `$[\Rdq]$'s que façam com que:

\bsk

a) $\SthreeI [\Rdq]$ vire algo como
   $\pmat{ \D \intx{2 \cos(3x+4)} \\
           \rotl{=} \\
           \D \intu {2 (\cos u) · \frac13} \\
         }$

\msk

b) $\pfo{S3I} \, [\Rdq]$ vire algo como
   $\pmat{ \D \intx{(\S)^5 (1 - \S^2) (\C)} \\
           \rotl{=} \\
           \D \ints{s^5 (1-s^2)} \\
         }$


\newpage

{\bf Gambiarras}

Em geral é mais prático a gente usar umas gambiarras

como ``$\frac{du}{dx}dx = du$'' ao invés do método ``mais honesto''

que a gente usou no exercício 2...

\msk

Às vezes essas gambiarras vão usar uma versão disfarçada

do teorema da derivada da função inversa: $\frac{du}{dx} = \frac{1}{\frac{dx}{du}}$,

e umas outras manipulações esquisitas de `$dx$'s e `$du$'s

que só aparecem explicadas direito nos capítulos sobre

``diferenciais'' dos livros de Cálculo.

\msk

Nós vamos começar usando elas como gambiarras mesmo,

e acho que nesse semestre não vai dar pra ver como

traduzir cada uma delas pra algo formal...

\newpage

% «gambiarras-2»  (to ".gambiarras-2")
% (c2m211isp 21 "gambiarras-2")
% (c2m211isa    "gambiarras-2")

{\bf Gambiarras (2)}

Quando a gente está começando e ainda não tem prática

este modo de por anotações embaixo de chaves ajuda muito:
%
$$%\begin{array}{c}
  \D \int  (\und{\S}{s})^5
           (1 - (\und{\S}{s})^2)
           \und{
           \und{(\C)}{\frac{ds}{dx}} \, dx
           }{ds}
            \\
  % \rotl{=} \\
  \;\; = \;\;
  \D \ints{s^5 (1-s^2)} \\
  % \end{array}
$$

Quando a gente já tem mais prática acaba sendo melhor

pôr todas as anotações dentro de caixinhas --- por exemplo:

$$\bmat{
  \sen x = s \\
  \frac{ds}{dx} = \frac{d}{dx} \sen x = \cos x \\
  \cos x \, dx = ds \\
  }
$$


\newpage

% «gambiarras-3»  (to ".gambiarras-3")
% (c2m211isp 22 "gambiarras-3")
% (c2m211isa    "gambiarras-3")

{\bf Gambiarras (3)}

Essas caixinhas, como
%
$$\bmat{
  \sen x = s \\
  \frac{ds}{dx} = \frac{d}{dx} \sen x = \cos x \\
  \cos x \, dx = ds \\
  }
$$

vão ser os únicos lugares em que nós vamos permitir

esses `$dx$'s e `$ds$' ``soltos'', que não estão nem em

derivadas e nem associados a um sinal `$∫$'...

\msk

E esses `$dx$'s e `$ds$' ``soltos'' só vão aparecer em linhas

que dizem como traduzir uma expressão que termina em `$dx$'

numa integral em $x$ pra uma expressão que termina em `$ds$'

numa integral na \ColorRed{variável} $s$.

\msk

Nós vamos \ColorRed{evitar} usar $s$ como uma \ColorRed{abreviação} para $\sen x$.


\newpage

{\bf Mais sobre as caixinhas de anotações}

Tudo numa caixinha de anotações é \ColorRed{consequência}

da primeira linha dela, que é a que define a variável

nova. Por exemplo, se definimos a variável nova como

$c=\cos x$ então $\frac{dc}{dx} = \frac{d}{dx} \cos x = - \sen x$, e podemos

reescrever isso na ``versão gambiarra'' como:

$dc = - \sen x \, dx$, \ColorRed{e também como} $\sen x \, dx = (-1) dc$.

\msk

A caixinha vai ser:
%
$$\bmat{c = \cos x \\
        \frac{dc}{dx} = \frac{d}{dx} \cos x = - \sen x \\
        dc = - \sen x \, dx \\
        \sen x \, dx = (-1) \, dc \\
       }
$$

\newpage

{\bf Mais sobre as caixinhas de anotações (2)}

\ColorRed{Muito importante:} cada linha das caixinhas

é uma série de igualdades --- por exemplo

$𝐬{expr}_1 = 𝐬{expr}_2 = 𝐬{expr}_3$ --- e cada uma dessas

expressões $𝐬{expr}_1, \ldots, 𝐬{expr}_n$ só pode mencionar

\ColorRed{ou} a variável antiga \ColorRed{ou} a variável nova...

\msk

Então:

\msk


\ColorRed{Bom:} $dc = - \sen x \, dx$

\ColorRed{Mau:} $\frac{1}{- \sen x} dc =  dx$

\ColorRed{Bom:} $\frac{dc}{dx} = \frac{d}{dx} \cos x$

\bsk

Truque: em $\frac{dc}{dx}$ o $c$ faz o papel de uma \ColorRed{abreviação}

para $\cos x$, não de uma variável.


\newpage

{\bf Mais sobre as caixinhas de anotações (3)}

Quando a gente faz algo como
%
$$\D \int  (\und{\S}{s})^5
           (1 - (\und{\S}{s})^2)
           \und{
           \und{(\C)}{\frac{ds}{dx}} \, dx
           }{ds}
            \\
  \;\; = \;\;
  \D \ints{s^5 (1-s^2)} \\
$$

Cada chave é como uma igualdade da caixa de anotações

``escrita na vertical''... por exemplo, ``$\und{\S}{s}$'' é $s = \sen x$.

\msk

As outras chaves correspondem a outras igualdades da

caixa de anotações --- \ColorRed{que têm que ser consequências

desse $s = \sen x$.}


\newpage

\vspace*{-0.5cm}

{\bf Mais sobre as caixinhas de anotações (3)}

Isto aqui está errado:
%
$$\D \int %(\und{\S}{s})^5
           (     \S    )^5
           (1 - (\und{\S}{s})^2)
           \und{
           \und{(\C)}{\frac{ds}{dx}} \, dx
           }{ds}
            \\
  \;\; = \;\;
  \D \ints{(\ColorRed{\S})^5 (1-s^2)} \\
$$

À esquerda do `$=$' a gente tem uma integral na qual

só aparece a ``variável antiga'', que é $x$, e à direita do `$=$'

a gente tem uma integral na qual aparecem tanto a variável

antiga, $x$, quanto a nova, que é $s$... \quad \frown

\msk

Lembre que tanto o truque das caixinhas quanto o truque das

chaves servem pra gente conseguir aplicar a $\pfo{S3I}$ de um jeito

mais fácil, e no $\pfo{S3I}$ uma integral usa só a variável antiga

e a outra usa só a nova.








\newpage

% «exercicio-3»  (to ".exercicio-3")
% (c2m211isp 27 "exercicio-3")
% (c2m211isa    "exercicio-3")

{\bf Exercício 3.}

Leia o início da seção 6.1 do APEX Calculus

e faça os exercíos 25 até 32 da página 280 dele. Link:

\ssk

{\scriptsize

% (find-books "__analysis/__analysis.el" "apex-calculus")
% (find-apexcalculuspage (+ 10 263) "6.1 Substitution")
% (find-apexcalculuspage (+ 10 280)     "Exercises 6.1")
% (find-twusfile "2021.1-C2/")
% http://angg.twu.net/2021.1-C2/APEX_Calculus_Version_4_BW_secs_6.1_6.2.pdf
\url{http://angg.twu.net/2021.1-C2/APEX_Calculus_Version_4_BW_secs_6.1_6.2.pdf}

}

\bsk
\bsk

% «exercicio-4»  (to ".exercicio-4")
% (c2m211isp 27 "exercicio-4")
% (c2m211isa    "exercicio-4")

{\bf Exercício 4.}

Leia o início da seção 6.1 do Martins/Martins

e refaça os exercícios resolvidos 1 a 6 dele

usando ou as nossas anotações sob chaves ou

as nossas anotações em caixinhas. Link:

\ssk

{\scriptsize

% (find-books "__analysis/__analysis.el" "martins-martins")
% (find-martinscdipage (+ 10 109) "4.2       Integral")
% (find-martinscditext (+ 10 109) "4.2       Integral")
% (find-martinscdipage (+ 10 165) "6" "Metodos de Integracao")
% (find-martinscditext (+ 10 165) "6" "Metodos de Integracao")
% (find-martinscdipage (+ 10 165) "6.1       Metodo da Substituicao")
% (find-martinscditext (+ 10 165) "6.1       Metodo da Substituicao")
% http://angg.twu.net/2021.1-C2/martins_martins__sec_6.1.pdf
\url{http://angg.twu.net/2021.1-C2/martins_martins__sec_6.1.pdf}

}


\newpage

% «exercicio-5»  (to ".exercicio-5")
% (c2m211isp 28 "exercicio-5")
% (c2m211isa    "exercicio-5")

{\bf Exercício 5.}

\msk

A questão 2 da P1 do semestre passado dizia que:
%
\begin{quote}
{\sl Toda integral que pode ser resolvida por uma sequência de
  mudanças de variável (ou: ``por uma sequência de integrações por
  substituição'') pode ser resolvida por uma mudança de variável só.}
\end{quote}

E ela pedia pra vocês verificarem isso num caso específico.

Tente fazer essa questão olhando poucas vezes pro gabarito dela.

Link:

\ssk

{\footnotesize

% (c2m202p1p 4)
%    http://angg.twu.net/LATEX/2020-2-C2-P1.pdf#page=4
\url{http://angg.twu.net/LATEX/2020-2-C2-P1.pdf#page=4}

}


% (c2m202p1p 4 "questao-2")
% (c2m202p1a   "questao-2")


\newpage


\sa{x}{xx}
\sa{u}{uu}
\sa{gx}{g(xx)}
\sa{nw}{F(g(x))}
\sa{ne}{f(g(x))g'(x)}
\sa{sw}{F(u)}
\sa{se}{f(u)}

\def\StwoIsetargs#1{\StwoIsetargsss#1}
\def\StwoIsetargsss#1#2#3#4#5#6#7{
  \sa{x}{#1} \sa{u}{#2} \sa{gx}{#3}
  \sa{nw}{#4} \sa{ne}{#5}
  \sa{sw}{#6} \sa{se}{#7}
  }

% (c2m202p1p 9 "gabarito-2")
% (c2m202p1a   "gabarito-2")

\StwoIsetargsss {xx} {uu} {gguu} {NW} {NE} {SW} {SE}
\StwoIsetargsss
    {v} {w} {\sqrt{v}}
    {F(\sqrt{v})} {\cos(2+\sqrt{v})·(2\sqrt{v})^{-1}}
    {F(w)}        {\cos(2+w)}

\def\StwoItmp{
  \isubstboxTT
    {\ga{nw}}  {\int \ga{ne} \, d\ga{x}}
    {\ph{m}}
    {\ga{sw}}  {\int \ga{se} \, d\ga{u}}
    {Se $F'(\ga{u})=\ga{se}$ então:}
    {Obs: $\ga{u}=\ga{gx}$.}
}
%
$$\scalebox{0.9}{$
  \begin{array}{c}
  \pfo{S2} \;\;=\;\; \Stwo \\
  [50pt]
  \pfo{S2}[f(u):=F'(u)] \;\;=\;\; \StwoItmp \\
  \end{array}
  $}
$$





%%%%%%%%%%%%%%%%%%%%%%%%%%%%%%%%%%%%%%%%%%%%%%%%%%%%%%%%%%%%
%%%%%%%%%%%%%%%%%%%%%%%%%%%%%%%%%%%%%%%%%%%%%%%%%%%%%%%%%%%%
%%%%%%%%%%%%%%%%%%%%%%%%%%%%%%%%%%%%%%%%%%%%%%%%%%%%%%%%%%%%
%%%%%%%%%%%%%%%%%%%%%%%%%%%%%%%%%%%%%%%%%%%%%%%%%%%%%%%%%%%%
%%%%%%%%%%%%%%%%%%%%%%%%%%%%%%%%%%%%%%%%%%%%%%%%%%%%%%%%%%%%
%%%%%%%%%%%%%%%%%%%%%%%%%%%%%%%%%%%%%%%%%%%%%%%%%%%%%%%%%%%%
%%%%%%%%%%%%%%%%%%%%%%%%%%%%%%%%%%%%%%%%%%%%%%%%%%%%%%%%%%%%



% «introducao»  (to ".introducao")
% (c2m212mvp 2 "introducao")
% (c2m212mva   "introducao")

{\bf Introdução}

No último PDF e na P1 a gente viu como fazer

``integração por substituição'' de um jeito mais ou menos

fácil de formalizar... agora a gente vai ver o método

que os livros usam, que nos permite fazer as contas bem

rápido, mas que usa várias gambiarras, algumas delas

bem difíceis de formalizar.

\msk

Os nomes ``integração por substituição'' e ``integração

por mudança de variável'' costumam ser equivalentes.

Vou me referir ao método que a gente vai ver agora como

``mudança de variável'', ``mudança de variável por

gambiarras'', ``MV'', ou ``MVG'', pra gente poder usar

o termo ``substituição'' pro `[:=]'.

\newpage

% «introducao-2»  (to ".introducao-2")
% (c2m212mvp 3 "introducao-2")
% (c2m212mva   "introducao-2")

{\bf Introdução (2)}

\scalebox{0.95}{\def\colwidth{12cm}\firstcol{

Cada livro usa convenções um pouco diferentes pra como

escrever as contas por MVG. Eu vou usar a convenção do

exemplo do próximo slide, em que a resolução da integral

fica à esquerda e as caixinhas indicando os truques que

usamos em \ColorRed{cada} MV ficam à direita, separadas da contas

da integral.

\msk

A primeira caixinha tem os truques pra mudar

da variável $x$ pra variável $u$ e pra voltar de $u$ pra $x$.

\ssk

A segunda caixinha tem os truques pra mudar

da variável $u$ pra variável $v$ e pra voltar de $v$ pra $u$.

\ssk

A terceira caixinha tem os truques pra mudar

da variável $v$ pra variável $w$ e pra voltar de $w$ pra $v$.

\ssk

A quarta caixinha tem os truques pra mudar

da variável $w$ pra variável $y$ e pra voltar de $y$ pra $w$.

%}\anothercol{
}}


\newpage

% «exemplo»  (to ".exemplo")
% (c2m212mvp 4 "exemplo")
% (c2m212mva   "exemplo")
% (c2m211isp 7 "exemplo-contas-2")
% (c2m211isa   "exemplo-contas-2")
% (c2m211p1p 15 "gabarito-2-2020.2")
% (c2m211p1a    "gabarito-2-2020.2")

$$\scalebox{1.25}{$
 \begin{array}{l}
   \begin{array}{l}
   \intx {\frac{3 \cos{\left (2 + \sqrt{3 x + 4} \right )}}
         {2 \sqrt{3 x + 4}}
         } \\
   = \intu {\frac{\cos{\left (2 + \sqrt{u + 4} \right )}}
           {2 \sqrt{u + 4}}
           } \\
   = \intv {\frac{\cos{\left (2 + \sqrt{v} \right )}}
           {2 \sqrt{v}}
           } \\
   = \intw {\cos{\left (2 + w \right )}
           } \\
   = \inty {\cos y}
           \\
   = \sen y \\
   = \sen \left( 2+w \right) \\
   = \sen \left( 2+\sqrt{v} \right) \\
   = \sen \left( 2+\sqrt{u+4} \right) \\
   = \sen \left( 2+\sqrt{3x+4} \right) \\
   \end{array}
   %
   \begin{array}{c}
     \bsm{u = 3x \\ \frac{du}{dx} = 3 \\ du = 3\,dx \\ dx = \frac13 du}
     \\[15pt]
     \bsm{v = u+4 \\ du=dv }
     \\[5pt]
     \bsm{w = \sqrt{v} \\ \frac{dw}{dv} = \frac12 v^{-1/2} = \frac{1}{2\sqrt{v}} \\}
     \\[5pt]
     \bsm{y = 2+w \\ dy=dw }
     \\[60pt]
   \end{array}
   %
  \end{array}
  $}
$$

\newpage

% «limites-int»  (to ".limites-int")
% (c2m212mvp 5 "limites-int")
% (c2m212mva   "limites-int")

{\bf Limites de integração}

A coluna da esquerda tem uma série de integrais sem

limites de integração --- a gente está trabalhando numa

notação abreviada em que os limites de integração foram

apagados. Eles podem ser recolocados de novo no final,

quando a gente for transformar essas contas abreviadas

numa versão ``desabreviada'' delas.

\msk

Os limites de integração em $x$ são diferentes

dos limites de integração em $u$, que são diferentes

dos limites de integração em $v$, que são diferentes

dos limites de integração em $w$, que são diferentes

dos limites de integração em $y$.

\msk

Detalhes em breve!


\newpage


\def\expr#1{〈\mathsf{expr_{#1}}〉}


A coluna da esquerda tem uma série de igualdades.

Ela é da forma $\expr1 = \expr2 = \ldots = \expr{n}$,

mas a gente escreve essa série de igualdades na

vertical.

\msk

Repare que na coluna da esquerda

``as variáveis não se misturam'':

$\expr{1}$ e $\expr{10}$ são ``expressões em $x$'',

$\expr{2}$ e $\expr{9}$ são ``expressões em $u$'',

$\expr{3}$ e $\expr{8}$ são ``expressões em $v$'',

$\expr{4}$ e $\expr{7}$ são ``expressões em $w$'',

$\expr{5}$ e $\expr{6}$ são ``expressões em $y$''.


\newpage

% «importante»  (to ".importante")
% (c2m212mvp 7 "importante")
% (c2m212mva   "importante")

{\bf A regra mais importante de todas}


\scalebox{0.85}{\def\colwidth{9cm}\firstcol{

Na coluna da esquerda cada expressão é

uma expressão ``em uma variável só''.

Se você escrever algo como
%
$$\inty{\cos(2+w)}$$

Isso é um \ColorRed{ERRO CONCEITUAL GRAVÍSSIMO}

e a sua questão é \ColorRed{ZERADA}.


\bsk



A gente não vai ter tempo de ver o porquê disso...

O motivo é que com essa proibição o método pra

``desabreviar'' as contas fica simples ---

sem essa proibição ele fica BEM mais complicado,

e a gente precisaria de uns truques de ``notação

de físicos'', que é um assunto bem difícil de

Cálculo 3, pra definir o método de desabreviação.

%}\anothercol{
}}

\newpage

% «caixinhas»  (to ".caixinhas")
% (c2m212mvp 8 "caixinhas")
% (c2m212mva   "caixinhas")

{\bf As caixinhas de truques}

As caixinhas de truques da MVG têm uma sintaxe

\ColorRed{BEM} diferente das caixinhas do `[:=]'.

Pra enfatizar isso a gente usa `$=$'s dentro delas,

não `$:=$'s, e a gente escreve elas separadas

do resto, à direita.


\msk

Dê uma olhada nas 9 primeiras páginas daqui:

\ssk

{\footnotesize

% (c3m212nfp 1 "title")
% (c3m212nfa   "title")
%    http://angg.twu.net/LATEX/2021-2-C3-notacao-de-fisicos.pdf
\url{http://angg.twu.net/LATEX/2021-2-C3-notacao-de-fisicos.pdf}

}

\msk

Dentro cada caixinha de truques da MVG a gente

vai usar algumas expressões que só podem ser

formalizadas \ColorRed{direito} usando a ``notação de físicos'',

que a gente vai ver com detalhes em C3...

Vou mostrar como ``ler em voz alta'' uma caixinha

e a gente vai tentar usar elas meio de improviso.


\newpage

{\bf Lendo uma caixinha de truques em voz alta}

$$\bmat{u = 3x \\ \frac{du}{dx} = 3 \\ du = 3\,dx \\ dx = \frac13 du}$$

Digamos que $u$ e $x$ são variáveis dependentes,

que obedecem a equação $u=3x$.

Então podemos tratar $u$ como uma função de $x$,

e temos $\frac{du}{dx} = \frac{d}{dx}(u(x)) = \frac{d}{dx}(3x) = \frac{d}{dx}(u(x)) = 3$.

Multiplicando os dois lados de $\frac{du}{dx} = 3$ por $dx$

obtemos $du = 3\,dx$; e multiplicando os dois lados de

$du = 3\,dx$ por $\frac13$ obtemos $dx = \frac{1}{3}$.

\newpage

Na caixinha
%
$$\bmat{u = 3x \\ \frac{du}{dx} = 3 \\ du = 3\,dx \\ dx = \frac13 du}$$

\msk

as duas últimas linhas são igualdades entre expressões

incompletas. Você viu na P1 como substituir expressões

incompletas, como parênteses, bananas e lentes...

\msk

Em expressões das formas `$\intx{\ldots}$' e `$\intu{\ldots}$' o `$dx$'

e o `$du$' fazem papel de ``fecha parênteses'', e as igualdades

$du = 3\,dx$ e $dx = \frac13 du$ indicam substituições que você vai

poder fazer nas integrais do lado esquerda que vão ser

\ColorRed{parecidas} com as da questão 2 da P1.



\newpage

{\bf Exercício 1.}

Reescreva os exemplos 1 a 4 da seção 6.2 do livro do

Daniel Miranda na notação que eu disse que nós vamos

usar, em que todas caixinhas de truques são escritas

explicitamente.

\msk

Link:

\ssk

{\scriptsize

\url{http://hostel.ufabc.edu.br/~daniel.miranda/calculo/calculo.pdf\#page=189}

}


% sobre
% 
% ;; (find-books "__analysis/__analysis.el" "miranda")
% ;; (find-es "ead" "daniel-miranda")
% ;; http://hostel.ufabc.edu.br/~daniel.miranda/calculo/calculo.pdf
% 
% 
% ;; (find-dmirandacalcpage 189 "6.2 Integração por Substituição")







%\printbibliography

\GenericWarning{Success:}{Success!!!}  % Used by `M-x cv'

\end{document}

%  ____  _             _         
% |  _ \(_)_   ___   _(_)_______ 
% | | | | \ \ / / | | | |_  / _ \
% | |_| | |\ V /| |_| | |/ /  __/
% |____// | \_/  \__,_|_/___\___|
%     |__/                       
%
% «djvuize»  (to ".djvuize")
% (find-LATEXgrep "grep --color -nH --null -e djvuize 2020-1*.tex")

 (eepitch-shell)
 (eepitch-kill)
 (eepitch-shell)
# (find-fline "~/2022.1-C2/")
# (find-fline "~/LATEX/2022-1-C2/")
# (find-fline "~/bin/djvuize")

cd /tmp/
for i in *.jpg; do echo f $(basename $i .jpg); done

f () { rm -v $1.pdf;  textcleaner -f 50 -o  5 $1.jpg $1.png; djvuize $1.pdf; xpdf $1.pdf }
f () { rm -v $1.pdf;  textcleaner -f 50 -o 10 $1.jpg $1.png; djvuize $1.pdf; xpdf $1.pdf }
f () { rm -v $1.pdf;  textcleaner -f 50 -o 20 $1.jpg $1.png; djvuize $1.pdf; xpdf $1.pdf }

f () { rm -fv $1.png $1.pdf; djvuize $1.pdf }
f () { rm -fv $1.png $1.pdf; djvuize WHITEBOARDOPTS="-m 1.0 -f 15" $1.pdf; xpdf $1.pdf }
f () { rm -fv $1.png $1.pdf; djvuize WHITEBOARDOPTS="-m 1.0 -f 30" $1.pdf; xpdf $1.pdf }
f () { rm -fv $1.png $1.pdf; djvuize WHITEBOARDOPTS="-m 1.0 -f 45" $1.pdf; xpdf $1.pdf }
f () { rm -fv $1.png $1.pdf; djvuize WHITEBOARDOPTS="-m 0.5" $1.pdf; xpdf $1.pdf }
f () { rm -fv $1.png $1.pdf; djvuize WHITEBOARDOPTS="-m 0.25" $1.pdf; xpdf $1.pdf }
f () { cp -fv $1.png $1.pdf       ~/2022.1-C2/
       cp -fv        $1.pdf ~/LATEX/2022-1-C2/
       cat <<%%%
% (find-latexscan-links "C2" "$1")
%%%
}

f 20201213_area_em_funcao_de_theta
f 20201213_area_em_funcao_de_x
f 20201213_area_fatias_pizza



%  __  __       _        
% |  \/  | __ _| | _____ 
% | |\/| |/ _` | |/ / _ \
% | |  | | (_| |   <  __/
% |_|  |_|\__,_|_|\_\___|
%                        
% <make>

 (eepitch-shell)
 (eepitch-kill)
 (eepitch-shell)
# (find-LATEXfile "2019planar-has-1.mk")
make -f 2019.mk STEM=2022-1-C2-algumas-t-ints veryclean
make -f 2019.mk STEM=2022-1-C2-algumas-t-ints pdf

% Local Variables:
% coding: utf-8-unix
% ee-tla: "c2at"
% ee-tla: "c2m221atis"
% End:
