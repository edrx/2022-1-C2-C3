% (find-LATEX "2022-1-C2-der-fun-inv.tex")
% (defun c () (interactive) (find-LATEXsh "lualatex -record 2022-1-C2-der-fun-inv.tex" :end))
% (defun C () (interactive) (find-LATEXsh "lualatex 2022-1-C2-der-fun-inv.tex" "Success!!!"))
% (defun D () (interactive) (find-pdf-page      "~/LATEX/2022-1-C2-der-fun-inv.pdf"))
% (defun d () (interactive) (find-pdftools-page "~/LATEX/2022-1-C2-der-fun-inv.pdf"))
% (defun e () (interactive) (find-LATEX "2022-1-C2-der-fun-inv.tex"))
% (defun o () (interactive) (find-LATEX "2022-1-C2-der-fun-inv.tex"))
% (defun u () (interactive) (find-latex-upload-links "2022-1-C2-der-fun-inv"))
% (defun v () (interactive) (find-2a '(e) '(d)))
% (defun d0 () (interactive) (find-ebuffer "2022-1-C2-der-fun-inv.pdf"))
% (defun cv () (interactive) (C) (ee-kill-this-buffer) (v) (g))
%          (code-eec-LATEX "2022-1-C2-der-fun-inv")
% (find-pdf-page   "~/LATEX/2022-1-C2-der-fun-inv.pdf")
% (find-sh0 "cp -v  ~/LATEX/2022-1-C2-der-fun-inv.pdf /tmp/")
% (find-sh0 "cp -v  ~/LATEX/2022-1-C2-der-fun-inv.pdf /tmp/pen/")
%     (find-xournalpp "/tmp/2022-1-C2-der-fun-inv.pdf")
%   file:///home/edrx/LATEX/2022-1-C2-der-fun-inv.pdf
%               file:///tmp/2022-1-C2-der-fun-inv.pdf
%           file:///tmp/pen/2022-1-C2-der-fun-inv.pdf
% http://angg.twu.net/LATEX/2022-1-C2-der-fun-inv.pdf
% (find-LATEX "2019.mk")
% (find-sh0 "cd ~/LUA/; cp -v Pict2e1.lua Pict2e1-1.lua Piecewise1.lua ~/LATEX/")
% (find-sh0 "cd ~/LUA/; cp -v Pict2e1.lua Pict2e1-1.lua Pict3D1.lua ~/LATEX/")
% (find-sh0 "cd ~/LUA/; cp -v {UbExpr1,UbExpr2,RAng1,RAngFormulas1}.lua ~/LATEX/")
% (find-CN-aula-links "2022-1-C2-der-fun-inv" "2" "c2m221dfi" "c2df")

% «.defs»			(to "defs")
% «.defs-DFIs»			(to "defs-DFIs")
% «.title»			(to "title")
% «.introducao»			(to "introducao")
% «.exercicio-1»		(to "exercicio-1")
% «.exercicio-1-resps»		(to "exercicio-1-resps")
% «.demonstracao-complicada»	(to "demonstracao-complicada")
% «.o-jogo»			(to "o-jogo")
%
% «.djvuize»	(to "djvuize")



% <videos>
% Video (not yet):
% (find-ssr-links     "c2m221dfi" "2022-1-C2-der-fun-inv")
% (code-eevvideo      "c2m221dfi" "2022-1-C2-der-fun-inv")
% (code-eevlinksvideo "c2m221dfi" "2022-1-C2-der-fun-inv")
% (find-c2m221dfivideo "0:00")

\documentclass[oneside,12pt]{article}
\usepackage[colorlinks,citecolor=DarkRed,urlcolor=DarkRed]{hyperref} % (find-es "tex" "hyperref")
\usepackage{amsmath}
\usepackage{amsfonts}
\usepackage{amssymb}
\usepackage{pict2e}
\usepackage[x11names,svgnames]{xcolor} % (find-es "tex" "xcolor")
\usepackage{colorweb}                  % (find-es "tex" "colorweb")
%\usepackage{tikz}
%
% (find-dn6 "preamble6.lua" "preamble0")
%\usepackage{proof}   % For derivation trees ("%:" lines)
%\input diagxy        % For 2D diagrams ("%D" lines)
%\xyoption{curve}     % For the ".curve=" feature in 2D diagrams
%
\usepackage{edrx21}               % (find-LATEX "edrx21.sty")
\input edrxaccents.tex            % (find-LATEX "edrxaccents.tex")
\input edrx21chars.tex            % (find-LATEX "edrx21chars.tex")
\input edrxheadfoot.tex           % (find-LATEX "edrxheadfoot.tex")
\input edrxgac2.tex               % (find-LATEX "edrxgac2.tex")
%\usepackage{emaxima}              % (find-LATEX "emaxima.sty")
%
%\usepackage[backend=biber,
%   style=alphabetic]{biblatex}            % (find-es "tex" "biber")
%\addbibresource{catsem-slides.bib}        % (find-LATEX "catsem-slides.bib")
%
% (find-es "tex" "geometry")
\usepackage[a6paper, landscape,
            top=1.5cm, bottom=.25cm, left=1cm, right=1cm, includefoot
           ]{geometry}
%
\begin{document}

\catcode`\^^J=10
\directlua{dofile "dednat6load.lua"}  % (find-LATEX "dednat6load.lua")
%L dofile "Piecewise1.lua"           -- (find-LATEX "Piecewise1.lua")
%L dofile "QVis1.lua"                -- (find-LATEX "QVis1.lua")
%L dofile "Pict3D1.lua"              -- (find-LATEX "Pict3D1.lua")
%L dofile "RAngFormulas1.lua"        -- (find-LATEX "RAngFormulas1.lua")
%L Pict2e.__index.suffix = "%"
\pu
\def\pictgridstyle{\color{GrayPale}\linethickness{0.3pt}}
\def\pictaxesstyle{\linethickness{0.5pt}}
\celllower=2.5pt

% «defs»  (to ".defs")
% (find-LATEX "edrx21defs.tex" "colors")
% (find-LATEX "edrx21.sty")

\def\u#1{\par{\footnotesize \url{#1}}}
\def\rq{\ColorRed{?}}

\def\drafturl{http://angg.twu.net/LATEX/2022-1-C2.pdf}
\def\drafturl{http://angg.twu.net/2022.1-C2.html}
\def\draftfooter{\tiny \href{\drafturl}{\jobname{}} \ColorBrown{\shorttoday{} \hours}}

% «defs-DFIs»  (to ".defs-DFIs")
% (c2m221dfip 3 "defs-DFIs")
% (c2m221dfia   "defs-DFIs")
\def\realeqnp#1{\overset{\scriptscriptstyle(#1)}{=}}
\def\eqnp{\realeqnp}
\def\eqnp#1{=}

\sa{DFI sem parenteses}{
  \begin{array}{lrcl}
    \text{Se:}    & f(g(x))       &\eqnp{1}& x \\
    \text{Então:} & \ddx f(g(x))  &\eqnp{2}& \ddx x \\
                                 &&\eqnp{3}& 1 \\
                  & \ddx f(g(x))  &\eqnp{4}& f'(g(x))g'(x) \\
                  & f'(g(x))g'(x) &\eqnp{5}& 1 \\
                  & g'(x)         &\eqnp{6}& \D \frac{1}{f'(g(x))} \\
  \end{array}}
\sa{DFI com parenteses}{\left( \ga{DFI sem parenteses} \right)}
\sa{[DFI]}{\ensuremath{[\text{DFI}]}}

\sa{DFI- sem parenteses}{
  \begin{array}{lrcl}
    \text{Se:}    & f(g(x))       &\eqnp{1}& x \\
    \text{Então:} & g'(x)         &\eqnp{6}& \D \frac{1}{f'(g(x))} \\
  \end{array}}
\sa{DFI- com parenteses}{\left( \ga{DFI- sem parenteses} \right)}
\sa{[DFI-]}{\ensuremath{[\text{DFI}^-]}}
 





%  _____ _ _   _                               
% |_   _(_) |_| | ___   _ __   __ _  __ _  ___ 
%   | | | | __| |/ _ \ | '_ \ / _` |/ _` |/ _ \
%   | | | | |_| |  __/ | |_) | (_| | (_| |  __/
%   |_| |_|\__|_|\___| | .__/ \__,_|\__, |\___|
%                      |_|          |___/      
%
% «title»  (to ".title")
% (c2m221dfip 1 "title")
% (c2m221dfia   "title")

\thispagestyle{empty}

\begin{center}

\vspace*{1.2cm}

{\bf \Large Cálculo 2 - 2022.1}

\bsk

Aula 28: a derivada

da função inversa

\bsk

Eduardo Ochs - RCN/PURO/UFF

\url{http://angg.twu.net/2022.1-C2.html}

\end{center}

\newpage

% «introducao»  (to ".introducao")
% (c2m221dfip 2 "introducao")
% (c2m221dfia    "introducao")

{\bf Introdução}


\scalebox{0.45}{\def\colwidth{11cm}\firstcol{

No curso de Cálculo 1 você deve ter visto uma fórmula para a derivada
da função inversa, e você deve ter visto que ela é sempre apresentada
com certas ``hipóteses''... tipo: ``se as condições tais e tais são
obedecidas então a derivada da função inversa é dada por esta formula
aqui: [bla]'' --- e fica implícito que quando essas condições não são
obedecidas a fórmula pode dar resultados errados. Dê uma olhada em:

\ssk

{\scriptsize

% (find-books "__analysis/__analysis.el" "miranda")
% (find-dmirandacalcpage 90 "3.6 Derivada da Função Inversa")
% http://hostel.ufabc.edu.br/~daniel.miranda/calculo/calculo.pdf#page=90
\url{http://hostel.ufabc.edu.br/~daniel.miranda/calculo/calculo.pdf\#page=90}

% https://en.wikipedia.org/wiki/Inverse_function_rule
% https://en.wikipedia.org/wiki/Integral_of_inverse_functions
\url{https://en.wikipedia.org/wiki/Inverse_function_rule}

}

\msk

Nós vimos --- por alto --- que existe uma versão do TFC1 pra funções
contínuas e uma outra, bem mais complicada, pra funções com
descontinuidades... e vimos que o TFC2 também tem várias versões, e
vimos que em muitas situações nós podemos fazer todas as contas que
nos interessam sem dizer explicitamente quais são os domínios das
nossas funções; se for necessário dizer os domínios nós podemos
descobrir os domínios certos no final, depois de fazer todas as
contas.

\msk

Em Cálculo 2 e Cálculo 3 é comum a gente fazer as contas primeiro e só
colocar os domínios e as ``condições necessárias'' no final. Neste PDF
eu vou fazer uma versão extrema disso: eu vou considerar que vocês só
vão ser capazes de entender bem as condições necessárias quando
tiverem bastante prática com as contas, então a gente vai sempre
começar ``chutando'' que as contas funcionam e ``testando'' elas
depois.

}\anothercol{

  Nós vamos usar esta versão aqui da demonstração da fórmula da
  derivada da função inversa (``DFI''):
  %
  $$\ga{DFI sem parenteses}$$

  O modo natural de numerar cada uma das igualdades dela é este:
  %
  $$\def\eqnp{\realeqnp}
    \ga{DFI sem parenteses}
  $$

  Por enquanto estamos fingindo que os domínios não importam e que as
  nossas funções são deriváveis ``onde precisar''.

  \msk

  Se $f:A→B$ e $g:B→A$ então ``$f$ e $g$ são inversas'' quer dizer:
  %
  $$\begin{array}{cc}
             & ∀a∈A.\;g(f(a)) = a \\
    \text{e} & ∀b∈B.\;f(g(b)) = b \\
    \end{array}
  $$

  A linha ``Se: $f(g(x))=x$'' diz que a nossa única hipótese
  \ColorRed{explícita} é que $∀x∈\dom(g). f(g(x))=x$...


}}

\newpage

% «exercicio-1»  (to ".exercicio-1")
% (c2m221dfip 3 "exercicio-1")
% (c2m221dfia   "exercicio-1")

{\bf Exercício 1.}

\scalebox{0.55}{\def\colwidth{12cm}\firstcol{

Sejam:
%
$$\begin{array}{rcl}
  \ga{[DFI]}  &=& \ga{DFI com parenteses} \\ \\[-5pt]
  \ga{[DFI-]} &=& \ga{DFI- com parenteses} \\
  \end{array}
$$

Repare que \ga{[DFI]} é a demonstração da fórmula da derivada da
função inversa e \ga{[DFI-]} é só a fórmula da derivada da função
inversa, sem a demonstração toda... e, de novo, lembre que eu vou usar
uma versão muito reduzida das condições necessárias pra essa fórmula
valer.

Diga os resultados das substituições abaixo.

\bsk

a) $\ga{[DFI]} \bmat{f(y) := e^y \\
                    f'(y) := e^y \\
                    g(x) := \ln x \\
                    g'(x) := \ln' x \\
                   } = \rq$


}\anothercol{

\def\Sqrt{\text{sqrt}}

b) $\ga{[DFI-]} \bmat{f (y) := y^2 \\
                    f'(y) := 2y \\
                    g (x) := \Sqrt(x) \\
                    g'(x) := \Sqrt'(x) \\
                   } = \rq
   $

\msk

c) $\ga{[DFI-]} \bmat{f (y) := \sen y \\
                    f'(y) := \cos y \\
                    g (x) := \arcsen(x) \\
                    g'(x) := \arcsen'(x) \\
                   } = \rq
   $

\msk

d) $\ga{[DFI-]} \bmat{x     := s \\
                    f (θ) := \sen θ \\
                    f'(θ) := \cos θ \\
                    g (s) := \arcsen(s) \\
                    g'(s) := \arcsen'(s) \\
                   } = \rq
   $

\msk

e) $\ga{[DFI-]} \bmat{x := c\\
                    f (θ) := \cos θ \\
                    f'(θ) := -\sen θ \\
                    g (c) := \cos^{-1}(c) \\
                    g'(c) := (\cos^{-1})'(c) \\
                   } = \rq
   $

\bsk

Repare que no item (b) eu usei `$\Sqrt(x)$' ao

invés de `$\sqrt{x}$'... isso é porque não há uma

notação boa pra derivada da raiz quadrada.

}}

\newpage

% «exercicio-1-resps»  (to ".exercicio-1-resps")
% (c2m221dfip 4 "exercicio-1-resps")
% (c2m221dfia   "exercicio-1-resps")

{\bf Exercício 1: respostas}

\msk

...ainda não digitei! Mas veja este PDF:


\ssk

{\footnotesize

% (c2m221dp1p 3 "underbraces")
% (c2m221dp1a   "underbraces")
%    http://angg.twu.net/LATEX/2022-1-C2-dicas-pra-P1.pdf
\url{http://angg.twu.net/LATEX/2022-1-C2-dicas-pra-P1.pdf}

}

\ssk




\newpage

% «demonstracao-complicada»  (to ".demonstracao-complicada")
% (c2m221dfip 5 "demonstracao-complicada")
% (c2m221dfia   "demonstracao-complicada")

{\bf Uma demonstração complicada}

\def\casespn#1#2{
  \begin{cases}
     #1 & \text{quando $0<x$}, \\
     #2 & \text{quando $x<0$} \\
  \end{cases}}

% (c2m212intsp 8 "dfi")
% (c2m212intsa   "dfi")

\def\eqnp#1{=}
\def\eqnp{\realeqnp}

\scalebox{0.55}{\def\colwidth{11cm}\firstcol{

$$\begin{array}{rcl}
  \exp(\ln(x)) &\eqnp1& x \\
        \ln' x &\eqnp2& 1/\exp'(\ln(x)) \\
               &\eqnp3& 1/\exp(\ln(x)) \\
               &\eqnp4& 1/x \\
  \ddx f(g(x)) &\eqnp5& f'(g(x))g'(x) \\
  \ddx \ln(-x) &\eqnp6& \ln'(-x)·-1 \\
               &\eqnp7& 1/(-x)·-1 \\
               &\eqnp8& 1/x \\
        \ln|x| &\eqnp9& \casespn{\ln x}{\ln -x} \\
   \ddx \ln|x| &\eqnp{10}& \ddx \casespn{\ln x}{\ln -x} \\
               &\eqnp{11}& \casespn{\ddx \ln x}{\ddx \ln -x} \\
               &\eqnp{12}& \casespn{1/x}{1/x} \\
               &\eqnp{13}& 1/x \\
           1/x &\eqnp{14}& \ddx \ln|x| \\
  \D \Intx{a}{b}{\frac1x}
               &\eqnp{15}& \D \difx{a}{b}{\big( \ln|x| \big)} \\
  \end{array}
$$

}\anothercol{
}}


\newpage

% «o-jogo»  (to ".o-jogo")
% (c2m221dfip 4 "o-jogo")
% (c2m221dfia   "o-jogo")

{\bf Outro jogo}

% (c2m221isp 10 "exercicio-2-dica")
% (c2m221isa    "exercicio-2-dica")


\scalebox{0.5}{\def\colwidth{11.5cm}\firstcol{

No final de maio nós usamos um jogo pra debugar

representações gráficas... esse aqui:

\ssk

{\footnotesize

% (c2m221isp 10)
%    http://angg.twu.net/LATEX/2022-1-C2-infs-e-sups.pdf#page=10
\url{http://angg.twu.net/LATEX/2022-1-C2-infs-e-sups.pdf\#page=10}

}

\ssk

Agora nós vamos fazer algo parecido pra debugar {\sl demonstrações}.
Nesse jogo novo os objetivos do jogador $O$ vão ser 1) garantir que o
jogador $P$ sabe justificar cada passo da demonstração que ele propôs,
e 2) ajudar o jogador $P$ a decobrir erros, 3) ajudar o jogador $P$ a
descobrir passos da demonstração que são saltos ``grandes demais'', e
que ficariam mais claros se fosses reescritos como vários sub-passos.

\bsk

{\sl Falta digitar isso aqui!}

\msk

Os exemplos de jogadas que eu pus no quadro em 30/jun/2022 foram
estes:

\msk

% (find-pdf-page "~/2022.1-C2/C2-quadros.pdf" 42)

$O$: Porque $\realeqnp{2}$?

$P$: Pela \ga{[DFI]}.

$O$: Qual caso particular da \ga{[DFI]}?

$P$: (Aqui o jogador $P$ responde mostrando uma substituição em
detalhes: o resultado do exercício 1a)

$O$: Porque $\realeqnp{4}$?

$P$: Por $\exp(\ln(x))=x$ --- portanto $1/\exp(\ln(x))=1/x$.

}\anothercol{

{\bf Exercício 2.}

\ssk

a) Justifique a igualdade $\realeqnp{5}$.

Obs: aqui você pode responder com o nome de uma fórmula bem conhecida.
Alguém que não lembre essa fórmula bem pode pesquisar ela pelo nome e
encontrar várias explicações grandes sobre ela, {\sl usando várias
  notações diferentes}, em livros e na internet.

\ssk

b) Justifique $\realeqnp{6}$.

c) Justifique $\realeqnp{7}$.

d) Justifique $\realeqnp{12}$.

\ssk

e) Justifique $\realeqnp{15}$.

Dicas: 1) a igualdade $\realeqnp{15}$ é consequência da igualdade
$\realeqnp{14}$; 2) aqui você vai ter que usar o TFC2! Encontre um
enunciado do TFC2 em algum lugar e mostre qual é a substituição que
você tem que usar pra obter $\realeqnp{15}$ a partir de
$\realeqnp{14}$.

\bsk

\sa{RM}{\ensuremath{[\text{RM}]}}

O item (f) abaixo é bem mais difícil ---

mas os livros fazem passos desse tipo a beça... $\frown$

\ssk

f) A igualdade $\realeqnp{11}$ é consequência de uma
``regra misteriosa'', \ga{RM}, que é ``óbvia''. Digamos que:
%
$$\ga{RM} \; = \;
  \left(\ddx
    \begin{cases}
      f(x) & \text{quando $P(x)$} \\
      g(x) & \text{quando $Q(x)$} \\
    \end{cases}
    = \rq
  \right)
$$

Descubra qual é o `$\rq$' certo e descubra qual é a substituição
$\ga{RM}\,[??] = ???$ que justifica $\realeqnp{11}$.



}}


\newpage

\sa{[DFI]}{{
   \sa{Mul(fp(g(x)),gp(x))}{f'(g(x)) · g'(x)}
  \sa{f(g(x))}{f(g(x))}
  \sa{fp(g(x))}{f'(g(x))}
  \sa{gp(x)}{g'(x)}
  \sa{x}{x}
  \begin{array}{lrcl}
    \text{Se:}    &                \ga{f(g(x))} &\eqnp{1}& \ga{x} \\
    \text{Então:} & \frac{d}{d\ga{x}} \ga{f(g(x))} &\eqnp{2}& \frac{d}{d\ga{x}} \ga{x} \\
                                            &&\eqnp{3}& 1 \\
                  & \frac{d}{d\ga{x}} \ga{f(g(x))} &\eqnp{4}& \ga{Mul(fp(g(x)),gp(x))} \\
                  & \ga{Mul(fp(g(x)),gp(x))}    &\eqnp{5}& 1 \\
                  & \ga{gp(x)}                  &\eqnp{6}& \D \frac{1}{\ga{fp(g(x))}} \\
  \end{array}
}}

\sa{[DFI][S1]}{{
  \sa{Mul(fp(g(x)),gp(x))}{f'(\ln(x)) · g'(x)}
  \sa{f(g(x))}{e^{g(x)}}
  \sa{fp(g(x))}{f'(\ln(x))}
  \sa{gp(x)}{g'(x)}
  \sa{x}{x}
  \begin{array}{lrcl}
    \text{Se:}    &                \ga{f(g(x))} &\eqnp{1}& \ga{x} \\
    \text{Então:} & \frac{d}{d\ga{x}} \ga{f(g(x))} &\eqnp{2}& \frac{d}{d\ga{x}} \ga{x} \\
                                            &&\eqnp{3}& 1 \\
                  & \frac{d}{d\ga{x}} \ga{f(g(x))} &\eqnp{4}& \ga{Mul(fp(g(x)),gp(x))} \\
                  & \ga{Mul(fp(g(x)),gp(x))}    &\eqnp{5}& 1 \\
                  & \ga{gp(x)}                  &\eqnp{6}& \D \frac{1}{\ga{fp(g(x))}} \\
  \end{array}
}}


$$\ga{[DFI]}$$
$$\ga{[DFI][S1]}$$




\GenericWarning{Success:}{Success!!!}  % Used by `M-x cv'

\end{document}

%  ____  _             _         
% |  _ \(_)_   ___   _(_)_______ 
% | | | | \ \ / / | | | |_  / _ \
% | |_| | |\ V /| |_| | |/ /  __/
% |____// | \_/  \__,_|_/___\___|
%     |__/                       
%
% «djvuize»  (to ".djvuize")
% (find-LATEXgrep "grep --color -nH --null -e djvuize 2020-1*.tex")


%  __  __       _        
% |  \/  | __ _| | _____ 
% | |\/| |/ _` | |/ / _ \
% | |  | | (_| |   <  __/
% |_|  |_|\__,_|_|\_\___|
%                        
% <make>

 (eepitch-shell)
 (eepitch-kill)
 (eepitch-shell)
# (find-LATEXfile "2019planar-has-1.mk")
make -f 2019.mk STEM=2022-1-C2-der-fun-inv veryclean
make -f 2019.mk STEM=2022-1-C2-der-fun-inv pdf

% Local Variables:
% coding: utf-8-unix
% ee-tla: "c2df"
% ee-tla: "c2m221dfi"
% End:
