% (find-LATEX "2022-1-C3-P1.tex")
% (defun c () (interactive) (find-LATEXsh "lualatex -record 2022-1-C3-P1.tex" :end))
% (defun C () (interactive) (find-LATEXsh "lualatex 2022-1-C3-P1.tex" "Success!!!"))
% (defun D () (interactive) (find-pdf-page      "~/LATEX/2022-1-C3-P1.pdf"))
% (defun d () (interactive) (find-pdftools-page "~/LATEX/2022-1-C3-P1.pdf"))
% (defun e () (interactive) (find-LATEX "2022-1-C3-P1.tex"))
% (defun o () (interactive) (find-LATEX "2022-1-C3-P1.tex"))
% (defun u () (interactive) (find-latex-upload-links "2022-1-C3-P1"))
% (defun v () (interactive) (find-2a '(e) '(d)))
% (defun d0 () (interactive) (find-ebuffer "2022-1-C3-P1.pdf"))
% (defun cv () (interactive) (C) (ee-kill-this-buffer) (v) (g))
%          (code-eec-LATEX "2022-1-C3-P1")
% (find-pdf-page   "~/LATEX/2022-1-C3-P1.pdf")
% (find-sh0 "cp -v  ~/LATEX/2022-1-C3-P1.pdf /tmp/")
% (find-sh0 "cp -v  ~/LATEX/2022-1-C3-P1.pdf /tmp/pen/")
%     (find-xournalpp "/tmp/2022-1-C3-P1.pdf")
%   file:///home/edrx/LATEX/2022-1-C3-P1.pdf
%               file:///tmp/2022-1-C3-P1.pdf
%           file:///tmp/pen/2022-1-C3-P1.pdf
% http://angg.twu.net/LATEX/2022-1-C3-P1.pdf
% (find-LATEX "2019.mk")
% (find-sh0 "cd ~/LUA/; cp -v Pict2e1.lua Pict2e1-1.lua Piecewise1.lua ~/LATEX/")
% (find-sh0 "cd ~/LUA/; cp -v Pict2e1.lua Pict2e1-1.lua Pict3D1.lua ~/LATEX/")
% (find-sh0 "cd ~/LUA/; cp -v C2Subst1.lua C2Formulas1.lua ~/LATEX/")
% (find-CN-aula-links "2022-1-C3-P1" "3" "c3m221p1" "c3p1")

% «.defs»		(to "defs")
% «.title»		(to "title")
% «.barranco-defs»	(to "barranco-defs")
% «.questao-1»		(to "questao-1")
%
% «.djvuize»		(to "djvuize")



% <videos>
% Video (not yet):
% (find-ssr-links     "c3m221p1" "2022-1-C3-P1")
% (code-eevvideo      "c3m221p1" "2022-1-C3-P1")
% (code-eevlinksvideo "c3m221p1" "2022-1-C3-P1")
% (find-c3m221p1video "0:00")

\documentclass[oneside,12pt]{article}
\usepackage[colorlinks,citecolor=DarkRed,urlcolor=DarkRed]{hyperref} % (find-es "tex" "hyperref")
\usepackage{amsmath}
\usepackage{amsfonts}
\usepackage{amssymb}
\usepackage{pict2e}
\usepackage[x11names,svgnames]{xcolor} % (find-es "tex" "xcolor")
\usepackage{colorweb}                  % (find-es "tex" "colorweb")
%\usepackage{tikz}
%
% (find-dn6 "preamble6.lua" "preamble0")
%\usepackage{proof}   % For derivation trees ("%:" lines)
%\input diagxy        % For 2D diagrams ("%D" lines)
%\xyoption{curve}     % For the ".curve=" feature in 2D diagrams
%
\usepackage{edrx21}               % (find-LATEX "edrx21.sty")
\input edrxaccents.tex            % (find-LATEX "edrxaccents.tex")
\input edrx21chars.tex            % (find-LATEX "edrx21chars.tex")
\input edrxheadfoot.tex           % (find-LATEX "edrxheadfoot.tex")
\input edrxgac2.tex               % (find-LATEX "edrxgac2.tex")
%\usepackage{emaxima}              % (find-LATEX "emaxima.sty")
%
%\usepackage[backend=biber,
%   style=alphabetic]{biblatex}            % (find-es "tex" "biber")
%\addbibresource{catsem-slides.bib}        % (find-LATEX "catsem-slides.bib")
%
% (find-es "tex" "geometry")
\usepackage[a6paper, landscape,
            top=1.5cm, bottom=.25cm, left=1cm, right=1cm, includefoot
           ]{geometry}
%
\begin{document}

\catcode`\^^J=10
\directlua{dofile "dednat6load.lua"}  % (find-LATEX "dednat6load.lua")
%L dofile "Piecewise1.lua"           -- (find-LATEX "Piecewise1.lua")
%L dofile "QVis1.lua"                -- (find-LATEX "QVis1.lua")
%L dofile "Pict3D1.lua"              -- (find-LATEX "Pict3D1.lua")
%L dofile "C2Formulas1.lua"          -- (find-LATEX "C2Formulas1.lua")
%L Pict2e.__index.suffix = "%"
\pu
\def\pictgridstyle{\color{GrayPale}\linethickness{0.3pt}}
\def\pictaxesstyle{\linethickness{0.5pt}}
\def\pictnaxesstyle{\color{GrayPale}\linethickness{0.5pt}}
\celllower=2.5pt

% «defs»  (to ".defs")
% (find-LATEX "edrx21defs.tex" "colors")
% (find-LATEX "edrx21.sty")

\def\u#1{\par{\footnotesize \url{#1}}}

\def\drafturl{http://angg.twu.net/LATEX/2022-1-C3.pdf}
\def\drafturl{http://angg.twu.net/2022.1-C3.html}
\def\draftfooter{\tiny \href{\drafturl}{\jobname{}} \ColorBrown{\shorttoday{} \hours}}

% «defs-T-and-B»  (to ".defs-T-and-B")
% (c3m202p1p 6 "questao-2")
% (c3m202p1a   "questao-2")
\long\def\ColorOrange#1{{\color{orange!90!black}#1}}
\def\T(Total: #1 pts){{\bf(Total: #1)}}
\def\T(Total: #1 pts){{\bf(Total: #1 pts)}}
\def\T(Total: #1 pts){\ColorRed{\bf(Total: #1 pts)}}
\def\B       (#1 pts){\ColorOrange{\bf(#1 pts)}}



%  _____ _ _   _                               
% |_   _(_) |_| | ___   _ __   __ _  __ _  ___ 
%   | | | | __| |/ _ \ | '_ \ / _` |/ _` |/ _ \
%   | | | | |_| |  __/ | |_) | (_| | (_| |  __/
%   |_| |_|\__|_|\___| | .__/ \__,_|\__, |\___|
%                      |_|          |___/      
%
% «title»  (to ".title")
% (c3m221p1p 1 "title")
% (c3m221p1a   "title")

\thispagestyle{empty}

\begin{center}

\vspace*{1.2cm}

{\bf \Large Cálculo 3 - 2022.1}

\bsk

P1 (Primeira prova)

\bsk

Eduardo Ochs - RCN/PURO/UFF

\url{http://angg.twu.net/2022.1-C3.html}

\end{center}

\newpage

% «barranco-defs»  (to ".barranco-defs")
% (c3m221nfp 26 "barranco")
% (c3m221nfa    "barranco")
%\printbibliography

%L Pict2e.bounds = PictBounds.new(v(-1,-1), v(9,9))
%L barranco = Numerozinhos.from(0, 0, [[
%L     4 4 4 4 4 4 4 4 4
%L     4 4 4 4 4 4 4 4 4
%L     3 3 3 3 4 4 4 4 4
%L     2 2 2 2 3 4 4 4 4
%L     1 1 1 1 2 3 4 4 4
%L     0 0 0 0 1 2 3 4 4
%L     0 0 0 0 0 1 2 2 2
%L     0 0 0 0 0 0 1 1 1
%L     0 0 0 0 0 0 0 0 0 ]])
%L barranco_spec0 = [[
%L     (0,7)--(3,7)--(7,3)--(8,3)
%L     (3,7)--(3,3)  (6,0)--(6,2)  (7,2)--(8,2)
%L     (0,3)--(3,3)--(6,0)--(8,0) ]]
%L barranco_spec  = barranco_spec0 -- .. [[ (7,3)--(6,2) ]]
%L barranco_spec  = barranco_spec0 .. [[ (7,3)--(6,2)--(7,2) ]]
%L barranco_spec2 = barranco_spec0 .. [[ (7,2)--(6,3)--(6,2) (6,3)--(7,3)--(7,2) ]]
%L barranco:topict(              ):sa("barranco"):output()
%L barranco:topict(barranco_spec ):sa("barranco com linhas"):output()
%L barranco:topict(barranco_spec2):sa("barranco com linhas 2"):output()
%L
%L barranco_Fa = [[ (3,3)--(3,7)--(7,3)--(6,2)--(6,0)--(3,3) ]]
%L barranco_Fb = [[ (8,2)--(6,2)--(7,3)--(8,3) ]]
%L barranco:topict(barranco_Fa):sa("barranco Fa"):output()
%L barranco:topict(barranco_Fb):sa("barranco Fb"):output()
\pu

\def\barra {\ga{barranco}}
\def\barra {\scalebox{0.9}{$\ga{barranco}$}}
\def\barrl {\scalebox{0.9}{$\ga{barranco com linhas}$}}
\def\barrFa{\scalebox{0.9}{$\ga{barranco Fa}$}}
\def\barrFb{\scalebox{0.9}{$\ga{barranco Fb}$}}

\newpage

% «questao-1»  (to ".questao-1")
% (c3m221p1p 99 "questao-1")
% (c3m221p1a    "questao-1")

{\bf Questão 1.}

% (c3m221nfp 12 "variaveis-novas")
% (c3m221nfa    "variaveis-novas")
% (c3m221nfp 30 "derivada-direcional")
% (c3m221nfa    "derivada-direcional")

\scalebox{0.55}{\def\colwidth{9cm}\firstcol{

\vspace*{-0.5cm}

\T(Total: 7.0 pts)

Quando nós fizemos os exercícios do barranco -- reproduzido na próxima
página -- nós vimos que as duas faces mais complicadas dele eram a) a
face que continha os pontos $(3,5,2)$, $(4,5,3)$ e $(3,6,3)$ e b) a
face que continha os pontos $(7,2,2)$, $(8,2,2)$ e $(7,3,4)$. Vou
chamar essas faces de $F_a$ e $F_b$ e usar os mesmos símbolos pras
funções dos planos associados a elas: quando $(x,y)∈F_a$ temos
$z(x,y)=F_a(x,y)$ e quando $(x,y)∈F_b$ temos $z(x,y)=F_b(x,y)$.

\msk

a) \B(0.2 pts) Dê a equação do plano $F_a(x,y)$.

\ssk

b) \B(0.2 pts) Dê a equação do plano $F_b(x,y)$.

\ssk

c) \B(2.0 pts) Mostre em qual região do barranco os numerozinhos obedecem
$z=F_a(x,y)$ e em qual região eles obedecem $z=F_b(x,y)$. As faces
$F_a$ e $F_b$ têm uma aresta em comum?

\ssk

d) \B(0.6 pts) Sejam $P_0=(6,2.5)$, $P_1=(6.5,2.5)$ e $P_2=(7,2.5)$.
Descubra - no olhômetro mesmo - quem são $z_x$ e $z_y$ nos pontos
$P_0$, $P_1$ e $P_2$.

\ssk

e) \B(2.0 pts) O Bortolossi define a derivada direcional por essa
fórmula aqui:
%
$$\frac{∂f}{∂𝐛v}(𝐛p) =
  \lim_{t→0} \frac{ f(𝐛p + t·𝐛v) - f(𝐛p) }{t}
$$

% (c3m221nfp 30 "derivada-direcional")
% (c3m221nfa    "derivada-direcional")

}\anothercol{

Calcule 
%
$$\frac{ f(𝐛p + t·𝐛v) - f(𝐛p) }{t}
$$

\msk

quando $𝐛p=P_1$ e $𝐛v=\VEC{0.5,-0.5}$, para os seguintes valores de
$t$: $t=3$, $t=2$, $t=1$, $t=0.5$, $t=-3$, $t=-2$, $t=-1$, $t=-0.5$.

\msk

f) \B(2.0 pts) Lembre que o gradiente de uma função de $\R^2$ em $\R$
é definido como $\Vec{∇}G(x,y) = \VEC{G_x(x,y), G_y(x,y)}$. Em
``notação de físicos'' isso vira $\Vec{∇}z = \VEC{z_x,z_y}$, e a nossa
convenção pra notação pra desenhar vetores gradientes é que cada
$\Vec{∇}G(x,y)$ é desenhado como $G(x,y) + \Vec{∇}G(x,y)$. Represente
em um dos diagramas de numerozinhos da próxima página $\Vec{∇}F$ para
estes valores de $(x,y)$: $(2,1)$, $(2,5)$, $(5,3)$, $(6,6)$, $(7,1)$,
$(7,2.5)$.


}}

\newpage

\def\barra{\ga{barranco}}
\def\barra{\scalebox{0.9}{$\ga{barranco}$}}
\def\barrl{\scalebox{0.9}{$\ga{barranco com linhas}$}}

$\begin{array}{rcl}
 \barra & \barra & \barra \\
 \barra & \barra & \barra \\
 \end{array}
$

\newpage

{\bf Questão 1: gabarito (muito incompleto)}


\scalebox{0.5}{\def\colwidth{9cm}\firstcol{

Dois modos de dividir o barranco em faces:

\msk

$\scalebox{0.9}{$
 \begin{array}{rcl}
 \ga{barranco com linhas} &
 \ga{barranco com linhas 2} \\
 \end{array}
 $}
$

\msk

Eu prefiro o primeiro modo porque ele tem

uma face a menos, mas vou aceitar respostas

que usavam o segundo modo.

As faces $F_a$ e $F_b$ são:

\msk

$\scalebox{0.9}{$
 \begin{array}{rcl}
 \ga{barranco Fa} &
 \ga{barranco Fb} \\
 \end{array}
 $}
$

\msk

a) $F_a(x,y) = x+y-6$

b) $F_b(x,y) = 2y-2$

c) Veja as figuras acima.

d) Em $P_0=(6,2.5)$ temos $z=2.5$;

Em $P_1=(6.5,2.5)$ temos $z=3$;

Em $P_2=(7,2.5)$ temos $z=3$.

}\anothercol{
}}


\newpage

{\bf Questão 2.}

\scalebox{0.6}{\def\colwidth{9cm}\firstcol{

% (c3m221nfp 12 "variaveis-novas")
% (c3m221nfa    "variaveis-novas")
% (find-sthompsonpage (+ 11  66) "IX. Introducing a Useful Dodge")
% (find-sthompsontext (+ 11  66)     "INTRODUCING A USEFUL DODGE")

\vspace*{-0.5cm}

\T(Total: 3.0 pts)

No capítulo VI o Thompson calcula $\ddx((x^2 + c) + (ax^4 + b))$
organizando as contas mais ou menos desta forma:

$$\begin{array}{rcl}
  y &=& (x^2 + c) + (ax^4 + b) \\
  \frac{dy}{dx} &=& \frac{d((x^2+c) + (ax^4+b))}{dx} \\
                &=& \frac{d(x^2+c)}{dx} + \frac{d(ax^4+b)}{dx} \\
                &=& 2x + 4ax^3 \\
  \end{array}
$$


E no capítulo IX -- ``Introducing a useful dodge'' -- o Thompson
mostra como a gente pode simplificar contas como essa introduzindo
``variáveis dependentes'' novas... por exemplo, $w = x^2+c$. Além
disso ele trata $dy$ e $dw$ como variáveis que dependem de $x$ e $dy$.

Use estes truques pra calcular $\frac{dy}{dx}$ quando:
%
$$ y = \sqrt   { \D \frac{a^2+x^2}{a^2-x^2} }
       \sqrt[3]{ \D \frac{a^2-x^2}{a^2+x^2} }
$$



}\anothercol{

{\bf Gabarito}

Veja o livro do Thompson! Ó:

\ssk

{\footnotesize

% (find-books "__analysis/__analysis.el" "thompson")
% (find-sthompsonpage (+ 11  66) "IX. Introducing a Useful Dodge")
%    https://www.gutenberg.org/files/33283/33283-pdf.pdf#page=81
\url{https://www.gutenberg.org/files/33283/33283-pdf.pdf\#page=81}

}

}}





\GenericWarning{Success:}{Success!!!}  % Used by `M-x cv'

\end{document}

%  ____  _             _         
% |  _ \(_)_   ___   _(_)_______ 
% | | | | \ \ / / | | | |_  / _ \
% | |_| | |\ V /| |_| | |/ /  __/
% |____// | \_/  \__,_|_/___\___|
%     |__/                       
%
% «djvuize»  (to ".djvuize")
% (find-LATEXgrep "grep --color -nH --null -e djvuize 2020-1*.tex")


%  __  __       _        
% |  \/  | __ _| | _____ 
% | |\/| |/ _` | |/ / _ \
% | |  | | (_| |   <  __/
% |_|  |_|\__,_|_|\_\___|
%                        
% <make>

 (eepitch-shell)
 (eepitch-kill)
 (eepitch-shell)
# (find-LATEXfile "2019planar-has-1.mk")
make -f 2019.mk STEM=2022-1-C3-P1 veryclean
make -f 2019.mk STEM=2022-1-C3-P1 pdf

% Local Variables:
% coding: utf-8-unix
% ee-tla: "c3p1"
% ee-tla: "c3m221p1"
% End:
