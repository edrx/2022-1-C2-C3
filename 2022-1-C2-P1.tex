% (find-LATEX "2022-1-C2-P1.tex")
% (defun c  () (interactive) (find-LATEXsh "lualatex -record 2022-1-C2-P1.tex" :end))
% (defun C  () (interactive) (find-LATEXsh "lualatex 2022-1-C2-P1.tex" "Success!!!"))
% (defun D  () (interactive) (find-pdf-page      "~/LATEX/2022-1-C2-P1.pdf"))
% (defun d  () (interactive) (find-pdftools-page "~/LATEX/2022-1-C2-P1.pdf"))
% (defun e  () (interactive) (find-LATEX "2022-1-C2-P1.tex"))
% (defun o  () (interactive) (find-LATEX "2021-2-C2-P1.tex"))
% (defun l2 () (interactive) (find-angg "LUA/Lazy2.lua"))
% (defun l3 () (interactive) (find-angg "LUA/Lazy3.lua"))
% (defun u  () (interactive) (find-latex-upload-links "2022-1-C2-P1"))
% (defun v  () (interactive) (find-2a '(e) '(d)))
% (defun d0 () (interactive) (find-ebuffer "2022-1-C2-P1.pdf"))
% (defun cv () (interactive) (C) (ee-kill-this-buffer) (v) (g))
%          (code-eec-LATEX "2022-1-C2-P1")
% (find-pdf-page   "~/LATEX/2022-1-C2-P1.pdf")
% (find-sh0 "cp -v  ~/LATEX/2022-1-C2-P1.pdf /tmp/")
% (find-sh0 "cp -v  ~/LATEX/2022-1-C2-P1.pdf /tmp/pen/")
%     (find-xournalpp "/tmp/2022-1-C2-P1.pdf")
%   file:///home/edrx/LATEX/2022-1-C2-P1.pdf
%               file:///tmp/2022-1-C2-P1.pdf
%           file:///tmp/pen/2022-1-C2-P1.pdf
% http://angg.twu.net/LATEX/2022-1-C2-P1.pdf
% (find-LATEX "2019.mk")
% (find-sh0 "cd ~/LUA/; cp -v Pict2e1.lua Pict2e1-1.lua Piecewise1.lua ~/LATEX/")
% (find-sh0 "cd ~/LUA/; cp -v Pict2e1.lua Pict2e1-1.lua Pict3D1.lua ~/LATEX/")
% (find-sh0 "cd ~/LUA/; cp -v C2Subst1.lua C2Formulas1.lua ~/LATEX/")
% (find-sh0 "cd ~/LUA/; cp -v Lazy2.lua Lazy3.lua Lazy4.lua Verbatim1.lua ~/LATEX/")
% (find-CN-aula-links "2022-1-C2-P1" "2" "c2m221p1" "c2p1")

% «.dofiles»			(to "dofiles")
% «.defs»			(to "defs")
% «.defs-T-and-B»		(to "defs-T-and-B")
% «.title»			(to "title")
% «.introducao»			(to "introducao")
% «.fracoes-parciais»		(to "fracoes-parciais")
% «.fracoes-parciais-gab»	(to "fracoes-parciais-gab")
% «.int-pots-sin-cos»		(to "int-pots-sin-cos")
% «.int-pots-sin-cos-gab»	(to "int-pots-sin-cos-gab")
% «.escadas-defs»		(to "escadas-defs")
% «.escadas»			(to "escadas")
% «.escadas-gab»		(to "escadas-gab")
% «.grids»			(to "grids")
% «.formulas»			(to "formulas")
% «.links»			(to "links")
%
% «.djvuize»			(to "djvuize")



% <videos>
% Video (not yet):
% (find-ssr-links     "c2m221p1" "2022-1-C2-P1")
% (code-eevvideo      "c2m221p1" "2022-1-C2-P1")
% (code-eevlinksvideo "c2m221p1" "2022-1-C2-P1")
% (find-c2m221p1video "0:00")

\documentclass[oneside,12pt]{article}
\usepackage[colorlinks,citecolor=DarkRed,urlcolor=DarkRed]{hyperref} % (find-es "tex" "hyperref")
\usepackage{amsmath}
\usepackage{amsfonts}
\usepackage{amssymb}
\usepackage{pict2e}
\usepackage[x11names,svgnames]{xcolor} % (find-es "tex" "xcolor")
\usepackage{colorweb}                  % (find-es "tex" "colorweb")
%\usepackage{tikz}
%
\usepackage{edrx21}               % (find-LATEX "edrx21.sty")
\input edrxaccents.tex            % (find-LATEX "edrxaccents.tex")
\input edrx21chars.tex            % (find-LATEX "edrx21chars.tex")
\input edrxheadfoot.tex           % (find-LATEX "edrxheadfoot.tex")
\input edrxgac2.tex               % (find-LATEX "edrxgac2.tex")
%\usepackage{emaxima}              % (find-LATEX "emaxima.sty")
%
% (find-es "tex" "geometry")
\usepackage[a6paper, landscape,
            top=1.5cm, bottom=.25cm, left=1cm, right=1cm, includefoot
           ]{geometry}
%
\begin{document}

% «dofiles»  (to ".dofiles")
% (c2m221p1p 1 "dofiles")
% (c2m221p1a   "dofiles")
\catcode`\^^J=10
\directlua{dofile "dednat6load.lua"}  % (find-LATEX "dednat6load.lua")
%L dofile "Piecewise1.lua"           -- (find-LATEX "Piecewise1.lua")
%L dofile "C2Formulas1.lua"          -- (find-LATEX "C2Formulas1.lua")
%L dofile "Lazy4.lua"                -- (find-LATEX "Lazy4.lua")
%L Pict2e.__index.suffix = "%"
\pu
\def\pictgridstyle{\color{GrayPale}\linethickness{0.3pt}}
\def\pictaxesstyle{\linethickness{0.5pt}}
\def\pictnaxesstyle{\color{GrayPale}\linethickness{0.5pt}}
\celllower=2.5pt

% «defs»  (to ".defs")
% (find-LATEX "edrx21defs.tex" "colors")
% (find-LATEX "edrx21.sty")

\def\u#1{\par{\footnotesize \url{#1}}}

\def\drafturl{http://angg.twu.net/LATEX/2022-1-C2.pdf}
\def\drafturl{http://angg.twu.net/2022.1-C2.html}
\def\draftfooter{\tiny \href{\drafturl}{\jobname{}} \ColorBrown{\shorttoday{} \hours}}

% «defs-T-and-B»  (to ".defs-T-and-B")
% (c3m202p1p 6 "questao-2")
% (c3m202p1a   "questao-2")
\long\def\ColorOrange#1{{\color{orange!90!black}#1}}
\def\T(Total: #1 pts){{\bf(Total: #1)}}
\def\T(Total: #1 pts){{\bf(Total: #1 pts)}}
\def\T(Total: #1 pts){\ColorRed{\bf(Total: #1 pts)}}
\def\B       (#1 pts){\ColorOrange{\bf(#1 pts)}}

% (c2m221ftp 1 "title")
% (c2m221fta   "title")
% (c2m221fda   "title")
\input 2022-1-C2-formulas-defs.tex  % (find-LATEX "2022-1-C2-formulas-defs.tex")



%  _____ _ _   _                               
% |_   _(_) |_| | ___   _ __   __ _  __ _  ___ 
%   | | | | __| |/ _ \ | '_ \ / _` |/ _` |/ _ \
%   | | | | |_| |  __/ | |_) | (_| | (_| |  __/
%   |_| |_|\__|_|\___| | .__/ \__,_|\__, |\___|
%                      |_|          |___/      
%
% «title»  (to ".title")
% (c2m221p1p 1 "title")
% (c2m221p1a   "title")

\thispagestyle{empty}

\begin{center}

\vspace*{1.2cm}

{\bf \Large Cálculo 2 - 2022.1}

\bsk

P1 (Primeira prova)

\bsk

Eduardo Ochs - RCN/PURO/UFF

\url{http://angg.twu.net/2022.1-C2.html}

\end{center}

\newpage

%  ___       _                 _                       
% |_ _|_ __ | |_ _ __ ___   __| |_   _  ___ __ _  ___  
%  | || '_ \| __| '__/ _ \ / _` | | | |/ __/ _` |/ _ \ 
%  | || | | | |_| | | (_) | (_| | |_| | (_| (_| | (_) |
% |___|_| |_|\__|_|  \___/ \__,_|\__,_|\___\__,_|\___/ 
%                                                      
% «introducao»  (to ".introducao")
% (c2m221p1p 2 "introducao")
% (c2m221p1a   "introducao")

{\bf Introdução}


\scalebox{0.6}{\def\colwidth{9cm}\firstcol{

Lembre que a gente não teve tempo no curso pra ``aprender a resolver
integrais'' no sentido usual... ``resolver integrais'' significa
começar com um problema como, sei lá, por exemplo,
%
$$\intx{\frac{2x^3 + 4x^2 + 5x + 6}{x^2 + 7x + 8}} \;=\; \Rq$$

e aí fazer um monte de contas até transformar a expressão original
numa expressão que ``é mais fácil de calcular'' porque não tem mais um
sinal de integral, e depois checar se todas as contas estão certas e
listar quais condições têm que ser verdade pra todas as igualdades
serem verdadeiras...

No curso só deu tempo da gente fazer o início disso, que é aprender e
ler contas de ``resolver integrais'', e entender, verificar e
justificar cada passo delas.

Em vários dos problemas desta prova você vai ter que justificar os
passos mais difíceis de certas ``contas de resolver integrais''. Eu
tentei fazer essa prova de um modo que valesse muito a pena vocês
relerem ela depois e tentarem entender e justificar os outros passos
dessas contas...

}\anothercol{

  ...mais precisamente: 1) eu escolhi integrais que usam técnicas que
  vão ser \underline{muito úteis} nas matérias que vêm depois
  (principalmente as de Física); 2) eu pus links pra mais textos (com
  exercícios!) sobre essas técnicas de integração; 3) e se vocês
  tiverem que fazer vista de prova eu talvez peça pra vocês
  justificarem outros passos da contas pra ver se vocês estão sabendo
  a matéria...

  \msk

  A prova vai ser posta neste link aqui em breve:

  \ssk

  {\footnotesize

    % (c2m221p1p 2 "introducao")
    % (c2m221p1a   "introducao")
    \url{http://angg.twu.net/LATEX/2022-1-C2-P1.pdf}

  }

  \msk

   Estudem por ela quando der!

   Boa prova! $\smile$

}}



\newpage

%  _____                                                     _       _     
% |  ___| __ __ _  ___ ___   ___  ___   _ __   __ _ _ __ ___(_) __ _(_)___ 
% | |_ | '__/ _` |/ __/ _ \ / _ \/ __| | '_ \ / _` | '__/ __| |/ _` | / __|
% |  _|| | | (_| | (_| (_) |  __/\__ \ | |_) | (_| | | | (__| | (_| | \__ \
% |_|  |_|  \__,_|\___\___/ \___||___/ | .__/ \__,_|_|  \___|_|\__,_|_|___/
%                                      |_|                                 
%
% «fracoes-parciais»  (to ".fracoes-parciais")
% (c2m221p1p 3 "fracoes-parciais")
% (c2m221p1a   "fracoes-parciais")

\def\qeqnp#1{`$\eqnp{#1}$'}
\def\eqnp{\eqnpfull}

{\bf Questão 1}

\scalebox{0.6}{\def\colwidth{10cm}\firstcol{

\hbox{\T(Total: 5.0 pts)}

(Sobre frações parciais)

Considere:

\msk

\def\myl{\\[-8pt]}
%
$\scalebox{0.8}{$
  \begin{array}{rcl}
  \D \intx{\frac1x\,}
  &\eqnp{1}& \ln|x| \\ \myl
  \D \intx{\frac1{x+a}\,}
  &\eqnp{2}& \D \intu{\frac1{u}} \\
  &\eqnp{3}& \ln|u| \\
  &\eqnp{4}& \ln|x+a| \\ \myl
      \D \frac{2}{x+3} + \frac{4}{x+5}
  &=& \D \frac{2(x+5)}{(x+3)(x+5)} + \frac{4(x+3)}{(x+3)(x+5)} \\ \myl
  &=& \D \frac {2(x+5) + 4(x+3)} {(x+3)(x+5)} \\ \myl
  &=& \D \frac {2x+10 + 4x+12} {(x+3)(x+5)} \\ \myl
  &=& \D \frac {6x+22} {x^2+8x+15} \\ \myl
      \D \intx{\frac {6x+22} {x^2+8x+15}}
  &=& \D \intx{\frac{2}{x+3} + \frac{4}{x+5}} \\ \myl
  &=& \D \intx{\frac{2}{x+3}} + \intx{\frac{4}{x+5}} \\ \myl
  &=& \D 2\intx{\frac{1}{x+3}} + 4\intx{\frac{1}{x+5}} \\ \myl
  &\eqnp{12} & 2\ln|x+3| + 4\ln|x+5| \\
  \end{array}
  $}
$

}\anothercol{

  % \vspace*{0.5cm}

  Sejam [FP1], [FP2] e [FP12] as igualdades

  numeradas à esquerda.

  \msk

  a) \B (2.0 pts) Mostre como justificar o \qeqnp{2}

  usando o [MV2].

  \msk

  b) \B (3.0) pts) Mostre como justificar o \qeqnp{12}.


}}


\newpage

%  _____                                                            _     
% |  ___| __ __ _  ___ ___   _ __   __ _ _ __ ___ ___    __ _  __ _| |__  
% | |_ | '__/ _` |/ __/ __| | '_ \ / _` | '__/ __/ __|  / _` |/ _` | '_ \ 
% |  _|| | | (_| | (__\__ \ | |_) | (_| | | | (__\__ \ | (_| | (_| | |_) |
% |_|  |_|  \__,_|\___|___/ | .__/ \__,_|_|  \___|___/  \__, |\__,_|_.__/ 
%                           |_|                         |___/             
%
% «fracoes-parciais-gab»  (to ".fracoes-parciais-gab")
% (c2m221p1p 4 "fracoes-parciais-gab")
% (c2m221p1a   "fracoes-parciais-gab")
% (find-angg "LUA/Lazy4.lua" "fracoes-parciais-test")

{\bf Questão 1: gabarito}


% \sa{[FP2]}{\CFname{FP}{_2}}
% \sa{[FP234]}{\CFname{FP}{_{234}}}
% \sa{[MV2]}{\CFname{MV}{_2}}

%L output(out)
%L
%L ang = Ang.from([[
%L   \begin{array}{rcl}
%L     <FP2_> &=& <Paren(FP2)> \\
%L     <MV2_> &=& <Paren(MV2)> \\
%L   \end{array}
%L ]])
%L ang:sa("fpgab1"):output()
%L
%L ang = Ang.from([[
%L   \begin{array}{rcl}
%L     <MV2_><SFPa.bsm> &=& <Paren(SFPa(MV2))> \\
%L     <MV2_><SFPb.bsm> &=& <Paren(SFPb(MV2))> \\
%L     <MV2_><SFPc.bsm> &=& <Paren(SFPc(MV2))> \\
%L   \end{array}
%L ]])
%L ang:sa("fpgab2"):output()
\pu

%L ang = Ang.from([[
%L   \begin{array}{rcl}
%L     <FP234_> &=& <Paren(FP234)> \\
%L   \end{array}
%L ]])
%L ang:sa("fpgab3"):output()
%L
%L ang = Ang.from([[
%L   \begin{array}{c}
%L     <FPa3_> = <FP234_><SFPa3.bmat> = <Paren(SFPa3(FP234))>  \\
%L     <FPa5_> = <FP234_><SFPa5.bmat> = <Paren(SFPa5(FP234))>  \\
%L   \end{array}
%L ]])
%L ang:sa("fpgab4"):output()
\pu





\scalebox{0.55}{\def\colwidth{10cm}\firstcol{

    Pra fazer o item (a) eu vou começar dando um nome curto --
    $\ga{[FP2]}$ -- pra igualdade \qeqnp2, e relembrando qual é a
    fórmula $\ga{[MV2]}$:

    \msk

    $\scalebox{0.8}{$\ga{fpgab1}$}
    $

    \msk

    ...e agora vou procurar uma substituição que transforma a
    $\ga{[MV2]}$ em algo parecido com a $\ga{[FP2]}$. Eu não consigo
    encontrar ela direto, então vou fazer vários chutes -- a escolha
    da substituição -- e testes -- escrever por extenso o resultado da
    substituição e ver se esse resultado é ``parecido'' com a
    $\ga{[FP2]}$:

    \msk

    $\scalebox{0.8}{$\ga{fpgab2}$}
    $

    \msk

    O último resultado acima é bastante bom. Note que ele tem um
    `$· 1$' e que ele descreve uma mudança de variável na integral
    definida, não na integral indefinida.

}\anothercol{

  Agora o item (b). Se combinarmos as igualdades \qeqnp2, \qeqnp3 e
  \qeqnp4 obtemos a igualdade abaixo:
  %
  $$\scalebox{0.8}{$\ga{fpgab3}$}
  $$

  Ela vale pra todo valor de $a$. Aqui estão dois casos particulares
  dela:
  %
  $$\scalebox{0.8}{$\ga{fpgab4}$}
  $$

  Considere esta série de igualdades:

  \def\myl{\\[-8pt]}
  % 
  $\scalebox{0.8}{$
    \begin{array}{rcl}
      \D \intx{\frac1{x+a}\,}
      &\eqnp{20}& \ln|x+a| \\ \myl
      \D \intx{\frac1{x+3}\,}
      &\eqnp{21}& \ln|x+3| \\ \myl
      \D \intx{\frac1{x+5}\,}
      &\eqnp{22}& \ln|x+5| \\ \myl
      \D 2\intx{\frac{1}{x+3}} + 4\intx{\frac{1}{x+5}}
      &\eqnp{23} & 2\ln|x+3| + 4\ln|x+5| \\
    \end{array}
  $}
$

As três primeiras são justificadas por fórmulas pras quais nós já
demos nomes, e a última é consequência das duas do meio.




}}






\newpage

%   ___                  _                ____      ___ ____  ____   ____ 
%  / _ \ _   _  ___  ___| |_ __ _  ___   |___ \ _  |_ _|  _ \/ ___| / ___|
% | | | | | | |/ _ \/ __| __/ _` |/ _ \    __) (_)  | || |_) \___ \| |    
% | |_| | |_| |  __/\__ \ || (_| | (_) |  / __/ _   | ||  __/ ___) | |___ 
%  \__\_\\__,_|\___||___/\__\__,_|\___/  |_____(_) |___|_|   |____/ \____|
%                                                                         
% «int-pots-sin-cos»  (to ".int-pots-sin-cos")
% (c2m221p1p 5 "int-pots-sin-cos")
% (c2m221p1a   "int-pots-sin-cos")
% (c2m201ipscp 2 "exemplo-1")
% (c2m201ipsc    "exemplo-1")

{\bf Questão 2}

\scalebox{0.55}{\def\colwidth{9cm}\firstcol{

\hbox{\T(Total: 3.0 pts)}

(Sobre ``integrais de potências

de senos e cossenos'')

\msk

A demonstração abaixo é do gabarito

de uma prova antiga minha:
\msk


\def\S{\sen x}
\def\C{\cos x}

$\begin{array}[t]{l}
  \D \intx{(\S)^5 (\C)^3} \\
  \D = \;\; \intx{(\S)^5 (\C)^2 (\C)} \\
  \D = \;\; \intx{(\S)^5 (1-\S^2) (\C)} \\
  \D \eqnp{3} \;\; \ints{s^5 (1-s^2)} \\
  \D = \;\; \ints{s^5 - s^7} \\
  \D = \;\; \frac{s^6}{6} - \frac{s^8}{8} \\
  \D = \;\; \frac{(\S)^6}{6} - \frac{(\S)^8}{8} \\
  \end{array}
$

\msk

Esta caixinha aqui
%
$$\bsm{s = \sen x \\
          \frac{ds}{dx} = \cos x \\
          \sen x = s \\
          (\cos x)^2 = 1 - s^2 \\
          \cos x \, dx = ds
    }
$$

``Explica'' a mudança de variáveis.

}\anothercol{

  a) \B (3.0 pts) Mostre como justificar o \qeqnp{3}

  usando o [MV2].

}}


% (setq eepitch-preprocess-regexp "^")
% (setq eepitch-preprocess-regexp "^%[TL] ?")
%
%L ee_dofile "~/LUA/Repl1.lua"
%L r = EdrxRepl.new()
%L -- r:repl()
\pu
%T  (eepitch-shell)
%T  (eepitch-kill)
%T  (eepitch-shell)
%T lualatex -record 2022-1-C2-P1.tex

\newpage

% «int-pots-sin-cos-gab»  (to ".int-pots-sin-cos-gab")
% (c2m221p1p 6 "int-pots-sin-cos-gab")
% (c2m221p1a   "int-pots-sin-cos-gab")

{\bf Questão 2: gabarito}

%L var "s"
%L defsubst("S2a", "S2a", [[
%L     if isvar(u)  then return s end
%L   ]], [[
%L     u := s \\
%L   ]])
%L defsubst("S2c", "S2c", [[
%L     if isapp(g)  then return sen(Sarg()) end
%L     if isapp(gp) then return cos(Sarg()) end
%L     if isvar(u)  then return s end
%L   ]], [[
%L     g(x) := \sen(x) \\
%L     g'(x) := \cos(x) \\
%L     u := s \\
%L   ]])
%L defsubst("S2d", "S2d", [[
%L     -- if isapp(fp) then return Paren(minus(pot(Sarg(),5),pot(Sarg(),3))) end
%L     if isapp(fp) then return Paren(mul(pot(Sarg(),5),Paren(minus(1,pot(Sarg(),2))))) end
%L     if isapp(g)  then return sen(Sarg()) end
%L     if isapp(gp) then return cos(Sarg()) end
%L     if isvar(u)  then return s end
%L   ]], [[
%L     f'(x) := x^5(1-x^2) \\
%L     g(x) := \sen(x) \\
%L     g'(x) := \cos(x) \\
%L     u := s \\
%L   ]])
%L ang = Ang.from([[
%L   \def\S{\sen x}
%L   \def\C{\cos x}
%L   \begin{array}{rcl}
%L     && \left( \D \intx{(\S)^5 (1-\S^2) (\C)}
%L               \eqnp{3} \ints{s^5 (1-s^2)} \right) \\
%L     <MV2_> &=& <Paren(MV2)>  \\
%L     <MV2_><S2a.bmat> &=& <Paren(S2a(MV2))>  \\
%L     <MV2_><S2c.bmat> &=& <Paren(S2c(MV2))>  \\
%L     <MV2_><S2d.bmat> &=& <Paren(S2d(MV2))>  \\
%L   \end{array}
%L ]])
%L ang:sa("gab 2"):output()
\pu
$$\scalebox{0.6}{$\ga{gab 2}$}$$


\newpage

%  _____                   _                 _       __     
% | ____|___  ___ __ _  __| | __ _ ___    __| | ___ / _|___ 
% |  _| / __|/ __/ _` |/ _` |/ _` / __|  / _` |/ _ \ |_/ __|
% | |___\__ \ (_| (_| | (_| | (_| \__ \ | (_| |  __/  _\__ \
% |_____|___/\___\__,_|\__,_|\__,_|___/  \__,_|\___|_| |___/
%                                                           
% «escadas-defs»  (to ".escadas-defs")
% (c2m221p1p 7 "escadas-defs")
% (c2m221p1a   "escadas-defs")
%
%L hx = function (x, y) return format(" (%s,%s)c--(%s,%s)o", x-1,y, x,y) end
%L hxs = function (ys)
%L     local str = ""
%L     for x,y in ipairs(ys) do str = str .. hx(x, y) end
%L     return str
%L   end
%L mtintegralspec = function (specf, xmax, y0)
%L     local pws = PwSpec.from(specf)
%L     local f = pws:fun()
%L     local ys = {[0] = y0}
%L     for x=1,xmax do
%L       PP("FOO", x, f(x-0.5), ys)
%L       ys[x] = ys[x - 1] + f(x - 0.5)
%L     end
%L     local strx = function (x) return tostring(v(x, ys[x])) end
%L     local specF = mapconcat(strx, seq(0, xmax), "--")
%L     return specF
%L   end
%L
%L ysf   = {1, 2, 1, 0, -1, -2, -1, 0, 1, 2, 1, 0}
%L specf = hxs(ysf)
%L ysg   = {0, 1, 2, 3, -2, -1, 0, -1, -2, 3, 2, 1, 0}
%L specg = hxs(ysg)
%L specF = mtintegralspec(specf, #ysf,  0)
%L specG = mtintegralspec(specf, #ysf, -3)
%L specI = mtintegralspec(specg, #ysg,  0)
%L pwsf  = PwSpec.from(specf)
%L pwsg  = PwSpec.from(specg)
%L pwsF  = PwSpec.from(specF)
%L pwsG  = PwSpec.from(specG)
%L pwsI  = PwSpec.from(specI)
%L pf    = pwsf:topict():setbounds(v(0,-2), v(#ysf,2)):pgat("pgatc")
%L pg    = pwsg:topict():setbounds(v(0,-2), v(#ysg,3)):pgat("pgatc")
%L pF    = pwsF:topict():setbounds(v(0,-0), v(#ysf,4)):pgat("pgatc")
%L pG    = pwsG:topict():setbounds(v(0,-3), v(#ysf,1)):pgat("pgatc")
%L pI    = pwsI:topict():setbounds(v(0,0),  v(#ysg,6)):pgat("pgatc")
%L pf:sa("Fig f"):output()
%L pg:sa("Fig g"):output()
%L pF:sa("Fig F"):output()
%L pG:sa("Fig G"):output()
%L pI:sa("Fig I"):output()
%L
%L PictList{}:setbounds(v(0,-4),v(13,4)):pgat("pgatc"):sa("respgrid"):output()
%L
%L mtintegralspec2 = function (x0, y0, Dys, dot0, dot1)
%L     local mkxy = function (x,y) return format("(%d,%d)", x, y) end
%L     local xys = { mkxy(x0,y0) .. (dot0 or "") }
%L     local x,y = x0,y0
%L     for i,Dy in ipairs(Dys) do
%L       x = x + 1
%L       y = y + Dy
%L       table.insert(xys, mkxy(x,y))
%L     end
%L     xys[#xys] = xys[#xys] .. (dot1 or "")
%L     return table.concat(xys, "--")
%L   end
%L
%L -- = mtintegralspec2(10, 20, {1, 2, -3, -3}, "a", "b")
%L ysf   = {1, 2, 1, 0, -1, -2, -1, 0, 1, 2, 1, 0}
%L ysf_  = {1, 2, 1, 0, -1, -2, -1}
%L ysg   = {0, 1, 2, 3, -2, -1,  0, -1, -2, 3, 2, 1, 0}
%L ysg_  =                         {-1, -2, 3, 2, 1}
%L specH = mtintegralspec2(0, -4, ysf_, "", "o\n") ..
%L         mtintegralspec2(7,  1, ysg_, "o", "")
%L specM = mtintegralspec2(0, -4, ysf_, "", "o\n") ..
%L         mtintegralspec2(7,  2, ysg_, "o", "")
%L -- = specH
%L -- = specM
%L pwsH  = PwSpec.from(specH)
%L pwsM  = PwSpec.from(specM)
%L pH    = pwsH:topict():setbounds(v(0,-4), v(12,4)):pgat("pgatc")
%L pM    = pwsM:topict():setbounds(v(0,-4),  v(12,5)):pgat("pgatc")
%L pH:sa("Fig H"):output()
%L pM:sa("Fig M"):output()
\pu


\newpage

%  _____                   _           
% | ____|___  ___ __ _  __| | __ _ ___ 
% |  _| / __|/ __/ _` |/ _` |/ _` / __|
% | |___\__ \ (_| (_| | (_| | (_| \__ \
% |_____|___/\___\__,_|\__,_|\__,_|___/
%                                      
% «escadas»  (to ".escadas")
% (c2m221p1p 7 "escadas")
% (c2m221p1a    "escadas")


{\bf Questão 3}

% (c2m221mt1p 2 "defs-figuras")
% (c2m221mt1a   "defs-figuras")



\scalebox{0.45}{\def\colwidth{9cm}\firstcol{

\hbox{\T(Total: 3.0 pts)}

(Sobre funções escada)

Sejam:

\unitlength=10pt



$f(x) \;=\; \ga{Fig f}$ ,

\msk

$g(x) \;=\; \ga{Fig g}$ ,

$$F(x) = \D\Intt{0}{x}{f(t)},$$

$$G(x) = \D\Intt{2}{x}{f(t)},$$

e seja $H(x)$ a função contínua cujo domínio

é o conjunto $D=[0,7)∪(7,12]$ e que obedece

estas três condições:

1) $H(x) = \Intt{3}{x}{f(t)}$ se $x∈[0,7)$,

2) $H(10)=1$, e

3) ``para todo $x∈(7,12]$ temos $H'(x) = g(x)$'',

onde eu pus a expressão acima entre aspas

porque ela é um abuso de linguagem comum

quando a gente fala de funções escada...

a tradução disto pra uma linguagem mais

formal é: ``$H'(x) = g(x)$ é verdade em

todos os pontos $x∈(7,12]$ nos quais

a $g(x)$ é contínua.



}\anothercol{

a) \B (0.5 pts) Faça o gráfico da $F(x)$.

b) \B (0.5 pts) Faça o gráfico da $G(x)$.

c) \B (1.0 pts) Faça o gráfico da $H(x)$.

\msk

d) \B (1.0 pts) Digamos que a função $M(x)$ é

definida exatamente da mesma forma que a

$H(x)$, mas mudando o ``$H(10)=1$'' por

``$M(10)=2$''. Faça o gráfico da $M(x)$.


}}




\newpage

%  _____                   _                         _     
% | ____|___  ___ __ _  __| | __ _ ___    __ _  __ _| |__  
% |  _| / __|/ __/ _` |/ _` |/ _` / __|  / _` |/ _` | '_ \ 
% | |___\__ \ (_| (_| | (_| | (_| \__ \ | (_| | (_| | |_) |
% |_____|___/\___\__,_|\__,_|\__,_|___/  \__, |\__,_|_.__/ 
%                                        |___/             
% «escadas-gab»  (to ".escadas-gab")
% (c2m221p1p 8 "escadas-gab")
% (c2m221p1a   "escadas-gab")

{\bf Questão 3: gabarito}

\unitlength=5pt

\def\defH{
 \begin{array}{r}
 H(x) = \Intt{3}{x}{f(t)}  \;\;\text{se}\;\; x∈[0,7), \\
 \text{``$H'(x) = g(x)$''} \;\;\text{se}\;\; x∈(7,12], \\
 H(10)=1
 \end{array}
 }
\def\defM{
 \begin{array}{r}
 M(x) = \Intt{3}{x}{f(t)}  \;\;\text{se}\;\; x∈[0,7), \\
 \text{``$M'(x) = g(x)$''} \;\;\text{se}\;\; x∈(7,12], \\
 M(10)=2
 \end{array}
 }

$\scalebox{0.8}{$
 \begin{array}{rcl}
                     f(x) &=& \ga{Fig f} \\
 F(x) = \Intt{0}{x}{f(t)} &=& \ga{Fig F} \\
 G(x) = \Intt{2}{x}{f(t)} &=& \ga{Fig G} \\
                     g(x) &=& \ga{Fig g} \\
        \Intt{0}{x}{g(t)} &=& \ga{Fig I} \\
                    \defH &⇒& \ga{Fig H} \\
                    \defM &⇒& \ga{Fig M} \\
 \end{array}
 $}
$



\newpage

% «grids»  (to ".grids")
% (c2m221p1p 9 "grids")
% (c2m221p1a   "grids")

\unitlength=6pt

\def\rg{\ga{respgrid}}

$\begin{array}{ccc}
 \rg & \rg & \rg \\
 \rg & \rg & \rg \\
 \rg & \rg & \rg \\
 \rg & \rg & \rg \\
 \end{array}
$



\newpage

% «formulas»  (to ".formulas")
% (c2m221p1p 10 "formulas")
% (c2m221p1a    "formulas")

% (c2m221ftp 2 "RC")
% (c2m221fta   "RC")
% (c2m221ftp 5 "MVs")
% (c2m221fta   "MVs")
% (c2m221fda   "MVs")

$\scalebox{0.55}{$
   \begin{array}{l}
   \ga{[RC]} \;=\;  \ga{(RC)} \\ \\[-5pt]
   \ga{[MV1]} \;=\; \ga{(MV1)} \\ \\[-5pt]
   \ga{[MV2]} \;=\; \ga{(MV2)} \\ \\[-5pt]
   \ga{[MV3]} \;=\; \ga{(MV3)}
    \quad
     \ga{[MV4]} \;=\; \ga{(MV4)} \\ \\[-5pt]
   \ga{[MVI3]} \;=\; \ga{(MVI3)}
    \quad
     \ga{[MVI4]} \;=\; \ga{(MVI4)} \\ \\[-5pt]
   \end{array}
 $}
$



\newpage

% «links»  (to ".links")
% (c2m221p1p 6 "links")
% (c2m221p1a    "links")

Links pra estudar esta matéria:

\ssk

{\scriptsize

% (find-angg ".emacs" "c2q192")
% (find-angg ".emacs" "c2q192" "funções racionais")
% (c2q192 37 "20190911 peq aula 8: Como integrar funções racionais, parte 1")
% (c2q192 40 "20190912 gde aula 8: Como integrar funções racionais, parte 1")
%    http://angg.twu.net/2019.2-C2/2019.2-C2.pdf#page=37
\url{http://angg.twu.net/2019.2-C2/2019.2-C2.pdf\#page=37}

% (c2m212fpa "title")
% (c2m212fpa "title" "Aula nn: frações parciais")
% (c2m212fpp 3 "together")
% (c2m212fpa   "together")
% (find-pdft-page   "~/LATEX/2021-2-C2-fracoes-parciais.pdf")
\url{http://angg.twu.net/LATEX/2021-2-C2-fracoes-parciais.pdf}

% (find-books "__analysis/__analysis.el" "miranda")
% (find-dmirandacalcpage 240 "8.1 Frações Parciais")
% (find-dmirandacalcpage 252 "8.2 Decomposição em Frações Parciais")
%    http://hostel.ufabc.edu.br/~daniel.miranda/calculo/calculo.pdf#page=240
\url{http://hostel.ufabc.edu.br/~daniel.miranda/calculo/calculo.pdf\#page=240}

}






%\printbibliography

\GenericWarning{Success:}{Success!!!}  % Used by `M-x cv'

\end{document}

%  ____  _             _         
% |  _ \(_)_   ___   _(_)_______ 
% | | | | \ \ / / | | | |_  / _ \
% | |_| | |\ V /| |_| | |/ /  __/
% |____// | \_/  \__,_|_/___\___|
%     |__/                       
%
% «djvuize»  (to ".djvuize")
% (find-LATEXgrep "grep --color -nH --null -e djvuize 2020-1*.tex")

 (eepitch-shell)
 (eepitch-kill)
 (eepitch-shell)
# (find-fline "~/2022.1-C2/")
# (find-fline "~/LATEX/2022-1-C2/")
# (find-fline "~/bin/djvuize")

cd /tmp/
for i in *.jpg; do echo f $(basename $i .jpg); done

f () { rm -v $1.pdf;  textcleaner -f 50 -o  5 $1.jpg $1.png; djvuize $1.pdf; xpdf $1.pdf }
f () { rm -v $1.pdf;  textcleaner -f 50 -o 10 $1.jpg $1.png; djvuize $1.pdf; xpdf $1.pdf }
f () { rm -v $1.pdf;  textcleaner -f 50 -o 20 $1.jpg $1.png; djvuize $1.pdf; xpdf $1.pdf }

f () { rm -fv $1.png $1.pdf; djvuize $1.pdf }
f () { rm -fv $1.png $1.pdf; djvuize WHITEBOARDOPTS="-m 1.0 -f 15" $1.pdf; xpdf $1.pdf }
f () { rm -fv $1.png $1.pdf; djvuize WHITEBOARDOPTS="-m 1.0 -f 30" $1.pdf; xpdf $1.pdf }
f () { rm -fv $1.png $1.pdf; djvuize WHITEBOARDOPTS="-m 1.0 -f 45" $1.pdf; xpdf $1.pdf }
f () { rm -fv $1.png $1.pdf; djvuize WHITEBOARDOPTS="-m 0.5" $1.pdf; xpdf $1.pdf }
f () { rm -fv $1.png $1.pdf; djvuize WHITEBOARDOPTS="-m 0.25" $1.pdf; xpdf $1.pdf }
f () { cp -fv $1.png $1.pdf       ~/2022.1-C2/
       cp -fv        $1.pdf ~/LATEX/2022-1-C2/
       cat <<%%%
% (find-latexscan-links "C2" "$1")
%%%
}

f 20201213_area_em_funcao_de_theta
f 20201213_area_em_funcao_de_x
f 20201213_area_fatias_pizza



%  __  __       _        
% |  \/  | __ _| | _____ 
% | |\/| |/ _` | |/ / _ \
% | |  | | (_| |   <  __/
% |_|  |_|\__,_|_|\_\___|
%                        
% <make>

 (eepitch-shell)
 (eepitch-kill)
 (eepitch-shell)
# (find-LATEXfile "2019planar-has-1.mk")
make -f 2019.mk STEM=2022-1-C2-P1 veryclean
make -f 2019.mk STEM=2022-1-C2-P1 pdf

% Local Variables:
% coding: utf-8-unix
% ee-tla: "c2p1"
% ee-tla: "c2m221p1"
% End:
