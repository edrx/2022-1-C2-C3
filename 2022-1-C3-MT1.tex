% (find-LATEX "2022-1-C3-MT1.tex")
% (defun c () (interactive) (find-LATEXsh "lualatex -record 2022-1-C3-MT1.tex" :end))
% (defun C () (interactive) (find-LATEXsh "lualatex 2022-1-C3-MT1.tex" "Success!!!"))
% (defun D () (interactive) (find-pdf-page      "~/LATEX/2022-1-C3-MT1.pdf"))
% (defun d () (interactive) (find-pdftools-page "~/LATEX/2022-1-C3-MT1.pdf"))
% (defun e () (interactive) (find-LATEX "2022-1-C3-MT1.tex"))
% (defun o () (interactive) (find-LATEX "2022-1-C3-notacao-de-fisicos.tex"))
% (defun u () (interactive) (find-latex-upload-links "2022-1-C3-MT1"))
% (defun v () (interactive) (find-2a '(e) '(d)))
% (defun d0 () (interactive) (find-ebuffer "2022-1-C3-MT1.pdf"))
% (defun cv () (interactive) (C) (ee-kill-this-buffer) (v) (g))
%          (code-eec-LATEX "2022-1-C3-MT1")
% (find-pdf-page   "~/LATEX/2022-1-C3-MT1.pdf")
% (find-sh0 "cp -v  ~/LATEX/2022-1-C3-MT1.pdf /tmp/")
% (find-sh0 "cp -v  ~/LATEX/2022-1-C3-MT1.pdf /tmp/pen/")
%     (find-xournalpp "/tmp/2022-1-C3-MT1.pdf")
%   file:///home/edrx/LATEX/2022-1-C3-MT1.pdf
%               file:///tmp/2022-1-C3-MT1.pdf
%           file:///tmp/pen/2022-1-C3-MT1.pdf
% http://angg.twu.net/LATEX/2022-1-C3-MT1.pdf
% (find-LATEX "2019.mk")
% (find-sh0 "cd ~/LUA/; cp -v Pict2e1.lua Pict2e1-1.lua Piecewise1.lua ~/LATEX/")
% (find-sh0 "cd ~/LUA/; cp -v Pict2e1.lua Pict2e1-1.lua Pict3D1.lua ~/LATEX/")
% (find-CN-aula-links "2022-1-C3-MT1" "3" "c3m221mt1" "c3mt1")

% «.defs»	(to "defs")
% «.title»	(to "title")
%
% «.djvuize»	(to "djvuize")



% <videos>
% Video (not yet):
% (find-ssr-links     "c3m221mt1" "2022-1-C3-MT1")
% (code-eevvideo      "c3m221mt1" "2022-1-C3-MT1")
% (code-eevlinksvideo "c3m221mt1" "2022-1-C3-MT1")
% (find-c3m221mt1video "0:00")

\documentclass[oneside,12pt]{article}
\usepackage[colorlinks,citecolor=DarkRed,urlcolor=DarkRed]{hyperref} % (find-es "tex" "hyperref")
\usepackage{amsmath}
\usepackage{amsfonts}
\usepackage{amssymb}
\usepackage{pict2e}
\usepackage[x11names,svgnames]{xcolor} % (find-es "tex" "xcolor")
\usepackage{colorweb}                  % (find-es "tex" "colorweb")
%\usepackage{tikz}
%
% (find-dn6 "preamble6.lua" "preamble0")
%\usepackage{proof}   % For derivation trees ("%:" lines)
%\input diagxy        % For 2D diagrams ("%D" lines)
%\xyoption{curve}     % For the ".curve=" feature in 2D diagrams
%
\usepackage{edrx21}               % (find-LATEX "edrx21.sty")
\input edrxaccents.tex            % (find-LATEX "edrxaccents.tex")
\input edrx21chars.tex            % (find-LATEX "edrx21chars.tex")
\input edrxheadfoot.tex           % (find-LATEX "edrxheadfoot.tex")
\input edrxgac2.tex               % (find-LATEX "edrxgac2.tex")
%\usepackage{emaxima}              % (find-LATEX "emaxima.sty")
%
%\usepackage[backend=biber,
%   style=alphabetic]{biblatex}            % (find-es "tex" "biber")
%\addbibresource{catsem-slides.bib}        % (find-LATEX "catsem-slides.bib")
%
% (find-es "tex" "geometry")
\usepackage[a6paper, landscape,
            top=1.5cm, bottom=.25cm, left=1cm, right=1cm, includefoot
           ]{geometry}
%
\begin{document}

\catcode`\^^J=10
\directlua{dofile "dednat6load.lua"}  % (find-LATEX "dednat6load.lua")
%L dofile "Pict2e1.lua"              -- (find-LATEX "Pict2e1.lua")
%L dofile "Pict2e1-1.lua"            -- (find-LATEX "Pict2e1-1.lua")
%L dofile "Piecewise1.lua"           -- (find-LATEX "Piecewise1.lua")
%L Pict2e.__index.suffix = "%"
%L Pict2e.__index.suffix = "%"
\pu
\def\pictgridstyle{\color{GrayPale}\linethickness{0.3pt}}
\def\pictaxesstyle{\linethickness{0.5pt}}
\def\pictnaxesstyle{\color{GrayPale}\linethickness{0.5pt}}
\celllower=3pt


% «defs»  (to ".defs")
% (find-LATEX "edrx21defs.tex" "colors")
% (find-LATEX "edrx21.sty")

\def\u#1{\par{\footnotesize \url{#1}}}

\def\drafturl{http://angg.twu.net/LATEX/2022-1-C3.pdf}
\def\drafturl{http://angg.twu.net/2022.1-C3.html}
\def\draftfooter{\tiny \href{\drafturl}{\jobname{}} \ColorBrown{\shorttoday{} \hours}}



%  _____ _ _   _                               
% |_   _(_) |_| | ___   _ __   __ _  __ _  ___ 
%   | | | | __| |/ _ \ | '_ \ / _` |/ _` |/ _ \
%   | | | | |_| |  __/ | |_) | (_| | (_| |  __/
%   |_| |_|\__|_|\___| | .__/ \__,_|\__, |\___|
%                      |_|          |___/      
%
% «title»  (to ".title")
% (c3m221mt1p 1 "title")
% (c3m221mt1a   "title")

\thispagestyle{empty}

\begin{center}

\vspace*{1.2cm}

{\bf \Large Cálculo 3 - 2022.1}

\bsk

Mini-teste 1

\bsk

Eduardo Ochs - RCN/PURO/UFF

\url{http://angg.twu.net/2022.1-C3.html}

\end{center}

\newpage

% (c3m221nfp 21 "low-poly")
% (c3m221nfa    "low-poly")

%L -- (find-angg "LUA/Pict2e1-1.lua" "Numerozinhos-test3")
%L Pict2e.bounds = PictBounds.new(v(-1,-1), v(11,11))
%L pyr = Numerozinhos.from(0, 0, [[
%L     0 0 0 0 0 0 0 0 0 0 0 0
%L     0 0 0 0 0 0 0 0 0 0 0 0
%L     0 0 0 0 0 0 0 1 0 0 0 0
%L     0 0 0 0 0 0 1 2 0 0 0 0
%L     0 0 0 0 0 1 2 3 1 0 0 0
%L     0 0 0 0 1 2 3 4 2 0 0 0
%L     0 0 0 1 2 3 4 5 3 1 0 0
%L     0 0 1 2 3 4 5 6 4 2 0 0
%L     0 0 0 0 1 2 3 4 2 0 0 0
%L     0 0 0 0 0 0 1 2 0 0 0 0
%L     0 0 0 0 0 0 0 0 0 0 0 0
%L     0 0 0 0 0 0 0 0 0 0 0 0 ]])
%L pyr_spec  = "(1,4)--(7,10)--(10,4)--(7,1)--(1,4) (1,4)--(10,4) (7,1)--(7,10)"
%L pyr:topict(pyr_spec ):sa("piramide com linhas"):output()
\pu



\scalebox{1.0}{\def\colwidth{7cm}\firstcol{

Seja $z=z(x,y)=F(x,y)$ a função

que dá a superfície desta pirâmide:

\msk

\unitlength=10pt

$\scalebox{0.75}{$\ga{piramide com linhas}$}
$

Lembre que:

$\frac{∂f}{∂𝐛v}(𝐛p) = \lim_{t→0} \frac{ f(𝐛p + t·𝐛v) - f(𝐛p) }{t} $

Digamos que $𝐛p = (5,3)$ e $𝐛v = \VEC{1,1}$.

Calcule $\frac{ f(𝐛p + t·𝐛v) - f(𝐛p) }{t}$ para $t=4$, $t=3$,

$t=2$, $t=1$, $t=\frac12$, $t=-1$, $t=-\frac12$.

}\anothercol{

\def\FA#1{\frac{ f(𝐛p + #1·𝐛v) - f(𝐛p) }{#1}}
\def\FB#1{\frac{ f((5,3) + #1·\VEC{1,1}) - f((5,3)) }{#1}}
\def\FC#1#2#3{\frac{ f((#2,#3)) - f((5,3)) }{#1}}



\hspace*{-1cm}
%
\scalebox{0.6}{$
\begin{tabular}[t]{l}
 {\bf Gabarito:} \\
 $\begin{array}{rcl}
  \FA{4} &=& \FB{4} \\ 
         &=& \FC 4 9 7 \\
         &=& \frac{0-2}{4} \\
         &=& -1/2 \\
  \FA{3} %&=& \FB{4} \\ 
         &=& \FC 3 8 6 \\
         &=& \frac{2-2}{3} \\
         &=& 0 \\
  \FA{2} %&=& \FB{4} \\ 
         &=& \FC 2 7 5 \\
         &=& \frac{5-2}{2} \\
         &=& 1 \\
  \FA{1} %&=& \FB{4} \\ 
         &=& \FC 1 6 4 \\
         &=& \frac{5-2}{1} \\
         &=& 3 \\
  \FA{0.5} %&=& \FB{4} \\ 
         &=& \FC {0.5} {5.5} {3.5} \\
         &=& \frac{3.5-2}{0.5} \\
         &=& 3 \\
  \FA{(-0.5)} %&=& \FB{4} \\ 
         &=& \FC {-0.5} {4.5} {2.5} \\
         &=& \frac{0.5-2}{0.5} \\
         &=& 3 \\
  \FA{(-1)} %&=& \FB{4} \\ 
         &=& \FC {-1} {4} {1} \\
         &=& \frac{0-2}{-1} \\
         &=& 2 \\
  \end{array}
 $
\end{tabular}
$}


}}

\newpage

Obs:

Falta eu escrever como a gente descobre que

$f(5.5, 3.5) = 3.5$ e que $f(4.5, 2.5) = 0.5$...




%\printbibliography

\GenericWarning{Success:}{Success!!!}  % Used by `M-x cv'

\end{document}

%  ____  _             _         
% |  _ \(_)_   ___   _(_)_______ 
% | | | | \ \ / / | | | |_  / _ \
% | |_| | |\ V /| |_| | |/ /  __/
% |____// | \_/  \__,_|_/___\___|
%     |__/                       
%
% «djvuize»  (to ".djvuize")
% (find-LATEXgrep "grep --color -nH --null -e djvuize 2020-1*.tex")

 (eepitch-shell)
 (eepitch-kill)
 (eepitch-shell)
# (find-fline "~/2022.1-C3/")
# (find-fline "~/LATEX/2022-1-C3/")
# (find-fline "~/bin/djvuize")

cd /tmp/
for i in *.jpg; do echo f $(basename $i .jpg); done

f () { rm -v $1.pdf;  textcleaner -f 50 -o  5 $1.jpg $1.png; djvuize $1.pdf; xpdf $1.pdf }
f () { rm -v $1.pdf;  textcleaner -f 50 -o 10 $1.jpg $1.png; djvuize $1.pdf; xpdf $1.pdf }
f () { rm -v $1.pdf;  textcleaner -f 50 -o 20 $1.jpg $1.png; djvuize $1.pdf; xpdf $1.pdf }

f () { rm -fv $1.png $1.pdf; djvuize $1.pdf }
f () { rm -fv $1.png $1.pdf; djvuize WHITEBOARDOPTS="-m 1.0 -f 15" $1.pdf; xpdf $1.pdf }
f () { rm -fv $1.png $1.pdf; djvuize WHITEBOARDOPTS="-m 1.0 -f 30" $1.pdf; xpdf $1.pdf }
f () { rm -fv $1.png $1.pdf; djvuize WHITEBOARDOPTS="-m 1.0 -f 45" $1.pdf; xpdf $1.pdf }
f () { rm -fv $1.png $1.pdf; djvuize WHITEBOARDOPTS="-m 0.5" $1.pdf; xpdf $1.pdf }
f () { rm -fv $1.png $1.pdf; djvuize WHITEBOARDOPTS="-m 0.25" $1.pdf; xpdf $1.pdf }
f () { cp -fv $1.png $1.pdf       ~/2022.1-C3/
       cp -fv        $1.pdf ~/LATEX/2022-1-C3/
       cat <<%%%
% (find-latexscan-links "C3" "$1")
%%%
}

f 20201213_area_em_funcao_de_theta
f 20201213_area_em_funcao_de_x
f 20201213_area_fatias_pizza



%  __  __       _        
% |  \/  | __ _| | _____ 
% | |\/| |/ _` | |/ / _ \
% | |  | | (_| |   <  __/
% |_|  |_|\__,_|_|\_\___|
%                        
% <make>

 (eepitch-shell)
 (eepitch-kill)
 (eepitch-shell)
# (find-LATEXfile "2019planar-has-1.mk")
make -f 2019.mk STEM=2022-1-C3-MT1 veryclean
make -f 2019.mk STEM=2022-1-C3-MT1 pdf

% Local Variables:
% coding: utf-8-unix
% ee-tla: "c3mt1"
% ee-tla: "c3m221mt1"
% End:
