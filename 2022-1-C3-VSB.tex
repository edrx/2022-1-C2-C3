% (find-LATEX "2022-1-C3-VSB.tex")
% (defun c () (interactive) (find-LATEXsh "lualatex -record 2022-1-C3-VSB.tex" :end))
% (defun C () (interactive) (find-LATEXsh "lualatex 2022-1-C3-VSB.tex" "Success!!!"))
% (defun D () (interactive) (find-pdf-page      "~/LATEX/2022-1-C3-VSB.pdf"))
% (defun d () (interactive) (find-pdftools-page "~/LATEX/2022-1-C3-VSB.pdf"))
% (defun e () (interactive) (find-LATEX "2022-1-C3-VSB.tex"))
% (defun o () (interactive) (find-LATEX "2022-1-C3-VSB.tex"))
% (defun u () (interactive) (find-latex-upload-links "2022-1-C3-VSB"))
% (defun v () (interactive) (find-2a '(e) '(d)))
% (defun d0 () (interactive) (find-ebuffer "2022-1-C3-VSB.pdf"))
% (defun cv () (interactive) (C) (ee-kill-this-buffer) (v) (g))
%          (code-eec-LATEX "2022-1-C3-VSB")
% (find-pdf-page   "~/LATEX/2022-1-C3-VSB.pdf")
% (find-sh0 "cp -v  ~/LATEX/2022-1-C3-VSB.pdf /tmp/")
% (find-sh0 "cp -v  ~/LATEX/2022-1-C3-VSB.pdf /tmp/pen/")
%     (find-xournalpp "/tmp/2022-1-C3-VSB.pdf")
%   file:///home/edrx/LATEX/2022-1-C3-VSB.pdf
%               file:///tmp/2022-1-C3-VSB.pdf
%           file:///tmp/pen/2022-1-C3-VSB.pdf
% http://angg.twu.net/LATEX/2022-1-C3-VSB.pdf
% (find-LATEX "2019.mk")
% (find-sh0 "cd ~/LUA/; cp -v Pict2e1.lua Pict2e1-1.lua Piecewise1.lua ~/LATEX/")
% (find-sh0 "cd ~/LUA/; cp -v Pict2e1.lua Pict2e1-1.lua Pict3D1.lua ~/LATEX/")
% (find-sh0 "cd ~/LUA/; cp -v C2Subst1.lua C2Formulas1.lua ~/LATEX/")
% (find-CN-aula-links "2022-1-C3-VSB" "3" "c3m221vsb" "c3vs")

% «.defs»		(to "defs")
% «.defs-T-and-B»	(to "defs-T-and-B")
% «.title»		(to "title")
% «.questao-1»		(to "questao-1")
% «.questao-2»		(to "questao-2")
% «.questao-1-gab»	(to "questao-1-gab")
% «.questao-2-gab»	(to "questao-2-gab")
%
% «.djvuize»		(to "djvuize")



% <videos>
% Video (not yet):
% (find-ssr-links     "c3m221vsb" "2022-1-C3-VSB")
% (code-eevvideo      "c3m221vsb" "2022-1-C3-VSB")
% (code-eevlinksvideo "c3m221vsb" "2022-1-C3-VSB")
% (find-c3m221vsbvideo "0:00")

\documentclass[oneside,12pt]{article}
\usepackage[colorlinks,citecolor=DarkRed,urlcolor=DarkRed]{hyperref} % (find-es "tex" "hyperref")
\usepackage{amsmath}
\usepackage{amsfonts}
\usepackage{amssymb}
\usepackage{pict2e}
\usepackage[x11names,svgnames]{xcolor} % (find-es "tex" "xcolor")
\usepackage{colorweb}                  % (find-es "tex" "colorweb")
%\usepackage{tikz}
%
% (find-dn6 "preamble6.lua" "preamble0")
%\usepackage{proof}   % For derivation trees ("%:" lines)
%\input diagxy        % For 2D diagrams ("%D" lines)
%\xyoption{curve}     % For the ".curve=" feature in 2D diagrams
%
\usepackage{edrx21}               % (find-LATEX "edrx21.sty")
\input edrxaccents.tex            % (find-LATEX "edrxaccents.tex")
\input edrx21chars.tex            % (find-LATEX "edrx21chars.tex")
\input edrxheadfoot.tex           % (find-LATEX "edrxheadfoot.tex")
\input edrxgac2.tex               % (find-LATEX "edrxgac2.tex")
%\usepackage{emaxima}              % (find-LATEX "emaxima.sty")
%
%\usepackage[backend=biber,
%   style=alphabetic]{biblatex}            % (find-es "tex" "biber")
%\addbibresource{catsem-slides.bib}        % (find-LATEX "catsem-slides.bib")
%
% (find-es "tex" "geometry")
\usepackage[a6paper, landscape,
            top=1.5cm, bottom=.25cm, left=1cm, right=1cm, includefoot
           ]{geometry}
%
\begin{document}

\catcode`\^^J=10
\directlua{dofile "dednat6load.lua"}  % (find-LATEX "dednat6load.lua")
%L dofile "Piecewise1.lua"           -- (find-LATEX "Piecewise1.lua")
%L dofile "QVis1.lua"                -- (find-LATEX "QVis1.lua")
%L dofile "Pict3D1.lua"              -- (find-LATEX "Pict3D1.lua")
%L dofile "C2Formulas1.lua"          -- (find-LATEX "C2Formulas1.lua")
%L Pict2e.__index.suffix = "%"
\pu
\def\pictgridstyle{\color{GrayPale}\linethickness{0.3pt}}
\def\pictaxesstyle{\linethickness{0.5pt}}
\def\pictnaxesstyle{\color{GrayPale}\linethickness{0.5pt}}
\celllower=2.5pt

% «defs»  (to ".defs")
% (find-LATEX "edrx21defs.tex" "colors")
% (find-LATEX "edrx21.sty")

\def\u#1{\par{\footnotesize \url{#1}}}

\def\drafturl{http://angg.twu.net/LATEX/2022-1-C3.pdf}
\def\drafturl{http://angg.twu.net/2022.1-C3.html}
\def\draftfooter{\tiny \href{\drafturl}{\jobname{}} \ColorBrown{\shorttoday{} \hours}}

% «defs-T-and-B»  (to ".defs-T-and-B")
% (c3m202p1p 6 "questao-2")
% (c3m202p1a   "questao-2")
\long\def\ColorOrange#1{{\color{orange!90!black}#1}}
\def\T(Total: #1 pts){{\bf(Total: #1)}}
\def\T(Total: #1 pts){{\bf(Total: #1 pts)}}
\def\T(Total: #1 pts){\ColorRed{\bf(Total: #1 pts)}}
\def\B       (#1 pts){\ColorOrange{\bf(#1 pts)}}



%  _____ _ _   _                               
% |_   _(_) |_| | ___   _ __   __ _  __ _  ___ 
%   | | | | __| |/ _ \ | '_ \ / _` |/ _` |/ _ \
%   | | | | |_| |  __/ | |_) | (_| | (_| |  __/
%   |_| |_|\__|_|\___| | .__/ \__,_|\__, |\___|
%                      |_|          |___/      
%
% «title»  (to ".title")
% (c3m221vsbp 1 "title")
% (c3m221vsba   "title")

\thispagestyle{empty}

\begin{center}

\vspace*{1.2cm}

{\bf \Large Cálculo 3 - 2022.1}

\bsk

VS extra - 31/ago/2022

\bsk

Eduardo Ochs - RCN/PURO/UFF

\url{http://angg.twu.net/2022.1-C3.html}

\end{center}

\newpage

% «questao-1»  (to ".questao-1")
% (c3m221vsbp 2 "questao-1")
% (c3m221vsba   "questao-1")
% (c3m221vsp 2 "questao-1")
% (c3m221vsa   "questao-1")

%L Pict2e.bounds = PictBounds.new(v(0,0), v(9,10))
%L barranco_VSB = Numerozinhos.from(0, 0, 
%L     [[ 6 6 6 6 6 6 6 6 6 6
%L        6 6 6 6 6 6 6 6 6 6
%L        6 6 6 6 6 6 6 5 5 5
%L        6 6 6 6 6 6 5 4 4 4
%L        6 6 6 6 6 5 4 3 3 3
%L        6 6 6 6 5 4 3 2 2 2
%L        6 6 6 5 4 3 2 1 1 1
%L        6 6 6 4 3 2 1 0 0 0
%L        6 6 6 4 2 1 0 0 0 0
%L        6 6 6 4 2 0 0 0 0 0
%L        6 6 6 4 2 0 0 0 0 0 ]])
%L barranco_VSB_spec = [[ (2,0)--(2,4)--(5,1)--(5,0)
%L                        (2,4)--(7,9)--(9,9)
%L                        (5,1)--(7,3)--(9,3)  (7,3)--(7,9) ]]
%L barranco_VSB:topictu("16pt"                   ):sa("barranco VSB")    :output()
%L barranco_VSB:topictu("16pt", barranco_VSB_spec):sa("barranco VSB gab"):output()
%L
%L -- = p:pgat("pN"):preunitlength("11pt"):bshow("")
% (find-angg "LUA/Pict2e1-1.lua" "Numerozinhos-test1")
\pu

{\bf Questão 1}

\scalebox{0.55}{\def\colwidth{9cm}\firstcol{

\vspace*{-0.4cm}

\T(Total: 7.0 pts)

O diagrama de numerozinhos da próxima folha corresponde a uma
superfície $z=F(x,y)$ que tem 5 faces. Também é possível interpretá-lo
como uma superfície com 7 ou mais faces, mas vamos considerar que a
superfície com só 5 faces é que é a correta.

\msk

a) \B (2.0 pts) Mostre como dividir o plano em 5 polígonos que são as
projeções destas faces.

% Use uma das cópias do diagrama de numerozinhos para a sua resposta.

\msk

b) \B (1.0 pts) Chame estas faces de face NW (``noroeste''), S
(``sul''), C (``centro''), E (``leste''), e SE (``sudeste''), e chame
as equações dos planos delas de $F_{NW}(x,y)$, $F_S(x,y)$,
$F_{C}(x,y)$, $F_E(x,y)$, e $F_{SE}(x,y)$. Dê as equações destes
planos.

\msk

c) \B (1.0 pts) Sejam:
%
$$\begin{array}{rcl}
  %P_{NW} &=& \setofxyzst{z = F_{NW}(x,y)}, \\
  P_S &=& \setofxyzst{z = F_S(x,y)}, \\
  P_C &=& \setofxyzst{z = F_C(x,y)}, \\
  %P_E &=& \setofxyzst{z = F_E(x,y)}, \\
  %P_{SE} &=& \setofxyzst{z = F_{SE}(x,y)}, \\
  r &=& P_S ∩ P_C. \\
  \end{array}
$$

Dê uma parametrização para a reta $r$.


}\anothercol{

  d) \B (2.0 pts) Seja
  %
  $$A \;=\; \{0,1,\ldots,9\} × \{0,1,\ldots,10\};$$

  note que os numerozinhos do diagrama de numerozinhos estão todos
  sobre pontos de $A$. Para cada ponto $(x,y)∈A$ represente
  graficamente $(x,y)+\frac12 \vec∇F(x,y)$.

  \ssk

  Obs: quando $\vec∇F(x,y)=0$ desenhe uma bolinha preta sobre o ponto
  $(x,y)$, e quando $\vec∇F(x,y)$ não existir não desenhe nada.

  \msk

  e) \B (1.0 pts) Sejam
  %
  $$\begin{array}{rcl}
    Q(t) &=& (1,0) + t\VEC{1,1}, \\
    (x(t),y(t)) &=& Q(t), \\
    h(t) &=& F(x(t),y(t)). \\ &
    \end{array}
  $$

  Faça o gráfico da função $h(t)$. Considere que o domínio dela é o
  intervalo $[0,8]$.


}}


% (find-angg "LUA/Pict2e1-1.lua" "Numerozinhos-test1")

\unitlength=15pt

\def\grd{\scalebox{0.6}{$\ga{barranco VSB}$}}

$\begin{array}{ccccc}
 \grd && \grd && \grd \\ \\
 \grd && \grd && \grd \\
 \end{array}
$

\newpage

% «questao-2»  (to ".questao-2")
% (c3m221vsbp 4 "questao-2")
% (c3m221vsba   "questao-2")

{\bf Questão 2}

\T(Total: 3.0 pts)

\msk

Seja $H(x,y) = \sqrt{x+ay}$.

Dê a aproximação de Taylor de ordem 2

para $H(x,y)$ em torno do ponto $(1,1)$.


% (setq eepitch-preprocess-regexp "^")
% (setq eepitch-preprocess-regexp "^%T ")
%
%T  (eepitch-maxima)
%T  (eepitch-kill)
%T  (eepitch-maxima)
%T H   : sqrt(x + a*y);
%T Hx  : diff(H,  x);
%T Hy  : diff(H,  y);
%T Hxx : diff(Hx, x);
%T Hxy : diff(Hx, y);
%T Hyy : diff(Hy, y);
%T H1s : [H, Hx, Hy];
%T H2s : [Hxx, Hxy, Hyy];
%T subst([x=1,y=1], H1s);
%T subst([x=1,y=1], H2s);
%T rat(Hyy);

% (c3m221p2p 5 "questao-3-gab")
% (c3m221p2a   "questao-3-gab")


\newpage

% «questao-1-gab»  (to ".questao-1-gab")
% (c3m221vsbp 5 "questao-1-gab")
% (c3m221vsba   "questao-1-gab")

%L spec  = [[ (0,6)--(1,6)--(2.5,3)--(6,3)--(8,5) ]]
%L pws   = PwSpec.from(spec)
%L p     = pws:topict():setbounds(v(0,0), v(8,6)):pgat("pgatc")
%L p:sa("gab 1e"):output()
\pu

{\bf Questão 1: gabarito}


\scalebox{0.6}{\def\colwidth{9cm}\firstcol{

\bsk

a)
$\scalebox{0.7}{$
  \ga{barranco VSB gab}
  $}
$

\bsk

b) $\begin{array}{rcl}
    F_{NW}(x,y) &=& 6 \\
    F_{S}(x,y) &=& 10-2x \\
    F_{C}(x,y) &=& y-x+4 \\
    F_{E}(x,y) &=& y-3 \\
    F_{SE}(x,y) &=& 0 \\
    \end{array}
   $

\bsk

c) $\begin{array}{rcl}
    P_S &=& \setofxyzst{z = 10-2x}, \\
    P_C &=& \setofxyzst{z = y-x+4}, \\
    r &=& \setofxyzst{z = 10-2x = y-x+4}, \\
      &=& \setofxyzst{z = 10-2x, \; y=6-x}, \\
      &=& \setofst{(x,6-x,10-2x)}{x∈\R}
    \end{array}
   $

}\anothercol{

\def\gra{\vec∇(x,y)}

d) No interior da região...

$NW$: temos $\gra = \VEC{0,0}$,

$S$: temos $\gra = \VEC{-2,0}$,

$C$: temos $\gra = \VEC{-1,-1}$,

$E$: temos $\gra = \VEC{0,1}$,

$SE$: temos $\gra = \VEC{0,0}$.

Nas fronteiras entre as regiões o gradiente não existe.

Vou fazer o desenho depois.

\bsk

e) $h(t) \;=\; \ga{gab 1e}$


}}



% (setq eepitch-preprocess-regexp "^")
% (setq eepitch-preprocess-regexp "^%T ")
%
%T  (eepitch-maxima)
%T  (eepitch-kill)
%T  (eepitch-maxima)
%T myeq : 10-2*x = y-x+4;
%T solve(myeq, y);
%T xyz(x) := [x, 6-x, 10-2*x];
%T xyz(0);
%T xyz(6);


\newpage

% «questao-2-gab»  (to ".questao-2-gab")
% (c3m221vsbp 6 "questao-2-gab")
% (c3m221vsba    "questao-2-gab")

{\bf Questão 2: gabarito}


\scalebox{0.8}{\def\colwidth{14cm}\firstcol{

Sejam $S=\sqrt{x+ay}$ e $S_0=\sqrt{1+a}$.
Então:
%
$$\begin{array}{rcl}
  H(x,y)      &=& S \\
  H_x(x,y)    &=& 1/(2S) \\
  H_y(x,y)    &=& a/(2S) \\
  H_{xx}(x,y) &=& -1/(4S^3) \\
  H_{xy}(x,y) &=& -a/(4S^3) \\
  H_{yy}(x,y) &=& -a^2/(4S^3) \\
  \end{array}
  \qquad
  \begin{array}{rcl}
  H(1,1)      &=& S_0 \\
  H_x(1,1)    &=& 1/(2S_0) \\
  H_y(1,1)    &=& a/(2S_0) \\
  H_{xx}(1,1) &=& -1/(4S_0^3) \\
  H_{xy}(1,1) &=& -a/(4S_0^3) \\
  H_{yy}(1,1) &=& -a^2/(4S_0^3) \\
  \end{array}
$$
%
$$\begin{array}{rcl}
  H(1+Δx,1+Δy) &=& H(1,1) \\
               &+& H_x(1,1)Δx + H_y(1,1)Δy \\
               &+& H_{xx}(1,1)\frac{Δx^2}{2}
                 + H_{xy}(1,1)ΔxΔy
                 + H_{yy}(1,1)\frac{Δy^2}{2} \\
               &=& S_0 \\
               &+& \frac{1}{2S_0}Δx + \frac{a}{2S_0}ΔxΔy \\
               &+& \frac{-1}{4S_0^3} \frac{Δx^2}{2}
                 + \frac{-a}{4S_0^3} ΔxΔy
                 + \frac{-a^2}{4S_0^3} \frac{Δy^2}{2} \\
  \end{array}
$$

}\anothercol{
}}


% (setq eepitch-preprocess-regexp "^")
% (setq eepitch-preprocess-regexp "^%T ")
%
%T  (eepitch-maxima)
%T  (eepitch-kill)
%T  (eepitch-maxima)
%T H    : sqrt(x+a*y);
%T H_x  : diff(H,   x);
%T H_xx : diff(H_x, x);
%T H_xy : diff(H_x, y);
%T H_y  : diff(H,   y);
%T H_yy : diff(H_y, y);
%T display(H, H_x, H_y, H_xx, H_xy, H_yy);
%T [H0, H_x0, H_y0, H_xx0, H_xy0, H_yy0] :
%T subst([x=1,y=1], [H, H_x, H_y, H_xx, H_xy, H_yy]);

\newpage

{\bf Critérios de correção}

\scalebox{0.5}{\def\colwidth{9cm}\firstcol{

Erros crassos exatamente iguais ao de colegas podem fazer a pessoa
perder mais pontos.

\msk

{\bf Questão 1}

\ssk

Item 1a:

Já fizemos exercícios deste tipo muitas vezes, então aqui a correção é
bem rigorosa. Se alguma das regiões da pessoa tem quatro pontos
não-coplanares ela leva zero neste item.

\msk

Item 1b:

Aqui as funções $F_{NW}$ e $F_{SE}$ são triviais e não contam pontos.
Se a pessoa obtiver três planos diferentes para $F_S$, $F_C$ e $F_E$
fazendo contas claras e legíveis pequenos erros de conta podem ser
perdoados.

\msk

Item 1c:

Aqui a reta correta é a reta

$r \;=\; \setofst{(x, 6-x, 10-2x)}{x∈\R}.$

\ssk

Se a pessoa obteve a projeção desta reta no plano $(x,y)$, que é
$r \;=\; \setofst{(x, 6-x, 10-2x)}{x∈\R}.$

então ela ganha só 0.6 pontos.

Repostas com retas dadas por equações ao invés de retas parametrizadas
são aceitas sem desconto de pontos.


}\anothercol{

Item 1d:

Nós fizemos exercícios deste tipo muitas vezes em sala --- tanto a
parte de calcular gradientes de regiões planas de diagramas de
numerozinhos quanto a parte de representar graficamente somas de
pontos e vetores seguindo esta convenção:


{\footnotesize

% (c3m212tudop 8)
%    http://angg.twu.net/LATEX/2021-2-C3-tudo.pdf#page=8
\url{http://angg.twu.net/LATEX/2021-2-C3-tudo.pdf\#page=8}

}

\ssk

então aqui a correção pode ser bem rigorosa. Eu esperava que os alunos
mostrassem que sabiam que nas regiões NW e SE os gradientes são zero,
que na região S ele aponta pra esquerda, que na região C ele aponta
pra noroeste que e na região E ele aponta pra cima; erros nisso são
considerados erros graves e descontam muitos pontos. Os erros que
descontam poucos pontos são: 1) não reconhecer que nos pontos de
fronteira entre os planos o gradiente não está definido e 2) não
reconhecer que na região S o módulo do gradiente é o dobro do módulo
na região E.



}}


\newpage

{\bf Critérios de correção (cont.)}

\scalebox{0.65}{\def\colwidth{16cm}\firstcol{

Item 1e:

Se a pessoa conseguiu desenhar a reta $Q(t)$ ela ganha 0.1 pontos.

Se além disso ela conseguiu fazer um gráfico da função $h(t)$ em que o
valor de $h(t)$ está correto em todos os valores de $t$ inteiros ela
ganha mais 0.5 pontos.

Se além disso a pessoa conseguiu ver que a inclinação da função $h(t)$
muda quando $t=2.5$ ela ganha os 0.4 que faltam.

\bsk
\bsk

{\bf Questão 2}

Aqui erros de conta no cálculo das derivadas parciais podem ser
perdoados. O mais importante é que a pessoa ponha o resultado final
numa destas formas:
%
$$\def\uu{\_\_}
  \begin{array}{rcl}
    G(1+Δx,1+Δy) &≈& \uu \\
                 &+& \uu Δx + \uu Δy \\
                 &+& \uu Δx^2 +  \uu ΔxΔy + \uu Δy^2 \\
  \\
  \text{ou:}  \\
    G(x,y) &≈& \uu \\
           &+& \uu (x-1)   + \uu (y-1) \\
           &+& \uu (x-1)^2 + \uu (x-1)(y-1) + \uu (y-1)^2 \\
  \end{array}
$$

e manipule os coeficientes de forma coerente.

}\anothercol{



}}


\msk







\newpage

\GenericWarning{Success:}{Success!!!}  % Used by `M-x cv'

\end{document}

% Local Variables:
% coding: utf-8-unix
% ee-tla: "c3vsb"
% ee-tla: "c3m221vsb"
% End:

