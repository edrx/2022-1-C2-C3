% (find-LATEX "2022-1-C3-funcoes-homogeneas.tex")
% (defun c () (interactive) (find-LATEXsh "lualatex -record 2022-1-C3-funcoes-homogeneas.tex" :end))
% (defun C () (interactive) (find-LATEXsh "lualatex 2022-1-C3-funcoes-homogeneas.tex" "Success!!!"))
% (defun D () (interactive) (find-pdf-page      "~/LATEX/2022-1-C3-funcoes-homogeneas.pdf"))
% (defun d () (interactive) (find-pdftools-page "~/LATEX/2022-1-C3-funcoes-homogeneas.pdf"))
% (defun e () (interactive) (find-LATEX "2022-1-C3-funcoes-homogeneas.tex"))
% (defun o () (interactive) (find-LATEX "2021-2-C3-funcoes-homogeneas.tex"))
% (defun u () (interactive) (find-latex-upload-links "2022-1-C3-funcoes-homogeneas"))
% (defun v () (interactive) (find-2a '(e) '(d)))
% (defun d0 () (interactive) (find-ebuffer "2022-1-C3-funcoes-homogeneas.pdf"))
% (defun cv () (interactive) (C) (ee-kill-this-buffer) (v) (g))
%          (code-eec-LATEX "2022-1-C3-funcoes-homogeneas")
% (find-pdf-page   "~/LATEX/2022-1-C3-funcoes-homogeneas.pdf")
% (find-sh0 "cp -v  ~/LATEX/2022-1-C3-funcoes-homogeneas.pdf /tmp/")
% (find-sh0 "cp -v  ~/LATEX/2022-1-C3-funcoes-homogeneas.pdf /tmp/pen/")
%     (find-xournalpp "/tmp/2022-1-C3-funcoes-homogeneas.pdf")
%   file:///home/edrx/LATEX/2022-1-C3-funcoes-homogeneas.pdf
%               file:///tmp/2022-1-C3-funcoes-homogeneas.pdf
%           file:///tmp/pen/2022-1-C3-funcoes-homogeneas.pdf
% http://angg.twu.net/LATEX/2022-1-C3-funcoes-homogeneas.pdf
% (find-LATEX "2019.mk")
% (find-sh0 "cd ~/LUA/; cp -v Pict2e1.lua Pict2e1-1.lua Piecewise1.lua ~/LATEX/")
% (find-sh0 "cd ~/LUA/; cp -v Pict2e1.lua Pict2e1-1.lua Pict3D1.lua ~/LATEX/")
% (find-CN-aula-links "2022-1-C3-funcoes-homogeneas" "3" "c3m221fh" "c3fh")

% «.videos-antigos»	(to "videos-antigos")
%
% «.defs»		(to "defs")
% «.title»		(to "title")
% «.exercicio-1»	(to "exercicio-1")
% «.exercicio-2»	(to "exercicio-2")
% «.exercicio-2-cont»	(to "exercicio-2-cont")
% «.exercicio-3»	(to "exercicio-3")
% «.exercicio-4»	(to "exercicio-4")
% «.exercicio-5»	(to "exercicio-5")
%
% «.djvuize»		(to "djvuize")



% «videos-antigos»  (to ".videos-antigos")
% (c3m212mt2a "video-a")
% (c3m212mt2a "video-b")
% (c3m212mt2a "video-c")


% <videos>
% Video (not yet):
% (find-ssr-links     "c3m221fh" "2022-1-C3-funcoes-homogeneas")
% (code-eevvideo      "c3m221fh" "2022-1-C3-funcoes-homogeneas")
% (code-eevlinksvideo "c3m221fh" "2022-1-C3-funcoes-homogeneas")
% (find-c3m221fhvideo "0:00")

\documentclass[oneside,12pt]{article}
\usepackage[colorlinks,citecolor=DarkRed,urlcolor=DarkRed]{hyperref} % (find-es "tex" "hyperref")
\usepackage{amsmath}
\usepackage{amsfonts}
\usepackage{amssymb}
\usepackage{pict2e}
\usepackage[x11names,svgnames]{xcolor} % (find-es "tex" "xcolor")
\usepackage{colorweb}                  % (find-es "tex" "colorweb")
%\usepackage{tikz}
%
\usepackage{edrx21}               % (find-LATEX "edrx21.sty")
\input edrxaccents.tex            % (find-LATEX "edrxaccents.tex")
\input edrx21chars.tex            % (find-LATEX "edrx21chars.tex")
\input edrxheadfoot.tex           % (find-LATEX "edrxheadfoot.tex")
\input edrxgac2.tex               % (find-LATEX "edrxgac2.tex")
%\usepackage{emaxima}              % (find-LATEX "emaxima.sty")
%
% (find-es "tex" "geometry")
\usepackage[a6paper, landscape,
            top=1.5cm, bottom=.25cm, left=1cm, right=1cm, includefoot
           ]{geometry}
%
\begin{document}

\catcode`\^^J=10
\directlua{dofile "dednat6load.lua"}  % (find-LATEX "dednat6load.lua")
%L dofile "Pict2e1-1.lua"            -- (find-LATEX "Pict2e1-1.lua")
%%L dofile "Piecewise1.lua"           -- (find-LATEX "Piecewise1.lua")
%%L dofile "QVis1.lua"                -- (find-LATEX "QVis1.lua")
%%L dofile "Pict3D1.lua"              -- (find-LATEX "Pict3D1.lua")
%%L Pict2e.__index.suffix = "%"
\pu
\def\pictgridstyle{\color{GrayPale}\linethickness{0.3pt}}
\def\pictaxesstyle{\linethickness{0.5pt}}
\def\pictnaxesstyle{\color{GrayPale}\linethickness{0.5pt}}
\celllower=2.5pt

% «defs»  (to ".defs")
% (find-LATEX "edrx21defs.tex" "colors")
% (find-LATEX "edrx21.sty")

\def\u#1{\par{\footnotesize \url{#1}}}

\def\drafturl{http://angg.twu.net/LATEX/2022-1-C3.pdf}
\def\drafturl{http://angg.twu.net/2022.1-C3.html}
\def\draftfooter{\tiny \href{\drafturl}{\jobname{}} \ColorBrown{\shorttoday{} \hours}}

\def\Rq{\ColorRed{?}}


%  _____ _ _   _                               
% |_   _(_) |_| | ___   _ __   __ _  __ _  ___ 
%   | | | | __| |/ _ \ | '_ \ / _` |/ _` |/ _ \
%   | | | | |_| |  __/ | |_) | (_| | (_| |  __/
%   |_| |_|\__|_|\___| | .__/ \__,_|\__, |\___|
%                      |_|          |___/      
%
% «title»  (to ".title")
% (c3m221fhp 1 "title")
% (c3m221fha   "title")

\thispagestyle{empty}

\begin{center}

\vspace*{1.2cm}

{\bf \Large Cálculo 3 - 2022.1}

\bsk

Aula 29: funções homogêneas

\bsk

Eduardo Ochs - RCN/PURO/UFF

\url{http://angg.twu.net/2022.1-C3.html}

\end{center}

\newpage

% «exercicio-1»  (to ".exercicio-1")
% (c3m221fhp 2 "exercicio-1")
% (c3m221fha   "exercicio-1")

{\bf Introdução}

\def\HH#1{[\text{H}_#1]}
\def\HP#1{[\text{H}'_#1]}
\def\HPP#1{[\text{H}''_#1]}
\def\HHH#1#2#3#4{
  \left(\begin{array}{lrcl}
        F(#3·#1,#3·#2) \;\; = \;\; #3^#4·F(#1,#2) \\
        \end{array}
  \right)}
\def\HHP#1#2#3#4#5#6{
  \left(\begin{array}{lrcl}
        \text{Se}    &  (#3,#4) &=& #5   · (#1,#2) \\
        \text{então} & F(#3,#4) &=& #5^#6·F(#1,#2) \\
        \end{array}
  \right)}

\def\HHPP#1#2#3#4#5#6{
  \left(\begin{array}{lrcl}
        F(#1+#5#3,#2+#5#4) \;\; = \;\; #5^#6F(#1+#3,#2+#4) \\
        \end{array}
  \right)}

\scalebox{0.4}{\def\colwidth{14cm}\firstcol{

No semestre passado eu apresentei funções homogêneas de um jeito que
demorou muito... esse aqui:

\ssk

{\footnotesize

% (c3m212fhp 3 "exercicio-1")
% (c3m212fha   "exercicio-1")
%    http://angg.twu.net/LATEX/2021-2-C3-funcoes-homogeneas.pdf
\url{http://angg.twu.net/LATEX/2021-2-C3-funcoes-homogeneas.pdf}

}

\ssk

Vou tentar outro jeito agora.

Normalmente a gente começa a ouvir falar de funções homogêneas por
polinômios homogêneos, que são polinômios que todos os monômios deles
têm o mesmo grau... por exemplo,
%
$$2x^3y^4 + 5x^4y^3 - 6x^7$$

é um polinômio em duas variáveis, $x$ e $y$, que é homogêneo de grau
7, porque $x^3y^4$, $x^4y^3$, e $x^7$ são monômios de grau 7. Qualquer
polinômio em duas variáveis pode ser decomposto em polinômios
homogêneos; por exemplo:
%
$$\def\gra#1{←\;\text{parte homogênea de grau #1}}
  \begin{array}{rcll}
  F(x,y) &=& a                           & \gra0 \\
         &+& bx + cy                     & \gra1 \\
         &+& dx^2 + exy + fy^2           & \gra2 \\
         &+& gx^3 + hxy^2 + jx^2y + ky^3 & \gra3 \\
         &+& \ldots
  \end{array}
$$

Repare que fica implícito que $a, b, \ldots, k, \ldots$ são
constantes.

\msk

Uma função em duas variáveis, $F(x,y)$, homogênea de grau $k$, é uma
que obedece isso aqui:
%
$$\HH{k} \;\;=\;\; \HHH{x_1}{y_1}{λ}{k}$$

ou, equivalentemente,

$$\HP{k} \;\;=\;\; \HHP{x_1}{y_1}{x_2}{y_2}{λ}{k}$$

O ``$∀x_1,y_1,x_2,y_2,λ∈\R$'' fica implícito. Veja estas páginas da
Wikipedia:

\ssk

{\footnotesize

\url{https://en.wikipedia.org/wiki/Homogeneous_polynomial}

\url{https://en.wikipedia.org/wiki/Homogeneous_function}

}


}\anothercol{

  Eu inventei nomes curtos --- $\HH{k}$ e $\HP{k}$ --- pra essas
  propriedades pra poder usar a operação `$[:=]$' de Cálculo 2,

  \ssk

  {\footnotesize

    % (c2m212introp 11 "exercicio-1-gab")
    % (c2m212introa    "exercicio-1-gab")
    % http://angg.twu.net/LATEX/2021-2-C2-intro.pdf#page=11
    \url{http://angg.twu.net/LATEX/2021-2-C2-intro.pdf\#page=11}
    
  }

  \ssk

  pra obter casos particulares. Por exemplo, se a função $F(x,y)$ é
  homogênea de grau 2 então estes casos particulares valem:
  %
  \sa{subst34}{\bsm{x_1:=3 \\ y_1:=4 \\ x_2:=30 \\ y_2:=40 \\ λ:=10 \\ k:=2 }}
  %
  $$\begin{array}{rcl}
    \HH{k} \ga{subst34} &=& \HHH{3}{4}        {10}{2} \\ \\[-5pt]
    \HP{k} \ga{subst34} &=& \HHP{3}{4}{30}{40}{10}{2}
    \end{array}
  $$

\bsk

Vamos definir
$\HH{1}, \HH{2}, \HH{3}, \ldots, \HP{1}, \HP{2}, \HP{3}, \ldots$ ``do
jeito óbvio'' --- $\HH{2}=\HH{k}[k:=2]$, $\HP{3}=\HH{k}[k:=3]$, etc.
No exercício abaixo você vai entender os detalhes disso.

\bsk
\bsk
\bsk

{\bf Exercício 1.}

Complete:

\ssk

a) $\HH{k} [k:=2] \;=\; \Rq$

\ssk

b) $\HP{k} [k:=3] \;=\; \Rq$

}}


\newpage

% «exercicio-2»  (to ".exercicio-2")
% (c3m221fhp 3 "exercicio-2")
% (c3m221fha   "exercicio-2")

{\bf Exercício 2}

% (find-angg "LUA/Pict2e1-1.lua" "Numerozinhos-test4")
%
%L Numerozinhos.xyn = function (x, y, n)
%L     if n == "." then return nil end
%L     if n == "?" then n = "\\ColorRed{?}" end
%L     return pformat("\\put(%s,%s){\\cell{\\text{%s}}}", x, y, n)
%L   end
%L Pict2e.bounds = PictBounds.new(v(-5,-5), v(5,5))
%L p = Numerozinhos.xynss(-4, -4, 
%L     [[ . . ? . . . . . .
%L        . . . . ? . . ? .
%L        . . . ? ? . ? . ?
%L        . . . . 3 4 5 . .
%L        . . ? ? ? 8 ? ? .
%L        . . ? ? ? . . . .
%L        ? . ? . ? 2 . . .
%L        . . . . . . . . .
%L        . . . . . . ? . .]])
%L p:pgat("pN"):preunitlength("11pt"):sa("Exercicio 2"):output()
\pu

\scalebox{0.7}{\def\colwidth{10cm}\firstcol{

Digamos que $F(x,y)$ é uma função de duas variáveis,

que não precisa ser um polinômio em duas variáveis ---

ela pode ser algo bem mais esquisito.

Digamos que $F(x,y)$ seja homogênea de grau 2.

Digamos que a figura à direita diz coisas que a gente

sabe sobre a função $F$ e coisas que a gente quer

descobrir sobre ela --- por exemplo, o numerozinho

2 na posição $(1,1)$ diz que sabemos que $F(1,1)=4$,

e o `$\Rq$' na posição $(3,3)$ diz que queremos saber

$F(3,3)$. Como $F$ obedece $\HP{2}$, este caso particular

do $\HP{2}$ tem que valer:
%
$$\begin{array}{l}
  \HHP{x_1}{y_1}{x_2}{y_2}{λ}{2} \bsm{x_1:=1 \\ y_1:=1 \\ x_2:=3 \\ y_2:=3 \\ λ:=3} \\
  = \; \HHP 113332
  \end{array}
$$

E portanto $F(3,3) = 3^2 · F(1,1) = 3^2 · 4 = 9·4 = 36$.

}\anothercol{

\vspace*{0cm}

$\ga{Exercicio 2}$

}}

\newpage

% «exercicio-2-cont»  (to ".exercicio-2-cont")
% (c3m221fhp 4 "exercicio-2-cont")
% (c3m221fha   "exercicio-2-cont")

{\bf Exercício 2 (cont.)}

\scalebox{0.7}{\def\colwidth{10cm}\firstcol{

a) O que você tem que pôr em cada `$\Rq$' daqui
%
$$\begin{array}{l}
  \HHP{x_1}{y_1}{x_2}{y_2}{λ}{2} \bsm{x_1:=\Rq \\ y_1:=\Rq \\ x_2:=\Rq \\ y_2:=\Rq \\ λ:=\Rq} \\
  \end{array}
$$

pra descobrir o valor de $F(3,0)$ a partir do valor

de $F(1,0)$?

\ssk

b) Qual vai ser o valor de $F(3,0)$?

\ssk

c) O que você tem que pôr em cada `$\Rq$' acima

pra descobrir $F(-2,4)$ a partir de $F(1,-2)$?

\ssk

d) Qual vai ser o valor de $F(-2,4)$?

\ssk

e) Qual vai ser o valor de $F(-1,-1)$?

\ssk

f) Escreva as contas do item (e) de forma que cada `$=$'

seja fácil de justificar. Baseie-se neste jogo aqui:

\ssk

{\scriptsize

% (c2m221dfip 4 "o-jogo")
% (c2m221dfia   "o-jogo")
%    http://angg.twu.net/LATEX/2022-1-C2-der-fun-inv.pdf#page=4
\url{http://angg.twu.net/LATEX/2022-1-C2-der-fun-inv.pdf\#page=4}

}

\ssk

g) Descubra os valores da $F$ nos outros `$\Rq$'s da

figura à direita.


}\anothercol{

\vspace*{0cm}

$\ga{Exercicio 2}$

}}

\newpage

% «exercicio-3»  (to ".exercicio-3")
% (c3m221fhp 5 "exercicio-3")
% (c3m221fha   "exercicio-3")

{\bf Exercício 3.}

\scalebox{0.8}{\def\colwidth{9cm}\firstcol{

    No exercício 2 nós supusemos que a função $F(x,y)$ era homogênea
    de grau 2. Agora nós vamos usar a mesma figura, \ColorRed{mas
      vamos supor que a $F(x,y)$ é homogênea de grau 1.}

    Digamos --- como no exercício anterior --- que os numerozinhos da
    figura à direita indicam coisas que a gente sabe sobre a função
    $F(x,y)$ e que os `$\Rq$' indicam coisas que a gente quer
    descobrir.

    \msk

    Descubra o valor dessa $F(x,y)$ em cada `$\Rq$' da figura.

}\anothercol{

\vspace*{0cm}

$\ga{Exercicio 2}$

}}


\newpage

% «exercicio-4»  (to ".exercicio-4")
% (c3m221fhp 6 "exercicio-4")
% (c3m221fha   "exercicio-4")

{\bf Exercício 4.}

\scalebox{0.8}{\def\colwidth{9cm}\firstcol{

    No exercício 2 nós supusemos que a função $F(x,y)$ era homogênea
    de grau 2, e no exercício 3 nós supusemos que ela era homogênea de
    grau 1. Agora nós vamos usar a mesma figura, mas vamos supor que a
    $F(x,y)$ é homogênea de grau \ColorRed{3}.

    \msk

    Digamos --- como no exercício anterior --- que os numerozinhos da
    figura à direita indicam coisas que a gente sabe sobre a função
    $F(x,y)$ e que os `$\Rq$' indicam coisas que a gente quer
    descobrir.

    \msk

    Descubra o valor dessa $F(x,y)$ em cada `$\Rq$' da figura.

}\anothercol{

\vspace*{0cm}

$\ga{Exercicio 2}$

}}


\newpage

% «exercicio-5»  (to ".exercicio-5")
% (c3m221fhp 7 "exercicio-5")
% (c3m221fha   "exercicio-5")
% (find-LATEX "Pict2e1-1.lua" "Numerozinhos-test2")
% (find-angg "LUA/Pict3D1.lua" "nff-test2")
% (find-angg "LUA/Pict3D1.lua" "nff")

{\bf Exercício 5.}

Vamos dizer que uma função $F(x,y)$ é {\sl homogênea de grau $k$ em
  torno do ponto $(x_0,y_0)$} quando ela obedece 

$\HHPP {x_0}{Δx}{y_0}{Δy}λk$

%L Pict2e.bounds = PictBounds.new(v(0,0), v(6,5))
%L x0,y0 = 4,3
%L nff = function (str)
%L     return Code.vc("x,y => local Dx,Dy = x-x0,y-y0; return "..str)
%L   end
%L p = Numerozinhos.fromf(v(x0-2,y0-2),v(x0+2,y0+2), nff "Dx*Dy")
%L p:pgat("pN"):preunitlength("11pt"):sa("Exercicio 5"):output()
\pu

$$\ga{Exercicio 5}$$



%\printbibliography

\GenericWarning{Success:}{Success!!!}  % Used by `M-x cv'

\end{document}

%  ____  _             _         
% |  _ \(_)_   ___   _(_)_______ 
% | | | | \ \ / / | | | |_  / _ \
% | |_| | |\ V /| |_| | |/ /  __/
% |____// | \_/  \__,_|_/___\___|
%     |__/                       
%
% «djvuize»  (to ".djvuize")
% (find-LATEXgrep "grep --color -nH --null -e djvuize 2020-1*.tex")


%  __  __       _        
% |  \/  | __ _| | _____ 
% | |\/| |/ _` | |/ / _ \
% | |  | | (_| |   <  __/
% |_|  |_|\__,_|_|\_\___|
%                        
% <make>

 (eepitch-shell)
 (eepitch-kill)
 (eepitch-shell)
# (find-LATEXfile "2019planar-has-1.mk")
make -f 2019.mk STEM=2022-1-C3-funcoes-homogeneas veryclean
make -f 2019.mk STEM=2022-1-C3-funcoes-homogeneas pdf

% Local Variables:
% coding: utf-8-unix
% ee-tla: "c3fh"
% ee-tla: "c3m221fh"
% End:
