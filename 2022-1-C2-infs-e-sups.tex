% (find-LATEX "2022-1-C2-infs-e-sups.tex")
% (defun c () (interactive) (find-LATEXsh "lualatex -record 2022-1-C2-infs-e-sups.tex" :end))
% (defun C () (interactive) (find-LATEXsh "lualatex 2022-1-C2-infs-e-sups.tex" "Success!!!"))
% (defun D () (interactive) (find-pdf-page      "~/LATEX/2022-1-C2-infs-e-sups.pdf"))
% (defun d () (interactive) (find-pdftools-page "~/LATEX/2022-1-C2-infs-e-sups.pdf"))
% (defun e () (interactive) (find-LATEX "2022-1-C2-infs-e-sups.tex"))
% (defun o () (interactive) (find-LATEX "2021-2-C2-infs-e-sups.tex"))
% (defun l () (interactive) (find-LATEX    "Piecewise1.lua"))
% (defun l () (interactive) (find-angg "LUA/Piecewise1.lua"))
% (defun u () (interactive) (find-latex-upload-links "2022-1-C2-infs-e-sups"))
% (defun v () (interactive) (find-2a '(e) '(d)))
% (defun d0 () (interactive) (find-ebuffer "2022-1-C2-infs-e-sups.pdf"))
% (defun cv () (interactive) (C) (ee-kill-this-buffer) (v) (g))
%          (code-eec-LATEX "2022-1-C2-infs-e-sups")
% (find-pdf-page   "~/LATEX/2022-1-C2-infs-e-sups.pdf")
% (find-sh0 "cp -v  ~/LATEX/2022-1-C2-infs-e-sups.pdf /tmp/")
% (find-sh0 "cp -v  ~/LATEX/2022-1-C2-infs-e-sups.pdf /tmp/pen/")
%     (find-xournalpp "/tmp/2022-1-C2-infs-e-sups.pdf")
%   file:///home/edrx/LATEX/2022-1-C2-infs-e-sups.pdf
%               file:///tmp/2022-1-C2-infs-e-sups.pdf
%           file:///tmp/pen/2022-1-C2-infs-e-sups.pdf
% http://angg.twu.net/LATEX/2022-1-C2-infs-e-sups.pdf
% (find-LATEX "2019.mk")
% (find-CN-aula-links "2022-1-C2-infs-e-sups" "2" "c2m221is" "c2is")
% (find-sh0 "cd ~/LUA/; cp -v Pict2e1.lua Pict2e1-1.lua Piecewise1.lua QVis1.lua ~/LATEX/")

% «.defs»			(to "defs")
% «.title»			(to "title")
% «.uma-figura»			(to "uma-figura")
% «.algumas-definicoes»		(to "algumas-definicoes")
% «.links»			(to "links")
% «.exercicio-1»		(to "exercicio-1")
% «.exercicio-1-dicas»		(to "exercicio-1-dicas")
% «.exercicio-1-dicas-2»	(to "exercicio-1-dicas-2")
% «.exercicio-1-dicas-3»	(to "exercicio-1-dicas-3")
% «.exercicio-2»		(to "exercicio-2")
% «.exercicio-2-dica»		(to "exercicio-2-dica")
% «.exercicio-3»		(to "exercicio-3")
% «.exercicio-4»		(to "exercicio-4")
% «.exercicio-5»		(to "exercicio-5")
% «.exercicio-6»		(to "exercicio-6")
% «.infs-e-sups-como-numeros»	(to "infs-e-sups-como-numeros")
% «.aproximacoes-por-cima»	(to "aproximacoes-por-cima")
%   «.particoes-preferidas»	(to "particoes-preferidas")
% «.aproximacoes-por-baixo»	(to "aproximacoes-por-baixo")
% «.exercicio-7»		(to "exercicio-7")
%
% «.djvuize»	(to "djvuize")



% <videos>
% Video (not yet):
% (find-ssr-links     "c2m221is" "2022-1-C2-infs-e-sups")
% (code-eevvideo      "c2m221is" "2022-1-C2-infs-e-sups")
% (code-eevlinksvideo "c2m221is" "2022-1-C2-infs-e-sups")
% (find-c2m221isvideo "0:00")

\documentclass[oneside,12pt]{article}
\usepackage[colorlinks,citecolor=DarkRed,urlcolor=DarkRed]{hyperref} % (find-es "tex" "hyperref")
\usepackage{amsmath}
\usepackage{amsfonts}
\usepackage{amssymb}
\usepackage{pict2e}
\usepackage[x11names,svgnames]{xcolor} % (find-es "tex" "xcolor")
\usepackage{colorweb}                  % (find-es "tex" "colorweb")
%\usepackage{tikz}
%
% (find-dn6 "preamble6.lua" "preamble0")
%\usepackage{proof}   % For derivation trees ("%:" lines)
%\input diagxy        % For 2D diagrams ("%D" lines)
%\xyoption{curve}     % For the ".curve=" feature in 2D diagrams
%
\usepackage{edrx21}               % (find-LATEX "edrx21.sty")
\input edrxaccents.tex            % (find-LATEX "edrxaccents.tex")
\input edrx21chars.tex            % (find-LATEX "edrx21chars.tex")
\input edrxheadfoot.tex           % (find-LATEX "edrxheadfoot.tex")
\input edrxgac2.tex               % (find-LATEX "edrxgac2.tex")
%\usepackage{emaxima}              % (find-LATEX "emaxima.sty")
%
%\usepackage[backend=biber,
%   style=alphabetic]{biblatex}            % (find-es "tex" "biber")
%\addbibresource{catsem-slides.bib}        % (find-LATEX "catsem-slides.bib")
%
% (find-es "tex" "geometry")
\usepackage[a6paper, landscape,
            top=1.5cm, bottom=.25cm, left=1cm, right=1cm, includefoot
           ]{geometry}
%
\begin{document}

\catcode`\^^J=10
\directlua{dofile "dednat6load.lua"}  % (find-LATEX "dednat6load.lua")
%L dofile "Piecewise1.lua"           -- (find-LATEX "Piecewise1.lua")
%L dofile "QVis1.lua"                -- (find-LATEX "QVis1.lua")
%L Pict2e.__index.suffix = "%"
\pu
\def\pictgridstyle{\color{GrayPale}\linethickness{0.3pt}}
\def\pictaxesstyle{\linethickness{0.5pt}}
\celllower=2.5pt

% «defs»  (to ".defs")
% (find-LATEX "edrx21defs.tex" "colors")
% (find-LATEX "edrx21.sty")

\def\u#1{\par{\footnotesize \url{#1}}}

% https://en.wikipedia.org/wiki/Extended_real_number_line
\def\Rext{\overline{\R}}

\celllower=2.5pt

\def\Em#1#2{\underbrace{\mathstrut #2}_{\text{em }#1}}
\def\Emx #1{\underbrace{\mathstrut #1}_{\text{em }x}}
\def\Emy #1{\underbrace{\mathstrut #1}_{\text{em }y}}
\def\Emxy#1{\underbrace{\mathstrut #1}_{\text{em }\R^2}}

\def\Intover     #1#2{\overline {∫}_{#1}#2\,dx}
\def\Intunder    #1#2{\underline{∫}_{#1}#2\,dx}
\def\Intoverunder#1#2{\Intover{#1}{#2} - \Intunder{#1}{#2}}

\def\Intxover     #1#2#3{\overline {∫}_{x=#1}^{x=#2}#3\,dx}
\def\Intxunder    #1#2#3{\underline{∫}_{x=#1}^{x=#2}#3\,dx}

\def\Intoverunder   #1#2{\overline{\underline{∫}}_{#1}      #2\,dx}
\def\Intxoverunder#1#2#3{\overline{\underline{∫}}_{x=#1}^{x=#2} #3\,dx}

\def\sumiN#1{\sum_{i=1}^N #1 (b_i-a_i)}
\def\mname#1{\text{[#1]}}

\def\drafturl{http://angg.twu.net/LATEX/2022-1-C2.pdf}
\def\drafturl{http://angg.twu.net/2022.1-C2.html}
\def\draftfooter{\tiny \href{\drafturl}{\jobname{}} \ColorBrown{\shorttoday{} \hours}}



%  _____ _ _   _                               
% |_   _(_) |_| | ___   _ __   __ _  __ _  ___ 
%   | | | | __| |/ _ \ | '_ \ / _` |/ _` |/ _ \
%   | | | | |_| |  __/ | |_) | (_| | (_| |  __/
%   |_| |_|\__|_|\___| | .__/ \__,_|\__, |\___|
%                      |_|          |___/      
%
% «title»  (to ".title")
% (c2m221isp 1 "title")
% (c2m221isa   "title")

\thispagestyle{empty}

\begin{center}

\vspace*{1.2cm}

{\bf \Large Cálculo 2 - 2022.1}

\bsk

Aula 15: infs e sups

\bsk

Eduardo Ochs - RCN/PURO/UFF

\url{http://angg.twu.net/2022.1-C2.html}

\end{center}

\newpage

% «uma-figura»  (to ".uma-figura")
% (c2m221isp 2 "uma-figura")
% (c2m221isa   "uma-figura")

%L fromep    = PwSpec.fromep
%L thick     = function (th) return "\\linethickness{"..th.."}" end
%L putcellat = function (xy, str) return pformat("\\put%s{\\cell{%s}}", xy, str) end
%L 
%L _minfy,_pinfy = -3,10  
%L _xD,_xL,_xU = -0.4,-0.8,-1.0
%L 
%L p = PictList {
%L   fromep(" (0,<_pinfy-1>)--(0,<_pinfy>)c "),
%L   fromep(" (0,<_minfy+1>)--(0,<_minfy>)c "),
%L   putcellat(v(1.2, _pinfy), "+\\infty"),
%L   putcellat(v(1.2, _minfy), "-\\infty"),
%L   thick("1pt"),
%L   fromep(" (-1,2)--(3,6)--(8,1)--(11,4)  "),
%L   thick("2pt"),
%L   fromep(" (1,0)c--(2,0)c                "):color("red"),
%L   fromep(" (1,4)c--(2,5)c                "):color("orange"),
%L   fromep(" (0,4)c--(0,5)c                "):color("green"),
%L   fromep(" (<_xL>,<_minfy>)c--(<_xL>,4)c "):color("blue"),
%L   fromep(" (<_xD>,4)c--(<_xD>,5)c        "):color("SpringDarkHard"),
%L   fromep(" (<_xU>,5)c--(<_xU>,<_pinfy>)c "):color("violet"),
%L }
%L p = (p
%L       :setbounds(v(-1,0), v(11,7))
%L       :pgat("gat")
%L       :setbounds(v(-1,_minfy), v(11,_pinfy))
%L       :pgat("p")
%L       :preunitlength("15pt")
%L       :sa("Figura com infinitos")
%L     )
%L p:output()
\pu

$\scalebox{1.05}{$\ga{Figura com infinitos}$}$

\newpage

% «algumas-definicoes»  (to ".algumas-definicoes")
% (c2m221isp 3 "algumas-definicoes")
% (c2m221isa   "algumas-definicoes")

{\bf Algumas definições}

Digamos que $f:\R→\R$ e $B⊂\R$.

Vamos definir $\inf(f(B))$ e $\sup(f(B))$

desta forma:

$$\begin{array}{rcl}
  \Rext &=& \R∪\{-∞,+∞\} \\
  C  &=& \setofst{(x,f(x))}{x∈B} \\
  D  &=& \setofst{f(x)}{x∈B} \\
  D' &=& \setofst{y∈\R}{∃x∈B.\ f(x)=y} \\
  L &=& \setofst{y∈\Rext}{∀d∈D.\;y≤d} \\
  U &=& \setofst{y∈\Rext}{∀d∈D.\;d≤y} \\
  (α=\inf(D)) &=& α∈L ∧ (∀ℓ∈L.\;ℓ \le α) \\
  (β=\sup(D)) &=& β∈U ∧ (∀u∈U.\;β \le u) \\
  \end{array}
$$

\newpage

%  _     _       _        
% | |   (_)_ __ | | _____ 
% | |   | | '_ \| |/ / __|
% | |___| | | | |   <\__ \
% |_____|_|_| |_|_|\_\___/
%                         
% «links»  (to ".links")
% (c2m221isp 4 "links")
% (c2m221isa   "links")

{\bf Como visualizar proposições}

O que nós vamos ver agora é uma versão reorganizada

das páginas 11 até 20 do ``Somas 2'' e de algumas

idéias do ``Somas 2 4'', que na verdade se chama

``Comentários sobre o exercício 4 do

``Integrais como somas de retângulos (2)''\,''...

\msk

Links:

\ssk

{\footnotesize

% (c2m212somas2p 11 "tipos")
% (c2m212somas2a    "tipos")
% (c2m212somas2p 12 "exercicio-3")
% (c2m212somas2a    "exercicio-3")
% (c2m212somas2p 13 "definindo-proposicoes")
% (c2m212somas2a    "definindo-proposicoes")
% (c2m212somas2p 14 "para-todo-e-existe")
% (c2m212somas2a    "para-todo-e-existe")
% (c2m212somas2p 15 "visualizando-fas-e-exs")
% (c2m212somas2a    "visualizando-fas-e-exs")
% (c2m212somas2p 16 "visualizando-fas-e-exs-2")
% (c2m212somas2a    "visualizando-fas-e-exs-2")
% (c2m212somas2p 17 "visualizando-fas-e-exs-3")
% (c2m212somas2a    "visualizando-fas-e-exs-3")
% (c2m212somas2p 18 "exercicio-4")
% (c2m212somas2a    "exercicio-4")
%    http://angg.twu.net/LATEX/2021-2-C2-somas-2.pdf#page=11
\url{http://angg.twu.net/LATEX/2021-2-C2-somas-2.pdf#page=11}

% (c2m212somas24p 1 "title")
% (c2m212somas24a   "title")
%    http://angg.twu.net/LATEX/2021-2-C2-somas-2-4.pdf
\url{http://angg.twu.net/LATEX/2021-2-C2-somas-2-4.pdf}

}

\newpage

%  _____                   _      _         _ 
% | ____|_  _____ _ __ ___(_) ___(_) ___   / |
% |  _| \ \/ / _ \ '__/ __| |/ __| |/ _ \  | |
% | |___ >  <  __/ | | (__| | (__| | (_) | | |
% |_____/_/\_\___|_|  \___|_|\___|_|\___/  |_|
%                                             
% «exercicio-1»  (to ".exercicio-1")
% (c2m221isp 5 "exercicio-1")
% (c2m221isa   "exercicio-1")

{\bf Exercício 1.}

% (c2m221somas3p 15 "exercicio-7-figs")
% (c2m221somas3a    "exercicio-7-figs")
%
%L Pict2e.bounds = PictBounds.new(v(0,0), v(11,7))
%L spec   = "(0,3)--(3,6)--(8,1)--(11,4)"
%L pws    = PwSpec.from(spec)
%L curve  = pws:topict()
%L p = PictList { curve:prethickness("2pt") }
%L p:pgat("pgatc"):preunitlength("15pt"):sa("Exercicio 1 bare"):output()
%L p:addputstrat(v(3,6.4), "\\cell{(3,6)}")
%L p:addputstrat(v(8,0.4), "\\cell{(8,1)}")
%L p:pgat("pgatc"):preunitlength("15pt"):sa("Exercicio 1"):output()
\pu
%
\def\EB{\scalebox{0.4}{$\ga{Exercicio 1 bare}$}}
\def\EC{\scalebox{0.4}{$\ga{Exercicio 1}$}}

\def\alefx#1#2{\Em{(x,f(x))}{\Em{(0,#1)}{#1}≤f(#2)}}
\def\alefx#1#2{\Em{(x,f(x))}{\Em{(0,#1)}{#1}≤f(\Em{(#2,0)}{#2})}}
\def\alefx#1#2{\Em{(x,f(x))}{            #1 ≤f(\Em{(#2,0)}{#2})}}
\def\alefx#1#2{\Em{(#2,f(#2))}{          #1 ≤f(\Em{(#2,0)}{#2})}}


Sejam: $f(x) = \EC$\;,

\msk

$B=\{\Emx{7,8,9}\}$,
%
$P(α) = \Em{(0,α)}{∀x∈B.\; \alefx{α}{x} }$.

\msk

Represente graficamente

\begin{tabular}{l}
a) $P(0)$, \\
b) $P(2)$, \\
\end{tabular}
\qquad
\begin{tabular}{l}
c) $P(4)$,   \\
d) $P(1.5)$. \\
\end{tabular}

\msk

Use uma cópia do gráfico da $f$ pra cada uma.

\newpage

% «exercicio-1-dicas»  (to ".exercicio-1-dicas")
% (c2m221isp 6 "exercicio-1-dicas")
% (c2m221isa   "exercicio-1-dicas")

{\bf Dicas pro Exercício 1}

%\scalebox{0.6}{\def\colwidth{15cm}\firstcol{

$$\begin{array}{rcl}
     B &=& \{7,8,9\} \\
  P(α) &=& ∀x∈B. \; α≤f(x) \\
       &=& ∀x∈\{7,8,9\}. \; α≤f(x) \\
       [2.5pt]
       &=& (α≤f(x)) [x:=7] \\
       &∧& (α≤f(x)) [x:=8] \\
       &∧& (α≤f(x)) [x:=9] \\
       [2.5pt]
       &=& (α≤f(7)) ∧ (α≤f(8)) ∧ (α≤f(9)) \\
       &=& (α≤2) ∧ (α≤1) ∧ (α≤2) \\
       [5pt]
  \end{array}
$$


%}\anothercol{
%}}


\newpage


% «exercicio-1-dicas-2»  (to ".exercicio-1-dicas-2")
% (c2m221isp 7 "exercicio-1-dicas-2")
% (c2m221isa   "exercicio-1-dicas-2")

{\bf Dicas pro Exercício 1 (2)}

\scalebox{0.7}{\def\colwidth{15cm}\firstcol{

\bsk

$\begin{array}{rcl}
 %P(α) &=& \Em{(0,α)}{∀\Em{(x,0)}{x}∈B.\; \alefx{α}{x} } \\
  P(α) &=& \Em{(0,α)}{∀           x ∈B.\; \alefx{α}{x} } \\
           [65pt]
       &=& \Em{(0,α)}{ (\alefx{α}{x})[x:=7]
                     ∧ (\alefx{α}{x})[x:=8]
                     ∧ (\alefx{α}{x})[x:=9] } \\
           [65pt]
       &=& \Em{(0,α)}{ (\alefx{α}{7})
                     ∧ (\alefx{α}{8})
                     ∧ (\alefx{α}{9}) } \\
  \end{array}
$

}\anothercol{
}}


\newpage

% «exercicio-1-dicas-3»  (to ".exercicio-1-dicas-3")
% (c2m221isp 7 "exercicio-1-dicas-3")
% (c2m221isa   "exercicio-1-dicas-3")

{\bf Dicas pro Exercício 1 (3)}

%L Pict2e.bounds = PictBounds.new(v(0,0), v(11,7))
%L ex1spec   = "(0,3)--(3,6)--(8,1)--(11,4)"
%L ex1pws    = PwSpec.from(ex1spec)
%L ex1f      = ex1pws:fun()
%L ex1curve  = ex1pws:topict()
%L pd        = PlotDots.new():dims(0.7, 0.4)
%L pd        = PlotDots.new():dims(0.5, 0.3)
%L pd:plot(v(7, 0),   "Red")
%L pd:plot(v(8, 0),   "Red")
%L pd:plot(v(9, 0),   "Red")
%L pd:plot(v(7, 2),   "Orange")
%L pd:plot(v(8, 1),   "Orange", "open")
%L pd:plot(v(9, 2),   "Orange")
%L pd:plot(v(0, 1.5), "Violet", "open")
%L ex1p      = PictList { ex1curve:prethickness("2pt"), pd:topict() }
%L ex1pdef0  = ex1p:pgat("pgatc"):preunitlength("15pt"):sa("P(1.5)")
%L ex1pdef0:output()
\pu

$$\scalebox{1.5}{$\ga{P(1.5)}$}$$




\newpage

%  _____                   _      _         ____  
% | ____|_  _____ _ __ ___(_) ___(_) ___   |___ \ 
% |  _| \ \/ / _ \ '__/ __| |/ __| |/ _ \    __) |
% | |___ >  <  __/ | | (__| | (__| | (_) |  / __/ 
% |_____/_/\_\___|_|  \___|_|\___|_|\___/  |_____|
%                                                 
% «exercicio-2»  (to ".exercicio-2")
% (c2m221isp 6 "exercicio-2")
% (c2m221isa   "exercicio-2")
% (find-pdf-page "~/2022.1-C2/C2-quadros-manha.pdf" 10)
% (c2m212somas24p 4 "subconjunto-do-plano")
% (c2m212somas24a   "subconjunto-do-plano")

{\bf Exercício 2.}


\scalebox{0.60}{\def\colwidth{12cm}\firstcol{

Represente graficamente os seguintes conjuntos:
%
$$\begin{array}{rcl}
  A &=& \setofxyst{x∈[1,2), \; y∈[1,2)} \\
  B &=& \setofst{(x,2x)}{x∈[1,2)} \\
  C &=& \setofxyst{0≤x \;∧\; x+y<2} \\
  \end{array}
$$

Dica: todos eles vão dar subconjuntos do plano feitos de

infinitos pontos, e você vai ter que adaptar as convenções

que usamos pra desenhar intervalos pra desenhar {\sl regiões}.

\msk

Use bolinhas cheias pra indicar ``este ponto pertence ao

conjunto'', bolinhas ocas pra indicar ``este ponto não

pertence ao conjunto'', linhas grossas contínuas pra

indicar ``esse trecho da fronteira pertence ao conjunto''

e linhas tracejadas pra indicar ``esse trecho da fronteira

não pertence ao conjunto''. Por exemplo:

%L Pict2e.bounds = PictBounds.new(v(-1,-1), v(4,3))
%L spec   = [[ (0,0)--(0,2)--(2,2)
%L             (0.3,0)--(0.75,0) (1.25,0)--(1.7,0)
%L             (2,0.3)--(2,0.75) (2,1.25)--(2,1.7)
%L             (0,0)o (2,0)o (2,2)c (0,2)c
%L          ]]
%L pws    = PwSpec.from(spec)
%L curve  = pws:topict()
%L p = PictList {
%L     [[ \def\closeddot{\circle*{0.4}}% ]],
%L     [[ \def\opendot  {\circle*{0.4}\color{white}\circle*{0.3}}% ]],
%L     Pict2e.region0(v(0,0), v(2,0), v(2,2), v(0,2)):Color("Orange"),
%L     curve:prethickness("3pt")
%L  }
%L p:pgat("pgatc"):preunitlength("20pt"):sa("Exercicio 2 exemplo"):output()
\pu
%
$$\ga{Exercicio 2 exemplo}$$



%}\anothercol{
}}

\newpage

% «exercicio-2-dica»  (to ".exercicio-2-dica")
% (c2m221isp 10 "exercicio-2-dica")
% (c2m221isa    "exercicio-2-dica")

{\bf Dica pro exercício 2}

%L Pict2e.bounds = PictBounds.new(v(0,0), v(2,2))
%L spec   = [[ (0,0)--(0,1)--(1,1)--(1,0)--(0,0)--(0,1)
%L          ]]
%L pws    = PwSpec.from(spec)
%L curve  = pws:topict()
%L p = PictList {
%L     [[ \def\closeddot{\circle*{0.4}}% ]],
%L     [[ \def\opendot  {\circle*{0.4}\color{white}\circle*{0.3}}% ]],
%L     Pict2e.region0(v(0,0), v(1,0), v(1,1), v(0,1)):Color("Orange"),
%L     -- curve:prethickness("3pt")
%L  }
%L p:pgat("pgatc"):preunitlength("10pt"):sa("Prop A'"):output()
%L
%L Pict2e.bounds = PictBounds.new(v(0,0), v(2,2))
%L spec   = [[ (0,0)--(0,1)--(1,1)--(1,0)--(0,0)--(0,1)
%L          ]]
%L pws    = PwSpec.from(spec)
%L curve  = pws:topict()
%L p = PictList {
%L     [[ \def\closeddot{\circle*{0.4}}% ]],
%L     [[ \def\opendot  {\circle*{0.4}\color{white}\circle*{0.3}}% ]],
%L     Pict2e.region0(v(0,0), v(1,0), v(1,1), v(0,1)):Color("Orange"),
%L     curve:prethickness("2pt")
%L  }
%L p:pgat("pgatc"):preunitlength("10pt"):sa("Prop A''"):output()
\pu

\scalebox{0.55}{\def\colwidth{10cm}\firstcol{

...ou: como debugar representações gráficas.

Pense num jogo. Os jogadores se chamam $P$ (``proponente''), e $O$
(``oponente''). O $P$ quer encontrar uma representação gráfica pro
conjunto $A$, e o $O$ quer mostrar que o $P$ está errado.

\msk

Digamos que
%
$$A = \setofxyst{x∈[1,2), \; y∈[1,2)}.$$

O $P$ desenha uma representação gráfica \ColorRed{com um nome
  diferente de $A$} e ``propõe'' ela --- por exemplo, o $P$ diz isso
aqui:
%
\pu
%
$$A' = \ga{Prop A'}$$

O oponente $O$ diz: ``verifica o ponto $(1,1)$''. Os dois verificam o
ponto $(1,1)$ do $A'$ e vêem que o desenho do $A'$ é ambíguo no ponto
$(1,1)$, já que esse é um ponto de fronteira e o $P$ não desenhou ele
nem como linha grossa sólida nem com linha tracejada... então a
resposta pra pergunta ``$(1,1)∈A'$?'' não é nem $\True$ nem $\False$,
é ``erro'', e portanto $A≠A'$, e o $P$ ainda não conseguiu a
representação gráfica certa. O oponente $O$ ganha essa rodada, e o $P$
tem que propôr outra representação gráfica.

}\anothercol{

  Aí o $P$ propõe uma outra representação gráfica, \ColorRed{com um
    outro nome, diferente de $A$ e de $A'$}. Por exemplo, $P$ propõe
  isso aqui:
%
$$A'' = \ga{Prop A''}$$

O oponente $O$ diz: ``verifica o ponto $(0,0)$''. Os dois verificam, e
vêem que:
%
$$(0,0)\not∈A, \quad (0,0)∈A''$$

E portanto $A≠A''$, e o $P$ ainda não conseguiu a representação
gráfica certa. O oponente $O$ ganha mais essa rodada.

\bsk

Quando o $P$ propõe um desenho que o $O$ não consegue mostrar que está
errado o $P$ ganha a rodada.

\bsk

Até vocês terem prática vocês vão jogar como o $P$, vão me mostrar as
representações gráficas de vocês, e eu vou jogar como o $O$. Quando
vocês tiverem mais prática vocês vão conseguir chutar representações
gráficas (como o jogador $P$) e testá-las (fazendo o papel do jogador
$O$ vocês mesmos).

}}




\newpage

%  _____                   _      _         _____ 
% | ____|_  _____ _ __ ___(_) ___(_) ___   |___ / 
% |  _| \ \/ / _ \ '__/ __| |/ __| |/ _ \    |_ \ 
% | |___ >  <  __/ | | (__| | (__| | (_) |  ___) |
% |_____/_/\_\___|_|  \___|_|\___|_|\___/  |____/ 
%                                                 
% «exercicio-3»  (to ".exercicio-3")
% (c2m221isp 8 "exercicio-3")
% (c2m221isa   "exercicio-3")

{\bf Exercício 3.}

Aqui as definições são as mesmas do exercício 1,

mas você só vai representar o resultado de cada $P(α)$

em $(0,α)$... não desenhe as coisas que ficavam

sobre o eixo $x$ ou sobre o gráfico da $f$.

\msk

a) Represente graficamente $P(y)$ (obs: em $(0,y)$!)

para $y=0, 0.5, 1, \ldots, 4$.

\msk

b) Represente graficamente $P(y)$ para $y∈[0,4]$.

\msk

c) Represente graficamente $\setofst{y∈[0,4]}{P(y)}$.

% (c2m211somas24p 34 "que-finja-ter-infinitas")
% (c2m211somas24a    "que-finja-ter-infinitas")


\newpage

%  _____                   _      _         _  _   
% | ____|_  _____ _ __ ___(_) ___(_) ___   | || |  
% |  _| \ \/ / _ \ '__/ __| |/ __| |/ _ \  | || |_ 
% | |___ >  <  __/ | | (__| | (__| | (_) | |__   _|
% |_____/_/\_\___|_|  \___|_|\___|_|\___/     |_|  
%                                                  
% «exercicio-4»  (to ".exercicio-4")
% (c2m221isp 11 "exercicio-4")
% (c2m221isa   "exercicio-4")

{\bf Exercício 4.}

Faça o exercício 4 das páginas 18 a 20 do ``Somas 2''.

Link:

\msk

% (c2m212somas2p 18 "exercicio-4")
% (c2m212somas2a    "exercicio-4")

{\footnotesize

% (c2m212somas2p 18)
%    http://angg.twu.net/LATEX/2021-2-C2-somas-2.pdf#page=18
\url{http://angg.twu.net/LATEX/2021-2-C2-somas-2.pdf#page=18}

}

\bsk
\bsk

Obs: esse exercício 4 do semestre passado é meio

bagunçado... todos os meus colegas de graduação

conheciam esse método de visualização, mas ele

era algo informal, que eu nunca vi descrito por

escrito em lugar nenhum...

\ssk

Acho que os exercícios 1, 2 e 3 deste semestre

estão bem mais claros do que o 4 do semestre

passado. ${=}/$


\newpage

%  _____                   _      _         ____  
% | ____|_  _____ _ __ ___(_) ___(_) ___   | ___| 
% |  _| \ \/ / _ \ '__/ __| |/ __| |/ _ \  |___ \ 
% | |___ >  <  __/ | | (__| | (__| | (_) |  ___) |
% |_____/_/\_\___|_|  \___|_|\___|_|\___/  |____/ 
%                                                 
% «exercicio-5»  (to ".exercicio-5")
% (c2m221isp 12 "exercicio-5")
% (c2m221isa    "exercicio-5")

{\bf Exercício 5.}

%L Pict2e.bounds = PictBounds.new(v(0,0), v(9,7))
%L spec   = "(0,3)--(2,1)o (2,3)c (2,5)o--(7,0)"
%L pws    = PwSpec.from(spec)
%L curve  = pws:topict()
%L p = PictList { curve:prethickness("2pt") }
%L p:addputstrat(v(2.7,5.5), "\\cell{(2,5)}")
%L p:addputstrat(v(7.7,0.5), "\\cell{(7,0)}")
%L p:pgat("pgatc"):preunitlength("17pt"):sa("Exercicio 5"):output()
\pu

\msk

Sejam
%
$f(x) = \scalebox{0.5}{$\ga{Exercicio 5}$}$

e $B=[1,3]$.

\msk

Represente graficamente estes conjuntos ---

as definições deles são as mesmas do slide 3:
%
$$\begin{array}{rcl}
  % \Rext &=& \R∪\{-∞,+∞\} \\
  C  &=& \setofst{(x,f(x))}{x∈B} \\
  D  &=& \setofst{f(x)}{x∈B} \\
  D' &=& \setofst{y∈\R}{∃x∈B.\ f(x)=y} \\
  L &=& \setofst{y∈\Rext}{∀d∈D.\;y≤d} \\
  U &=& \setofst{y∈\Rext}{∀d∈D.\;d≤y} \\
  % (α=\inf(D)) &=& α∈L ∧ (∀ℓ∈L.\;ℓ \le α) \\
  % (β=\sup(D)) &=& β∈U ∧ (∀u∈U.\;β \le u) \\
  \end{array}
$$

\newpage

%  _____                   _      _          __   
% | ____|_  _____ _ __ ___(_) ___(_) ___    / /_  
% |  _| \ \/ / _ \ '__/ __| |/ __| |/ _ \  | '_ \ 
% | |___ >  <  __/ | | (__| | (__| | (_) | | (_) |
% |_____/_/\_\___|_|  \___|_|\___|_|\___/   \___/ 
%                                                 
% «exercicio-6»  (to ".exercicio-6")
% (c2m221isp 13 "exercicio-6")
% (c2m221isa    "exercicio-6")

{\bf Exercício 6}

Obs: se você tiver muita dificuldade com o Exercício 5

faça este exercício antes do 5... e se você conseguir

fazer o Exercício 5 direto não faça este aqui.

\msk

Sejam $f(x)=x+2$ e $B=[1,2]$.

\msk

Represente graficamente estes conjuntos ---

as definições deles são as mesmas do slide 3:
%
$$\begin{array}{rcl}
  % \Rext &=& \R∪\{-∞,+∞\} \\
  C  &=& \setofst{(x,f(x))}{x∈B} \\
  D  &=& \setofst{f(x)}{x∈B} \\
  D' &=& \setofst{y∈\R}{∃x∈B.\ f(x)=y} \\
  L &=& \setofst{y∈\Rext}{∀d∈D.\;y≤d} \\
  U &=& \setofst{y∈\Rext}{∀d∈D.\;d≤y} \\
  % (α=\inf(D)) &=& α∈L ∧ (∀ℓ∈L.\;ℓ \le α) \\
  % (β=\sup(D)) &=& β∈U ∧ (∀u∈U.\;β \le u) \\
  \end{array}
$$

\newpage

% «infs-e-sups-como-numeros»  (to ".infs-e-sups-como-numeros")
% (c2m221isp 14 "infs-e-sups-como-numeros")
% (c2m221isa    "infs-e-sups-como-numeros")

{\bf Infs e sups como números}


\scalebox{0.9}{\def\colwidth{9cm}\firstcol{

Dá pra provar que
%
$$\begin{array}{l}
  ∀D∈\Rext.\;∃!α∈\Rext.\;(α=\inf(D)) \\
  ∀D∈\Rext.\;∃!β∈\Rext.\;(β=\sup(D)) \\
  \end{array}
$$

Vamos chamar esses valores de $α$ e $β$

de $\inf(D)$ e $\sup(D)$.

\bsk
\bsk

{\bf Exercício 6.5.}

Calcule:

\msk

\begin{tabular}{ll}
a) $\inf([3,4])$ \qquad & b) $\sup([3,4])$ \\
c) $\inf((3,4))$ \qquad & d) $\sup((3,4))$ \\
e) $\inf(\R)$    \qquad & f) $\sup(\R)$    \\
g) $\inf(\Rext)$ \qquad & h) $\sup(\Rext)$ \\
i) $\inf(∅)$     \qquad & j) $\sup(∅)$ \\
\end{tabular}

}\anothercol{
}}




\newpage

% «aproximacoes-por-cima»  (to ".aproximacoes-por-cima")
% (c2m221isp 16 "aproximacoes-por-cima")
% (c2m221isa    "aproximacoes-por-cima")
% (c2m212dip 15 "aproximacoes-por-cima")
% (c2m212dia    "aproximacoes-por-cima")

{\bf Aproximações por cima}

\scalebox{0.85}{\def\colwidth{12cm}\firstcol{

Mais duas definições:

A ``melhor aproximação por cima'' para a integral de $f$

na partição $P$ é:
%
$$\Intover{P}{f(x)} = \mname{sup}_P,$$

O ``limite das aproximações por cima'' pra integral de $f$

no intervalo $[a,b]$ é:
%
$$\Intxover{a}{b}{f(x)} = \lim_{k→∞} \mname{sup}_{[a,b]_{2^k}},$$

Esse limite também é chamado de a ``integral por cima de $f$

no intervalo $[a,b]$''.

\bsk

A notação $[a,b]_{2^k}$ está explicada aqui:

\ssk

{\scriptsize

% «particoes-preferidas»  (to ".particoes-preferidas")
% (c2m212somas2p 35 "exercicio-13")
% (c2m212somas2a    "exercicio-13")
%    http://angg.twu.net/LATEX/2021-2-C2-somas-2.pdf#page=35
\url{http://angg.twu.net/LATEX/2021-2-C2-somas-2.pdf\#page=35}

}

}\anothercol{
}}



\newpage

% «aproximacoes-por-baixo»  (to ".aproximacoes-por-baixo")
% (c2m221isp 17 "aproximacoes-por-baixo")
% (c2m221isa    "aproximacoes-por-baixo")
% (c2m212dip 16 "aproximacoes-por-baixo")
% (c2m212dia    "aproximacoes-por-baixo")

{\bf Aproximações por baixo}

\scalebox{0.85}{\def\colwidth{12cm}\firstcol{

Mais duas definições:

A ``melhor aproximação por baixo'' para a integral de $f$

na partição $P$ é:
%
$$\Intunder{P}{f(x)} = \mname{inf}_P,$$

O ``limite das aproximações por baixo'' pra integral de $f$

no intervalo $[a,b]$ é:
%
$$\Intxunder{a}{b}{f(x)} = \lim_{k→∞} \mname{inf}_{[a,b]_{2^k}},$$

Esse limite também é chamado de a ``integral por baixo de $f$

no intervalo $[a,b]$''.

}\anothercol{
}}

\newpage

%  _____                   _      _         _____ 
% | ____|_  _____ _ __ ___(_) ___(_) ___   |___  |
% |  _| \ \/ / _ \ '__/ __| |/ __| |/ _ \     / / 
% | |___ >  <  __/ | | (__| | (__| | (_) |   / /  
% |_____/_/\_\___|_|  \___|_|\___|_|\___/   /_/   
%                                                 
% «exercicio-7»  (to ".exercicio-7")
% (c2m221isp 18 "exercicio-7")
% (c2m221isa    "exercicio-7")


{\bf Exercício 7.}

%L Pict2e.bounds = PictBounds.new(v(0,0), v(9,7))
%L spec   = "(0,3)--(2,1)o (2,3)c (2,5)o--(7,0)"
%L pws    = PwSpec.from(spec)
%L curve  = pws:topict()
%L p = PictList { curve:prethickness("2pt") }
%L p:addputstrat(v(2.7,5.5), "\\cell{(2,5)}")
%L p:addputstrat(v(7.7,0.5), "\\cell{(7,0)}")
%L p:pgat("pgatc"):preunitlength("17pt"):sa("Exercicio 5"):output()

Seja:
%
$$f(x) = \scalebox{0.5}{$\ga{Exercicio 5}$}$$

\def\io#1{\Intover {[1,5]_{2^#1}}{f(x)}}
\def\iu#1{\Intunder{[1,5]_{2^#1}}{f(x)}}

Represente graficamente:

\begin{tabular}{ll}
a) $\io0$ & \qquad b) $\iu0$ \\
c) $\io1$ & \qquad d) $\iu1$ \\
e) $\io2$ & \qquad f) $\iu2$ \\
\end{tabular}



\newpage

% «definicao-integral»  (to ".definicao-integral")
% (c2m221isp 19 "definicao-integral")
% (c2m221isa    "definicao-integral")
% (c2m211somas2p 34 "definicao-integral")
% (c2m211somas2a    "definicao-integral")

{\bf A definição de integral}

\ssk

\def\eqa{\overset{\ColorRed{\Downarrow}}{=}}

A nossa definição de $\Intx{a}{b}{f(x)}$ vai ser:
%
$$\Intx     {a}{b}{f(x)} \;\;=\;\;
  \Intxover {a}{b}{f(x)} \;\; \eqa \;\;
  \Intxunder{a}{b}{f(x)}
$$

se a igualdade marcada com `$\eqa$' for verdade.

\msk
\msk

Se a igualdade `$\eqa$' for falsa vamos dizer que:

``$f(x)$ não é integrável no intervalo $[a,b]$'',

``$\Intx{a}{b}{f(x)}$ não está definida'', ou

``$\Intx{a}{b}{f(x)}$ dá erro''.

\msk
\msk

(Compare com $\frac{42}{0}$, que também ``não está definido'', ou ``dá erro''...)

\newpage

% «intoverunder»  (to ".intoverunder")
% (c2m211somas2p 35 "intoverunder")
% (c2m211somas2a    "intoverunder")
%
% «exercicio-15»  (to ".exercicio-15")
% (c2m211somas2p 35 "exercicio-15")
% (c2m211somas2a    "exercicio-15")

{\bf Como esses limites funcionam?}

Em Cálculo 1 você viu que algumas funções não são deriváveis.

Agora nós vamos ver que algumas funções não são integráveis.

O melhor modo de visualizar isso é usando estas definições:
%
$$\begin{array}{rcl}
  \D \Intoverunder{P}{f(x)} &=&
  \D \Intover     {P}{f(x)} -
     \Intunder    {P}{f(x)}
  \\[15pt]
  \D \Intxoverunder{a}{b}{f(x)} &=&
  \D \Intxover     {a}{b}{f(x)} -
     \Intxunder    {a}{b}{f(x)}
  \end{array}
$$


As notações com `$\overline{\underline{∫}}$' representam a diferença
entre uma aproximação

por cima e uma aproximação por baixo, e a gente
vai desenhar os

`$\overline{\underline{∫}}_P$'s como \ColorRed{retângulos flutuando no
  ar}. O `$\overline{\underline{∫}}_{x=a}^{x=b}$' é o limite de
figuras

tipo `$\overline{\underline{∫}}_P$', e ele pode dar algo mais
complicado do que retângulos.

\newpage

{\bf Exercício 8.}

\def\Intoverunder #1#2{\overline{\underline{∫}}_{#1} #2\,dx}


Seja:
%
$$f(x) = \scalebox{0.5}{$\ga{Exercicio 5}$}$$

\def\iou#1{\overline{\underline{∫}}_{[1,5]_{2^#1}} {f(x)} \, dx}

Represente graficamente:

\ssk

a) $\iou0$

b) $\iou1$

c) $\iou2$






\newpage



% {\bf Exercício 15.}
% 
% \def\iou#1{\Intoverunder{[2,10]_{2^#1}}{f(x)}}
% 
% a) Verifique que no exercício 14 você desenhou $\iou0$,
% 
% $\iou1$, $\iou2$, e $\iou3$.
% 
% \msk
% 
% b) Calcule a área dessas quatro diferenças. \ColorRed{Veja o vídeo!}








% (c2m211somas24p 35 "conjuntos-vazios")
% (c2m211somas24a    "conjuntos-vazios")



% (find-pdf-page "~/2022.1-C2/C2-quadros-manha.pdf" 10)


% (c2m212somas2p 11 "tipos")
% (c2m212somas2a    "tipos")

%\printbibliography

% (c2m212somas24p 1 "title")
% (c2m212somas24a   "title")



% https://en.wikipedia.org/wiki/Extended_real_number_line
% (c2m212somas2p 13 "definindo-proposicoes")
% (c2m212somas2a    "definindo-proposicoes")
% (c2m212isp 11 "uma-figura")
% (c2m212isa    "uma-figura")
% (find-LATEXgrep "grep --color=auto -nH --null -e Aroundfwithinftys *.tex *.lua")
% (find-LATEX "2021-2-C2-infs-e-sups.tex" "programa-2")

\GenericWarning{Success:}{Success!!!}  % Used by `M-x cv'

\end{document}

%  ____  _             _         
% |  _ \(_)_   ___   _(_)_______ 
% | | | | \ \ / / | | | |_  / _ \
% | |_| | |\ V /| |_| | |/ /  __/
% |____// | \_/  \__,_|_/___\___|
%     |__/                       
%
% «djvuize»  (to ".djvuize")
% (find-LATEXgrep "grep --color -nH --null -e djvuize 2020-1*.tex")

 (eepitch-shell)
 (eepitch-kill)
 (eepitch-shell)
# (find-fline "~/2022.1-C2/")
# (find-fline "~/LATEX/2022-1-C2/")
# (find-fline "~/bin/djvuize")

cd /tmp/
for i in *.jpg; do echo f $(basename $i .jpg); done

f () { rm -v $1.pdf;  textcleaner -f 50 -o  5 $1.jpg $1.png; djvuize $1.pdf; xpdf $1.pdf }
f () { rm -v $1.pdf;  textcleaner -f 50 -o 10 $1.jpg $1.png; djvuize $1.pdf; xpdf $1.pdf }
f () { rm -v $1.pdf;  textcleaner -f 50 -o 20 $1.jpg $1.png; djvuize $1.pdf; xpdf $1.pdf }

f () { rm -fv $1.png $1.pdf; djvuize $1.pdf }
f () { rm -fv $1.png $1.pdf; djvuize WHITEBOARDOPTS="-m 1.0 -f 15" $1.pdf; xpdf $1.pdf }
f () { rm -fv $1.png $1.pdf; djvuize WHITEBOARDOPTS="-m 1.0 -f 30" $1.pdf; xpdf $1.pdf }
f () { rm -fv $1.png $1.pdf; djvuize WHITEBOARDOPTS="-m 1.0 -f 45" $1.pdf; xpdf $1.pdf }
f () { rm -fv $1.png $1.pdf; djvuize WHITEBOARDOPTS="-m 0.5" $1.pdf; xpdf $1.pdf }
f () { rm -fv $1.png $1.pdf; djvuize WHITEBOARDOPTS="-m 0.25" $1.pdf; xpdf $1.pdf }
f () { cp -fv $1.png $1.pdf       ~/2022.1-C2/
       cp -fv        $1.pdf ~/LATEX/2022-1-C2/
       cat <<%%%
% (find-latexscan-links "C2" "$1")
%%%
}

f 20201213_area_em_funcao_de_theta
f 20201213_area_em_funcao_de_x
f 20201213_area_fatias_pizza



%  __  __       _        
% |  \/  | __ _| | _____ 
% | |\/| |/ _` | |/ / _ \
% | |  | | (_| |   <  __/
% |_|  |_|\__,_|_|\_\___|
%                        
% <make>

 (eepitch-shell)
 (eepitch-kill)
 (eepitch-shell)
# (find-LATEXfile "2019planar-has-1.mk")
make -f 2019.mk STEM=2022-1-C2-infs-e-sups veryclean
make -f 2019.mk STEM=2022-1-C2-infs-e-sups pdf

% Local Variables:
% coding: utf-8-unix
% ee-tla: "c2is"
% ee-tla: "c2m221is"
% End:
