% (find-LATEX "2022-1-C2-dicas-pra-P2.tex")
% (defun c () (interactive) (find-LATEXsh "lualatex -record 2022-1-C2-dicas-pra-P2.tex" :end))
% (defun C () (interactive) (find-LATEXsh "lualatex 2022-1-C2-dicas-pra-P2.tex" "Success!!!"))
% (defun D () (interactive) (find-pdf-page      "~/LATEX/2022-1-C2-dicas-pra-P2.pdf"))
% (defun d () (interactive) (find-pdftools-page "~/LATEX/2022-1-C2-dicas-pra-P2.pdf"))
% (defun e () (interactive) (find-LATEX "2022-1-C2-dicas-pra-P2.tex"))
% (defun o () (interactive) (find-LATEX "2022-1-C2-dicas-pra-P2.tex"))
% (defun u () (interactive) (find-latex-upload-links "2022-1-C2-dicas-pra-P2"))
% (defun v () (interactive) (find-2a '(e) '(d)))
% (defun d0 () (interactive) (find-ebuffer "2022-1-C2-dicas-pra-P2.pdf"))
% (defun cv () (interactive) (C) (ee-kill-this-buffer) (v) (g))
%          (code-eec-LATEX "2022-1-C2-dicas-pra-P2")
% (find-pdf-page   "~/LATEX/2022-1-C2-dicas-pra-P2.pdf")
% (find-sh0 "cp -v  ~/LATEX/2022-1-C2-dicas-pra-P2.pdf /tmp/")
% (find-sh0 "cp -v  ~/LATEX/2022-1-C2-dicas-pra-P2.pdf /tmp/pen/")
%     (find-xournalpp "/tmp/2022-1-C2-dicas-pra-P2.pdf")
%   file:///home/edrx/LATEX/2022-1-C2-dicas-pra-P2.pdf
%               file:///tmp/2022-1-C2-dicas-pra-P2.pdf
%           file:///tmp/pen/2022-1-C2-dicas-pra-P2.pdf
% http://angg.twu.net/LATEX/2022-1-C2-dicas-pra-P2.pdf
% (find-LATEX "2019.mk")
% (find-sh0 "cd ~/LUA/; cp -v Pict2e1.lua Pict2e1-1.lua Piecewise1.lua ~/LATEX/")
% (find-sh0 "cd ~/LUA/; cp -v Pict2e1.lua Pict2e1-1.lua Pict3D1.lua ~/LATEX/")
% (find-sh0 "cd ~/LUA/; cp -v C2Subst1.lua C2Formulas1.lua ~/LATEX/")
% (find-CN-aula-links "2022-1-C2-dicas-pra-P2" "2" "c2m221dp2" "c2d2")

% «.defs»		(to "defs")
% «.title»		(to "title")
% «.campos-de-direcoes»	(to "campos-de-direcoes")
% «.edovs»		(to "edovs")
% «.osc-am»		(to "osc-am")
% «.osc-am-links»	(to "osc-am-links")
% «.osc-am-email»	(to "osc-am-email")
%
% «.djvuize»		(to "djvuize")



% <videos>
% Video (not yet):
% (find-ssr-links     "c2m221dp2" "2022-1-C2-dicas-pra-P2")
% (code-eevvideo      "c2m221dp2" "2022-1-C2-dicas-pra-P2")
% (code-eevlinksvideo "c2m221dp2" "2022-1-C2-dicas-pra-P2")
% (find-c2m221dp2video "0:00")

\documentclass[oneside,12pt]{article}
\usepackage[colorlinks,citecolor=DarkRed,urlcolor=DarkRed]{hyperref} % (find-es "tex" "hyperref")
\usepackage{amsmath}
\usepackage{amsfonts}
\usepackage{amssymb}
\usepackage{pict2e}
\usepackage[x11names,svgnames]{xcolor} % (find-es "tex" "xcolor")
\usepackage{colorweb}                  % (find-es "tex" "colorweb")
%\usepackage{tikz}
%
% (find-dn6 "preamble6.lua" "preamble0")
%\usepackage{proof}   % For derivation trees ("%:" lines)
%\input diagxy        % For 2D diagrams ("%D" lines)
%\xyoption{curve}     % For the ".curve=" feature in 2D diagrams
%
\usepackage{edrx21}               % (find-LATEX "edrx21.sty")
\input edrxaccents.tex            % (find-LATEX "edrxaccents.tex")
\input edrx21chars.tex            % (find-LATEX "edrx21chars.tex")
\input edrxheadfoot.tex           % (find-LATEX "edrxheadfoot.tex")
\input edrxgac2.tex               % (find-LATEX "edrxgac2.tex")
%\usepackage{emaxima}              % (find-LATEX "emaxima.sty")
%
%\usepackage[backend=biber,
%   style=alphabetic]{biblatex}            % (find-es "tex" "biber")
%\addbibresource{catsem-slides.bib}        % (find-LATEX "catsem-slides.bib")
%
% (find-es "tex" "geometry")
\usepackage[a6paper, landscape,
            top=1.5cm, bottom=.25cm, left=1cm, right=1cm, includefoot
           ]{geometry}
%
\begin{document}

\catcode`\^^J=10
\directlua{dofile "dednat6load.lua"}  % (find-LATEX "dednat6load.lua")
%L dofile "Piecewise1.lua"           -- (find-LATEX "Piecewise1.lua")
%L dofile "QVis1.lua"                -- (find-LATEX "QVis1.lua")
%L dofile "Pict3D1.lua"              -- (find-LATEX "Pict3D1.lua")
%L dofile "C2Formulas1.lua"          -- (find-LATEX "C2Formulas1.lua")
%L Pict2e.__index.suffix = "%"
\pu
\def\pictgridstyle{\color{GrayPale}\linethickness{0.3pt}}
\def\pictaxesstyle{\linethickness{0.5pt}}
\def\pictnaxesstyle{\color{GrayPale}\linethickness{0.5pt}}
\celllower=2.5pt

% «defs»  (to ".defs")
% (find-LATEX "edrx21defs.tex" "colors")
% (find-LATEX "edrx21.sty")

\def\u#1{\par{\footnotesize \url{#1}}}

\def\drafturl{http://angg.twu.net/LATEX/2022-1-C2.pdf}
\def\drafturl{http://angg.twu.net/2022.1-C2.html}
\def\draftfooter{\tiny \href{\drafturl}{\jobname{}} \ColorBrown{\shorttoday{} \hours}}



%  _____ _ _   _                               
% |_   _(_) |_| | ___   _ __   __ _  __ _  ___ 
%   | | | | __| |/ _ \ | '_ \ / _` |/ _` |/ _ \
%   | | | | |_| |  __/ | |_) | (_| | (_| |  __/
%   |_| |_|\__|_|\___| | .__/ \__,_|\__, |\___|
%                      |_|          |___/      
%
% «title»  (to ".title")
% (c2m221dp2p 1 "title")
% (c2m221dp2a   "title")

\thispagestyle{empty}

\begin{center}

\vspace*{1.2cm}

{\bf \Large Cálculo 2 - 2022.1}

\bsk

Aula 31.5: dicas pra P2

\bsk

Eduardo Ochs - RCN/PURO/UFF

\url{http://angg.twu.net/2022.1-C2.html}

\end{center}

\newpage


% «campos-de-direcoes»  (to ".campos-de-direcoes")
% (c2m221dp2p 2 "campos-de-direcoes")
% (c2m221dp2a   "campos-de-direcoes")
% (find-books "__analysis/__analysis.el" "trench" "Direction Fields")
% (c2m211edovsp 4 "campos-dirs")
% (c2m211edovsa   "campos-dirs")

{\bf Campos de direções}

A P2 vai ter uma questão de ``desenhe o campo de direções''

que vai ser muito parecida com os itens do Exercício 1 daqui:

\ssk

{\scriptsize

% (c2m211edovsp 5 "exercicio-1")
% (c2m211edovsa   "exercicio-1")
%    http://angg.twu.net/LATEX/2021-1-C2-edovs.pdf#page=5
\url{http://angg.twu.net/LATEX/2021-1-C2-edovs.pdf#page=5}

}

\ssk

A página 7 desse PDF tem links pra três vídeos:

\ssk

{\scriptsize

% (c2m211edovsp 7 "links")
% (c2m211edovsa   "links")
%    http://angg.twu.net/LATEX/2021-1-C2-edovs.pdf#page=7
\url{http://angg.twu.net/LATEX/2021-1-C2-edovs.pdf#page=7}

}

\ssk

Tem uma explicação pra esse Exercício 1 no segundo

vídeo, a partir do 5:54. Use o terceiro link abaixo:

{\scriptsize

\ssk

% (c2m202edovsa "video-1")
% (find-c2m202edovsvideo "5:54" "a gente vai interpretar esse f'(x)")
% (find-c2m202edovsvideo "6:30" "desenhando os coefs angs só desses 10 pontos daqui")
% (find-c2m202edovsvideo "7:37" "Exercício 1")
\url{http://angg.twu.net/eev-videos/2020-2-C2-edovs.mp4}

\url{https://www.youtube.com/watch?v=bNfZUomf1xg}

\url{https://www.youtube.com/watch?v=bNfZUomf1xg\#t=5m54s} (Exercício 1)

}

\msk

Essa questão vai valer um ponto.


\newpage

% «edovs»  (to ".edovs")
% (c2m221dp2p 2 "edovs")
% (c2m221dp2a   "edovs")
% (c2m221p2p 2 "edovs")
% (c2m221p2a   "edovs")



\newpage


% «osc-am»  (to ".osc-am")

% «osc-am-links»  (to ".osc-am-links")
% (find-angg ".emacs" "c2q191")
% (find-angg ".emacs" "c2q191" "(D-(a+ib))(D-(a-ib))f")
% (find-angg ".emacs" "c2q192")
% (find-angg ".emacs" "c2q192" "f''+7f+10f = 0")
% (find-books "__analysis/__analysis.el" "trench")
% (find-books "__analysis/__analysis.el" "trench" "5.1 Homogeneous Linear Equations")
% (find-books "__analysis/__analysis.el" "trench" "6.1 Spring Problems I")
% (find-books "__analysis/__analysis.el" "boyce-diprima")
% (find-books "__analysis/__analysis.el" "boyce-diprima" "3 Second-Order Linear Differential Equations")
% (find-books "__analysis/__analysis.el" "hernandez")
% (find-books "__analysis/__analysis.el" "hernandez" "24 EDOs homogêneas lineares de ordem n 2")
% (find-books "__analysis/__analysis.el" "thomas")
% (find-books "__analysis/__analysis.el" "thomas" "16 Second-Order Differential")

% «osc-am-email»  (to ".osc-am-email")
% Oi! Tudo bem?
% Deixa eu te pedir mais uma referência...
% Digamos que a gente tem uma equação diferencial homogênea de segunda
% ordem com coeficientes constantes, como por exemplo essa aqui:
% 
%   f'' + 7f + 10f = 0
% 
% Se a gente encara isso como
% 
%   (D^2 + 7D + 10) f = 0
%   (D + 5) (D + 2) f = 0
%   (D + 2) (D + 5) f = 0
% 
% a gente descobre num instante as soluções básicas, que são f(x) =
% exp(-5x) e f(x) = exp(-2x), e aí a gente combina elas pra obter
% soluções não-básicas.
% 
% Você sabe de algum livro que explica isso? Eu aprendi isso num curso
% seu e eu costumava apresentar isso sem dar referências, e agora eu fui
% procurar isso nos livros de EDOs que eu tenho aqui e nenhum deles tem
% essa prova via álgebra linear...
% 
% Torcendo pros tsunamis te darem alguns períodos de paz,
%   [[]] =),
%     E.

% {\bf Oscilações amortecidas}




\GenericWarning{Success:}{Success!!!}  % Used by `M-x cv'

\end{document}

%  ____  _             _         
% |  _ \(_)_   ___   _(_)_______ 
% | | | | \ \ / / | | | |_  / _ \
% | |_| | |\ V /| |_| | |/ /  __/
% |____// | \_/  \__,_|_/___\___|
%     |__/                       
%
% «djvuize»  (to ".djvuize")
% (find-LATEXgrep "grep --color -nH --null -e djvuize 2020-1*.tex")

 (eepitch-shell)
 (eepitch-kill)
 (eepitch-shell)
# (find-fline "~/2022.1-C2/")
# (find-fline "~/LATEX/2022-1-C2/")
# (find-fline "~/bin/djvuize")

cd /tmp/
for i in *.jpg; do echo f $(basename $i .jpg); done

f () { rm -v $1.pdf;  textcleaner -f 50 -o  5 $1.jpg $1.png; djvuize $1.pdf; xpdf $1.pdf }
f () { rm -v $1.pdf;  textcleaner -f 50 -o 10 $1.jpg $1.png; djvuize $1.pdf; xpdf $1.pdf }
f () { rm -v $1.pdf;  textcleaner -f 50 -o 20 $1.jpg $1.png; djvuize $1.pdf; xpdf $1.pdf }

f () { rm -fv $1.png $1.pdf; djvuize $1.pdf }
f () { rm -fv $1.png $1.pdf; djvuize WHITEBOARDOPTS="-m 1.0 -f 15" $1.pdf; xpdf $1.pdf }
f () { rm -fv $1.png $1.pdf; djvuize WHITEBOARDOPTS="-m 1.0 -f 30" $1.pdf; xpdf $1.pdf }
f () { rm -fv $1.png $1.pdf; djvuize WHITEBOARDOPTS="-m 1.0 -f 45" $1.pdf; xpdf $1.pdf }
f () { rm -fv $1.png $1.pdf; djvuize WHITEBOARDOPTS="-m 0.5" $1.pdf; xpdf $1.pdf }
f () { rm -fv $1.png $1.pdf; djvuize WHITEBOARDOPTS="-m 0.25" $1.pdf; xpdf $1.pdf }
f () { cp -fv $1.png $1.pdf       ~/2022.1-C2/
       cp -fv        $1.pdf ~/LATEX/2022-1-C2/
       cat <<%%%
% (find-latexscan-links "C2" "$1")
%%%
}

f 20201213_area_em_funcao_de_theta
f 20201213_area_em_funcao_de_x
f 20201213_area_fatias_pizza



%  __  __       _        
% |  \/  | __ _| | _____ 
% | |\/| |/ _` | |/ / _ \
% | |  | | (_| |   <  __/
% |_|  |_|\__,_|_|\_\___|
%                        
% <make>

 (eepitch-shell)
 (eepitch-kill)
 (eepitch-shell)
# (find-LATEXfile "2019planar-has-1.mk")
make -f 2019.mk STEM=2022-1-C2-dicas-pra-P2 veryclean
make -f 2019.mk STEM=2022-1-C2-dicas-pra-P2 pdf

% Local Variables:
% coding: utf-8-unix
% ee-tla: "c2d2"
% ee-tla: "c2m221dp2"
% End:
