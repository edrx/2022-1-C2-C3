% (find-LATEX "2022-1-C2-TFC1.tex")
% (defun c () (interactive) (find-LATEXsh "lualatex -record 2022-1-C2-TFC1.tex" :end))
% (defun C () (interactive) (find-LATEXsh "lualatex 2022-1-C2-TFC1.tex" "Success!!!"))
% (defun D () (interactive) (find-pdf-page      "~/LATEX/2022-1-C2-TFC1.pdf"))
% (defun d () (interactive) (find-pdftools-page "~/LATEX/2022-1-C2-TFC1.pdf"))
% (defun e () (interactive) (find-LATEX "2022-1-C2-TFC1.tex"))
% (defun o () (interactive) (find-LATEX "2021-2-C2-TFC1.tex"))
% (defun u () (interactive) (find-latex-upload-links "2022-1-C2-TFC1"))
% (defun v () (interactive) (find-2a '(e) '(d)))
% (defun d0 () (interactive) (find-ebuffer "2022-1-C2-TFC1.pdf"))
% (defun cv () (interactive) (C) (ee-kill-this-buffer) (v) (g))
%          (code-eec-LATEX "2022-1-C2-TFC1")
% (find-pdf-page   "~/LATEX/2022-1-C2-TFC1.pdf")
% (find-sh0 "cp -v  ~/LATEX/2022-1-C2-TFC1.pdf /tmp/")
% (find-sh0 "cp -v  ~/LATEX/2022-1-C2-TFC1.pdf /tmp/pen/")
%     (find-xournalpp "/tmp/2022-1-C2-TFC1.pdf")
%   file:///home/edrx/LATEX/2022-1-C2-TFC1.pdf
%               file:///tmp/2022-1-C2-TFC1.pdf
%           file:///tmp/pen/2022-1-C2-TFC1.pdf
% http://angg.twu.net/LATEX/2022-1-C2-TFC1.pdf
% (find-LATEX "2019.mk")
% (find-sh0 "cd ~/LUA/; cp -v Pict2e1.lua Pict2e1-1.lua Piecewise1.lua ~/LATEX/")
% (find-CN-aula-links "2022-1-C2-TFC1" "2" "c2m221tfc1" "c2t1")

% «.defs»		(to "defs")
% «.title»		(to "title")
% «.intro-2022»		(to "intro-2022")
%
% «.defs»		(to "defs")
% «.defs-parabola»	(to "defs-parabola")
% «.title»		(to "title")
% «.intro-1»		(to "intro-1")
% «.intro-2»		(to "intro-2")
% «.intro-3»		(to "intro-3")
% «.exemplo-1»		(to "exemplo-1")
% «.exemplo-1-left»	(to "exemplo-1-left")
% «.exercicio-1»	(to "exercicio-1")
% «.exercicio-3»	(to "exercicio-3")
% «.exercicio-4»	(to "exercicio-4")
% «.exercicio-4-dicas»	(to "exercicio-4-dicas")
% «.exercicio-5»	(to "exercicio-5")
% «.exercicio-6»	(to "exercicio-6")
% «.descontinuidades»	(to "descontinuidades")
% «.descontinuidades-2»	(to "descontinuidades-2")
% «.tfc1-complicado-1»	(to "tfc1-complicado-1")
% «.tfc1-complicado-2»	(to "tfc1-complicado-2")
% «.exercicio-7»	(to "exercicio-7")
% «.exercicio-2»	(to "exercicio-2")
% «.tfc1-complicado-3»	(to "tfc1-complicado-3")
% «.TFC2-exemplo»	(to "TFC2-exemplo")
% «.TFC2-exemplo-2»	(to "TFC2-exemplo-2")
% «.exercicio-8»	(to "exercicio-8")
%
% «.djvuize»	(to "djvuize")



% <videos>
% (c2m221tfc1a "video-1")
% Video (not yet):
% (find-ssr-links     "c2m221tfc1" "2022-1-C2-TFC1")
% (code-eevvideo      "c2m221tfc1" "2022-1-C2-TFC1")
% (code-eevlinksvideo "c2m221tfc1" "2022-1-C2-TFC1")
% (find-c2m221tfc1video "0:00")

\documentclass[oneside,12pt]{article}
\usepackage[colorlinks,citecolor=DarkRed,urlcolor=DarkRed]{hyperref} % (find-es "tex" "hyperref")
\usepackage{amsmath}
\usepackage{amsfonts}
\usepackage{amssymb}
\usepackage{pict2e}
\usepackage[x11names,svgnames]{xcolor} % (find-es "tex" "xcolor")
\usepackage{colorweb}                  % (find-es "tex" "colorweb")
%\usepackage{tikz}
%
% (find-dn6 "preamble6.lua" "preamble0")
%\usepackage{proof}   % For derivation trees ("%:" lines)
%\input diagxy        % For 2D diagrams ("%D" lines)
%\xyoption{curve}     % For the ".curve=" feature in 2D diagrams
%
\usepackage{edrx21}               % (find-LATEX "edrx21.sty")
\input edrxaccents.tex            % (find-LATEX "edrxaccents.tex")
\input edrx21chars.tex            % (find-LATEX "edrx21chars.tex")
\input edrxheadfoot.tex           % (find-LATEX "edrxheadfoot.tex")
\input edrxgac2.tex               % (find-LATEX "edrxgac2.tex")
%\usepackage{emaxima}              % (find-LATEX "emaxima.sty")
%
%\usepackage[backend=biber,
%   style=alphabetic]{biblatex}            % (find-es "tex" "biber")
%\addbibresource{catsem-slides.bib}        % (find-LATEX "catsem-slides.bib")
%
% (find-es "tex" "geometry")
\usepackage[a6paper, landscape,
            top=1.5cm, bottom=.25cm, left=1cm, right=1cm, includefoot
           ]{geometry}
%
\begin{document}

\catcode`\^^J=10
\directlua{dofile "dednat6load.lua"}  % (find-LATEX "dednat6load.lua")
%L dofile "Piecewise1.lua"           -- (find-LATEX "Piecewise1.lua")
%L dofile "QVis1.lua"                -- (find-LATEX "QVis1.lua")
%L Pict2e.__index.suffix = "%"
\pu
\def\pictgridstyle{\color{GrayPale}\linethickness{0.3pt}}
\def\pictaxesstyle{\linethickness{0.5pt}}
\celllower=2.5pt

% «defs»  (to ".defs")
% (find-LATEX "edrx21defs.tex" "colors")
% (find-LATEX "edrx21.sty")

% (find-LATEX "2022-1-C2-infs-e-sups.tex" "defs")
\def\Intover     #1#2{\overline {∫}_{#1}#2\,dx}
\def\Intunder    #1#2{\underline{∫}_{#1}#2\,dx}
\def\Intoverunder#1#2{\Intover{#1}{#2} - \Intunder{#1}{#2}}
\def\Intxover     #1#2#3{\overline {∫}_{x=#1}^{x=#2}#3\,dx}
\def\Intxunder    #1#2#3{\underline{∫}_{x=#1}^{x=#2}#3\,dx}
\def\Intoverunder   #1#2{\overline{\underline{∫}}_{#1}      #2\,dx}
\def\Intxoverunder#1#2#3{\overline{\underline{∫}}_{x=#1}^{x=#2} #3\,dx}
\def\IntPoverunder  #1#2{\overline{\underline{∫}}_{#1} #2\,dx}

\def\u#1{\par{\footnotesize \url{#1}}}

\def\drafturl{http://angg.twu.net/LATEX/2022-1-C2.pdf}
\def\drafturl{http://angg.twu.net/2022.1-C2.html}
\def\draftfooter{\tiny \href{\drafturl}{\jobname{}} \ColorBrown{\shorttoday{} \hours}}



%  _____ _ _   _                               
% |_   _(_) |_| | ___   _ __   __ _  __ _  ___ 
%   | | | | __| |/ _ \ | '_ \ / _` |/ _` |/ _ \
%   | | | | |_| |  __/ | |_) | (_| | (_| |  __/
%   |_| |_|\__|_|\___| | .__/ \__,_|\__, |\___|
%                      |_|          |___/      
%
% «title»  (to ".title")
% (c2m221tfc1p 1 "title")
% (c2m221tfc1a   "title")

\thispagestyle{empty}

\begin{center}

\vspace*{1.2cm}

{\bf \Large Cálculo 2 - 2022.1}

\bsk

Aula 23: o TFC1

\bsk

Eduardo Ochs - RCN/PURO/UFF

\url{http://angg.twu.net/2022.1-C2.html}

\end{center}

\newpage

%  ___       _               ____   ___ ____  ____  
% |_ _|_ __ | |_ _ __ ___   |___ \ / _ \___ \|___ \ 
%  | || '_ \| __| '__/ _ \    __) | | | |__) | __) |
%  | || | | | |_| | | (_) |  / __/| |_| / __/ / __/ 
% |___|_| |_|\__|_|  \___/  |_____|\___/_____|_____|
%                                                   
% «intro-2022»  (to ".intro-2022")

{\bf Introdução (2022.1)}


\scalebox{0.9}{\def\colwidth{12cm}\firstcol{

Este PDF é uma versão reescrita deste aqui, de 2021.2:

\ssk

{\footnotesize

% (c2m212tfc1p 2 "intro-1")
% (c2m212tfc1a   "intro-1")
\url{http://angg.twu.net/LATEX/2021-2-C2-TFC1.pdf}

}

\msk

Neste semestre o nosso primeiro mini-teste vai ser na

última aula da semana de 22 a 24 de junho/2022, e ele

vai ser parecido com o mini-teste 3 do semestre passado:

\ssk

{\footnotesize

% (c2m212mt3p 3 "questao")
% (c2m212mt3a   "questao")
\url{http://angg.twu.net/LATEX/2021-2-C2-MT3.pdf}

}

\msk

Os primeiros slides deste PDF são novos e são uma

preparação pra vocês conseguirem fazer o mini-teste.

Depois que vocês estiverem preparados pro mini-teste

a gente provavelmente vai ver o resto do material daqui

mais ou menos na mesma ordem do semestre passado.

\msk

{\bf Dica:} assista este vídeo do semestre passado:

\ssk

{\footnotesize

% (c2m212tfc1a "video-1")

\url{http://www.youtube.com/watch?v=XvzrNtle-c0}

\url{http://angg.twu.net/eev-videos/2021-2-C2-TFC1.mp4}

}

%}\anothercol{
}}


\newpage

{\bf Algumas propriedades da integral}

Dê uma olhada na seção 7.4 do Daniel Miranda:

\ssk

{\scriptsize

% (find-books "__analysis/__analysis.el" "miranda")
% (find-dmirandacalcpage 220 "7.4 Propriedades da Integral")
%    http://hostel.ufabc.edu.br/~daniel.miranda/calculo/calculo.pdf#page=220
\url{http://hostel.ufabc.edu.br/~daniel.miranda/calculo/calculo.pdf\#page=220}

}

Nós queremos que estas três propriedades aqui valham sempre:
%
$$\begin{array}{rclc}
  k\Intx{a}{b}{f(x)} &=& \Intx{a}{b}{kf(x)} & (*) \\[5pt]
  \Intx{a}{b}{f(x)} + \Intx{b}{c}{f(x)} &=& \Intx{a}{c}{f(x)} & (**) \\[5pt]
  \Intx{a}{b}{k} &=& k(b-a) & (***) \\
  \end{array}
$$

ou seja, queremos que elas valham tanto em casos ``normais''

como em casos ``estranhos''...

\newpage



%L para = function (x) return 4*x - x^2 end
%L vex  = function (str) return Code.ve("x => "..str) end
%L rievex = function (str) return Riemann.fromf(vex(str), seq(0, 4, 0.125)) end
%L rievexa = function (str, a, b)
%L     local rie = rievex(str)
%L     return PictList {
%L       rie.pwf:areaify(a, b):Color("Orange"),
%L       rie:lineify(0, 4),
%L     }
%L   end
%L rievexaout = function (str, a, b, name)
%L     local p = rievexa(str, a, b)
%L     p:pgat("pgatc"):sa(name):output()
%L   end
%L
%L Pict2e.bounds = PictBounds.new(v(0,0), v(4,4))
%L rievexaout("para(x)/2", 0, 4, "Fig 1a 1")
%L rievexaout("para(x)",   0, 4, "Fig 1a 2")
%L
%L Pict2e.bounds = PictBounds.new(v(0,-2), v(4,2))
%L rievexaout(" para(x)/2", 0, 4, "Fig 1b 1")
%L rievexaout("-para(x)/2", 0, 4, "Fig 1b 2")
%L
%L Pict2e.bounds = PictBounds.new(v(0,0), v(4,2))
%L rievexaout("para(x)/2", 0, 3, "Fig 2a 1")
%L rievexaout("para(x)/2", 3, 4, "Fig 2a 2")
%L rievexaout("para(x)/2", 0, 4, "Fig 2a 3")
%L
%L Pict2e.bounds = PictBounds.new(v(0,0), v(4,2))
%L rievexaout("para(x)/2", 0, 4, "Fig 2b 1")
%L rievexaout("para(x)/2", 3, 4, "Fig 2b 2")
%L rievexaout("para(x)/2", 0, 3, "Fig 2b 3")
\pu

\unitlength=10pt
\unitlength=7.5pt

\def\undga#1#2{\underbrace{\textstyle #2}_{\ga{#1}}}
\def\undqq#1#2{\underbrace{\textstyle #2}_{???}}

$$\begin{array}{lrclc}
  (*): & k\Intx{a}{b}{f(x)} &=& \Intx{a}{b}{kf(x)} & \\[10pt]
  (*) \bsm{ a:=0 \\
            b:=4 \\
            k=2 }:
      &   2· \undga{Fig 1a 1}{\Intx{0}{4}{f(x)}}
      &=& \undga{Fig 1a 2}{\Intx{0}{4}{2· f(x)}} \\[10pt]
  (*) \bsm{ a:=0 \\
            b:=4 \\
            k=-1 }:
      &   (-1)· \undga{Fig 1b 1}{\Intx{0}{4}{f(x)}}
      &=& \undga{Fig 1b 2}{\Intx{0}{4}{(-1)· f(x)}} \\[10pt]
  \end{array}
$$


$$\begin{array}{lrclc}
  (**): & \Intx{a}{b}{f(x)} + \Intx{b}{c}{f(x)} &=& \Intx{a}{c}{f(x)} \\[5pt]
  (**)  \bsm{ a:=0 \\
              b:=3 \\
              c:=4 }:
        &   \undga{Fig 2a 1}{\Intx{0}{3}{f(x)}}
          + \undga{Fig 2a 2}{\Intx{3}{4}{f(x)}}
        &=& \undga{Fig 2a 3}{\Intx{0}{4}{f(x)}} \\[5pt]
  (**) \bsm{ a:=0 \\
             b:=4 \\
             c:=3 }:
        &   \undga{Fig 2b 1}{\Intx{0}{4}{f(x)}}
          + \undqq{Fig 2b 2}{\Intx{4}{3}{f(x)}}
        &=& \undga{Fig 2b 3}{\Intx{0}{3}{f(x)}} \\[50pt]
        &   \undga{Fig 2b 1}{\Intx{0}{4}{f(x)}}
          - \undga{Fig 2b 2}{\Intx{3}{4}{f(x)}}
        &=& \undga{Fig 2b 3}{\Intx{0}{3}{f(x)}} \\[5pt]
  \end{array}
$$


\newpage

A terceira regra que queremos que valha sempre,

inclusive em casos estranhos, é essa aqui:
%
$$\begin{array}{crclc}
  \Intx{a}{b}{k} &=& k(b-a) & (***) \\
  \end{array}
$$

Ela vai valer também quando $k<0$,

quando $b<a$, e também vamos ter
%
$$\begin{array}{crclc}
  \Intx{a}{b}{f(x)} &=& k(b-a) \\
  \end{array}
$$

a) quando $a≤b$ e $f(x) = k$ em todo ponto de $[a,b]$,

b) quando $a≤b$ e $f(x) = k$ em todo ponto de $(a,b)$,

c) quando $a≤b$ e $f(x) = k$ em todo ponto de $[a,b]$

{\sl exceto num conjunto finito de pontos.}


\newpage

% «exercicio-1»  (to ".exercicio-1")
% (c2m221tfc1p 7 "exercicio-1")
% (c2m221tfc1a   "exercicio-1")

{\bf Exercício 1.}

%L Pict2e.bounds = PictBounds.new(v(0,-2), v(7,4))
%L spec = "(0,0)--(1,0)o (1,2)c--(2,2)o (2,3)c--(4,3)c (4,-1)o--(6,-1)o (6,0)c--(7,0)"
%L pws = PwSpec.from(spec)
%L pws:topict():prethickness("1.5pt"):pgat("pgatc"):sa("Ex 1"):output()
\pu

\unitlength=7.5pt

\ssk

Seja $f(x) \; = \;\; \ga{Ex 1}$ \; .

Note que:

$\Intx{1}{2}{f(x)} = 2·(2-1)$,

$\Intx{3}{4}{f(x)} = 3·(4-3)$,

$\Intx{4}{6}{f(x)} = -1·(6-4)$,

\msk

Calcule:

a) $\Intx{1.5}{2}{f(x)}$

b) $\Intx{2}{4}{f(x)}$

c) $\Intx{1.5}{4}{f(x)}$

d) $\Intx{1.5}{6}{f(x)}$




\newpage

{\bf Exercício 2.}

\msk

Sejam $f(x) \; = \;\; \ga{Ex 1}$

e $F(β) = \Intx{2}{β}{f(x)}$.

\msk

a) Calcule $F(2), F(2.5), F(3), \ldots, F(6)$.

b) Calcule $F(1.5), F(1), F(0.5), F(0)$.


\newpage

% «exercicio-3»  (to ".exercicio-3")
% (c2m221tfc1p 9 "exercicio-3")
% (c2m221tfc1a   "exercicio-3")

{\bf Exercício 3.}

No exercício 2 você obteve alguns valores da função $F(β)$,

mas não todos... por exemplo, você {\sl ainda} não calculou $F(2.1)$.

\msk

a) Desenhe num gráfico só todos os pontos $(x,F(x))$

que você calculou nos itens (a) e (b) do exercício 2.

Dica: o conjunto que você quer desenhar é este aqui:

$\{(0,F(0)), \, (0.5,F(0.5)), \ldots, (6,F(6))\}$.

\msk

b) Tente descobrir --- lendo os próximos slides, assitindo

o vídeo, e discutindo com os seus colegas --- qual é o jeito

certo de ligar os pontos do item (a).


\newpage

% «exercicio-4»  (to ".exercicio-4")
% (c2m221tfc1p 10 "exercicio-4")
% (c2m221tfc1a    "exercicio-4")

{\bf Exercício 4.}

A função $G(x)$ do mini-teste 3 do semestre passado é esta aqui:
%
%L putcellat = function (xy, str) return pformat("\\put%s{\\cell{%s}}", xy, str) end
%L Pict2e.bounds = PictBounds.new(v(0,-2), v(15,5))
%L spec = 
%L   "(0,-1)--(1,-1)--(2,0)--(3,-1)--(3.5,0)--" ..
%L   "(4,1)--(5,0)--(6,1)--(8,1)--(9,5)--(10,2)--(11,4)--(12,3)--(13,4)--(15,2)"
%L pws = PwSpec.from(spec)
%L p = PictList {
%L   pws:topict():prethickness("1.5pt"),
%L   putcellat(v(5, -0.7), "5"),
%L   putcellat(v(10,-0.7), "10")
%L }
%L p:pgat("pgatc"):sa("Ex 4"):output()
\pu
%
$$G(x) \;\;=\;\;\,
    \unitlength=15pt
    \scalebox{0.7}{$\ga{Ex 4}$}
$$

Relembre como calcular coeficientes angulares e derivadas

no olhômetro e faça um gráfico da função $G'(x)$.

\ssk

Dica 1: $G'(3.5)=2$.

Dica 2: $G'(4)$ não existe --- use uma bolinha

vazia pra representar isso no seu gráfico.





\newpage

% «exercicio-4-dicas»  (to ".exercicio-4-dicas")
% (c2m221tfc1p 11 "exercicio-4-dicas")
% (c2m221tfc1a    "exercicio-4-dicas")

{\bf Dicas pro exercício 4}

\scalebox{0.8}{\def\colwidth{12cm}\firstcol{

Se o gráfico da $G(x)$ é um segmento de reta no intervalo $[a,b]$

então a derivada $G'(x)$ é constante no intervalo aberto $(a,b)$,

e podemos calculá-la pelo coeficiente angular de uma reta secante...

\msk

Escolha dois pontos $x_0,x_1∈[a,b]$ com $x_0≠x_1$, e aí faça isto
aqui:
%
$$\begin{array}{rcl}
  (x_0,y_0) &=& (x_0,G(x_0)) \\
  (x_1,y_1) &=& (x_1,G(x_1)) \\
  Δx &=& x_1-x_0 \\
  Δy &=& y_1-y_0 \\
  G'(c) &=& \frac{Δy}{Δx} \\
  \end{array}
$$

A última linha, $G'(c) = \frac{Δy}{Δx}$, vai ser verdade para qualquer
$c∈(a,b)$.

Dê uma olhada no capítulo 2 do Daniel Miranda se precisar:

\ssk

{\scriptsize

% (find-books "__analysis/__analysis.el" "miranda")
% (find-dmirandacalcpage 65 "II Derivadas")
%    http://hostel.ufabc.edu.br/~daniel.miranda/calculo/calculo.pdf#page=65
\url{http://hostel.ufabc.edu.br/~daniel.miranda/calculo/calculo.pdf\#page=65}

}

\ssk

E se você precisar relembrar limites laterais e derivadas

laterais, dê uma olhada das seções 1.4 e 3.2.3 do livro:

\ssk

{\scriptsize

% (find-books "__analysis/__analysis.el" "miranda")
% (find-dmirandacalcpage 22 "1.4 Limites Laterais")
% (find-dmirandacalcpage 74 "3.2.3 Derivadas Laterais")
%    http://hostel.ufabc.edu.br/~daniel.miranda/calculo/calculo.pdf#page=22
\url{http://hostel.ufabc.edu.br/~daniel.miranda/calculo/calculo.pdf\#page=22}

%    http://hostel.ufabc.edu.br/~daniel.miranda/calculo/calculo.pdf#page=74
\url{http://hostel.ufabc.edu.br/~daniel.miranda/calculo/calculo.pdf\#page=74}

}



}\anothercol{
}}



\newpage

%  ___       _                 _                       
% |_ _|_ __ | |_ _ __ ___   __| |_   _  ___ __ _  ___  
%  | || '_ \| __| '__/ _ \ / _` | | | |/ __/ _` |/ _ \ 
%  | || | | | |_| | | (_) | (_| | |_| | (_| (_| | (_) |
% |___|_| |_|\__|_|  \___/ \__,_|\__,_|\___\__,_|\___/ 
%                                                      
% «intro-1»  (to ".intro-1")
% (c2m221tfc1p 2 "intro-1")
% (c2m221tfc1a   "intro-1")

{\bf Introdução (2021.2)}

\scalebox{0.75}{\def\colwidth{12cm}\firstcol{

Digamos que $f:[a,b] \to \R$ é uma função integrável.

Digamos que $c∈[a,b]$.

Digamos que a função $F:[a,b] \to \R$ é \ColorRed{definida} por:
%
$$F(t) \;\; = \Intx{c}{t}{f(x)}.$$

O TFC1 tem duas versões.

A versão mais simples diz o seguinte:

se a função $f$ é contínua então para todo $t∈(a,b)$ vale:
%
$$F'(t) \;\; = f(t). \qquad \qquad (*)$$

A versão mais complicada do TFC1, que vamos ver

depois, não supõe que a função $f$ é contínua.

\msk

Nós vamos ver um argumento visual que mostra que

a igualdade $(*)$ é verdade. Esse argumento visual é

\ColorRed{quase} uma demonstração formal, num sentido que eu

vou explicar depois.

}}



\newpage

% «intro-2»  (to ".intro-2")
% (c2m221tfc1p 3 "intro-2")
% (c2m221tfc1a   "intro-2")

{\bf Introdução (2)}

\scalebox{0.75}{\def\colwidth{12cm}\firstcol{

Digamos que $f:[a,b] \to \R$ é uma função \ColorRed{contínua}.

Digamos que $c∈[a,b]$.

Digamos que a função $F:[a,b] \to \R$ é \ColorRed{definida} por:
%
$$F(t) \;\; = \Intx{c}{t}{f(x)}.$$

\def\eqq{\overset{\ColorRed{???}}{=}}

Então:
%
$$\begin{array}{rcl}
  F'(t) &=& \D \lim_{ε→0} \frac{F(t+ε)-F(t)}{ε} \\
        &=& \D \lim_{ε→0} \frac{ \Intx{c}{t+ε}{f(x)} - \Intx{c}{t}{f(x)} }{ε} \\
        &=& \D \lim_{ε→0} \frac{ \Intx{t}{t+ε}{f(x)} }{ε} \\[12pt]
        &=& \D \lim_{ε→0} \frac{1}{ε} \Intx{t}{t+ε}{f(x)}  \\[12pt]
        &\eqq& f(t) \\
  \end{array}
$$


}}


\newpage

% «intro-3»  (to ".intro-3")
% (c2m221tfc1p 4 "intro-3")
% (c2m221tfc1a   "intro-3")

{\bf Introdução (3)}

Digamos que $f:[a,b] \to \R$ é uma função \ColorRed{contínua}.

Digamos que $c∈[a,b]$.

Digamos que a função $F:[a,b] \to \R$ é \ColorRed{definida} por:
%
$$F(t) \;\; = \Intx{c}{t}{f(x)}.$$

O nosso argumento visual vai mostrar que:
%
$$\begin{array}{rcl}
  \D \lim_{ε→0} \frac{1}{ε} \Intx{t}{t+ε}{f(x)}
  &=& f(t). \\
  \end{array}
$$



\newpage

%  _____                          _         _ 
% | ____|_  _____ _ __ ___  _ __ | | ___   / |
% |  _| \ \/ / _ \ '_ ` _ \| '_ \| |/ _ \  | |
% | |___ >  <  __/ | | | | | |_) | | (_) | | |
% |_____/_/\_\___|_| |_| |_| .__/|_|\___/  |_|
%                          |_|                
%
% «exemplo-1»  (to ".exemplo-1")
% (c2m221tfc1p 15 "exemplo-1")
% (c2m221tfc1a    "exemplo-1")

% (find-angg "LUA/Piecewise1.lua" "TFC1-tests")
%
%L Pict2e.bounds = PictBounds.new(v(0,0), v(7,5))
%L tfc1_fig_parabola = function (scale)
%L     local f = function (x) return 4*x - x^2 end
%L     local tfc1 = TFC1.fromf(f, seqn(0, 4, 64))
%L     tfc1:setxts(0,1,4, 5, scale):setpwg()
%L     local p = PictList {
%L         tfc1:areaify_f():Color("Orange"),
%L         tfc1:areaify_g():Color("Orange"),
%L         tfc1:lineify_f(),
%L         tfc1:lineify_g(),
%L       }
%L     return p
%L   end
%L
%L tfc1_fig_parabola(1/2):pgat("pgat"):sa("TFC1 parabola 1/2"):output()
%L tfc1_fig_parabola(1)  :pgat("pgat"):sa("TFC1 parabola 1"):output()
%L tfc1_fig_parabola(2)  :pgat("pgat"):sa("TFC1 parabola 2"):output()
%L tfc1_fig_parabola(4)  :pgat("pgat"):sa("TFC1 parabola 4"):output()
%L tfc1_fig_parabola(8)  :pgat("pgat"):sa("TFC1 parabola 8"):output()
%L tfc1_fig_parabola(16) :pgat("pgat"):sa("TFC1 parabola 16"):output()
%L tfc1_fig_parabola(32) :pgat("pgat"):sa("TFC1 parabola 32"):output()
%L tfc1_fig_parabola(64) :pgat("pgat"):sa("TFC1 parabola 64"):output()
%L tfc1_fig_parabola(-1) :pgat("pgat"):sa("TFC1 parabola -1"):output()
%L tfc1_fig_parabola(-2) :pgat("pgat"):sa("TFC1 parabola -2"):output()
%L tfc1_fig_parabola(-4) :pgat("pgat"):sa("TFC1 parabola -4"):output()
%L tfc1_fig_parabola(-8) :pgat("pgat"):sa("TFC1 parabola -8"):output()
%L tfc1_fig_parabola(-16):pgat("pgat"):sa("TFC1 parabola -16"):output()
%L tfc1_fig_parabola(-32):pgat("pgat"):sa("TFC1 parabola -32"):output()
%L tfc1_fig_parabola(-64):pgat("pgat"):sa("TFC1 parabola -64"):output()
\pu

\unitlength=10pt

\scalebox{1.0}{\def\colwidth{5cm}\firstcol{

{\bf Primeiro exemplo:}

$f(x)$ é a nossa parábola

preferida, e $t=1$.

\msk

Primeira figura: $ε=2$.

Segunda figura: $ε=1$.

Terceira figura: $ε=1/2$.

\msk

À esquerda: $\Intx{t}{t+ε}{f(x)}$.

À direita: $\frac{1}{ε}\Intx{t}{t+ε}{f(x)}$.

\msk

Repare que a área em

laranja à esquerda sempre

tem base $ε$ e a área em

laranja à direita sempre

tem base $ε·\frac{1}{ε}=1$.


}\anothercol{

\unitlength=10pt

$$\ga{TFC1 parabola 1/2}$$
$$\ga{TFC1 parabola 1}$$
$$\ga{TFC1 parabola 2}$$

}}




\newpage

\unitlength=25pt

\def\myint{\Intx{1}{1+ε}{f(x)}}
\def\myinte#1{
  $\begin{array}{rl}
   \D             \myint & \text{e} \\[15pt]
   \D \frac{1}{ε} \myint & \text{quando $ε=#1$:} \\
   \end{array}
  $}

\msk

\myinte{2}
$$\ga{TFC1 parabola 1/2}$$
\newpage
\myinte{1}
$$\ga{TFC1 parabola 1}$$
\newpage
\myinte{1/2}
$$\ga{TFC1 parabola 2}$$
\newpage
\myinte{1/4}
$$\ga{TFC1 parabola 4}$$
\newpage
\myinte{1/8}
$$\ga{TFC1 parabola 8}$$
\newpage
\myinte{1/16}
$$\ga{TFC1 parabola 16}$$
\newpage
\myinte{1/32}
$$\ga{TFC1 parabola 32}$$
\newpage
\myinte{1/64}
$$\ga{TFC1 parabola 64}$$


\newpage

% «exemplo-1-left»  (to ".exemplo-1-left")
% (c2m221tfc1p 14 "exemplo-1-left")
% (c2m221tfc1a    "exemplo-1-left")

\scalebox{1.0}{\def\colwidth{5cm}\firstcol{

{\bf Agora com $ε$ negativo!...}

\msk

$f(x)$ é a nossa parábola

preferida, e $t=1$.

\msk

Primeira figura: $ε=-1$.

Segunda figura: $ε=-1/2$.

Terceira figura: $ε=-1/4$.

\msk

À esquerda: $\Intx{t}{t+ε}{f(x)}$.

À direita: $\frac{1}{ε}\Intx{t}{t+ε}{f(x)}$.

% \msk
% 
% Repare que a área em
% 
% laranja à esquerda sempre
% 
% tem base $ε$ e a área em
% 
% laranja à direita sempre
% 
% tem base $ε·\frac{1}{ε}=1$.


}\anothercol{

\unitlength=10pt

$$\ga{TFC1 parabola -1}$$
$$\ga{TFC1 parabola -2}$$
$$\ga{TFC1 parabola -4}$$

}}

\newpage

\myinte{-1}
$$\ga{TFC1 parabola -1}$$
\newpage
\myinte{-1/2}
$$\ga{TFC1 parabola -2}$$
\newpage
\myinte{-1/4}
$$\ga{TFC1 parabola -4}$$
\newpage
\myinte{-1/8}
$$\ga{TFC1 parabola -8}$$
\newpage
\myinte{-1/16}
$$\ga{TFC1 parabola -16}$$
\newpage
\myinte{-1/32}
$$\ga{TFC1 parabola -32}$$
\newpage
\myinte{-1/64}
$$\ga{TFC1 parabola -64}$$




\newpage

%  _____                   _      _         ____  
% | ____|_  _____ _ __ ___(_) ___(_) ___   | ___| 
% |  _| \ \/ / _ \ '__/ __| |/ __| |/ _ \  |___ \ 
% | |___ >  <  __/ | | (__| | (__| | (_) |  ___) |
% |_____/_/\_\___|_|  \___|_|\___|_|\___/  |____/ 
%                                                 
% «exercicio-5»  (to ".exercicio-5")
% (c2m221tfc1p 32 "exercicio-5")
% (c2m221tfc1a    "exercicio-5")
% (c2m221tfc1p 22 "exercicio-1")
% (c2m221tfc1a    "exercicio-1")
% (find-angg "LUA/Piecewise1.lua" "TFC1-tests")

%
%L Pict2e.bounds = PictBounds.new(v(0,0), v(7,5))

%L exerc_1_spec = "(0,2)--(1,1)--(2,3)--(3,4)--(4,3)"
%L exerc_2_spec = "(0,2)--(1,0)--(2,1)o (2,2)c (2,3)o--(3,4)--(4,3)"
%L
%L tfc1_exercs_1_2 = function (spec, scale)
%L     local tfc1 = TFC1.fromspec(spec)
%L     tfc1:setxts(0,2,4, 5,  scale)
%L     local p = PictList {
%L         tfc1:areaify_f():Color("Orange"),
%L         tfc1.pws:topict(),
%L       }
%L     return p
%L   end
%L tfc1_exerc1 = function (scale) return tfc1_exercs_1_2(exerc_1_spec, scale) end
%L tfc1_exerc2 = function (scale) return tfc1_exercs_1_2(exerc_2_spec, scale) end
%L tfc1_exerc1(1/2) :pgat("pgat"):sa("TFC1 exerc1 1/2"):output()
%L tfc1_exerc1(1)   :pgat("pgat"):sa("TFC1 exerc1 1"):output()
%L tfc1_exerc1(2)   :pgat("pgat"):sa("TFC1 exerc1 2"):output()
%L tfc1_exerc1(-1/2):pgat("pgat"):sa("TFC1 exerc1 -1/2"):output()
%L tfc1_exerc1(-1)  :pgat("pgat"):sa("TFC1 exerc1 -1"):output()
%L tfc1_exerc1(-2)  :pgat("pgat"):sa("TFC1 exerc1 -2"):output()
%L tfc1_exerc2(1/2):pgat("pgat"):sa("TFC1 exerc2 1/2"):output()
%L tfc1_exerc2(1)  :pgat("pgat"):sa("TFC1 exerc2 1"):output()
%L tfc1_exerc2(2)  :pgat("pgat"):sa("TFC1 exerc2 2"):output()
%L tfc1_exerc2(-1/2):pgat("pgat"):sa("TFC1 exerc2 -1/2"):output()
%L tfc1_exerc2(-1)  :pgat("pgat"):sa("TFC1 exerc2 -1"):output()
%L tfc1_exerc2(-2)  :pgat("pgat"):sa("TFC1 exerc2 -2"):output()
\pu

\scalebox{1.0}{\def\colwidth{6cm}\firstcol{

{\bf Exercício 5.}

Seja $f(x)$ a função à direita.

Seja $t=2$.

\msk

a) Desenhe $\frac{1}{ε}\Intx{t}{t+ε}{f(x)}$

para $ε=2$, $ε=1$, $ε=1/2$. 

\msk

b) Desenhe $\frac{1}{ε}\Intx{t}{t+ε}{f(x)}$

para $ε=-2$, $ε=-1$, $ε=-1/2$. 

\msk

Dica: comece entendendo as

áreas em laranja à direita!

\msk

c) Quanto você acha que dá

$\lim_{ε→0^+} \frac{1}{ε} \Intx{t}{t+ε}{f(x)}$?

\msk

d) Quanto você acha que dá

$\lim_{ε→0^-} \frac{1}{ε} \Intx{t}{t+ε}{f(x)}$?

}\hspace*{-1cm}\anothercol{

\unitlength=7.5pt

$$\ga{TFC1 exerc1 1/2} \quad \ga{TFC1 exerc1 -1/2}$$
$$\ga{TFC1 exerc1 1} \quad \ga{TFC1 exerc1 -1}$$
$$\ga{TFC1 exerc1 2} \quad \ga{TFC1 exerc1 -2}$$

}}


\newpage

% «exercicio-6»  (to ".exercicio-6")
% (c2m221tfc1p 33 "exercicio-6")
% (c2m221tfc1a    "exercicio-6")
% (find-LATEX "edrx21defs.tex" "firstcol-anothercol")

\scalebox{1.0}{\def\colwidth{6cm}\firstcol{

{\bf Exercício 6.}

Seja $f(x)$ a função à direita.

Seja $t=2$.

\msk

a) Desenhe $\frac{1}{ε}\Intx{t}{t+ε}{f(x)}$

para $ε=2$, $ε=1$, $ε=1/2$. 

\msk

b) Desenhe $\frac{1}{ε}\Intx{t}{t+ε}{f(x)}$

para $ε=-2$, $ε=-1$, $ε=-1/2$. 

\msk

Dica: comece entendendo as

áreas em laranja à direita!

\msk

c) Quanto você acha que dá

$\lim_{ε→0^+} \frac{1}{ε} \Intx{t}{t+ε}{f(x)}$?

\msk

d) Quanto você acha que dá

$\lim_{ε→0^-} \frac{1}{ε} \Intx{t}{t+ε}{f(x)}$?

}\hspace*{-1cm}\anothercol{

\unitlength=7.5pt

\def\PPP#1{\ParR{\expr{Pwil(#1)}}}

$$\ga{TFC1 exerc2 1/2} \quad \ga{TFC1 exerc2 -1/2}$$
$$\ga{TFC1 exerc2 1} \quad \ga{TFC1 exerc2 -1}$$
$$\ga{TFC1 exerc2 2} \quad \ga{TFC1 exerc2 -2}$$


}}



\newpage

% «descontinuidades»  (to ".descontinuidades")
% (c2m221tfc1p 34 "descontinuidades")
% (c2m221tfc1a    "descontinuidades")

{\bf Descontinuidades}

% (find-angg "LUA/Piecewise1.lua" "PwFunction-tests")
% (find-angg "LUA/Piecewise1.lua" "PwFunction-tests" "f_parabola_complicada")

%L Pict2e.bounds = PictBounds.new(v(0,0), v(8,5))
%L f_parabola_preferida = function (x)
%L     return 4 - (x-2)^2
%L   end
%L f_parabola_complicada = function (x)
%L     if x <= 4 then return f_parabola_preferida(x) end
%L     if x <  5 then return 5 - x end
%L     if x <  6 then return 7 - x end
%L     if x <  7 then return 3 end
%L     if x == 7 then return 4 end
%L     return 0.5
%L   end
%L pwf = PwFunction.from(f_parabola_complicada, seqn(0, 4, 32), seq(4, 8))
%L pwf:pw(0, 8):pgat("pgatc"):sa("Parabola complicada"):output()
%L
%L f_parabola_complicada_2 = function (x)
%L     if x <= 4 then return f_parabola_preferida(x) end
%L     if x <  5 then return 5 - x end
%L     if x <  6 then return 7 - x end
%L     if x <  7 then return 3 end
%L     if x == 7 then return 5 end
%L     return 0.5
%L   end
%L pwf = PwFunction.from(f_parabola_complicada_2, seqn(0, 4, 32), seq(4, 8))
%L pwf:pw(0, 8):pgat("pgatc"):sa("Parabola complicada 2"):output()
%L
\pu


\scalebox{0.65}{\def\colwidth{13cm}\firstcol{

Digamos que $f:[a,b]→\R$ é uma função qualquer.

Vamos definir o conjunto dos pontos de descontinuidade da $f$,

ou, pra abreviar, o ``conjunto das descontinuidades da $f$'', assim:
%
$$\mathsf{desc}(f) \;\; = \;\;
  \setofst{x∈[a,b]}{f \text{ é descontinua em $x$}}
$$

A expressão ``$f$ tem um número finito de pontos de descontinuidade'',

que eu vou abreviar pra ``$f$ tem finitas descontinuidades'' apesar

disso soar bem estranho em português, vai querer dizer:
%
$$\mathsf{desc}(f) \text{\;\;é um conjunto finito}$$

O conjunto vazio é finito, então toda $f$ contínua ``tem finitas

descontinuidades''. Essa função aqui tem finitas descontinuidades:
%
$$\unitlength=7.5pt
  \ga{Parabola complicada}
$$


A função de Dirichlet, que nós vimos aqui,

% (c2m211somas2p 46 "dirichlet")
% (c2m211somas2a    "dirichlet")

\ssk

{\footnotesize

\url{http://angg.twu.net/LATEX/2021-1-C2-somas-2.pdf\#page=46}

}

tem infinitas descontinuidades.


%}\anothercol{
}}


\newpage

% «descontinuidades-2»  (to ".descontinuidades-2")
% (c2m221tfc1p 35 "descontinuidades-2")
% (c2m221tfc1a    "descontinuidades-2")
% No semestre passado esta figura foi um exercício:
%   (c2m211somas2p 45 "exercicio-18")
%   (c2m211somas2a    "exercicio-18")


{\bf Descontinuidades (2)}

%L Pict2e.bounds = PictBounds.new(v(0,0), v(8,5))
%L rie  = Riemann.fromf(f_parabola_complicada,   seqn(0, 4, 32), seq(4, 8))
%L rie2 = Riemann.fromf(f_parabola_complicada_2, seqn(0, 4, 32), seq(4, 8))
%L rie :setab(0, 7.5)
%L rie2:setab(0, 7.5)
%L parabola_complicada_inf_sup_def = function (n, name) 
%L     local p = PictList {
%L       -- rie:areainfsup(n):Color("Orange"),
%L       -- rie:areasup(n):Color("Orange"),
%L       -- rie:areasup(n):color("orange!50!yellow"),
%L       rie:areasup(n):color("orange!75!white"),
%L       rie:areainf(n):Color("Red"),
%L       rie.pwf:pw(0, 8)
%L     }
%L     p:pgat("pgatc"):sa(name):output()
%L   end
%L parabola_complicada_2_inf_sup_def = function (n, name) 
%L     local p = PictList {
%L       -- rie2:areainfsup(n):Color("Orange"),
%L       -- rie2:areasup(n):Color("Orange"),
%L       -- rie2:areasup(n):color("orange!50!yellow"),
%L       rie2:areasup(n):color("orange!75!white"),
%L       rie2:areainf(n):Color("Red"),
%L       rie2.pwf:pw(0, 8)
%L     }
%L     p:pgat("pgatc"):sa(name):output()
%L   end
%L parabola_complicada_inf_sup_def( 4, "Parabola complicada diff 4")
%L parabola_complicada_inf_sup_def( 8, "Parabola complicada diff 8")
%L parabola_complicada_inf_sup_def(16, "Parabola complicada diff 16")
%L parabola_complicada_inf_sup_def(32, "Parabola complicada diff 32")
%L parabola_complicada_inf_sup_def(64, "Parabola complicada diff 64")
%L parabola_complicada_inf_sup_def(128, "Parabola complicada diff 128")
%L parabola_complicada_2_inf_sup_def( 4, "Parabola complicada 2 diff 4")
%L parabola_complicada_2_inf_sup_def( 8, "Parabola complicada 2 diff 8")
%L parabola_complicada_2_inf_sup_def(16, "Parabola complicada 2 diff 16")
%L parabola_complicada_2_inf_sup_def(32, "Parabola complicada 2 diff 32")
%L parabola_complicada_2_inf_sup_def(64, "Parabola complicada 2 diff 64")
%L parabola_complicada_2_inf_sup_def(128, "Parabola complicada 2 diff 128")
\pu

\unitlength=10pt

Sejam
%
$$f(x) \;=\; \ga{Parabola complicada}
  \quad
  \text{e}
  \quad
  g(x) \;=\; \ga{Parabola complicada 2}
  \; .
$$

  

As figuras dos próximos slides mostram
%
$$\IntPoverunder{[0,7.5]_{2^k}}{f(x)}
  \quad
  \text{e}
  \quad
  \IntPoverunder{[0,7.5]_{2^k}}{g(x)}
$$

para vários valores de $k$. Use-as pra entender porque

``na integral as descontinuidades não importam'' ---

se só tivermos um número finito de descontinuidades.

\unitlength=10pt
\def\paracompn#1{\newpage $$\scalebox{3}{$\ga{Parabola complicada diff #1}$}$$}
\unitlength=20pt
\def\paracompn#1{\newpage $$\scalebox{1.5}{$\ga{Parabola complicada diff #1}$}$$}
\def\paracompn#1{\newpage
  \vspace*{0.4cm}
  $$\scalebox{0.85}{$\ga{Parabola complicada diff #1}$}
    \qquad
    \scalebox{0.85}{$\ga{Parabola complicada 2 diff #1}$}
  $$
}

\paracompn{4}

\paracompn{8}

\paracompn{16}

\paracompn{32}

\paracompn{64}

\paracompn{128}


\newpage

% «tfc1-complicado-1»  (to ".tfc1-complicado-1")
% (c2m221tfc1p 42 "tfc1-complicado-1")
% (c2m221tfc1a    "tfc1-complicado-1")

{\bf A versão complicada do TFC1}

Vou dizer que uma função $f:[a,b]→\R$ é ``boa''

quando ela é integrável e tem finitas descontinuidades.

\msk

(O termo ``função boa'' é péssimo de propósito ---

é pra deixar óbvio que essa é uma definição temporária,

que vai valer só durante poucos slides...)

\msk

Vou dizer que uma função $G:[a,b]→\R$ obedece
%
$$G'(x) = f(x)$$

quando $G$ for contínua em $[a,b]$ e $G$ obedecer isto aqui:
%
$$∀x∈((a,b) \; ∖ \; \mathsf{desc}(f)). \; G'(x)=f(x)$$

ou seja, neste caso ``$G'(x) = f(x)$'' é uma abreviação

pra algo complicado.



\newpage

% «tfc1-complicado-2»  (to ".tfc1-complicado-2")
% (c2m221tfc1p 43 "tfc1-complicado-2")
% (c2m221tfc1a    "tfc1-complicado-2")
% «exercicio-7»  (to ".exercicio-7")
% (c2m221tfc1p 43 "exercicio-7")
% (c2m221tfc1a    "exercicio-7")

{\bf A versão complicada do TFC1 (2)}

%L Pict2e.bounds = PictBounds.new(v(0,-2), v(4,3))
%L spec = "(0,1)--(1,1)o (1,2)c--(2,2)o (2,-1)c--(4,-1)"
%L pws = PwSpec.from(spec)
%L pws:topict():prethickness("1pt"):pgat("pgatc"):sa("TFCcomplicEx"):output()
\pu

\scalebox{0.95}{\def\colwidth{12cm}\firstcol{

Antes de prosseguir vamos fazer um exercício.

\bsk

{\bf Exercício 7.}

Seja:
$$f(x) \;\;=\;\;
  \unitlength=7.5pt
  \ga{TFCcomplicEx}
$$

a) Qual é o domínio da $f$? (Ele está ``implícito no gráfico''...)

b) Encontre uma função $G$ que obedece $G'(x)=f(x)$ e $G(0)=0$.

c) Encontre uma função $H$ que obedece $H'(x)=f(x)$ e $H(0)=1$.

d) Faça o gráfico da função $M(x) = H(x) - G(x)$.

e) Encontre uma função $K$ que obedece $K'(x)=f(x)$ e $K(4)=-1$.

%}\anothercol{
}}


\newpage

% «tfc1-complicado-3»  (to ".tfc1-complicado-3")
% (c2m221tfc1p 43 "tfc1-complicado-3")
% (c2m221tfc1a    "tfc1-complicado-3")

{\bf A versão complicada do TFC1 (3)}

\scalebox{0.6}{\def\colwidth{9cm}\firstcol{

Digamos que $f:[a,b]→\R$ é ``boa''.

Digamos que $c∈[a,b]$ e que $G'(x) = f(x)$.

Digamos que
%
$$F(x) \;\; = \;\; \Intt{c}{x}{f(t)}.$$

Então $F$ e $G$ ``diferem por uma constante'',

como as funções $G$, $H$ e $K$ do exercício 3.

Isso é o ``TFC1 na versão complicada''.

Eu não vou demonstrá-lo. \quad $\frown$

\msk

Seja $k$ essa constante. Temos:
%
$$∀x∈[a,b]. \; G(x) = F(x) + k.$$

\msk

Isso tem um monte de consequências bacanas.

Por exemplo: $F(c) = 0$, $G(c) = k$, e,

se $α,β∈[a,b]$,
%
$$\begin{array}{rcl}
  \Intt{α}{β}{f(t)} &=& \Intt{c}{β}{f(t)} - \Intt{c}{α}{f(t)} \\
                    &=& F(β) - F(α) \\
                    &=& (G(β) - k) - (G(α) - k) \\
                    &=& G(β) - G(α). \\
  \end{array}
$$

}\anothercol{

Isso nos dá um \ColorRed{método} pra calcular integrais

da função $f$. Se $α,β∈[a,b]$,

\msk

1) encontramos \ColorRed{uma} solução $G(x)$

da EDO $G'(x) = f(x)$,

\msk

2) usamos a fórmula
%
$$\Intt{α}{β}{f(t)} \;\; = \;\; G(β) - G(α).$$

\msk

Você viu no exercício anterior que a EDO

$G'(x) = f(x)$ tem infinitas soluções...

Qualquer solução serve, e não precisamos

calcular a constante $k$.

\bsk
\bsk

\ColorRed{Esse método é o TFC2.}

}}

\newpage

{\bf O TFC2}

\msk

Digamos que $f:[a,b]→\R$ é ``boa''.

Digamos que $α,β∈[a,b]$ e que $G'(x) = f(x)$.

\msk

Então:

$$\Intt{α}{β}{f(t)} \;\; = \;\; G(β) - G(α).$$

% \qquad \mname{TFC2}


\newpage

% «TFC2-exemplo»  (to ".TFC2-exemplo")

{\bf TFC2: um exemplo}

\ssk

A nossa parábola preferida é $f(x) = 4 - (x-2)^2$,

ou seja, $f(x) = 4x - x^2$.

Digamos que $G(x) = 2x^2 - \frac{x^3}{3}$.

Então $G'(x) = f(x)$, e o resultado desta

substituição aqui vai dar uma igualdade verdadeira...


$$\left(
  \Intt{α}{β}{f(t)} \;\; = \;\; G(β) - G(α)
  \right)
  \;
  \bmat{
    f(x) := 4x - x^2 \\
    G(x) := 2x^2 - \frac{x^3}{3} \\
    β := 4 \\
    α := 0 \\
  }
$$

\newpage

% «TFC2-exemplo-2»  (to ".TFC2-exemplo-2")

{\bf TFC2: um exemplo (2)}

\scalebox{0.8}{\def\colwidth{12cm}\firstcol{

Temos:
%
$$\begin{array}{c}
    \left(
    \D \Intt{α}{β}{f(t)} \;\; = \;\; G(β) - G(α)
    \right)
    \;
    \bmat{
      f(x) := 4 - (x-2)^2 \\
      G(x) := 2x^2 - \frac{x^3}{3} \\
      β := 4 \\
      α := 0 \\
    }
  \\[25pt]
    = \;\;
    \left( \D
    \Intt{0}{4}{4-(t-2)^2} \;\; = \;\;
      \left(2·4^2-\frac{4^3}{3}\right) -
      \left(2·0^2-\frac{0^3}{3}\right)
    \right)
    \;
  \\
  \end{array}
$$

e:
%
$$\begin{array}{rcl}
    \D \Intt{0}{4}{4-(t-2)^2}
    &=&
      \left(2·4^2-\frac{4^3}{3}\right) -
      \left(2·0^2-\frac{0^3}{3}\right) \\
    &=&
      \left(32 - \frac{64}{3}\right) - 0 \\[5pt]
    &=& \frac{96}{3} - \frac{64}{3} \\[5pt]
    &=& \frac{32}{3}. \\
  \\
  \end{array}
$$

%}\anothercol{
}}


\newpage

% «exercicio-8»  (to ".exercicio-8")
% (c2m221tfc1p 48 "exercicio-8")
% (c2m221tfc1a    "exercicio-8")

{\bf Exercício 8.}


\scalebox{0.8}{\def\colwidth{9cm}\firstcol{

Este exercício vai servir pra explicar porque é

que eu não uso o ``$+C$'' na fórmula do TFC2 ---

mas isso só daqui a várias páginas.

%L Pict2e.bounds = PictBounds.new(v(-5,-4), v(5,4))
%L spec = [[  (-5,0)--(-3,0)o
%L          (-3,-1)c--(-2,-1)o
%L          (-2,-2)c--(-1,-2)o
%L          (-1,-3)c--(0,-3)o
%L          (0,3)o--(1,3)c
%L          (1,2)o--(2,2)c
%L          (2,1)o--(3,1)c
%L          (3,0)o--(5,0)
%L        ]]
%L pws = PwSpec.from(spec)
%L pws:topict():prethickness("1.5pt"):pgat("pgatc"):sa("Intro +C"):output()
\pu

\msk

\unitlength=7.5pt


Sejam
%
$$f(x) \;=\; \ga{Intro +C}
 \quad
 \text{e}
$$
%
$$F(x) \;=\;
  \begin{cases}
    \D α + \Intt{β}{x}{f(t)} & \text{quando $x<0$}, \\[10pt]
    \D γ + \Intt{δ}{x}{f(t)} & \text{quando $0<x$}. \\
  \end{cases}
$$

\msk

Sejam $α=4$, $β=-1$, $γ=3$, $δ=1$.

Faça os gráficos de $F(x)$ de $F'(x)$.

}\anothercol{

{\bf Exercício 9.}

Sejam $f(x)$ e $F(x)$ as

mesmas do exercício 8,

mas agora considere

que $α=3$, $β=-2$,

$γ=6$, $δ=2$.

\msk

Faça os gráficos de

$F(x)$ e de $F'(x)$.


}}





%\printbibliography

\GenericWarning{Success:}{Success!!!}  % Used by `M-x cv'

\end{document}

%  ____  _             _         
% |  _ \(_)_   ___   _(_)_______ 
% | | | | \ \ / / | | | |_  / _ \
% | |_| | |\ V /| |_| | |/ /  __/
% |____// | \_/  \__,_|_/___\___|
%     |__/                       
%
% «djvuize»  (to ".djvuize")
% (find-LATEXgrep "grep --color -nH --null -e djvuize 2020-1*.tex")

 (eepitch-shell)
 (eepitch-kill)
 (eepitch-shell)
# (find-fline "~/2022.1-C2/")
# (find-fline "~/LATEX/2022-1-C2/")
# (find-fline "~/bin/djvuize")

cd /tmp/
for i in *.jpg; do echo f $(basename $i .jpg); done

f () { rm -v $1.pdf;  textcleaner -f 50 -o  5 $1.jpg $1.png; djvuize $1.pdf; xpdf $1.pdf }
f () { rm -v $1.pdf;  textcleaner -f 50 -o 10 $1.jpg $1.png; djvuize $1.pdf; xpdf $1.pdf }
f () { rm -v $1.pdf;  textcleaner -f 50 -o 20 $1.jpg $1.png; djvuize $1.pdf; xpdf $1.pdf }

f () { rm -fv $1.png $1.pdf; djvuize $1.pdf }
f () { rm -fv $1.png $1.pdf; djvuize WHITEBOARDOPTS="-m 1.0 -f 15" $1.pdf; xpdf $1.pdf }
f () { rm -fv $1.png $1.pdf; djvuize WHITEBOARDOPTS="-m 1.0 -f 30" $1.pdf; xpdf $1.pdf }
f () { rm -fv $1.png $1.pdf; djvuize WHITEBOARDOPTS="-m 1.0 -f 45" $1.pdf; xpdf $1.pdf }
f () { rm -fv $1.png $1.pdf; djvuize WHITEBOARDOPTS="-m 0.5" $1.pdf; xpdf $1.pdf }
f () { rm -fv $1.png $1.pdf; djvuize WHITEBOARDOPTS="-m 0.25" $1.pdf; xpdf $1.pdf }
f () { cp -fv $1.png $1.pdf       ~/2022.1-C2/
       cp -fv        $1.pdf ~/LATEX/2022-1-C2/
       cat <<%%%
% (find-latexscan-links "C2" "$1")
%%%
}

f 20201213_area_em_funcao_de_theta
f 20201213_area_em_funcao_de_x
f 20201213_area_fatias_pizza



%  __  __       _        
% |  \/  | __ _| | _____ 
% | |\/| |/ _` | |/ / _ \
% | |  | | (_| |   <  __/
% |_|  |_|\__,_|_|\_\___|
%                        
% <make>

 (eepitch-shell)
 (eepitch-kill)
 (eepitch-shell)
# (find-LATEXfile "2019planar-has-1.mk")
make -f 2019.mk STEM=2022-1-C2-TFC1 veryclean
make -f 2019.mk STEM=2022-1-C2-TFC1 pdf

% Local Variables:
% coding: utf-8-unix
% ee-tla: "c2t1"
% ee-tla: "c2m221tfc1"
% End:

